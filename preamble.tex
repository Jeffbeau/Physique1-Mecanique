% =============================================================================
% PRÉAMBULE - Notes de cours Physique IMQ
% Jean-François - Institut Maritime du Québec
% =============================================================================

% -----------------------------------------------------------------------------
% ENCODAGE ET LANGUE
% -----------------------------------------------------------------------------
\usepackage[utf8]{inputenc}
\usepackage[T1]{fontenc}
\usepackage[french, provide=*]{babel}

% -----------------------------------------------------------------------------
% MISE EN PAGE
% -----------------------------------------------------------------------------
\usepackage[
    letterpaper,
    top=2.5cm,
    bottom=2.5cm,
    left=2.5cm,
    right=2.5cm,
    headheight=14pt
]{geometry}

\usepackage{fancyhdr}
\pagestyle{fancy}
\fancyhf{}
\fancyhead[LE]{\leftmark}
\fancyhead[RO]{\rightmark}
\fancyfoot[C]{\thepage}
\renewcommand{\headrulewidth}{0.4pt}

% -----------------------------------------------------------------------------
% MATHÉMATIQUES
% -----------------------------------------------------------------------------
\usepackage{amsmath, amssymb, amsfonts}
\usepackage{mathtools}
\usepackage{siunitx}
\sisetup{
    locale = FR,
    inter-unit-product = \ensuremath{{}\cdot{}},
    per-mode = symbol,
    group-separator = {\,},
    output-decimal-marker = {,}
}

% -----------------------------------------------------------------------------
% VECTEURS - Notation parenthétique (Ax, Ay) plutôt que vecteurs unitaires
% -----------------------------------------------------------------------------
% Commande pour les vecteurs avec flèche
\newcommand{\vect}[1]{\overrightarrow{#1}}

% Commande pour écrire un vecteur en notation composantes
% Usage: \vectcomp{A}{A_x}{A_y} donne A = (Ax, Ay)
\newcommand{\vectcomp}[3]{\vect{#1} = (#2,\, #3)}

% Commande pour les composantes seules
% Usage: \comp{3}{4} donne (3, 4)
\newcommand{\comp}[2]{(#1,\, #2)}

% Module d'un vecteur
\newcommand{\module}[1]{\left\| \vect{#1} \right\|}

% -----------------------------------------------------------------------------
% TABLEAUX
% -----------------------------------------------------------------------------
\usepackage{array}
\usepackage{booktabs}
\usepackage{tabularx}
\usepackage{multirow}
\usepackage{longtable}
\usepackage{colortbl}

% Colonnes centrées avec largeur fixe
\newcolumntype{C}[1]{>{\centering\arraybackslash}p{#1}}
\newcolumntype{L}[1]{>{\raggedright\arraybackslash}p{#1}}
\newcolumntype{R}[1]{>{\raggedleft\arraybackslash}p{#1}}

% -----------------------------------------------------------------------------
% FIGURES ET GRAPHIQUES
% -----------------------------------------------------------------------------
\usepackage{graphicx}
\usepackage{float}
\usepackage{wrapfig}
\usepackage{subcaption}

\graphicspath{{figures/}}

% TikZ
\usepackage{tikz}
\usetikzlibrary{
    calc,
    angles,
    quotes,
    arrows.meta,
    positioning,
    decorations.pathmorphing,
    decorations.markings,
    patterns,
    shapes.geometric
}

% Style pour les vecteurs en TikZ
\tikzset{
    vecteur/.style={-{Stealth[length=3mm, width=2mm]}, thick, blue},
    vecteur rouge/.style={-{Stealth[length=3mm, width=2mm]}, thick, red},
    vecteur vert/.style={-{Stealth[length=3mm, width=2mm]}, thick, green!60!black},
    vecteur orange/.style={-{Stealth[length=3mm, width=2mm]}, thick, orange},
    axe/.style={-{Stealth[length=2mm]}, thin},
    pointille/.style={dashed, gray},
    angle mark/.style={draw, thick, blue!50},
}

% -----------------------------------------------------------------------------
% COULEURS
% -----------------------------------------------------------------------------
\usepackage{xcolor}
\definecolor{bleuimq}{RGB}{0, 82, 147}
\definecolor{orangeimq}{RGB}{232, 119, 34}
\definecolor{grisimq}{RGB}{128, 130, 133}
\definecolor{vertexemple}{RGB}{34, 139, 34}
\definecolor{rougealerte}{RGB}{178, 34, 34}
\definecolor{bleuclair}{RGB}{230, 242, 255}
\definecolor{grisclair}{RGB}{245, 245, 245}

% --- Commande de difficulté ---
\newcommand{\dif}[1]{%
  \ifcase#1\or$\bigstar$\or$\bigstar\bigstar$\or$\bigstar\bigstar\bigstar$\or\textbf{Synthèse}\fi
}

% -----------------------------------------------------------------------------
% ENCADRÉS ET ENVIRONNEMENTS PERSONNALISÉS
% -----------------------------------------------------------------------------
\usepackage{tcolorbox}
\tcbuselibrary{skins, breakable, theorems}

% D\'efinition
\newtcolorbox{definition}[1][]{
    colback=bleuclair,
    colframe=bleuimq,
    fonttitle=\bfseries,
    title=D\'efinition,
    #1
}

% Exemple (avec compteur num\'erot\'e par chapitre)
\newtcbtheorem[number within=chapter]{exemple}{Exemple}{
    colback=grisclair,
    colframe=grisimq,
    fonttitle=\bfseries,
    breakable,
    separator sign={~--}
}{ex}

% Remarque
\newtcolorbox{remarque}[1][]{
    colback=white,
    colframe=orangeimq,
    fonttitle=\bfseries,
    title=Remarque,
    #1
}

% Attention / Alerte
\newtcolorbox{attention}[1][]{
    colback=red!5,
    colframe=rougealerte,
    fonttitle=\bfseries,
    title=Attention,
    #1
}

% Lois de Newton
\newtcolorbox{loinewton}[1][]{
    colback=bleuclair,
    colframe=bleuimq,
    fonttitle=\bfseries,
    breakable,
    separator sign={~--},
    title={Loi de Newton},
    #1
}

% \'Equation importante
\newtcolorbox{equationimportante}[1][]{
    colback=bleuclair,
    colframe=bleuimq,
    boxrule=1.5pt,
    arc=0pt,
    #1
}

% Pratique autonome (exercices en classe avec espace de résolution)
\definecolor{vertpratique}{RGB}{46, 125, 50}
\definecolor{vertclair}{RGB}{232, 245, 233}

\newcounter{pratique}[chapter]
\renewcommand{\thepratique}{\thechapter.\arabic{pratique}}

\newtcolorbox{pratiqueautonome}[1][]{
    colback=vertclair,
    colframe=vertpratique,
    fonttitle=\bfseries,
    title={$\vartriangleright$ Pratique autonome~\stepcounter{pratique}\thepratique},
    breakable,
    before skip=1em,
    after skip=1em,
    boxrule=1.5pt,
    arc=2pt,
    left=8pt,
    right=8pt,
    top=8pt,
    bottom=8pt,
    #1
}

% Commande pour l'espace de résolution avec lignes pointillées optionnelles
\newcommand{\espaceresolution}[1][6cm]{%
    \par\vspace{0.5em}
    \noindent\textit{Résolution :}
    \par\vspace{#1}
}

% Commande pour afficher la réponse discrètement
\newcommand{\reponsepratique}[1]{%
    \par\vspace{0.3em}
    \hfill{\small\textcolor{grisimq}{Rép. : #1}}
}

% --- Environnement exercice ---
\newcounter{exercice}[chapter]
\renewcommand{\theexercice}{\thechapter.\arabic{exercice}}

\newtcolorbox{exercice}[1][]{
    enhanced,
    colback=white,
    colframe=bleuimq,
    fonttitle=\bfseries,
    title={Exercice~\stepcounter{exercice}\theexercice\hfill #1},
    breakable,
    before skip=1em,
    after skip=1em,
    boxrule=0.8pt,
    arc=2pt,
    left=8pt,
    right=8pt
}

% -----------------------------------------------------------------------------
% LISTES
% -----------------------------------------------------------------------------
\usepackage{enumitem}
\setlist[itemize]{topsep=0.5em, itemsep=0.25em}
\setlist[enumerate]{topsep=0.5em, itemsep=0.25em}

% -----------------------------------------------------------------------------
% RÉFÉRENCES ET LIENS
% -----------------------------------------------------------------------------
\usepackage{hyperref}
\hypersetup{
    colorlinks=true,
    linkcolor=bleuimq,
    urlcolor=bleuimq,
    citecolor=bleuimq
}
\usepackage{cleveref}
\crefname{figure}{figure}{figures}
\crefname{table}{tableau}{tableaux}
\crefname{equation}{équation}{équations}
\crefname{chapter}{chapitre}{chapitres}
\crefname{section}{section}{sections}

% -----------------------------------------------------------------------------
% DIVERS
% -----------------------------------------------------------------------------
\usepackage{lipsum} % Pour le texte de remplissage (à retirer en production)
\usepackage{cancel} % Pour barrer des termes dans les équations
\usepackage{pifont} % Symboles spéciaux (checkmark, etc.)
\newcommand{\cmark}{\ding{51}} % Checkmark compatible pdfLaTeX

% Espacement des paragraphes
\setlength{\parindent}{0pt}
\setlength{\parskip}{0.8em}

% -----------------------------------------------------------------------------
% COMMANDES PERSONNALISÉES
% -----------------------------------------------------------------------------
% Unités courantes
\newcommand{\ms}{\si{\meter\per\second}}
\newcommand{\mss}{\si{\meter\per\second\squared}}
\newcommand{\kmh}{\si{\kilo\meter\per\hour}}
\newcommand{\rpm}{\text{tr/min}}

% Dérivées
\newcommand{\ddt}[1]{\frac{d#1}{dt}}
\newcommand{\ddx}[1]{\frac{d#1}{dx}}

% Delta
\newcommand{\Dt}{\Delta t}
\newcommand{\Dx}{\Delta x}
\newcommand{\Dv}{\Delta v}

% Grandeurs physiques en italique
\newcommand{\grandeur}[1]{\textit{#1}}

% -----------------------------------------------------------------------------
% INFORMATIONS DU DOCUMENT
% -----------------------------------------------------------------------------
\title{\textbf{Physique 1 -- M\'ecanique}\\[0.5cm]
       \Large Notes de cours}
\author{Institut Maritime du Qu\'ebec}
\date{Hiver 2026}

% =============================================================================