% =============================================================================
% CHAPITRE 1 - CINÉMATIQUE
% Partie 5 : Accélération
% Version maritime pour l'IMQ
% =============================================================================

% =============================================================================
\section{Acc\'el\'eration}
% =============================================================================

\subsection{Introduction : la vitesse peut varier}

Jusqu'\`a pr\'esent, nous avons d\'efini la vitesse comme une mesure du changement de position. Mais la vitesse elle-m\^eme peut \textbf{changer}! 

Un navire qui quitte le port voit sa vitesse augmenter progressivement. Un navire qui approche d'un quai voit sa vitesse diminuer. Dans les deux cas, la vitesse \textbf{varie} au cours du temps.

\begin{remarque}[title=Le taux de variation de la vitesse]
De plus, la vitesse peut varier \textbf{rapidement} ou \textbf{lentement} :
\begin{itemize}
    \item Une voiture de course peut passer de 0 \`a $\SI{100}{km/h}$ en quelques secondes
    \item Un p\'etrolier met plusieurs minutes pour atteindre sa vitesse de croisi\`ere
\end{itemize}

Le \textbf{taux de variation de la vitesse} --- c'est-\`a-dire \`a quelle rapidit\'e la vitesse change --- est ce qu'on appelle l'\textbf{acc\'el\'eration}.
\end{remarque}

\begin{remarque}[title=L'acc\'el\'eration : une grandeur vectorielle]
L'acc\'el\'eration est \`a la vitesse ce que la vitesse est \`a la position :
\begin{center}
\begin{tabular}{|l|l|}
\hline
\rowcolor{bleuclair}
\textbf{Grandeur} & \textbf{D\'efinition} \\
\hline
Vitesse & Taux de variation de la \textbf{position} \\
\hline
Acc\'el\'eration & Taux de variation de la \textbf{vitesse} \\
\hline
\end{tabular}
\end{center}

Comme la vitesse, l'acc\'el\'eration est une grandeur \textbf{vectorielle}. En une dimension, cela signifie qu'elle a un \textbf{signe} qui indique sa direction.
\end{remarque}

% =============================================================================
\subsection{Le signe de l'acc\'el\'eration : attention aux pi\`eges!}
% =============================================================================

Avant de d\'efinir formellement l'acc\'el\'eration, clarifions une source fr\'equente de confusion.

\begin{attention}[title=Acc\'el\'eration n\'egative $\neq$ ralentissement!]
Dans le langage courant, \guillemotleft~acc\'el\'erer~\guillemotright{} signifie aller plus vite et \guillemotleft~d\'ec\'el\'erer~\guillemotright{} signifie ralentir.

En physique, c'est \textbf{plus subtil} : le signe de l'acc\'el\'eration indique sa \textbf{direction}, pas si l'objet acc\'el\`ere ou ralentit!

\begin{center}
\renewcommand{\arraystretch}{1.5}
\begin{tabular}{|c|c|c|}
\hline
\rowcolor{bleuclair}
\textbf{Signe de $v$} & \textbf{Signe de $a$} & \textbf{Effet sur le mouvement} \\
\hline
$v > 0$ & $a > 0$ & Acc\'el\`ere (m\^eme sens) \\
\hline
$v > 0$ & $a < 0$ & \textbf{Ralentit} (freinage) \\
\hline
$v < 0$ & $a < 0$ & Acc\'el\`ere (m\^eme sens) \\
\hline
$v < 0$ & $a > 0$ & \textbf{Ralentit} (freinage) \\
\hline
\end{tabular}
\end{center}

\textbf{R\`egle simple :}
\begin{itemize}
    \item $v$ et $a$ de \textbf{m\^eme signe} $\Rightarrow$ l'objet \textbf{acc\'el\`ere} (vitesse augmente en valeur absolue)
    \item $v$ et $a$ de \textbf{signes oppos\'es} $\Rightarrow$ l'objet \textbf{ralentit} (freinage)
\end{itemize}
\end{attention}

\begin{exemple}{Traversier se d\'epla\c{c}ant vers l'est}{}
Un traversier se d\'eplace vers l'est (sens positif de l'axe). Sa vitesse est $v = \SI{+8}{m/s}$.

\textbf{Cas 1 :} Le capitaine acc\'el\`ere. L'acc\'el\'eration est $a = \SI{+0,5}{m/s^2}$.
\begin{itemize}
    \item $v > 0$ et $a > 0$ : m\^eme signe $\Rightarrow$ le traversier va \textbf{de plus en plus vite} vers l'est
\end{itemize}

\textbf{Cas 2 :} Le capitaine freine. L'acc\'el\'eration est $a = \SI{-0,5}{m/s^2}$.
\begin{itemize}
    \item $v > 0$ et $a < 0$ : signes oppos\'es $\Rightarrow$ le traversier \textbf{ralentit}
    \item Le signe n\'egatif de $a$ indique un \textbf{freinage}, pas un d\'eplacement vers l'ouest!
\end{itemize}
\end{exemple}

\begin{exemple}{Traversier se d\'epla\c{c}ant vers l'ouest}{}
Maintenant, le traversier se d\'eplace vers l'ouest (sens n\'egatif). Sa vitesse est $v = \SI{-8}{m/s}$.

\textbf{Cas 3 :} Le capitaine acc\'el\`ere. L'acc\'el\'eration est $a = \SI{-0,5}{m/s^2}$.
\begin{itemize}
    \item $v < 0$ et $a < 0$ : m\^eme signe $\Rightarrow$ le traversier va \textbf{de plus en plus vite} vers l'ouest
    \item Une acc\'el\'eration n\'egative peut donc \^etre une \guillemotleft~vraie~\guillemotright{} acc\'el\'eration!
\end{itemize}

\textbf{Cas 4 :} Le capitaine freine. L'acc\'el\'eration est $a = \SI{+0,5}{m/s^2}$.
\begin{itemize}
    \item $v < 0$ et $a > 0$ : signes oppos\'es $\Rightarrow$ le traversier \textbf{ralentit}
    \item Une acc\'el\'eration positive peut donc \^etre un freinage!
\end{itemize}
\end{exemple}

\begin{pratiqueautonome}
Un navire se déplace vers l'ouest (sens négatif de l'axe) à $\SI{12}{\knots}$. Déterminez le signe de l'accélération dans chaque cas :

\begin{enumerate}[label=\alph*)]
    \item Le capitaine augmente la puissance des moteurs pour aller plus vite.
    \item Le capitaine ordonne la marche arrière pour freiner.
\end{enumerate}

Dans chaque cas, le navire accélère-t-il ou ralentit-il?

\espaceresolution[5cm]
\reponsepratique{a) $a < 0$ (accélère vers l'ouest) \quad b) $a > 0$ (ralentit)}
\end{pratiqueautonome}

% =============================================================================
\subsection{Acc\'el\'eration moyenne}
% =============================================================================

\begin{definition}[title=Acc\'el\'eration moyenne]
L'\textbf{acc\'el\'eration moyenne} $a_{moy}$ d'un objet est le rapport entre la variation de sa vitesse et l'intervalle de temps correspondant :
\begin{equationimportante}
\begin{equation}
a_{moy} = \frac{\Delta v}{\Delta t} = \frac{v_f - v_i}{t_f - t_i}
\end{equation}
\end{equationimportante}

L'unit\'e SI est le \textbf{m\`etre par seconde au carr\'e} (\si{m/s^2}).
\end{definition}

\begin{exemple}{Acc\'el\'eration d'un vraquier au d\'epart}{}
Un vraquier quitte le port de Montr\'eal. Sa vitesse passe de $\SI{0}{\knots}$ \`a $\SI{10}{\knots}$ en $\SI{15}{minutes}$.

\textbf{Conversion en unit\'es SI :}
\begin{align*}
v_i &= \SI{0}{m/s} \\[0.2cm]
v_f &= \SI{10}{\knots} \times \frac{\SI{0,5144}{m/s}}{\SI{1}{\knots}} = \SI{5,14}{m/s} \\[0.2cm]
\Delta t &= \SI{15}{min} \times \frac{\SI{60}{s}}{\SI{1}{min}} = \SI{900}{s}
\end{align*}

\textbf{Acc\'el\'eration moyenne :}
\[ a_{moy} = \frac{v_f - v_i}{\Delta t} = \frac{\SI{5,14}{m/s} - \SI{0}{m/s}}{\SI{900}{s}} = \SI{0,0057}{m/s^2} \]

C'est une acc\'el\'eration tr\`es faible compar\'ee \`a celle d'une voiture ($\sim \SI{3}{m/s^2}$), ce qui est typique des gros navires en raison de leur masse \'enorme.
\end{exemple}

\begin{exemple}{Freinage d'un p\'etrolier}{}
Un p\'etrolier naviguant \`a $\SI{15}{\knots}$ (vers l'est, donc $v_i > 0$) doit s'arr\^eter. En raison de sa grande inertie, le freinage prend $\SI{20}{minutes}$.

\textbf{Conversion en unit\'es SI :}
\begin{align*}
v_i &= \SI{15}{\knots} \times \frac{\SI{0,5144}{m/s}}{\SI{1}{\knots}} = \SI{+7,72}{m/s} \\[0.2cm]
v_f &= \SI{0}{m/s} \\[0.2cm]
\Delta t &= \SI{20}{min} \times \frac{\SI{60}{s}}{\SI{1}{min}} = \SI{1200}{s}
\end{align*}

\textbf{Acc\'el\'eration :}
\[ a_{moy} = \frac{v_f - v_i}{\Delta t} = \frac{\SI{0}{m/s} - \SI{7,72}{m/s}}{\SI{1200}{s}} = \SI{-0,0064}{m/s^2} \]

\textbf{Analyse :} Le signe n\'egatif de $a$ et le signe positif de $v_i$ indiquent qu'il s'agit bien d'un \textbf{freinage} (signes oppos\'es). Cette d\'ec\'el\'eration\footnote{Rappel : la d\'ec\'el\'eration est une acc\'el\'eration dont le vecteur est oppos\'e au vecteur vitesse.} tr\`es faible explique pourquoi les p\'etroliers n\'ecessitent de tr\`es longues distances pour s'arr\^eter.
\end{exemple}

\begin{pratiqueautonome}
Un vraquier naviguant à $\SI{8}{\knots}$ vers l'est doit s'arrêter. Son accélération de freinage est de $a = \SI{-0,004}{m/s^2}$.

\begin{enumerate}[label=\alph*)]
    \item Combien de temps faut-il pour s'immobiliser complètement?
    \item Quelle distance parcourt-il pendant le freinage?
\end{enumerate}

\textit{Indice : Utilisez les équations du MRUA que nous verrons en détail plus loin, ou raisonnez avec la définition de l'accélération.}

\espaceresolution[6cm]
\reponsepratique{a) $\Delta t \approx \SI{17}{min}$ \quad b) $\Delta x \approx \SI{2,1}{km}$}
\end{pratiqueautonome}

\begin{exemple}{Acc\'el\'eration d'une voiture (exemple terrestre)}{}
Une voiture passe de $\SI{0}{km/h}$ \`a $\SI{100}{km/h}$ en $\SI{8}{s}$.

\textbf{Conversion :}
\[ v_f = \SI{100}{km/h} \times \frac{\SI{1}{h}}{\SI{3600}{s}} \times \frac{\SI{1000}{m}}{\SI{1}{km}} = \SI{100}{km/h} \times \frac{\SI{1}{m/s}}{\SI{3,6}{km/h}} = \SI{27,8}{m/s} \]

\textbf{Acc\'el\'eration :}
\[ a = \frac{v_f - v_i}{\Delta t} = \frac{\SI{27,8}{m/s} - \SI{0}{m/s}}{\SI{8}{s}} = \SI{3,47}{m/s^2} \]

Comparons : la voiture acc\'el\`ere environ \textbf{600 fois plus vite} que le vraquier! C'est parce que le rapport puissance/masse est beaucoup plus favorable pour une voiture.
\end{exemple}

% =============================================================================
\subsection{Acc\'el\'eration instantan\'ee}
% =============================================================================

\begin{definition}[title=Acc\'el\'eration instantan\'ee]
L'\textbf{acc\'el\'eration instantan\'ee} est l'acc\'el\'eration \`a un instant pr\'ecis. C'est la limite de l'acc\'el\'eration moyenne lorsque $\Delta t$ tend vers z\'ero :
\begin{equationimportante}
\begin{equation}
a = \lim_{\Delta t \to 0} \frac{\Delta v}{\Delta t}
\end{equation}
\end{equationimportante}

\textbf{Interpr\'etation graphique :} Sur un graphique $v(t)$, l'acc\'el\'eration instantan\'ee correspond \`a la \textbf{pente de la tangente} \`a la courbe vitesse-temps.
\end{definition}

\begin{remarque}[title=Acc\'el\'eration constante]
Lorsque l'acc\'el\'eration est \textbf{constante}, l'acc\'el\'eration instantan\'ee est \'egale \`a l'acc\'el\'eration moyenne \`a tout instant :
\[ \text{Si } a = \text{constante} \Rightarrow a_{instantan\acute{e}e} = a_{moyenne} \]

Dans ce cas, le graphique $v(t)$ est une \textbf{droite}.
\end{remarque}

% =============================================================================
\subsection{Interpr\'etation graphique de l'acc\'el\'eration}
% =============================================================================

Sur un graphique \textbf{vitesse-temps} $v(t)$ :

\begin{center}
\renewcommand{\arraystretch}{1.6}
\begin{tabular}{|c|c|c|}
\hline
\rowcolor{bleuclair}
\textbf{Forme de la courbe} & \textbf{Acc\'el\'eration} & \textbf{Type de mouvement} \\
\hline
Droite horizontale & $a = 0$ & Vitesse constante \\
\hline
Droite inclin\'ee vers le haut & $a > 0$ constante & Acc\'el\'eration uniforme \\
\hline
Droite inclin\'ee vers le bas & $a < 0$ constante & Acc\'el\'eration uniforme \\
\hline
Courbe & $a$ variable & Mouvement non uniforme \\
\hline
\end{tabular}
\end{center}

\begin{exemple}{Lecture d'un graphique $v(t)$}{}
Le graphique suivant montre la vitesse d'un cargo pendant une man\oe{}uvre :

\begin{center}
\begin{tikzpicture}[scale=0.8]
% Axes
\draw[axe, thick] (0,0) -- (9,0) node[right] {$t$ (min)};
\draw[axe, thick] (0,0) -- (0,5) node[above] {$v$ (m/s)};
% Graduations
\foreach \x in {1,2,3,4,5,6,7,8} {\draw (\x,0.1) -- (\x,-0.1) node[below] {\x};}
\foreach \y in {1,2,3,4} {\draw (0.1,\y) -- (-0.1,\y) node[left] {\y};}
% Courbe
\draw[very thick, blue] (0,0) -- (2,4) -- (5,4) -- (8,0);
% Annotations
\node[blue, above] at (1,2) {I};
\node[blue, above] at (3.5,4.3) {II};
\node[blue, above] at (6.5,2) {III};
\end{tikzpicture}
\end{center}

\textbf{Phase I (0 \`a 2 min) :} Droite montante $\Rightarrow$ acc\'el\'eration positive constante
\[ a_I = \frac{4 - 0}{2 - 0} = \SI{2}{m/s/min} = \SI{0,033}{m/s^2} \]

\textbf{Phase II (2 \`a 5 min) :} Droite horizontale $\Rightarrow$ vitesse constante, $a = 0$

\textbf{Phase III (5 \`a 8 min) :} Droite descendante $\Rightarrow$ freinage (car $v > 0$ et $a < 0$)
\[ a_{III} = \frac{0 - 4}{8 - 5} = \SI{-1,33}{m/s/min} = \SI{-0,022}{m/s^2} \]
\end{exemple}
