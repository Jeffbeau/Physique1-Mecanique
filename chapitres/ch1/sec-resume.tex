% =============================================================================
% CHAPITRE 1 - CINÉMATIQUE
% Partie 7 : Résumé, compétences et exercices
% Version maritime pour l'IMQ
% =============================================================================

% =============================================================================
\section*{Résumé du chapitre}
\addcontentsline{toc}{section}{Résumé}
% =============================================================================

\begin{center}
\renewcommand{\arraystretch}{2.0}
\begin{longtable}{|L{8cm}|C{6cm}|}
\hline
\rowcolor{bleuclair}
\textbf{Concept} & \textbf{Formule / Définition} \\
\hline
\endfirsthead
\hline
\rowcolor{bleuclair}
\textbf{Concept} & \textbf{Formule / Définition} \\
\hline
\endhead

La \textbf{cinématique} est la branche de la mécanique qui \textbf{décrit} le mouvement des corps, sans s'intéresser à ses causes. & Description qualitative et mathématique du mouvement \\
\hline

Le \textbf{déplacement} est la variation de position. C'est une grandeur vectorielle (avec signe en 1D). & $\Delta x = x_f - x_i$ \\
\hline

La \textbf{distance parcourue} est la longueur totale du trajet. C'est une grandeur scalaire (toujours $\geq 0$). & $d = $ longueur du trajet \\
\hline

La \textbf{vitesse moyenne} est le rapport entre le déplacement et l'intervalle de temps. & $v_{moy} = \dfrac{\Delta x}{\Delta t}$ \\
\hline

La \textbf{vitesse scalaire moyenne} est le rapport entre la distance parcourue et l'intervalle de temps. & $v_{scalaire} = \dfrac{d}{\Delta t}$ \\
\hline

La \textbf{vitesse instantanée} est la vitesse à un instant précis. Graphiquement, c'est la pente de la tangente à $x(t)$. & $v = \text{pente de la tangente à } x(t)$ \\
\hline

L'\textbf{accélération moyenne} est le rapport entre la variation de vitesse et l'intervalle de temps. & $a_{moy} = \dfrac{\Delta v}{\Delta t}$ \\
\hline

L'\textbf{accélération instantanée} est l'accélération à un instant précis. C'est la pente de la tangente à $v(t)$. & $a = \text{pente de la tangente à } v(t)$ \\
\hline

\textbf{Équations du MRUA} (mouvement à accélération constante) & 
$v_f = v_i + a\Delta t$ \newline
$v_{moy} = \dfrac{v_i + v_f}{2}$ \newline
$\Delta x = \dfrac{1}{2}(v_i + v_f)\Delta t$ \newline
$\Delta x = v_i\Delta t + \dfrac{1}{2}a(\Delta t)^2$ \newline
$v_f^2 = v_i^2 + 2a\Delta x$ \\
\hline

\textbf{Conversion maritime} : Le n\oe{}ud est l'unité de vitesse en navigation. & $\SI{1}{\knots} = \SI{1,852}{km/h} = \SI{0,5144}{m/s}$ \\
\hline

\textbf{Mouvement rectiligne uniforme (MRU)} : mouvement en ligne droite à vitesse constante. & $\Delta x = v \cdot \Delta t$ \\
\hline

\textbf{Chute libre} : mouvement sous la seule action de la gravité (MRUA vertical). & $g = \SI{9,81}{m/s^2}$ (vers le bas) \\
\hline

\textbf{Mouvement en 2D} : décomposer en $x$ et $y$, traiter séparément, relier par le temps $\Delta t$. & $v_x = v\cos\theta$ \newline $v_y = v\sin\theta$ \\
\hline

\textbf{Projectile} : en $x$ c'est un MRU, en $y$ c'est une chute libre. & $a_x = 0$ \newline $a_y = -g$ \\
\hline

\textbf{Cinématique de rotation} : analogie complète avec la translation. & $\theta = \dfrac{s}{r}$ \newline $\omega = \dfrac{\Delta\theta}{\Delta t}$ \newline $\alpha = \dfrac{\Delta\omega}{\Delta t}$ \\
\hline

\textbf{Relations linéaire-angulaire} & $s = r\theta$ \newline $v = r\omega$ \newline $a_t = r\alpha$ \\
\hline

\textbf{Équations rotation} (accélération angulaire constante) & 
$\omega_f = \omega_i + \alpha\Delta t$ \newline
$\Delta\theta = \omega_i\Delta t + \dfrac{1}{2}\alpha(\Delta t)^2$ \newline
$\omega_f^2 = \omega_i^2 + 2\alpha\Delta\theta$ \\
\hline

\end{longtable}
\end{center}

% =============================================================================
\section*{Compétences}
\addcontentsline{toc}{section}{Compétences}
% =============================================================================

À la fin de ce chapitre, vous devriez être en mesure de :

\begin{center}
\renewcommand{\arraystretch}{1.4}
\begin{tabular}{|L{10cm}|C{1cm}|C{1cm}|C{1cm}|C{1cm}|}
\hline
\rowcolor{bleuclair}
\textbf{Compétence} & \rotatebox{90}{\textbf{Difficile}} & \rotatebox{90}{\textbf{Familier}} & \rotatebox{90}{\textbf{Minimum}} & \rotatebox{90}{\textbf{Maîtrise}} \\
\hline
\multicolumn{5}{|l|}{\cellcolor{gray!20}\textbf{Cinématique de translation (1D)}} \\
\hline
Expliquer ce qu'est la cinématique (description du mouvement) & & & & \\
\hline
Différencier un déplacement et une distance parcourue & & & & \\
\hline
Calculer la vitesse moyenne à partir de données ou d'un graphique & & & & \\
\hline
Calculer la vitesse scalaire moyenne & & & & \\
\hline
Interpréter un graphique position-temps $x(t)$ & & & & \\
\hline
Déterminer la vitesse instantanée à partir d'un graphique & & & & \\
\hline
Calculer l'accélération moyenne & & & & \\
\hline
Interpréter un graphique vitesse-temps $v(t)$ & & & & \\
\hline
Choisir et appliquer les équations du MRUA & & & & \\
\hline
\multicolumn{5}{|l|}{\cellcolor{gray!20}\textbf{Chute libre et projectile (2D)}} \\
\hline
Appliquer les équations du MRUA à la chute libre & & & & \\
\hline
Résoudre des problèmes de chute libre (objet lâché ou lancé) & & & & \\
\hline
Décomposer la vitesse initiale d'un projectile en composantes & & & & \\
\hline
Résoudre des problèmes de projectile en 2D & & & & \\
\hline
\multicolumn{5}{|l|}{\cellcolor{gray!20}\textbf{Cinématique de rotation}} \\
\hline
Convertir des angles entre degrés, radians et tours & & & & \\
\hline
Calculer la vitesse angulaire et convertir entre rad/s et RPM & & & & \\
\hline
Utiliser les relations $s = r\theta$, $v = r\omega$, $a_t = r\alpha$ & & & & \\
\hline
Appliquer les équations du mouvement circulaire uniformément accéléré & & & & \\
\hline
\multicolumn{5}{|l|}{\cellcolor{gray!20}\textbf{Applications}} \\
\hline
Convertir entre les unités SI et les unités maritimes (n\oe{}uds, NM) & & & & \\
\hline
Résoudre des problèmes de navigation impliquant la cinématique & & & & \\
\hline
\end{tabular}
\end{center}
