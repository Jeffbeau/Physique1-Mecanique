% =============================================================================
% CHAPITRE 1 - CINÉMATIQUE
% Partie 3 : Description graphique du mouvement
% Version maritime pour l'IMQ
% =============================================================================

% =============================================================================
\section{Description graphique du mouvement}
% =============================================================================

La description d'un mouvement peut se faire de plusieurs fa\c{c}ons : par des \'equations, par des tableaux de valeurs, ou par des \textbf{graphiques}. Les graphiques sont particuli\`erement utiles car ils permettent de visualiser l'ensemble du mouvement d'un seul coup d'\oe{}il.

\begin{attention}[title=L'importance de savoir lire les graphiques]
Les graphiques sont des outils puissants en physique, mais encore faut-il \textbf{apprendre \`a les lire correctement}! 

Un graphique position-temps $x(t)$ ne montre \textbf{pas} la trajectoire de l'objet dans l'espace --- il montre comment sa position varie dans le temps. De m\^eme, un graphique vitesse-temps $v(t)$ ne montre pas la vitesse ``dans l'espace'' mais son \'evolution temporelle.

Dans cette section, nous allons d\'evelopper les comp\'etences n\'ecessaires pour :
\begin{itemize}
    \item Extraire des informations quantitatives d'un graphique (vitesse, acc\'el\'eration)
    \item Interpr\'eter qualitativement un mouvement (acc\'el\'er\'e, ralenti, au repos)
    \item Relier les diff\'erents types de graphiques entre eux
\end{itemize}
\end{attention}

\subsection{Le graphique position-temps}

Le graphique \textbf{position en fonction du temps}, not\'e $x(t)$, montre comment la position d'un objet varie au fil du temps.

\begin{remarque}[title=Lecture d'un graphique $x(t)$]
Sur un graphique position-temps :
\begin{itemize}
    \item L'axe horizontal repr\'esente le \textbf{temps} $t$
    \item L'axe vertical repr\'esente la \textbf{position} $x$
    \item Chaque point de la courbe donne la position de l'objet \`a un instant donn\'e
    \item La \textbf{pente} de la courbe repr\'esente la \textbf{vitesse}
\end{itemize}
\end{remarque}

\begin{exemple}{Approche d'un navire vers un quai}{}
Le graphique suivant montre la position d'un navire lors de son approche vers un quai. L'origine ($x = 0$) est plac\'ee au quai, et l'axe $x$ est positif vers le large.

\begin{center}
\begin{tikzpicture}[scale=0.9]
% Axes
\draw[axe, thick] (0,0) -- (8,0) node[right] {$t$ (min)};
\draw[axe, thick] (0,0) -- (0,5) node[above] {$x$ (m)};
% Graduations
\foreach \x in {1,2,3,4,5,6,7} {\draw (\x,0.1) -- (\x,-0.1) node[below] {\x};}
\foreach \y in {1,2,3,4} {\draw (0.1,\y) -- (-0.1,\y) node[left] {\pgfmathparse{int(\y*100)}\pgfmathresult};}
% Courbe
\draw[very thick, blue] (0,4) -- (2,4) -- (5,1) -- (7,1);
% Points
\fill[blue] (0,4) circle (3pt) node[above right] {A};
\fill[blue] (2,4) circle (3pt) node[above right] {B};
\fill[blue] (5,1) circle (3pt) node[above right] {C};
\fill[blue] (7,1) circle (3pt) node[above right] {D};
% Annotations
\node[right] at (7.5,4) {\small Segment AB : navire immobile};
\node[right] at (7.5,2.5) {\small Segment BC : approche};
\node[right] at (7.5,1) {\small Segment CD : navire immobile};
\end{tikzpicture}
\end{center}

\textbf{Interprétation :}
\begin{itemize}
    \item \textbf{A à B} (0 à 2 min) : Le navire est \textbf{immobile} à $\SI{400}{m}$ du quai. La courbe est horizontale (pente = 0, donc vitesse = 0).
    \item \textbf{B à C} (2 à 5 min) : Le navire \textbf{s'approche} du quai. La courbe descend (pente négative, donc vitesse négative = mouvement vers les $x$ décroissants).
    \item \textbf{C à D} (5 à 7 min) : Le navire est \textbf{immobile} à $\SI{100}{m}$ du quai, en attente du pilote.
\end{itemize}
\end{exemple}

\subsection{Calcul de la vitesse moyenne à partir du graphique}

\begin{definition}[title=Vitesse moyenne et pente de la sécante]
Sur un graphique $x(t)$, la \textbf{vitesse moyenne} entre deux instants $t_1$ et $t_2$ correspond à la \textbf{pente de la droite (sécante)} reliant les deux points correspondants :
\begin{equationimportante}
\begin{equation}
v_{moy} = \frac{\Delta x}{\Delta t} = \frac{x_2 - x_1}{t_2 - t_1} = \text{pente de la sécante}
\end{equation}
\end{equationimportante}
\end{definition}

\begin{exemple}{Calcul de la vitesse d'approche}{}
À partir du graphique précédent, calculons la vitesse moyenne du navire pendant la phase d'approche (B à C).

\textbf{Données lues sur le graphique :}
\begin{itemize}
    \item Point B : $t_1 = \SI{2}{min}$, $x_1 = \SI{400}{m}$
    \item Point C : $t_2 = \SI{5}{min}$, $x_2 = \SI{100}{m}$
\end{itemize}

\textbf{Calcul :}
\begin{align*}
v_{moy} &= \frac{x_2 - x_1}{t_2 - t_1} = \frac{100 - 400}{5 - 2} = \frac{-300}{3} = \SI{-100}{m/min}
\end{align*}

Conversion en m/s : $v_{moy} = \dfrac{-100}{60} = \SI{-1,67}{m/s}$

Le signe négatif indique que le navire se déplace vers les $x$ décroissants (vers le quai). En valeur absolue, c'est environ $\SI{3,2}{n\oe{}uds}$, une vitesse typique pour une approche finale.
\end{exemple}

\subsection{Interprétation de la forme de la courbe}

La forme de la courbe $x(t)$ nous renseigne sur le type de mouvement :

\begin{center}
\renewcommand{\arraystretch}{1.6}
\begin{tabular}{|c|c|c|}
\hline
\rowcolor{bleuclair}
\textbf{Forme de la courbe} & \textbf{Type de mouvement} & \textbf{Vitesse} \\
\hline
Droite horizontale & Repos (immobile) & $v = 0$ \\
\hline
Droite inclinée vers le haut & MRU dans le sens $+x$ & $v > 0$ constante \\
\hline
Droite inclinée vers le bas & MRU dans le sens $-x$ & $v < 0$ constante \\
\hline
Courbe (parabole) vers le haut & Mouvement accéléré & $v$ augmente \\
\hline
Courbe (parabole) vers le bas & Mouvement décéléré & $v$ diminue \\
\hline
\end{tabular}
\end{center}

\begin{exemple}{Départ d'un navire du port -- Les trois graphiques}{}
Un navire quitte le port. Pendant les 4 premières minutes, il accélère uniformément. Ensuite, il maintient sa vitesse de croisière. Voici les trois graphiques qui décrivent ce mouvement :

\begin{center}
\begin{tikzpicture}[scale=0.65]
% === GRAPHIQUE x(t) ===
\begin{scope}[shift={(0,0)}]
\draw[axe, thick] (0,0) -- (7,0) node[right] {$t$};
\draw[axe, thick] (0,0) -- (0,5) node[above] {$x$};
\node[above] at (3.5,5) {\textbf{Position}};
% Courbe : parabole puis droite
\draw[very thick, blue, domain=0:3, samples=50] plot (\x, {0.167*\x*\x});
\draw[very thick, blue] (3,1.5) -- (6,4.5);
% Points
\fill[blue] (0,0) circle (2pt);
\fill[blue] (3,1.5) circle (2pt);
% Annotations
\draw[dashed, gray] (3,0) -- (3,1.5);
\node[below] at (3,0) {\small $t_1$};
\node[left] at (1.5,1) {\small parabole};
\node[right] at (4.5,3) {\small droite};
\end{scope}

% === GRAPHIQUE v(t) ===
\begin{scope}[shift={(8,0)}]
\draw[axe, thick] (0,0) -- (7,0) node[right] {$t$};
\draw[axe, thick] (0,0) -- (0,5) node[above] {$v$};
\node[above] at (3.5,5) {\textbf{Vitesse}};
% Courbe : droite montante puis horizontale
\draw[very thick, red] (0,0) -- (3,3) -- (6,3);
% Points
\fill[red] (0,0) circle (2pt);
\fill[red] (3,3) circle (2pt);
% Annotations
\draw[dashed, gray] (3,0) -- (3,3);
\node[below] at (3,0) {\small $t_1$};
\node[left] at (3,3) {\small $v_{max}$};
\end{scope}

% === GRAPHIQUE a(t) ===
\begin{scope}[shift={(16,0)}]
\draw[axe, thick] (0,0) -- (7,0) node[right] {$t$};
\draw[axe, thick] (0,0) -- (0,5) node[above] {$a$};
\node[above] at (3.5,5) {\textbf{Accélération}};
% Courbe : constante puis nulle
\draw[very thick, green!60!black] (0,2.5) -- (3,2.5);
\draw[very thick, green!60!black] (3,0) -- (6,0);
% Points
\fill[green!60!black] (0,2.5) circle (2pt);
\fill[green!60!black] (3,2.5) circle (2pt);
\draw[green!60!black, dashed] (3,2.5) -- (3,0);
\fill[green!60!black] (3,0) circle (2pt);
% Annotations
\node[below] at (3,0) {\small $t_1$};
\node[left] at (0,2.5) {\small $a$};
\end{scope}
\end{tikzpicture}
\end{center}

\textbf{Interprétation des liens entre les graphiques :}

\begin{center}
\renewcommand{\arraystretch}{1.4}
\begin{tabular}{|c|c|c|c|}
\hline
\rowcolor{bleuclair}
\textbf{Phase} & \textbf{Position $x(t)$} & \textbf{Vitesse $v(t)$} & \textbf{Accélération $a(t)$} \\
\hline
$0 \to t_1$ (accélération) & Parabole (courbée vers le haut) & Droite montante & Constante positive \\
\hline
$t_1 \to $ fin (croisière) & Droite inclinée & Horizontale & Nulle \\
\hline
\end{tabular}
\end{center}

\begin{attention}[title=Id\'ealisation des graphiques -- Limites du mod\`ele]
Les graphiques avec des \textbf{angles vifs} (changements de pente instantan\'es) sont une \textbf{id\'ealisation math\'ematique}. Dans la r\'ealit\'e :

\begin{itemize}
    \item L'\textbf{inertie} du navire emp\^eche tout changement instantan\'e de vitesse ou d'acc\'el\'eration
    \item Les courbes r\'eelles sont toujours \textbf{plus lisses} (pas de discontinuit\'es)
    \item Un navire de $\SI{100000}{tonnes}$ ne peut pas passer d'une acc\'el\'eration constante \`a une acc\'el\'eration nulle en un instant
\end{itemize}

Ces mod\`eles simplifi\'es restent tr\`es utiles pour comprendre les principes fondamentaux et faire des calculs approch\'es.
\end{attention}
\end{exemple}

\begin{exemple}{Arriv\'ee d'un navire au port -- Les trois graphiques}{}
Un navire approche du quai. Il navigue d'abord à vitesse constante, puis freine uniformément jusqu'à l'arrêt.

\begin{center}
\begin{tikzpicture}[scale=0.65]
% === GRAPHIQUE x(t) ===
\begin{scope}[shift={(0,0)}]
\draw[axe, thick] (0,0) -- (7,0) node[right] {$t$};
\draw[axe, thick] (0,0) -- (0,5) node[above] {$x$};
\node[above] at (3.5,5) {\textbf{Position}};
% Courbe : droite puis parabole inversée
\draw[very thick, blue] (0,0) -- (2,1.5);
\draw[very thick, blue, domain=2:6, samples=50] plot (\x, {1.5 + 0.75*(\x-2) - 0.09375*(\x-2)*(\x-2)});
% Points
\fill[blue] (0,0) circle (2pt);
\fill[blue] (2,1.5) circle (2pt);
\fill[blue] (6,4.5) circle (2pt);
% Annotations
\draw[dashed, gray] (2,0) -- (2,1.5);
\node[below] at (2,0) {\small $t_1$};
\node[left] at (1,0.8) {\small droite};
\node[right] at (4.5,3.5) {\small parabole};
\node[right] at (6,4.5) {\small quai};
\end{scope}

% === GRAPHIQUE v(t) ===
\begin{scope}[shift={(8,0)}]
\draw[axe, thick] (0,0) -- (7,0) node[right] {$t$};
\draw[axe, thick] (0,0) -- (0,5) node[above] {$v$};
\node[above] at (3.5,5) {\textbf{Vitesse}};
% Courbe : horizontale puis droite descendante
\draw[very thick, red] (0,3) -- (2,3) -- (6,0);
% Points
\fill[red] (0,3) circle (2pt);
\fill[red] (2,3) circle (2pt);
\fill[red] (6,0) circle (2pt);
% Annotations
\draw[dashed, gray] (2,0) -- (2,3);
\node[below] at (2,0) {\small $t_1$};
\node[left] at (0,3) {\small $v_i$};
\end{scope}

% === GRAPHIQUE a(t) ===
\begin{scope}[shift={(16,0)}]
\draw[axe, thick] (0,0) -- (7,0) node[right] {$t$};
\draw[axe, thick] (0,-2.5) -- (0,2.5) node[above] {$a$};
\node[above] at (3.5,2.5) {\textbf{Accélération}};
% Courbe : nulle puis constante négative
\draw[very thick, green!60!black] (0,0) -- (2,0);
\draw[very thick, green!60!black, dashed] (2,0) -- (2,-2);
\draw[very thick, green!60!black] (2,-2) -- (6,-2);
% Points
\fill[green!60!black] (0,0) circle (2pt);
\fill[green!60!black] (2,0) circle (2pt);
\fill[green!60!black] (2,-2) circle (2pt);
\fill[green!60!black] (6,-2) circle (2pt);
% Annotations
\node[below] at (2,-0.3) {\small $t_1$};
\node[left] at (0,-2) {\small $-a$};
% Ligne de référence
\draw[gray, thin] (0,0) -- (6,0);
\end{scope}
\end{tikzpicture}
\end{center}

\textbf{Interprétation :}

\begin{center}
\renewcommand{\arraystretch}{1.4}
\begin{tabular}{|c|c|c|c|}
\hline
\rowcolor{bleuclair}
\textbf{Phase} & \textbf{Position $x(t)$} & \textbf{Vitesse $v(t)$} & \textbf{Accélération $a(t)$} \\
\hline
$0 \to t_1$ (approche) & Droite inclinée & Horizontale & Nulle \\
\hline
$t_1 \to $ fin (freinage) & Parabole (s'aplatit) & Droite descendante & Constante négative \\
\hline
\end{tabular}
\end{center}

\begin{attention}
Notez que pendant le freinage :
\begin{itemize}
    \item La \textbf{position continue d'augmenter} (le navire avance toujours vers le quai)
    \item La \textbf{vitesse diminue} (le navire ralentit)
    \item L'\textbf{accélération est négative} (elle s'oppose au mouvement)
\end{itemize}
La position augmente de moins en moins vite : c'est pourquoi la parabole s'aplatit.
\end{attention}
\end{exemple}

\subsection{Le graphique vitesse-temps}

Le graphique \textbf{vitesse en fonction du temps}, noté $v(t)$, est complémentaire au graphique position-temps.

\begin{remarque}[title=Lecture d'un graphique $v(t)$]
Sur un graphique vitesse-temps :
\begin{itemize}
    \item L'axe horizontal représente le \textbf{temps} $t$
    \item L'axe vertical représente la \textbf{vitesse} $v$
    \item La \textbf{pente} de la courbe représente l'\textbf{accélération}
\end{itemize}
\end{remarque}

\begin{exemple}{Man\oe{}uvre d'accostage d'un traversier}{}
Voici le graphique $v(t)$ d'un traversier lors de son accostage :

\begin{center}
\begin{tikzpicture}[scale=0.9]
% Axes
\draw[axe, thick] (0,0) -- (8,0) node[right] {$t$ (s)};
\draw[axe, thick] (0,0) -- (0,4.5) node[above] {$v$ (m/s)};
% Graduations
\foreach \x in {20,40,60,80,100,120} {\draw (\x/20,0.1) -- (\x/20,-0.1) node[below] {\tiny\x};}
\foreach \y in {1,2,3,4} {\draw (0.1,\y) -- (-0.1,\y) node[left] {\y};}
% Courbe
\draw[very thick, blue] (0,4) -- (2,4) -- (6,0);
% Points
\fill[blue] (0,4) circle (3pt);
\fill[blue] (2,4) circle (3pt);
\fill[blue] (6,0) circle (3pt);
\end{tikzpicture}
\end{center}

\textbf{Interprétation :}
\begin{itemize}
    \item \textbf{0 à 40 s} : Vitesse constante de $\SI{4}{m/s}$ (MRU pendant l'approche). La pente est nulle, donc l'accélération est nulle.
    \item \textbf{40 à 120 s} : La vitesse diminue linéairement jusqu'à zéro (freinage). La pente est négative et constante, donc l'accélération est constante et négative (MRUA).
\end{itemize}

\textbf{Calcul de l'accélération pendant le freinage :}
\[ a = \frac{\Delta v}{\Delta t} = \frac{0 - 4}{120 - 40} = \frac{-4}{80} = \SI{-0,05}{m/s^2} \]
\end{exemple}
