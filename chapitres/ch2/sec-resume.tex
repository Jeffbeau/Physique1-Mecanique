% =============================================================================
% SECTION 6 - RÉSUMÉ ET SYNTHÈSE
% Bloc 8 (intégré) - Semaine 8
% =============================================================================

\section{Résumé du chapitre}
\label{sec:resume}

% =============================================================================
\subsection{Les trois lois de Newton}
\label{subsec:resume-lois}
% =============================================================================

\begin{center}
\renewcommand{\arraystretch}{2.2}
\begin{tabular}{|c|L{5.5cm}|L{6cm}|}
\hline
\rowcolor{bleuimq} \textcolor{white}{\textbf{Loi}} & \textcolor{white}{\textbf{Énoncé}} & \textcolor{white}{\textbf{Formulation mathématique}} \\
\hline
\textbf{1\textsuperscript{re} loi} \newline (Inertie) & 
Un objet reste au repos ou en MRU si aucune force résultante n'agit sur lui. & 
$\displaystyle \sum \vec{F} = \vec{0} \Leftrightarrow \vec{a} = \vec{0}$ \newline (équilibre) \\
\hline
\textbf{2\textsuperscript{e} loi} \newline (Fondamentale) & 
L'accélération est proportionnelle à la force résultante et inversement proportionnelle à la masse. & 
$\displaystyle \sum \vec{F} = m\vec{a}$ \newline
En composantes : \newline
$\sum F_x = ma_x$ \quad $\sum F_y = ma_y$ \\
\hline
\textbf{3\textsuperscript{e} loi} \newline (Action-réaction) & 
Toute force exercée par A sur B s'accompagne d'une force égale et opposée de B sur A. & 
$\displaystyle \vec{F}_{A \to B} = -\vec{F}_{B \to A}$ \newline
(même module, sens opposés, \newline objets différents) \\
\hline
\end{tabular}
\end{center}

% =============================================================================
\subsection{Catalogue des forces}
\label{subsec:resume-forces}
% =============================================================================

\begin{center}
\renewcommand{\arraystretch}{1.8}
\begin{tabular}{|L{2.2cm}|C{2.8cm}|L{4cm}|L{4.5cm}|}
\hline
\rowcolor{bleuimq} \textcolor{white}{\textbf{Force}} & \textcolor{white}{\textbf{Formule}} & \textcolor{white}{\textbf{Direction}} & \textcolor{white}{\textbf{Remarques}} \\
\hline
\textbf{Poids} & $F_g = mg$ & Verticale, vers le bas & Toujours présente! \\
\hline
\textbf{Normale} & À déterminer & $\perp$ à la surface, vers l'extérieur & $N \neq mg$ en général \\
\hline
\textbf{Tension} & À déterminer & // à la corde, tire l'objet & Corde idéale : $T$ uniforme \\
\hline
\textbf{Frottement statique} & $f_s \leq \mu_s N$ & // à la surface, oppose la tendance & Inégalité! Force adaptative \\
\hline
\textbf{Frottement cinétique} & $f_c = \mu_c N$ & // à la surface, oppose le mouvement & Égalité, $\mu_c < \mu_s$ \\
\hline
\textbf{Centripète} & $F_c = \dfrac{mv^2}{r}$ & Vers le centre du cercle & Ce n'est PAS une nouvelle force! \\
\hline
\end{tabular}
\end{center}

% =============================================================================
\subsection{L'algorithme de résolution}
\label{subsec:resume-algorithme}
% =============================================================================

\begin{center}
\begin{tikzpicture}[
    node distance=0.8cm,
    box/.style={rectangle, draw=bleuimq, fill=bleuclair, thick, text width=10cm, minimum height=1.2cm, align=left, font=\small},
    arrow/.style={-{Stealth[length=3mm]}, thick, bleuimq}
]

% Étape 1
\node[box] (e1) {
    \textbf{ÉTAPE 1 --- SCHÉMA et DCL}
    \begin{itemize}[leftmargin=*, topsep=0pt, itemsep=0pt]
        \item Dessiner la situation physique
        \item Isoler l'objet d'intérêt
        \item Tracer le DCL : représenter \textbf{toutes} les forces
        \item Identifier les grandeurs connues et inconnues
    \end{itemize}
};

% Étape 2
\node[box, below=of e1] (e2) {
    \textbf{ÉTAPE 2 --- AXES}
    \begin{itemize}[leftmargin=*, topsep=0pt, itemsep=0pt]
        \item Choisir un système $(x, y)$ adapté
        \item Plan incliné? $\rightarrow$ $x$ // pente, $y$ $\perp$ pente
        \item Indiquer clairement les directions positives
    \end{itemize}
};

% Étape 3
\node[box, below=of e2] (e3) {
    \textbf{ÉTAPE 3 --- ÉQUATIONS DE NEWTON}
    \begin{itemize}[leftmargin=*, topsep=0pt, itemsep=0pt]
        \item Décomposer chaque force selon $x$ et $y$
        \item Écrire : $\sum F_x = ma_x$ et $\sum F_y = ma_y$
        \item Équilibre? $\rightarrow$ $\sum F_x = 0$ et $\sum F_y = 0$
    \end{itemize}
};

% Étape 4
\node[box, below=of e3] (e4) {
    \textbf{ÉTAPE 4 --- ALGÈBRE}
    \begin{itemize}[leftmargin=*, topsep=0pt, itemsep=0pt]
        \item Résoudre \textbf{algébriquement} (avec symboles)
        \item Substituer les valeurs numériques à la fin
        \item Vérifier : unités? ordre de grandeur? signes?
    \end{itemize}
};

% Flèches
\draw[arrow] (e1) -- (e2);
\draw[arrow] (e2) -- (e3);
\draw[arrow] (e3) -- (e4);

\end{tikzpicture}
\end{center}

% =============================================================================
\subsection{Formules essentielles}
\label{subsec:resume-formules}
% =============================================================================

\subsubsection*{Équilibre (statique ou MRU)}

\begin{equationimportante}
\begin{align}
    \sum F_x &= 0 \\
    \sum F_y &= 0
\end{align}
\end{equationimportante}

\subsubsection*{Dynamique (mouvement accéléré)}

\begin{equationimportante}
\begin{align}
    \sum F_x &= ma_x \\
    \sum F_y &= ma_y
\end{align}
\end{equationimportante}

\subsubsection*{Plan incliné}

\begin{center}
\begin{tikzpicture}[scale=0.8]
    % Plan incliné
    \fill[gray!15] (0, 0) -- (5, 0) -- (5, 2.5) -- cycle;
    \draw[thick] (0, 0) -- (5, 2.5);
    \draw[thick] (0, 0) -- (5, 0);
    
    % Angle
    \draw[thick] (1.2, 0) arc (0:26.57:1.2);
    \node at (1.6, 0.35) {$\theta$};
    
    % Bloc
    \begin{scope}[rotate=26.57, shift={(2, 0)}]
        \draw[thick, fill=blue!20] (0, 0) rectangle (0.8, 0.6);
    \end{scope}
    
    % Axes
    \draw[-{Stealth}, thick, bleuimq] (1.5, 0.75) -- ({1.5 + 1.5*cos(26.57)}, {0.75 + 1.5*sin(26.57)}) node[right] {$x$};
    \draw[-{Stealth}, thick, bleuimq] (1.5, 0.75) -- ({1.5 - 0.8*sin(26.57)}, {0.75 + 0.8*cos(26.57)}) node[above] {$y$};
    
    % Formules
    \node[right, text width=6cm] at (6, 2) {
        \textbf{Avec axes // et $\perp$ à la pente :}
        \begin{align*}
            N &= mg\cos\theta \\[0.3em]
            F_{g,x} &= mg\sin\theta \\[0.3em]
            f_{s,max} &= \mu_s mg\cos\theta \\[0.3em]
            f_c &= \mu_c mg\cos\theta
        \end{align*}
    };
\end{tikzpicture}
\end{center}

\subsubsection*{Mouvement circulaire}

\begin{equationimportante}
\[
    F_c = \frac{mv^2}{r} = m\omega^2 r \qquad \text{(dirigée vers le centre)}
\]
\end{equationimportante}

\subsubsection*{Angle limite de glissement}

\begin{equationimportante}
\[
    \theta_{max} = \arctan(\mu_s)
\]
\end{equationimportante}

% =============================================================================
\subsection{Pièges à éviter}
\label{subsec:resume-pieges}
% =============================================================================

\begin{attention}[title=Les erreurs les plus fréquentes]

\begin{enumerate}
    \item \textbf{Dessiner l'accélération sur le DCL}
    
    L'accélération n'est \textbf{pas} une force! Elle est la \textit{conséquence} des forces.
    
    \item \textbf{Dessiner l'inertie ou la « force du mouvement »}
    
    L'inertie est une propriété, pas une force. Un objet n'a pas besoin de force pour maintenir sa vitesse.
    
    \item \textbf{Confondre $N$ et $mg$}
    
    $N = mg$ seulement sur surface horizontale sans autres forces verticales. En général, $N$ se calcule!
    
    \item \textbf{Oublier le poids}
    
    La Terre attire toujours l'objet. Le poids est \textbf{toujours} présent!
    
    \item \textbf{Ajouter $F_c$ comme force supplémentaire}
    
    La force centripète est le \textbf{nom} de la résultante vers le centre, pas une force additionnelle.
    
    \item \textbf{Confondre paires action-réaction}
    
    Les forces d'une paire action-réaction agissent sur des objets \textbf{différents}.
    
    \item \textbf{Se tromper de sens pour le frottement}
    
    Le frottement s'oppose au \textbf{mouvement} (ou à la tendance), pas à la force appliquée.
    
    \item \textbf{Utiliser $f_s = \mu_s N$ au lieu de $f_s \leq \mu_s N$}
    
    Le frottement statique est une inégalité! Il prend la valeur \textit{nécessaire}, jusqu'au maximum.
\end{enumerate}
\end{attention}

% =============================================================================
\subsection{Auto-évaluation des compétences}
\label{subsec:resume-competences}
% =============================================================================

Cochez les compétences que vous maîtrisez :

\begin{center}
\renewcommand{\arraystretch}{1.6}
\begin{tabular}{|L{11cm}|c|c|c|}
\hline
\rowcolor{bleuclair} \textbf{Compétence} & \textbf{Acquis} & \textbf{En cours} & \textbf{À revoir} \\
\hline
Je peux énoncer et expliquer les trois lois de Newton. & $\square$ & $\square$ & $\square$ \\
\hline
Je peux identifier toutes les forces sur un objet et tracer un DCL correct. & $\square$ & $\square$ & $\square$ \\
\hline
Je peux appliquer l'algorithme en 4 étapes pour résoudre un problème. & $\square$ & $\square$ & $\square$ \\
\hline
Je peux résoudre un problème d'équilibre (statique). & $\square$ & $\square$ & $\square$ \\
\hline
Je peux résoudre un problème de dynamique à une dimension. & $\square$ & $\square$ & $\square$ \\
\hline
Je peux résoudre un problème sur plan incliné (avec ou sans frottement). & $\square$ & $\square$ & $\square$ \\
\hline
Je peux analyser un système de deux objets reliés (Atwood, poulie). & $\square$ & $\square$ & $\square$ \\
\hline
Je peux appliquer le concept de force centripète au mouvement circulaire. & $\square$ & $\square$ & $\square$ \\
\hline
Je peux expliquer pourquoi la « force centrifuge » n'est pas une vraie force. & $\square$ & $\square$ & $\square$ \\
\hline
\end{tabular}
\end{center}

% =============================================================================
\subsection{Ce qu'il faut retenir}
\label{subsec:resume-essentiel}
% =============================================================================

\begin{remarque}[title=L'essentiel du chapitre en quelques phrases]

\begin{enumerate}
    \item \textbf{La dynamique répond au « pourquoi »} : Les forces causent les accélérations, pas les vitesses.
    
    \item \textbf{L'inertie explique la résistance au changement} : Plus un objet est massif, plus il est difficile de modifier son mouvement.
    
    \item \textbf{Le DCL est votre meilleur outil} : Un DCL correct et complet garantit presque une solution correcte.
    
    \item \textbf{L'algorithme en 4 étapes fonctionne toujours} : Schéma $\rightarrow$ Axes $\rightarrow$ Équations $\rightarrow$ Algèbre.
    
    \item \textbf{La normale se calcule, ne se devine pas} : $N = mg$ n'est vrai que dans des cas particuliers.
    
    \item \textbf{Le frottement s'oppose au mouvement relatif} : Pas à la force appliquée, pas à la vitesse absolue.
    
    \item \textbf{La force centripète est une résultante} : Ce n'est pas une nouvelle force à ajouter.
    
    \item \textbf{Résolvez algébriquement d'abord} : Les symboles révèlent la physique, les chiffres la cachent.
\end{enumerate}
\end{remarque}
