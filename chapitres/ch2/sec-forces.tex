% =============================================================================
% SECTION 2 - LES FORCES
% Blocs 2 et 3 (suite) - Semaine 5
% =============================================================================

\section{Les forces}
\label{sec:forces}

Maintenant que nous comprenons les lois de Newton, il est temps de dresser l'inventaire des forces que nous rencontrerons en mécanique. Connaître ces forces --- leur origine, leur direction, leur formule --- est essentiel pour résoudre tout problème de dynamique ou de statique.

% =============================================================================
\subsection{Le poids et la gravitation}
\label{subsec:poids}
% =============================================================================

\subsubsection{La gravitation universelle}

En 1687, Newton a proposé une loi décrivant l'attraction gravitationnelle entre deux masses quelconques :

\begin{definition}[title=Loi de la gravitation universelle]
Deux objets de masses $m_1$ et $m_2$, séparés par une distance $r$ (mesurée entre leurs centres), s'attirent mutuellement avec une force :

\begin{equationimportante}
\begin{equation}
    F_g = G \frac{m_1 m_2}{r^2}
    \label{eq:gravitation}
\end{equation}
\end{equationimportante}

où $G = \SI{6,674e-11}{N \cdot m^2/kg^2}$ est la \textbf{constante gravitationnelle universelle}.
\end{definition}

Cette loi explique aussi bien la chute d'une pomme que le mouvement des planètes autour du Soleil. C'est une force \textbf{toujours attractive} : les masses s'attirent, elles ne se repoussent jamais.

\subsubsection{Le poids : cas particulier près de la Terre}

Près de la surface de la Terre, la distance $r$ entre un objet et le centre de la Terre est approximativement constante (le rayon terrestre $R_T \approx \SI{6371}{km}$). On peut alors simplifier l'expression de la force gravitationnelle :

\begin{definition}[title=Le poids]
Le \textbf{poids} d'un objet est la force gravitationnelle exercée par la Terre sur cet objet :

\begin{equationimportante}
\begin{equation}
    \boxed{F_g = mg}
    \label{eq:poids}
\end{equation}
\end{equationimportante}

où :
\begin{itemize}
    \item $m$ est la masse de l'objet (en kg)
    \item $g = \SI{9,81}{m/s^2}$ est l'accélération gravitationnelle près de la surface terrestre
    \item $F_g$ est exprimé en newtons (N)
\end{itemize}

\textbf{Direction :} Le poids est \textbf{toujours dirigé vers le centre de la Terre}, c'est-à-dire verticalement vers le bas.
\end{definition}

\begin{remarque}[title=D'où vient $g = \SI{9,81}{m/s^2}$?]
En combinant la loi de la gravitation universelle avec $F_g = mg$, on obtient :
\[
    g = \frac{G M_T}{R_T^2} = \frac{(\SI{6,674e-11}{}) \times (\SI{5,972e24}{kg})}{(\SI{6,371e6}{m})^2} \approx \SI{9,81}{m/s^2}
\]

La valeur de $g$ varie légèrement selon l'altitude et la latitude :
\begin{itemize}
    \item Au niveau de la mer à l'équateur : $g \approx \SI{9,78}{m/s^2}$
    \item Au niveau de la mer aux pôles : $g \approx \SI{9,83}{m/s^2}$
    \item À $\SI{10}{km}$ d'altitude : $g \approx \SI{9,78}{m/s^2}$
\end{itemize}

Pour nos calculs, nous utiliserons $g = \SI{9,81}{m/s^2}$ ou $g \approx \SI{10}{m/s^2}$ pour les estimations rapides.
\end{remarque}

\begin{attention}[title=Masse vs Poids : la distinction fondamentale]
\begin{center}
\renewcommand{\arraystretch}{1.4}
\begin{tabular}{|L{6cm}|C{4cm}|C{4cm}|}
\hline
\rowcolor{bleuclair} & \textbf{Masse} & \textbf{Poids} \\
\hline
Nature & Propriété intrinsèque & Force \\
\hline
Symbole & $m$ & $F_g$ ou $P$ \\
\hline
Unité SI & kilogramme (kg) & newton (N) \\
\hline
Dépend du lieu? & Non & Oui \\
\hline
Type de grandeur & Scalaire & Vecteur \\
\hline
\end{tabular}
\end{center}

Un astronaute de $\SI{80}{kg}$ a toujours une masse de $\SI{80}{kg}$, que ce soit sur Terre, sur la Lune ou dans l'espace. Mais son poids est :
\begin{itemize}
    \item Sur Terre : $F_g = 80 \times 9,81 = \SI{785}{N}$
    \item Sur la Lune : $F_g = 80 \times 1,62 = \SI{130}{N}$
    \item En orbite (« apesanteur ») : $F_g \approx \SI{0}{N}$ (en chute libre)
\end{itemize}
\end{attention}

% =============================================================================
\subsection{La force normale}
\label{subsec:normale}
% =============================================================================

Quand un objet est en contact avec une surface, cette surface exerce une force sur l'objet. La composante de cette force qui est \textbf{perpendiculaire} à la surface s'appelle la \textbf{force normale}.

\begin{definition}[title=Force normale]
La \textbf{force normale} $\vect{N}$ est la force exercée par une surface sur un objet en contact avec elle.

\begin{itemize}
    \item \textbf{Direction :} Toujours \textbf{perpendiculaire} à la surface
    \item \textbf{Sens :} Vers l'\textbf{extérieur} de la surface (pousse l'objet, ne le tire pas)
    \item \textbf{Module :} À déterminer par les conditions d'équilibre ou la 2e loi
\end{itemize}
\end{definition}

\begin{center}
\begin{tikzpicture}[scale=1.0]
    % Surface horizontale
    \begin{scope}[xshift=-4cm]
        \fill[gray!20] (-1.5, 0) rectangle (1.5, -0.3);
        \draw[thick] (-1.5, 0) -- (1.5, 0);
        
        % Bloc
        \draw[thick, fill=blue!20] (-0.5, 0) rectangle (0.5, 0.7);
        \node at (0, 0.35) {\small $m$};
        
        % Normale
        \draw[-{Stealth[length=3mm]}, very thick, green!60!black] (0, 0.35) -- (0, 1.5);
        \node[green!60!black, right] at (0.1, 1.0) {$\vect{N}$};
        
        % Poids
        \draw[-{Stealth[length=3mm]}, very thick, red] (0, 0.35) -- (0, -0.8);
        \node[red, right] at (0.1, -0.3) {$\vect{F}_g$};
        
        \node[below] at (0, -1.3) {\small Surface horizontale};
        \node[below] at (0, -1.8) {\small $N = mg$};
    \end{scope}
    
    % Surface inclinée
    \begin{scope}[xshift=0cm]
        % Plan incliné
        \fill[gray!20] (-1.5, 0) -- (1.5, 0) -- (1.5, 1.5) -- cycle;
        \draw[thick] (-1.5, 0) -- (1.5, 1.5);
        
        % Angle
        \draw[thick] (-0.8, 0) arc (0:26.57:0.7);
        \node at (-0.4, 0.2) {\small $\theta$};
        
        % Bloc (sur la pente)
        \begin{scope}[rotate=26.57, shift={(0, 0.35)}]
            \draw[thick, fill=blue!20] (-0.35, 0) rectangle (0.35, 0.5);
            \node at (0, 0.25) {\small $m$};
        \end{scope}
        
        % Normale (perpendiculaire à la pente)
        \draw[-{Stealth[length=3mm]}, very thick, green!60!black] (-0.15, 0.75) -- (-0.75, 1.95);
        \node[green!60!black, above left] at (-0.5, 1.4) {$\vect{N}$};
        
        % Poids (vertical vers le bas)
        \draw[-{Stealth[length=3mm]}, very thick, red] (-0.15, 0.75) -- (-0.15, -0.45);
        \node[red, right] at (-0.05, 0.1) {$\vect{F}_g$};
        
        \node[below] at (0, -1.3) {\small Plan incliné};
        \node[below] at (0, -1.8) {\small $N = mg\cos\theta$};
    \end{scope}
    
    % Mur vertical
    \begin{scope}[xshift=4cm]
        % Mur
        \fill[gray!20] (0.5, -0.5) rectangle (0.8, 2);
        \draw[thick] (0.5, -0.5) -- (0.5, 2);
        
        % Bloc poussé contre le mur
        \draw[thick, fill=blue!20] (-0.2, 0.5) rectangle (0.5, 1.2);
        \node at (0.15, 0.85) {\small $m$};
        
        % Force appliquée
        \draw[-{Stealth[length=3mm]}, very thick, purple] (-1.0, 0.85) -- (-0.3, 0.85);
        \node[purple, above] at (-0.65, 0.9) {$\vect{F}$};
        
        % Normale (horizontale, vers la gauche)
        \draw[-{Stealth[length=3mm]}, very thick, green!60!black] (0.15, 0.85) -- (-0.65, 0.85);
        \node[green!60!black, below] at (-0.25, 0.75) {$\vect{N}$};
        
        \node[below] at (0.15, -1.3) {\small Mur vertical};
        \node[below] at (0.15, -1.8) {\small $N = F$};
    \end{scope}
\end{tikzpicture}
\end{center}

\subsubsection*{L'origine physique de la normale}

À l'échelle microscopique, la force normale provient de la \textbf{répulsion électromagnétique} entre les nuages électroniques des atomes. Quand vous posez un livre sur une table, les électrons du livre repoussent les électrons de la table --- c'est cette répulsion qui empêche le livre de traverser la table!

\begin{attention}[title=La normale n'est pas toujours égale au poids!]
Sur une surface horizontale sans autres forces verticales, on a effectivement $N = mg$. Mais ce n'est \textbf{pas} une formule générale!

\begin{itemize}
    \item Sur un plan incliné : $N = mg\cos\theta < mg$
    \item Si on pousse l'objet vers le bas : $N > mg$
    \item Si on tire l'objet vers le haut : $N < mg$
    \item Dans un ascenseur qui accélère vers le haut : $N > mg$
\end{itemize}

\textbf{La normale est toujours déterminée par les équations de Newton}, jamais par une formule mémorisée.
\end{attention}

% =============================================================================
\subsection{La tension}
\label{subsec:tension}
% =============================================================================

\begin{definition}[title=Tension]
La \textbf{tension} $\vect{T}$ est la force exercée par une corde, un câble, une chaîne ou tout objet flexible sur un objet auquel il est attaché.

\begin{itemize}
    \item \textbf{Direction :} Parallèle à la corde, le long de celle-ci
    \item \textbf{Sens :} La tension \textbf{tire} toujours l'objet (une corde ne peut pas pousser!)
    \item \textbf{Module :} À déterminer par les conditions d'équilibre ou la 2e loi
\end{itemize}
\end{definition}

\begin{center}
\begin{tikzpicture}[scale=1.0]
    % Support
    \fill[gray!50] (-0.3, 2.5) rectangle (0.3, 3);
    \draw[thick] (-0.5, 2.5) -- (0.5, 2.5);
    
    % Corde
    \draw[thick, brown] (0, 2.5) -- (0, 1);
    
    % Masse suspendue
    \draw[thick, fill=blue!20] (-0.4, 0.3) rectangle (0.4, 1);
    \node at (0, 0.65) {\small $m$};
    
    % Tension sur la masse
    \draw[-{Stealth[length=3mm]}, very thick, orange] (0, 0.65) -- (0, 1.8);
    \node[orange, right] at (0.1, 1.3) {$\vect{T}$};
    
    % Poids
    \draw[-{Stealth[length=3mm]}, very thick, red] (0, 0.65) -- (0, -0.5);
    \node[red, right] at (0.1, 0.0) {$\vect{F}_g$};
    
    % Annotation
    \node[text width=5cm, align=center] at (0, -1.5) {\small La corde tire la masse vers le haut avec une tension $T = mg$ (équilibre)};
\end{tikzpicture}
\end{center}

\subsubsection*{Hypothèse de la corde idéale}

Dans ce cours, nous considérerons généralement des cordes \textbf{idéales} :

\begin{itemize}
    \item \textbf{Masse négligeable} : La corde elle-même ne contribue pas à l'inertie du système.
    \item \textbf{Inextensible} : La corde ne s'étire pas sous la tension.
    \item \textbf{Tension uniforme} : La tension est la même partout dans la corde.
\end{itemize}

\begin{remarque}[title=Application maritime : les amarres]
Les amarres d'un navire sont soumises à des tensions considérables. Un navire de croisière de 100\,000 tonnes amarré dans un port peut exercer des forces de plusieurs centaines de kilonewtons sur ses amarres lors de vents forts ou de courants.

La connaissance des tensions dans les amarres est essentielle pour :
\begin{itemize}
    \item Choisir des cordages de résistance appropriée
    \item Répartir les amarres de façon à équilibrer les forces
    \item Éviter la rupture des amarres (danger mortel!)
\end{itemize}
\end{remarque}

% =============================================================================
\subsection{Le frottement}
\label{subsec:frottement}
% =============================================================================

Quand deux surfaces sont en contact, une force s'oppose au glissement (ou à la tendance au glissement) de l'une sur l'autre : c'est le \textbf{frottement}.

\begin{definition}[title=Force de frottement]
La \textbf{force de frottement} $\vect{f}$ est une force qui s'oppose au mouvement relatif (ou à la tendance au mouvement) entre deux surfaces en contact.

\begin{itemize}
    \item \textbf{Direction :} Parallèle à la surface de contact
    \item \textbf{Sens :} Opposé au mouvement (ou à la tendance au mouvement)
\end{itemize}
\end{definition}

On distingue deux types de frottement sec :

\subsubsection{Le frottement statique}

Le frottement \textbf{statique} agit entre deux surfaces qui \textbf{ne glissent pas} l'une sur l'autre. C'est une force « adaptative » : elle s'ajuste pour empêcher le glissement, jusqu'à une valeur maximale.

\begin{equationimportante}
\begin{equation}
    \boxed{f_s \leq \mu_s N}
    \label{eq:frottement-statique}
\end{equation}
\end{equationimportante}

où :
\begin{itemize}
    \item $\mu_s$ est le \textbf{coefficient de frottement statique} (sans dimension)
    \item $N$ est la force normale
    \item L'inégalité signifie que $f_s$ prend la valeur \textit{nécessaire} pour empêcher le glissement, jusqu'à un maximum de $\mu_s N$
\end{itemize}

\subsubsection{Le frottement cinétique}

Le frottement \textbf{cinétique} (ou dynamique) agit entre deux surfaces qui \textbf{glissent} l'une sur l'autre. Son module est essentiellement constant :

\begin{equationimportante}
\begin{equation}
    \boxed{f_c = \mu_c N}
    \label{eq:frottement-cinetique}
\end{equation}
\end{equationimportante}

où $\mu_c$ est le \textbf{coefficient de frottement cinétique}.

\begin{remarque}[title=Relation entre $\mu_s$ et $\mu_c$]
En général, $\mu_s > \mu_c$ : il est plus difficile de \textit{mettre} un objet en mouvement que de le \textit{maintenir} en mouvement.

C'est pourquoi un objet qu'on pousse semble « partir » brusquement une fois qu'il commence à glisser --- on passe soudainement d'un frottement statique maximal à un frottement cinétique plus faible.
\end{remarque}

\subsubsection*{Coefficients de frottement typiques}

\begin{center}
\renewcommand{\arraystretch}{1.3}
\begin{tabular}{|l|c|c|}
\hline
\rowcolor{bleuclair} \textbf{Surfaces en contact} & $\mu_s$ & $\mu_c$ \\
\hline
Acier sur acier (sec) & 0,74 & 0,57 \\
\hline
Acier sur acier (lubrifié) & 0,15 & 0,06 \\
\hline
Caoutchouc sur asphalte (sec) & 0,9 & 0,7 \\
\hline
Caoutchouc sur asphalte (mouillé) & 0,7 & 0,5 \\
\hline
Bois sur bois & 0,5 & 0,3 \\
\hline
Glace sur glace & 0,1 & 0,03 \\
\hline
Téflon sur acier & 0,04 & 0,04 \\
\hline
\end{tabular}
\end{center}

\begin{remarque}[title=Ce dont le frottement ne dépend PAS]
En première approximation (modèle de Coulomb), le frottement :
\begin{itemize}
    \item \textbf{Ne dépend pas de l'aire de contact} : Une brique posée à plat ou sur la tranche a le même frottement.
    \item \textbf{Ne dépend pas de la vitesse de glissement} (pour le frottement cinétique).
\end{itemize}

Ces approximations sont valables pour le « frottement sec ». Le frottement dans les fluides (air, eau) obéit à des lois différentes.
\end{remarque}

% =============================================================================
\subsection{La force centripète}
\label{subsec:force-centripete}
% =============================================================================

Au chapitre précédent, nous avons vu qu'un objet en mouvement circulaire possède une \textbf{accélération centripète} dirigée vers le centre du cercle, même si sa vitesse (en module) reste constante. Selon la deuxième loi de Newton, cette accélération \textit{doit} être causée par une force.

\begin{definition}[title=Force centripète]
La \textbf{force centripète} est la force résultante nécessaire pour maintenir un objet sur une trajectoire circulaire. Elle est toujours dirigée \textbf{vers le centre} de la trajectoire.

\begin{equationimportante}
\begin{equation}
    \boxed{F_c = ma_c = m\frac{v^2}{r} = m\omega^2 r}
    \label{eq:force-centripete}
\end{equation}
\end{equationimportante}
\end{definition}

\begin{attention}[title=La force centripète n'est PAS un nouveau type de force!]
La force centripète n'est \textbf{pas} une force supplémentaire à ajouter sur un diagramme de corps libre. C'est simplement le \textbf{nom} qu'on donne à la composante de la force résultante qui est dirigée vers le centre du cercle.

Cette force peut être fournie par :
\begin{itemize}
    \item La \textbf{tension} d'une corde (balle au bout d'une ficelle)
    \item Le \textbf{frottement} (voiture dans un virage)
    \item La \textbf{gravité} (satellite en orbite)
    \item La \textbf{normale} (dans un virage bancé)
    \item Une \textbf{combinaison} de forces
\end{itemize}
\end{attention}

\begin{center}
\begin{tikzpicture}[scale=0.9]
    % Cercle de la trajectoire
    \draw[dashed, gray] (0,0) circle (2.5);
    
    % Centre
    \fill (0,0) circle (2pt);
    \node[below] at (0, -0.2) {\small Centre};
    
    % Voiture (rectangle)
    \begin{scope}[rotate=45, shift={(2.5, 0)}]
        \draw[thick, fill=blue!30, rounded corners=1pt] (-0.3, -0.15) rectangle (0.3, 0.15);
    \end{scope}
    
    % Vitesse (tangente)
    \draw[-{Stealth[length=3mm]}, very thick, purple] (1.77, 1.77) -- (0.77, 2.77);
    \node[purple, above] at (1.1, 2.4) {$\vect{v}$};
    
    % Force centripète (vers le centre)
    \draw[-{Stealth[length=3mm]}, very thick, red] (1.77, 1.77) -- (0.6, 0.6);
    \node[red, below left] at (1.1, 1.1) {$\vect{F}_c$};
    
    % Rayon
    \draw[thick] (0,0) -- (1.77, 1.77);
    \node at (0.6, 1.1) {$r$};
    
    % Annotation
    \node[text width=6cm, align=center] at (0, -3.5) {\small La force centripète est toujours perpendiculaire à la vitesse et dirigée vers le centre.};
\end{tikzpicture}
\end{center}

\subsubsection*{Et la « force centrifuge »?}

Vous avez certainement ressenti cette sensation d'être « poussé vers l'extérieur » dans un virage serré. Pourtant, aucune force ne vous pousse réellement vers l'extérieur!

\begin{remarque}[title=La force centrifuge n'existe pas (dans un référentiel inertiel)]
Ce que vous ressentez comme une « force centrifuge » est en réalité votre \textbf{inertie} : votre corps tend à continuer en ligne droite, tandis que la voiture tourne sous vous.

Du point de vue du sol (référentiel inertiel), aucune force ne vous pousse vers l'extérieur. C'est le siège et la ceinture qui vous \textit{tirent} vers l'intérieur pour vous faire tourner avec la voiture.

La « force centrifuge » n'apparaît que si vous analysez le mouvement depuis la voiture (référentiel non inertiel), ce que nous n'avons pas besoin de faire dans ce cours.
\end{remarque}

\begin{exemple}{Cargaison dans un virage}{cargaison-virage}
Un navire effectue un virage à vitesse constante. Une caisse mal arrimée sur le pont glisse vers l'extérieur du virage. Que s'est-il passé?

\textbf{Analyse incorrecte :} « La force centrifuge a poussé la caisse vers l'extérieur. »

\textbf{Analyse correcte :} Le navire tourne, ce qui nécessite une force centripète pour tout ce qui est à bord. Pour la caisse, cette force devrait être fournie par le \textbf{frottement} avec le pont. Mais si le frottement est insuffisant (caisse mal arrimée, pont glissant), la caisse ne peut pas tourner avec le navire. Selon la première loi de Newton, elle \textbf{continue donc en ligne droite} --- ce qui, vu depuis le navire qui tourne, ressemble à un mouvement vers l'extérieur.

\textbf{Leçon :} La cargaison doit être correctement arrimée pour que les forces de friction puissent fournir la force centripète nécessaire lors des manœuvres!
\end{exemple}

% =============================================================================
\subsection{Tableau récapitulatif des forces}
\label{subsec:tableau-forces}
% =============================================================================

\begin{center}
\renewcommand{\arraystretch}{2.0}
\begin{tabular}{|L{2.5cm}|C{3.5cm}|L{7.5cm}|}
\hline
\rowcolor{bleuclair} \textbf{Force} & \textbf{Formule} & \textbf{Direction et caractéristiques} \\
\hline
\textbf{Poids} & $F_g = mg$ & Toujours verticale, vers le bas (centre de la Terre) \\
\hline
\textbf{Normale} & $N$ (à déterminer) & Perpendiculaire à la surface, vers l'extérieur \\
\hline
\textbf{Tension} & $T$ (à déterminer) & Parallèle à la corde, tire sur l'objet \\
\hline
\textbf{Frottement statique} & $f_s \leq \mu_s N$ & Parallèle à la surface, oppose la tendance au glissement \\
\hline
\textbf{Frottement cinétique} & $f_c = \mu_c N$ & Parallèle à la surface, oppose le mouvement \\
\hline
\textbf{Force centripète} & $F_c = \dfrac{mv^2}{r}$ & Vers le centre de la trajectoire circulaire \\
\hline
\end{tabular}
\end{center}

\begin{pratiqueautonome}[title=Identification des forces]

Pour chaque situation, identifiez \textbf{toutes} les forces agissant sur l'objet indiqué en gras. Précisez la direction de chaque force.

\begin{enumerate}
    \item Un \textbf{livre} posé sur une table horizontale.
    
    \item Une \textbf{caisse} tirée sur le sol par une corde faisant un angle de 30° avec l'horizontale.
    
    \item Un \textbf{bloc} immobile sur un plan incliné à 25°.
    
    \item Un \textbf{bloc} glissant vers le bas d'un plan incliné (avec frottement).
    
    \item Une \textbf{voiture} prenant un virage horizontal à vitesse constante.
    
    \item Un \textbf{conteneur} suspendu par deux câbles faisant des angles différents avec l'horizontale.
\end{enumerate}

\tcblower
\textbf{Réponses :}
\begin{enumerate}
    \item Poids (vers le bas), Normale (vers le haut)
    
    \item Poids (vers le bas), Normale (vers le haut), Tension (30° vers le haut-avant), Frottement (opposé au mouvement)
    
    \item Poids (vertical, vers le bas), Normale (perpendiculaire au plan), Frottement statique (parallèle au plan, vers le haut)
    
    \item Poids (vertical, vers le bas), Normale (perpendiculaire au plan), Frottement cinétique (parallèle au plan, vers le haut)
    
    \item Poids (vers le bas), Normale (vers le haut), Frottement statique (horizontal, vers le centre du virage)
    
    \item Poids (vers le bas), Tension 1 (vers le câble 1), Tension 2 (vers le câble 2)
\end{enumerate}
\end{pratiqueautonome}
