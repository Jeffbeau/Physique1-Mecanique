% =============================================================================
% SECTION 2 - LA BOÎTE À OUTILS DES FORCES
% =============================================================================

\section{La « boîte à outils » des forces}
\label{sec:forces}

Maintenant que nous comprenons les trois lois de Newton, nous devons identifier les forces courantes que nous rencontrerons dans les problèmes de mécanique. Ces forces sont des manifestations macroscopiques d'interactions fondamentales entre les particules qui constituent la matière.

Avant de détailler chaque force, rappelons ce qu'est une force.

\begin{definition}[title=La force]
Une \textbf{force} est une action exercée par un objet sur un autre, qui se manifeste par une poussée ou une traction. Une force tend à modifier l'état de mouvement d'un objet ou à le déformer.

Les forces sont \textbf{invisibles}. On ne peut pas voir une force, la toucher ou la peser directement. Ce qu'on observe, ce sont les \textit{effets} des forces : un objet qui accélère, qui ralentit, qui change de direction, ou qui se déforme.

L'unité de force dans le système international est le \textbf{newton} (N).
\end{definition}

Les forces sont des \textbf{grandeurs vectorielles}. Une force possède :
\begin{itemize}
    \item Un \textbf{module} (son intensité) — combien de newtons
    \item Une \textbf{direction} — vers où elle pointe
    \item Un \textbf{point d'application} — où elle s'exerce sur l'objet
\end{itemize}

La somme de toutes les forces agissant sur un objet s'appelle la \textbf{force résultante} :
\begin{equation}
    \vect{F}_{\text{rés}} = \sum \vect{F} = \vect{F}_1 + \vect{F}_2 + \vect{F}_3 + \cdots
\end{equation}

% -----------------------------------------------------------------------------
\subsection{La gravité et le poids}
\label{subsec:gravite}
% -----------------------------------------------------------------------------

\subsubsection*{La loi de la gravitation universelle}

L'une des plus grandes réalisations de Newton fut de comprendre que la force qui fait tomber une pomme est \textit{la même} que celle qui maintient la Lune en orbite autour de la Terre. Avant Newton, on pensait que les « lois célestes » étaient différentes des « lois terrestres ». Newton unifia ces deux mondes en une seule loi.

\begin{definition}[title=Loi de la gravitation universelle]
Deux objets de masses $m_1$ et $m_2$, séparés par une distance $r$ (mesurée entre leurs centres), s'attirent mutuellement avec une force :
\begin{equation}
    \boxed{F_g = G\frac{m_1 m_2}{r^2}}
    \label{eq:gravitation}
\end{equation}
où $G = 6{,}67 \times 10^{-11}~\text{N}\cdot\text{m}^2/\text{kg}^2$ est la \textbf{constante de gravitation universelle}.
\end{definition}

Cette loi est remarquable à plusieurs égards :
\begin{itemize}
    \item Elle est \textbf{universelle} : elle s'applique à tous les objets possédant une masse, des pommes aux galaxies.
    \item La force est \textbf{attractive} : les masses s'attirent toujours, jamais ne se repoussent.
    \item La force décroît avec le carré de la distance : si on double la distance, la force est divisée par 4.
    \item Elle respecte la troisième loi de Newton : $m_1$ attire $m_2$ avec la même force que $m_2$ attire $m_1$.
\end{itemize}

\begin{remarque}[title=Newton et Kepler]
Avec cette loi et ses trois lois du mouvement, Newton put démontrer mathématiquement les trois lois empiriques que Kepler avait découvertes en observant les planètes. C'est l'un des plus grands triomphes de la physique : des phénomènes aussi différents que la chute d'une pierre et l'orbite de Mars s'expliquent par les mêmes principes fondamentaux.
\end{remarque}

\subsubsection*{Le poids : la gravité près de la surface terrestre}

Considérons un objet de masse $m$ situé près de la surface de la Terre. La Terre (masse $M_T = 5{,}97 \times 10^{24}$~kg, rayon $R_T = 6{,}37 \times 10^6$~m) attire cet objet avec une force :
\[
F_g = G\frac{M_T \cdot m}{R_T^2} = \left(G\frac{M_T}{R_T^2}\right) \cdot m
\]

Le terme entre parenthèses ne dépend que des propriétés de la Terre. On le note $g$ :
\[
g = G\frac{M_T}{R_T^2} = 6{,}67 \times 10^{-11} \times \frac{5{,}97 \times 10^{24}}{(6{,}37 \times 10^6)^2} = 9{,}81~\text{m/s}^2
\]

C'est l'\textbf{accélération gravitationnelle} à la surface de la Terre.

\begin{definition}[title=Le poids]
Le \textbf{poids} $\vect{F}_g$ (ou $\vect{P}$) est la force gravitationnelle exercée par la Terre sur un objet situé près de sa surface :
\begin{equation}
    \boxed{F_g = mg \qquad \text{où} \quad g = 9{,}81~\text{m/s}^2}
    \label{eq:poids}
\end{equation}
Cette force est appliquée au centre de masse de l'objet et est toujours dirigée vers le centre de la Terre (c'est-à-dire vers le bas).
\end{definition}

\begin{attention}[title=Masse et poids : ne pas confondre!]
\begin{center}
\renewcommand{\arraystretch}{1.5}
\begin{tabular}{|l|c|c|}
\hline
\rowcolor{bleuclair} & \textbf{Masse} & \textbf{Poids} \\
\hline
Nature & Propriété intrinsèque & Force \\
\hline
Mesure & L'inertie & L'attraction gravitationnelle \\
\hline
Unité SI & kilogramme (kg) & newton (N) \\
\hline
Varie-t-elle selon le lieu? & Non & Oui \\
\hline
\end{tabular}
\end{center}

Un astronaute de 80~kg a une masse de 80~kg partout dans l'univers. Sur Terre, son poids est $80 \times 9{,}81 = 785$~N. Sur la Lune, où $g \approx 1{,}6~\text{m/s}^2$, son poids n'est que de 128~N. Dans l'espace loin de tout astre, son poids est essentiellement nul — mais sa masse reste 80~kg.
\end{attention}

% -----------------------------------------------------------------------------
\subsection{La tension}
\label{subsec:tension}
% -----------------------------------------------------------------------------

\begin{definition}[title=La tension]
La \textbf{tension} $\vect{T}$ est la force exercée par une corde, un câble, une chaîne ou une amarre sur un objet auquel elle est attachée. La tension est :
\begin{itemize}
    \item Appliquée au point d'attache
    \item Toujours parallèle à la corde
    \item Dirigée de l'objet vers la corde (elle « tire » sur l'objet, jamais ne pousse)
\end{itemize}
\end{definition}

Dans les situations idéalisées (corde de masse négligeable, sans friction dans les poulies), la tension est la même en tout point de la corde. On dit que la corde « transmet » la force d'un bout à l'autre.

\begin{exemple}
Sur un navire à quai, les amarres exercent des tensions sur le navire et sur les bollards du quai. Si une amarre exerce une tension de 50~kN sur le navire, elle exerce également une tension de 50~kN sur le bollard (troisième loi de Newton). La corde est « sous tension » de 50~kN.
\end{exemple}

% -----------------------------------------------------------------------------
\subsection{La normale}
\label{subsec:normale}
% -----------------------------------------------------------------------------

Quand vous posez un livre sur une table, pourquoi ne traverse-t-il pas la table? Quelque chose doit s'opposer au poids du livre et le maintenir en place. Cette force s'appelle la \textbf{normale}.

\begin{definition}[title=La normale]
La \textbf{normale} $\vect{N}$ est la force exercée par une surface sur un objet en contact avec elle. Cette force :
\begin{itemize}
    \item Est toujours \textbf{perpendiculaire} à la surface de contact (d'où son nom)
    \item Est dirigée vers l'extérieur de la surface
    \item Empêche les objets solides de se traverser mutuellement
\end{itemize}
\end{definition}

\subsubsection*{L'origine microscopique de la normale}

À l'échelle atomique, les surfaces solides sont constituées d'atomes dont les électrons forment des « nuages » chargés négativement. Quand deux surfaces s'approchent, leurs nuages électroniques se repoussent mutuellement par répulsion électrostatique.

\begin{remarque}[title=Vous ne touchez jamais vraiment rien!]
Quand vous posez votre main sur une table, vos atomes ne « touchent » jamais vraiment les atomes de la table. Les nuages électroniques de votre main et de la table se repoussent avant de se toucher réellement. Vous « flottez » à une distance d'environ $10^{-10}$~m (un dixième de nanomètre) au-dessus de la table!

Ce que vous percevez comme le « contact » est en fait cette répulsion électromagnétique entre les électrons des deux surfaces.
\end{remarque}

\subsubsection*{La normale est une force de réaction passive}

Un point crucial à comprendre : la normale n'est pas une force « active » que la surface décide d'exercer. C'est une force de \textbf{réaction passive} qui s'ajuste automatiquement pour empêcher la pénétration.

\begin{itemize}
    \item Si vous posez un livre de 1~kg sur une table, la normale est d'environ 9{,}8~N.
    \item Si vous posez un livre de 2~kg, la normale devient environ 19{,}6~N.
    \item Si vous appuyez sur le livre avec votre main (disons 10~N supplémentaires), la normale augmente de 10~N pour compenser.
\end{itemize}

La normale « fait ce qu'il faut » pour empêcher les objets de se traverser, ni plus, ni moins.

\begin{attention}
Sur un plan incliné, la normale n'est \textbf{pas} égale au poids! Elle est égale à la composante du poids perpendiculaire à la surface, soit :
\[ N = mg\cos\theta \]
où $\theta$ est l'angle d'inclinaison du plan par rapport à l'horizontale.
\end{attention}

% -----------------------------------------------------------------------------
\subsection{Le frottement}
\label{subsec:frottement}
% -----------------------------------------------------------------------------

\begin{definition}[title=La force de frottement]
Le \textbf{frottement} $\vect{f}$ est la force qui s'oppose au glissement (ou à la tendance au glissement) entre deux surfaces en contact. Cette force est :
\begin{itemize}
    \item Toujours \textbf{parallèle} à la surface de contact
    \item Toujours \textbf{opposée} au mouvement (ou à la tendance au mouvement)
\end{itemize}
\end{definition}

Le frottement a également une origine électromagnétique : les aspérités microscopiques des deux surfaces s'accrochent les unes aux autres, et les liaisons chimiques temporaires qui se forment doivent être brisées pour permettre le glissement.

On distingue deux types de frottement sec :

\subsubsection*{Le frottement statique}

Le frottement statique agit entre deux surfaces qui ne glissent \textbf{pas} l'une sur l'autre. C'est une force « adaptative » qui s'ajuste pour empêcher le glissement, jusqu'à une valeur maximale :
\begin{equation}
    \boxed{f_s \leq \mu_s N}
    \label{eq:frottement_stat}
\end{equation}
où $\mu_s$ est le \textbf{coefficient de frottement statique} (sans dimension) et $N$ la normale.

\subsubsection*{Le frottement cinétique}

Le frottement cinétique agit entre deux surfaces qui glissent l'une sur l'autre. Son module est essentiellement constant :
\begin{equation}
    \boxed{f_c = \mu_c N}
    \label{eq:frottement_cin}
\end{equation}
où $\mu_c$ est le \textbf{coefficient de frottement cinétique}.

\begin{remarque}
En général, $\mu_s > \mu_c$ : il est plus difficile de mettre un objet en mouvement que de le maintenir en mouvement. C'est pourquoi un objet qu'on pousse semble « partir » brusquement une fois qu'il commence à glisser — on passe soudainement d'un frottement statique maximal à un frottement cinétique plus faible.
\end{remarque}

Les coefficients de frottement dépendent de la nature des deux surfaces en contact (acier sur acier, caoutchouc sur asphalte, etc.), mais pas de l'aire de contact ni de la vitesse de glissement (en première approximation).

\begin{center}
\renewcommand{\arraystretch}{1.3}
\begin{tabular}{|l|c|c|}
\hline
\rowcolor{bleuclair} \textbf{Surfaces en contact} & $\mu_s$ & $\mu_c$ \\
\hline
Acier sur acier (sec) & 0,74 & 0,57 \\
\hline
Caoutchouc sur asphalte (sec) & 0,9 & 0,7 \\
\hline
Bois sur bois & 0,5 & 0,3 \\
\hline
Glace sur glace & 0,1 & 0,03 \\
\hline
\end{tabular}
\end{center}

% -----------------------------------------------------------------------------
\subsection{La force centripète}
\label{subsec:force_centripete}
% -----------------------------------------------------------------------------

Au chapitre précédent, nous avons vu qu'un objet qui se déplace sur une trajectoire circulaire possède une accélération centripète dirigée vers le centre du cercle, même si sa vitesse (en module) reste constante. Selon la deuxième loi de Newton, cette accélération doit être causée par une force.

\begin{definition}[title=La force centripète]
La \textbf{force centripète} est la force résultante nécessaire pour maintenir un objet sur une trajectoire circulaire. Elle est toujours dirigée vers le centre de la trajectoire et a pour module :
\begin{equation}
    \boxed{F_c = ma_c = m\frac{v^2}{r} = m\omega^2 r}
    \label{eq:force_centripete}
\end{equation}
\end{definition}

Il est crucial de comprendre que la force centripète n'est \textbf{pas} un nouveau type de force. C'est simplement le nom qu'on donne à la composante de la force résultante dirigée vers le centre du cercle. Cette force peut être fournie par :
\begin{itemize}
    \item La tension d'une corde (balle attachée qu'on fait tourner)
    \item Le frottement (voiture dans un virage)
    \item La gravité (satellite en orbite, Lune autour de la Terre)
    \item La normale (manège, mur d'un cylindre en rotation)
    \item Une combinaison de plusieurs forces
\end{itemize}

\subsubsection*{La « force centrifuge » : ce qu'elle est vraiment}

Quand vous êtes passager dans une voiture qui prend un virage serré, vous avez la sensation d'être « poussé » vers l'extérieur du virage. Cette sensation est souvent attribuée à une « force centrifuge ». Mais cette force existe-t-elle vraiment?

\begin{attention}[title=La force centrifuge n'est pas une force newtonienne]
La « force centrifuge » n'est \textbf{pas} une force au sens de Newton. Aucun objet extérieur ne vous pousse vers l'extérieur du virage.

Ce que vous ressentez, c'est votre \textbf{inertie}. Selon la première loi de Newton, votre corps veut continuer en ligne droite. Mais la voiture tourne, et le siège vous force à tourner avec elle en exerçant une force centripète sur vous (vers le centre du virage).

Depuis votre point de vue (référentiel de la voiture), vous semblez être poussé vers l'extérieur. Mais depuis le point de vue d'un observateur immobile sur le bord de la route (référentiel inertiel), c'est simplement la voiture qui tourne autour de vous pendant que vous tentez d'aller tout droit.

La « force centrifuge » est une manifestation de l'inertie observée depuis un référentiel non inertiel.
\end{attention}

\begin{exemple}[title=Cargaison dans un virage]
Un navire effectue un virage serré. Une caisse mal arrimée sur le pont glisse vers l'extérieur du virage. Que s'est-il passé?

\textbf{Analyse incorrecte :} « La force centrifuge a poussé la caisse vers l'extérieur. »

\textbf{Analyse correcte :} Le navire tourne, ce qui nécessite une force centripète pour tout ce qui est à bord. Pour la caisse, cette force devrait être fournie par le frottement avec le pont. Mais si le frottement est insuffisant (caisse mal arrimée, pont glissant), la caisse ne peut pas tourner avec le navire. Selon la première loi de Newton, elle continue donc en ligne droite — ce qui, vu depuis le navire qui tourne, ressemble à un mouvement vers l'extérieur.

\textbf{Morale :} La cargaison doit être correctement arrimée pour que les forces de friction puissent fournir la force centripète nécessaire lors des manœuvres!
\end{exemple}

% -----------------------------------------------------------------------------
\subsection{Tableau récapitulatif des forces}
\label{subsec:tableau_forces}
% -----------------------------------------------------------------------------

\begin{center}
\renewcommand{\arraystretch}{1.8}
\begin{tabular}{|L{2.8cm}|C{3.5cm}|L{7cm}|}
\hline
\rowcolor{bleuclair} \textbf{Force} & \textbf{Formule} & \textbf{Direction et caractéristiques} \\
\hline
Poids & $F_g = mg$ & Toujours vers le bas (centre de la Terre) \\
\hline
Tension & $T$ (à déterminer) & Parallèle à la corde, tire sur l'objet \\
\hline
Normale & $N$ (à déterminer) & Perpendiculaire à la surface, vers l'extérieur \\
\hline
Frottement statique & $f_s \leq \mu_s N$ & Parallèle à la surface, oppose la tendance au glissement \\
\hline
Frottement cinétique & $f_c = \mu_c N$ & Parallèle à la surface, oppose le mouvement \\
\hline
Force centripète & $F_c = m\dfrac{v^2}{r}$ & Vers le centre de la trajectoire circulaire \\
\hline
\end{tabular}
\end{center}

% -----------------------------------------------------------------------------
% PRATIQUE AUTONOME - Section 2
% -----------------------------------------------------------------------------

\begin{pratiqueautonome}[title=Pratique autonome 2.2 — Identification des forces]
Pour chaque situation, identifiez \textbf{toutes} les forces agissant sur l'objet indiqué et dessinez un DCL approximatif.

\begin{enumerate}
    \item Un livre posé sur une table horizontale.
    \item Une caisse tirée sur le sol par une corde faisant un angle de 30° avec l'horizontale.
    \item Un bloc immobile sur un plan incliné à 25°.
    \item Un bloc glissant vers le bas d'un plan incliné à 25° (avec frottement).
    \item Une voiture prenant un virage horizontal à vitesse constante.
    \item Un conteneur suspendu par deux câbles faisant des angles différents avec l'horizontale.
\end{enumerate}

\tcblower
\textit{Forces à identifier :}
\begin{enumerate}
    \item Poids (bas), Normale (haut)
    \item Poids (bas), Normale (haut), Tension (30° vers le haut), Frottement (opposé au mouvement)
    \item Poids (bas), Normale ($\perp$ plan), Frottement statique (// plan, vers le haut)
    \item Poids (bas), Normale ($\perp$ plan), Frottement cinétique (// plan, vers le haut)
    \item Poids (bas), Normale (haut), Frottement statique (vers le centre du virage)
    \item Poids (bas), Tension 1 (vers le câble 1), Tension 2 (vers le câble 2)
\end{enumerate}
\end{pratiqueautonome}
