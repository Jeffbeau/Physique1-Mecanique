% =============================================================================
% SECTION 4 - DYNAMIQUE
% Blocs 5 et 6 (4 heures - Semaine 7)
% =============================================================================

\section{Dynamique : la deuxième loi en action}
\label{sec:dynamique}

% =============================================================================
\subsection{De l'équilibre à l'accélération}
\label{subsec:equilibre-acceleration}
% =============================================================================

Jusqu'à maintenant, nous avons étudié des situations où les objets sont en équilibre : la force résultante est nulle, donc l'accélération est nulle. Mais que se passe-t-il quand $\sum \vect{F} \neq \vect{0}$?

\begin{remarque}[title={Maintenant, ça bouge!}]
\begin{center}
\renewcommand{\arraystretch}{1.5}
\begin{tabular}{|c|c|c|}
\hline
\rowcolor{bleuclair} & \textbf{Équilibre (statique)} & \textbf{Dynamique} \\
\hline
Force résultante & $\sum \vect{F} = \vect{0}$ & $\sum \vect{F} \neq \vect{0}$ \\
\hline
Accélération & $\vect{a} = \vect{0}$ & $\vect{a} \neq \vect{0}$ \\
\hline
Mouvement & Repos ou MRU & Accéléré (MRUA ou autre) \\
\hline
Équations & $\sum F_x = 0$, $\sum F_y = 0$ & $\sum F_x = ma_x$, $\sum F_y = ma_y$ \\
\hline
\end{tabular}
\end{center}

\textbf{L'algorithme de résolution reste le même!} Seules les équations de Newton changent.
\end{remarque}

% =============================================================================
\subsection{Problèmes à une dimension}
\label{subsec:dynamique-1d}
% =============================================================================

Commençons par des situations où toutes les forces et le mouvement sont alignés sur une seule direction.

% -----------------------------------------------------------------------------
\subsubsection{Une seule force}
% -----------------------------------------------------------------------------

\begin{exemple}{Remorqueur en accélération}{remorqueur-accel}
Un remorqueur de masse $m = \SI{800}{tonnes}$ démarre du repos. Ses moteurs fournissent une poussée de $F = \SI{120}{kN}$. En négligeant la résistance de l'eau au démarrage, quelle est son accélération initiale?

\tcblower

\textbf{\underline{Étape 1 — SCHÉMA et DCL}}

\begin{center}
\begin{tikzpicture}[scale=0.8]
    % Schéma de situation (gauche)
    \begin{scope}[xshift=-4cm]
        \node[above, font=\bfseries] at (0, 2.2) {Situation};
        
        % Eau
        \fill[blue!10] (-2, -0.5) rectangle (2, 0);
        \draw[thick, blue!50] (-2, 0) -- (2, 0);
        
        % Remorqueur simplifié
        \draw[thick, fill=gray!30] (-1.2, 0) -- (1.5, 0) -- (1.8, 0.3) -- (1.5, 0.6) -- (-1.2, 0.6) -- (-1.4, 0.3) -- cycle;
        \node at (0.2, 0.3) {\small $m$};
        
        % Flèche mouvement
        \draw[-{Stealth}, thick, purple] (2, 0.3) -- (3, 0.3);
        \node[purple, above] at (2.5, 0.4) {\small $\vect{v}$};
    \end{scope}
    
    % DCL (droite)
    \begin{scope}[xshift=3cm]
        \node[above, font=\bfseries] at (0, 2.2) {DCL};
        
        % Point
        \fill[blue!50] (0, 0) circle (5pt);
        
        % Axes
        \draw[-{Stealth}, thick] (0, 0) -- (3, 0) node[right] {$x$};
        \draw[-{Stealth}, thick] (0, 0) -- (0, 2) node[above] {$y$};
        
        % Forces
        \draw[-{Stealth[length=3mm]}, very thick, red] (0, 0) -- (0, -1.2);
        \node[red, left] at (-0.1, -0.6) {$\vect{F}_g$};
        
        \draw[-{Stealth[length=3mm]}, very thick, green!60!black] (0, 0) -- (0, 1.2);
        \node[green!60!black, left] at (-0.1, 0.6) {$\vect{F}_A$};
        
        \draw[-{Stealth[length=3mm]}, very thick, purple] (0, 0) -- (2, 0);
        \node[purple, above] at (1.2, 0.15) {$\vect{F}$};
    \end{scope}
\end{tikzpicture}
\end{center}

\textbf{Forces :} Poids $F_g$ (bas), Poussée d'Archimède $F_A$ (haut), Poussée des moteurs $F$ (avant)

En négligeant la résistance, seule la poussée $F$ contribue à l'accélération horizontale.

\textbf{\underline{Étapes 2-3 — Équation de Newton}}

Selon $x$ : $\sum F_x = ma_x$
\[
    F = ma \quad \Rightarrow \quad a = \frac{F}{m}
\]

\textbf{\underline{Étape 4 — Calcul}}
\[
    a = \frac{\SI{120e3}{N}}{\SI{800e3}{kg}} = \SI{0{,}15}{m/s^2} = \boxed{\SI{0{,}15}{m/s^2}}
\]

\textbf{Interprétation :} Le remorqueur gagne $\SI{0{,}15}{m/s}$ chaque seconde. Pour atteindre $\SI{6}{m/s}$ (environ 12 nœuds), il faudrait $t = \frac{6}{0{,}15} = \SI{40}{s}$.
\end{exemple}

% -----------------------------------------------------------------------------
\subsubsection{Deux forces opposées : introduction du frottement}
% -----------------------------------------------------------------------------

En réalité, la résistance de l'eau ou le frottement s'opposent au mouvement. Voyons comment les intégrer.

\begin{exemple}{Caisse poussée avec frottement}{caisse-frottement}
Une caisse de $m = \SI{50}{kg}$ est poussée sur un plancher horizontal avec une force horizontale $F = \SI{200}{N}$. Le coefficient de frottement cinétique est $\mu_c = 0{,}25$. Quelle est l'accélération de la caisse?

\tcblower

\textbf{\underline{Étape 1 — SCHÉMA et DCL}}

\begin{center}
\begin{tikzpicture}[scale=0.8]
    % Schéma de situation (gauche)
    \begin{scope}[xshift=-4cm]
        \node[above, font=\bfseries] at (0, 2.2) {Situation};
        
        % Sol
        \fill[gray!20] (-2, 0) rectangle (2.5, -0.2);
        \draw[thick] (-2, 0) -- (2.5, 0);
        \fill[pattern=north east lines, pattern color=gray] (-2, -0.2) rectangle (2.5, -0.4);
        
        % Caisse
        \draw[thick, fill=blue!20] (-0.5, 0) rectangle (0.5, 0.6);
        \node at (0, 0.3) {\small $m$};
        
        % Force appliquée
        \draw[-{Stealth}, very thick, purple] (-1.5, 0.3) -- (-0.6, 0.3);
        \node[purple, above] at (-1.05, 0.4) {$\vec{F}$};
        
        % Mouvement
        \draw[-{Stealth}, thick, gray] (0.6, 0.3) -- (1.4, 0.3);
        \node[gray, above] at (1.0, 0.35) {\small $\vec{v}$};
    \end{scope}
    
    % DCL (droite)
    \begin{scope}[xshift=2.5cm]
        \node[above, font=\bfseries] at (0, 2.2) {DCL};
        
        % Point
        \fill[blue!50] (0, 0) circle (5pt);
        
        % Axes
        \draw[-{Stealth}, thick] (0, 0) -- (3, 0) node[right] {$x$};
        \draw[-{Stealth}, thick] (0, 0) -- (0, 2) node[above] {$y$};
        
        % Forces
        \draw[-{Stealth[length=3mm]}, very thick, red] (0, 0) -- (0, -1.2);
        \node[red, right] at (0.1, -0.6) {$\vect{F}_g$};
        
        \draw[-{Stealth[length=3mm]}, very thick, green!60!black] (0, 0) -- (0, 1.2);
        \node[green!60!black, right] at (0.1, 0.6) {$\vect{N}$};
        
        \draw[-{Stealth[length=3mm]}, very thick, purple] (0, 0) -- (2, 0);
        \node[purple, above] at (1.2, 0.15) {$\vect{F}$};
        
        \draw[-{Stealth[length=3mm]}, very thick, orange] (0, 0) -- (-1.2, 0);
        \node[orange, below] at (-0.6, -0.15) {$\vect{f}_c$};
    \end{scope}
\end{tikzpicture}
\end{center}

\textbf{\underline{Étapes 2-3 — Équations de Newton}}

Selon $y$ (équilibre vertical) :
\[
    N - mg = 0 \quad \Rightarrow \quad N = mg = 50 \times 9{,}81 = \SI{490{,}5}{N}
\]

Frottement cinétique :
\[
    f_c = \mu_c N = 0{,}25 \times 490{,}5 = \SI{122{,}6}{N}
\]

Selon $x$ :
\[
    F - f_c = ma
\]

\textbf{\underline{Étape 4 — Calcul}}
\[
    a = \frac{F - f_c}{m} = \frac{200 - 122{,}6}{50} = \frac{77{,}4}{50} = \boxed{\SI{1{,}55}{m/s^2}}
\]

\textbf{Remarque :} Sans frottement, l'accélération serait $a = \frac{200}{50} = \SI{4}{m/s^2}$. Le frottement réduit l'accélération de plus de 60\%!
\end{exemple}

\begin{pratiqueautonome}[title=Chariot de manutention]
Un chariot de manutention de $m = \SI{120}{kg}$ est poussé sur le pont d'un navire avec une force horizontale de $\SI{180}{N}$. Le coefficient de frottement cinétique entre les roues et le pont est $\mu_c = 0{,}08$.

\begin{enumerate}
    \item Calculez l'accélération du chariot.
    \item Quelle force minimale faudrait-il pour déplacer le chariot à vitesse constante?
    \item Si le chariot part du repos, quelle sera sa vitesse après $\SI{5}{s}$?
\end{enumerate}

\espaceresolution[4cm]

\tcblower
\textbf{Réponses :}
\begin{enumerate}
    \item $f_c = 0{,}08 \times 120 \times 9{,}81 = \SI{94{,}2}{N}$; $a = \frac{180 - 94{,}2}{120} = \SI{0{,}72}{m/s^2}$
    \item Pour $a = 0$ : $F = f_c = \SI{94{,}2}{N}$
    \item $v = v_0 + at = 0 + 0{,}72 \times 5 = \SI{3{,}6}{m/s}$
\end{enumerate}
\end{pratiqueautonome}

% =============================================================================
\subsection{Problèmes à deux dimensions}
\label{subsec:dynamique-2d}
% =============================================================================

Quand les forces ne sont pas toutes alignées, il faut décomposer selon deux axes.

% -----------------------------------------------------------------------------
\subsubsection{Force appliquée à un angle}
% -----------------------------------------------------------------------------

\begin{exemple}{Caisse tirée par une corde à angle}{caisse-corde-angle}
Une caisse de $m = \SI{40}{kg}$ est tirée sur un plancher horizontal par une corde faisant un angle $\theta = 25°$ avec l'horizontale. La tension dans la corde est $T = \SI{150}{N}$ et le coefficient de frottement cinétique est $\mu_c = 0{,}30$. Calculez l'accélération de la caisse.

\tcblower

\textbf{\underline{Étape 1 — SCHÉMA et DCL}}

\begin{center}
\begin{tikzpicture}[scale=0.85]
    % Schéma de situation (gauche)
    \begin{scope}[xshift=-4.5cm]
        \node[above, font=\bfseries] at (0, 2.5) {Situation};
        
        % Sol
        \fill[gray!20] (-2, 0) rectangle (2.5, -0.2);
        \draw[thick] (-2, 0) -- (2.5, 0);
        
        % Caisse
        \draw[thick, fill=blue!20] (-0.5, 0) rectangle (0.5, 0.6);
        \node at (0, 0.3) {\small $m$};
        
        % Corde
        \draw[thick, brown] (0.5, 0.5) -- (2.2, 1.3);
        
        % Angle
        \draw[thick] (1.2, 0.5) arc (0:25:0.7);
        \node at (1.5, 0.7) {\small $\theta$};
        
        % Mouvement
        \draw[-{Stealth}, thick, gray] (0.7, 0.3) -- (1.5, 0.3);
        \node[gray, below] at (1.1, 0.2) {\small $\vec{v}$};
    \end{scope}
    
    % DCL (droite)
    \begin{scope}[xshift=2.5cm]
        \node[above, font=\bfseries] at (0, 2.5) {DCL};
        
        % Point
        \fill[blue!50] (0, 0) circle (5pt);
        
        % Axes
        \draw[-{Stealth}, thick] (0, 0) -- (3.5, 0) node[right] {$x$};
        \draw[-{Stealth}, thick] (0, 0) -- (0, 2.5) node[above] {$y$};
        
        % Poids
        \draw[-{Stealth[length=3mm]}, very thick, red] (0, 0) -- (0, -1.5);
        \node[red, right] at (0.1, -0.8) {$\vect{F}_g$};
        
        % Normale
        \draw[-{Stealth[length=3mm]}, very thick, green!60!black] (0, 0) -- (0, 1.3);
        \node[green!60!black, right] at (0.1, 0.7) {$\vect{N}$};
        
        % Tension (à angle)
        \draw[-{Stealth[length=3mm]}, very thick, orange] (0, 0) -- ({2*cos(25)}, {2*sin(25)});
        \node[orange, above right] at (1.4, 0.65) {$\vect{T}$};
        
        % Composantes de T (pointillés)
        \draw[dashed, orange] ({2*cos(25)}, {2*sin(25)}) -- ({2*cos(25)}, 0);
        \draw[dashed, orange] ({2*cos(25)}, {2*sin(25)}) -- (0, {2*sin(25)});
        \node[orange, below] at ({cos(25)}, -0.15) {\scriptsize $T\cos\theta$};
        \node[orange, left] at (-0.15, {sin(25)}) {\scriptsize $T\sin\theta$};
        
        % Angle
        \draw[thick] (0.6, 0) arc (0:25:0.6);
        \node at (0.85, 0.2) {\small $\theta$};
        
        % Frottement
        \draw[-{Stealth[length=3mm]}, very thick, blue] (0, 0) -- (-1.3, 0);
        \node[blue, below] at (-0.65, -0.15) {$\vect{f}_c$};
    \end{scope}
\end{tikzpicture}
\end{center}

\textbf{\underline{Étape 2 — Décomposition}}

La tension a deux composantes :
\begin{align*}
    T_x &= T\cos\theta = 150 \times \cos 25° = \SI{136{,}0}{N} \\
    T_y &= T\sin\theta = 150 \times \sin 25° = \SI{63{,}4}{N}
\end{align*}

\textbf{\underline{Étape 3 — Équations de Newton}}

Selon $y$ (pas d'accélération verticale) :
\[
    N + T\sin\theta - mg = 0
\]
\[
    N = mg - T\sin\theta = 40 \times 9{,}81 - 63{,}4 = 392{,}4 - 63{,}4 = \SI{329{,}0}{N}
\]

\textbf{Observation importante :} $N < mg$ ! La composante verticale de la tension « soulage » partiellement le poids, réduisant la normale et donc le frottement.

Frottement :
\[
    f_c = \mu_c N = 0{,}30 \times 329{,}0 = \SI{98{,}7}{N}
\]

Selon $x$ :
\[
    T\cos\theta - f_c = ma
\]

\textbf{\underline{Étape 4 — Calcul}}
\[
    a = \frac{T\cos\theta - f_c}{m} = \frac{136{,}0 - 98{,}7}{40} = \frac{37{,}3}{40} = \boxed{\SI{0{,}93}{m/s^2}}
\]

\textbf{Comparaison :} Si la corde était horizontale ($\theta = 0$), on aurait $N = mg = \SI{392{,}4}{N}$, $f_c = \SI{117{,}7}{N}$, et $a = \frac{150 - 117{,}7}{40} = \SI{0{,}81}{m/s^2}$. Tirer à angle est plus efficace!
\end{exemple}

% -----------------------------------------------------------------------------
\subsubsection{Plan incliné avec mouvement}
% -----------------------------------------------------------------------------

\begin{exemple}{Bloc lancé vers le haut d'un plan incliné}{bloc-pousse}
Un bloc de $m = \SI{10}{kg}$ est lancé vers le haut d'un plan incliné à $\theta = 35°$ avec une vitesse initiale de $\SI{4{,}0}{m/s}$. Le coefficient de frottement cinétique est $\mu_c = 0{,}20$. Calculez la décélération du bloc pendant sa montée.

\tcblower

\textbf{\underline{Étape 1 — DCL}}

\begin{center}
\begin{tikzpicture}[scale=0.9]
    % Point
    \fill[blue!50] (0, 0) circle (5pt);
    
    % Axes inclinés (x positif vers le haut de la pente = sens du mouvement)
    \draw[-{Stealth}, thick] (0, 0) -- ({2.5*cos(35)}, {2.5*sin(35)}) node[right] {$x$};
    \draw[-{Stealth}, thick] (0, 0) -- ({-1.8*sin(35)}, {1.8*cos(35)}) node[above] {$y$};
    \node[font=\scriptsize] at (2.4, 0.7) {(+ vers haut)};
    
    % Vitesse initiale (vers le haut de la pente)
    \draw[-{Stealth[length=3mm]}, thick, purple, dashed] (0, 0) -- ({1.2*cos(35)}, {1.2*sin(35)});
    \node[purple, font=\scriptsize, above] at ({1.2*cos(35)}, {1.2*sin(35)}) {$\vect{v}_0$};
    
    % Poids (vertical)
    \draw[-{Stealth[length=3mm]}, very thick, red] (0, 0) -- (0, -1.8);
    \node[red, right] at (0.1, -1.0) {$\vect{F}_g$};
    
    % Normale
    \draw[-{Stealth[length=3mm]}, very thick, green!60!black] (0, 0) -- ({-1.5*sin(35)}, {1.5*cos(35)});
    \node[green!60!black, above left] at (-0.6, 1.0) {$\vect{N}$};
    
    % Frottement (vers le BAS de la pente = oppose le mouvement vers le haut)
    \draw[-{Stealth[length=3mm]}, very thick, blue] (0, 0) -- ({-1.0*cos(35)}, {-1.0*sin(35)});
    \node[blue, below left] at (-0.6, -0.3) {$\vect{f}_c$};
    
    % Composantes du poids
    \node[red, font=\scriptsize] at (1.0, -0.8) {$mg\sin\theta$};
    \node[red, font=\scriptsize] at (-0.7, -1.4) {$mg\cos\theta$};
\end{tikzpicture}
\end{center}

\textbf{\underline{Étape 2 — Axes}}

$x$ : parallèle à la pente, positif vers le haut (sens de la vitesse initiale)

$y$ : perpendiculaire à la pente, positif vers l'extérieur

\textbf{Le frottement est vers le bas de la pente}, car il s'oppose au mouvement du bloc qui monte.

\textbf{\underline{Étape 3 — Équations de Newton}}

Selon $y$ (pas de mouvement perpendiculaire à la pente) :
\[
    N - mg\cos\theta = 0 \quad \Rightarrow \quad N = mg\cos\theta
\]

Selon $x$ (attention : les deux forces sont dans le sens $-x$) :
\[
    -mg\sin\theta - f_c = ma
\]

Avec $f_c = \mu_c N = \mu_c mg\cos\theta$ :
\[
    -mg\sin\theta - \mu_c mg\cos\theta = ma
\]
\[
    a = -g(\sin\theta + \mu_c\cos\theta)
\]

\textbf{\underline{Étape 4 — Calcul}}
\begin{align*}
    a &= -9{,}81 \times (\sin 35° + 0{,}20 \times \cos 35°) \\
    &= -9{,}81 \times (0{,}574 + 0{,}20 \times 0{,}819) \\
    &= -9{,}81 \times (0{,}574 + 0{,}164) \\
    &= -9{,}81 \times 0{,}738 = \boxed{\SI{-7{,}24}{m/s^2}}
\end{align*}

Le signe négatif confirme que l'accélération est dirigée vers le bas de la pente : le bloc décélère.

\textbf{Vérification :} Comparons avec le cas sans frottement : $a = -g\sin\theta = \SI{-5{,}63}{m/s^2}$. Le frottement \textit{augmente} la décélération de 29\%, ce qui est logique puisque frottement et gravité agissent \textbf{tous les deux} vers le bas de la pente.

\textbf{Distance parcourue avant l'arrêt :}
\[
    v^2 = v_0^2 + 2a\Delta x \quad \Rightarrow \quad \Delta x = \frac{-v_0^2}{2a} = \frac{-(4{,}0)^2}{2 \times (-7{,}24)} = \SI{1{,}10}{m}
\]
\end{exemple}

\begin{attention}[title=Sens du frottement sur un plan incliné]
Le frottement s'oppose \textbf{toujours} au mouvement relatif :
\begin{itemize}
    \item Si le bloc glisse vers le \textbf{bas} : le frottement est vers le \textbf{haut} de la pente.
    \item Si le bloc est poussé vers le \textbf{haut} : le frottement est vers le \textbf{bas} de la pente.
\end{itemize}

Attention à ne pas confondre « s'oppose au mouvement » avec « vers le bas » ou « vers le haut »!
\end{attention}

% =============================================================================
\subsection{Systèmes à plusieurs objets}
\label{subsec:systemes}
% =============================================================================

Jusqu'ici, nous avons analysé des objets isolés. Mais souvent, plusieurs objets sont reliés et interagissent. Comment traiter ces systèmes?

\textbf{Deux approches possibles :}
\begin{enumerate}
    \item \textbf{Analyse séparée :} Tracer un DCL pour chaque objet, écrire les équations de Newton pour chacun, puis résoudre le système d'équations.
    
    \item \textbf{Analyse globale :} Traiter le système comme un seul objet (utile pour trouver l'accélération rapidement, mais ne donne pas les forces internes).
\end{enumerate}

% -----------------------------------------------------------------------------
\subsubsection{Objets reliés sur surface horizontale}
% -----------------------------------------------------------------------------

\begin{exemple}{Deux blocs reliés par une corde}{deux-blocs}
Deux blocs de masses $m_1 = \SI{5}{kg}$ et $m_2 = \SI{3}{kg}$ sont reliés par une corde légère et posés sur une surface horizontale sans frottement. On tire le bloc 1 avec une force $F = \SI{24}{N}$.

a) Quelle est l'accélération du système?

b) Quelle est la tension dans la corde?

\tcblower

\begin{center}
\begin{tikzpicture}[scale=0.8]
    % Surface
    \fill[gray!20] (-1, -0.2) rectangle (7, -0.4);
    \draw[thick] (-1, -0.2) -- (7, -0.2);
    
    % Bloc 2
    \draw[thick, fill=blue!20] (0, -0.2) rectangle (1.2, 0.6);
    \node at (0.6, 0.2) {$m_2$};
    
    % Corde
    \draw[thick, brown] (1.2, 0.2) -- (2.5, 0.2);
    
    % Bloc 1
    \draw[thick, fill=blue!30] (2.5, -0.2) rectangle (4, 0.6);
    \node at (3.25, 0.2) {$m_1$};
    
    % Force F
    \draw[-{Stealth[length=3mm]}, very thick, purple] (4, 0.2) -- (5.5, 0.2);
    \node[purple, above] at (4.75, 0.3) {$\vect{F}$};
    
    % Direction du mouvement
    \draw[-{Stealth}, thick, gray] (2, -0.8) -- (4, -0.8);
    \node[gray, below] at (3, -0.9) {\small mouvement};
\end{tikzpicture}
\end{center}

\textbf{Méthode 1 : Analyse globale (pour trouver $a$)}

Considérons les deux blocs comme un seul système de masse $m_{tot} = m_1 + m_2 = \SI{8}{kg}$.

La seule force extérieure horizontale est $F$ (la tension est une force \textit{interne} au système).
\[
    F = m_{tot} \cdot a \quad \Rightarrow \quad a = \frac{F}{m_1 + m_2} = \frac{24}{8} = \boxed{\SI{3{,}0}{m/s^2}}
\]

\textbf{Méthode 2 : Analyse séparée (pour trouver $T$)}

\begin{center}
\begin{tikzpicture}[scale=0.75]
    % DCL bloc 2
    \begin{scope}[xshift=0cm]
        \fill[blue!30] (0, 0) circle (4pt);
        \draw[-{Stealth}, thick] (0, 0) -- (2, 0) node[right] {$x$};
        
        \draw[-{Stealth[length=3mm]}, very thick, orange] (0, 0) -- (1.2, 0);
        \node[orange, above] at (0.6, 0.1) {$\vect{T}$};
        
        \node[below] at (0, -0.8) {DCL de $m_2$};
    \end{scope}
    
    % DCL bloc 1
    \begin{scope}[xshift=5cm]
        \fill[blue!30] (0, 0) circle (4pt);
        \draw[-{Stealth}, thick] (0, 0) -- (2.5, 0) node[right] {$x$};
        
        \draw[-{Stealth[length=3mm]}, very thick, purple] (0, 0) -- (1.8, 0);
        \node[purple, above] at (0.9, 0.1) {$\vect{F}$};
        
        \draw[-{Stealth[length=3mm]}, very thick, orange] (0, 0) -- (-1.2, 0);
        \node[orange, above] at (-0.6, 0.1) {$\vect{T}$};
        
        \node[below] at (0, -0.8) {DCL de $m_1$};
    \end{scope}
\end{tikzpicture}
\end{center}

Pour le bloc 2 : $T = m_2 a = 3 \times 3{,}0 = \boxed{\SI{9{,}0}{N}}$

Vérification avec le bloc 1 : $F - T = m_1 a$ → $24 - 9 = 5 \times 3 = 15$ \cmark
\end{exemple}

% -----------------------------------------------------------------------------
\subsubsection{La machine d'Atwood}
% -----------------------------------------------------------------------------

\begin{exemple}{Machine d'Atwood}{atwood}
Deux masses $m_1 = \SI{5}{kg}$ et $m_2 = \SI{3}{kg}$ sont reliées par une corde légère passant sur une poulie sans frottement. On lâche le système du repos.

a) Quelle est l'accélération du système?

b) Quelle est la tension dans la corde?

\tcblower

\begin{center}
\begin{tikzpicture}[scale=0.8]
    % Support
    \fill[gray!50] (-0.3, 3) rectangle (0.3, 3.5);
    \draw[thick] (-0.5, 3) -- (0.5, 3);
    
    % Poulie
    \draw[thick, fill=gray!20] (0, 2.5) circle (0.5);
    \fill (0, 2.5) circle (2pt);
    
    % Cordes
    \draw[thick, brown] (-0.5, 2.5) -- (-0.5, 0.8);
    \draw[thick, brown] (0.5, 2.5) -- (0.5, 1.5);
    
    % Masse 1 (plus lourde, à gauche)
    \draw[thick, fill=blue!30] (-1, 0) rectangle (0, 0.8);
    \node at (-0.5, 0.4) {$m_1$};
    
    % Masse 2 (plus légère, à droite)
    \draw[thick, fill=blue!20] (0, 0.7) rectangle (1, 1.5);
    \node at (0.5, 1.1) {$m_2$};
    
    % Flèches de mouvement
    \draw[-{Stealth}, thick, red] (-0.5, -0.3) -- (-0.5, -1);
    \node[red, right] at (-0.4, -0.65) {\small $\vec{a}$};
    
    \draw[-{Stealth}, thick, red] (0.5, 1.8) -- (0.5, 2.5);
    \node[red, right] at (0.6, 2.15) {\small $\vec{a}$};
    
    % Labels
    \node[left] at (-1.2, 0.4) {\small descend};
    \node[right] at (1.2, 1.1) {\small monte};
\end{tikzpicture}
\end{center}

\textbf{Contraintes importantes :}
\begin{itemize}
    \item La corde est \textbf{inextensible} : les deux masses ont la même accélération (en module).
    \item La poulie est \textbf{sans masse et sans frottement} : la tension est la même des deux côtés.
\end{itemize}

\textbf{Convention :} Prenons positif vers le bas pour $m_1$ et positif vers le haut pour $m_2$ (les deux dans le sens du mouvement attendu).

\begin{center}
\begin{tikzpicture}[scale=0.7]
    % DCL m1
    \begin{scope}[xshift=0cm]
        \fill[blue!30] (0, 0) circle (4pt);
        \draw[-{Stealth}, thick] (0, 0) -- (0, -2) node[below] {$+$};
        
        \draw[-{Stealth[length=3mm]}, very thick, red] (0, 0) -- (0, -1.3);
        \node[red, right] at (0.1, -0.65) {$m_1 g$};
        
        \draw[-{Stealth[length=3mm]}, very thick, orange] (0, 0) -- (0, 0.9);
        \node[orange, right] at (0.1, 0.45) {$T$};
        
        \node[below] at (0, -2.5) {DCL de $m_1$};
    \end{scope}
    
    % DCL m2
    \begin{scope}[xshift=4cm]
        \fill[blue!30] (0, 0) circle (4pt);
        \draw[-{Stealth}, thick] (0, 0) -- (0, 2) node[above] {$+$};
        
        \draw[-{Stealth[length=3mm]}, very thick, red] (0, 0) -- (0, -0.9);
        \node[red, right] at (0.1, -0.45) {$m_2 g$};
        
        \draw[-{Stealth[length=3mm]}, very thick, orange] (0, 0) -- (0, 1.3);
        \node[orange, right] at (0.1, 0.65) {$T$};
        
        \node[below] at (0, -2.5) {DCL de $m_2$};
    \end{scope}
\end{tikzpicture}
\end{center}

\textbf{Équations de Newton :}

Pour $m_1$ (positif vers le bas) :
\begin{equation}
    m_1 g - T = m_1 a \tag{1}
\end{equation}

Pour $m_2$ (positif vers le haut) :
\begin{equation}
    T - m_2 g = m_2 a \tag{2}
\end{equation}

\textbf{Résolution :}

Additionnons (1) et (2) :
\[
    m_1 g - T + T - m_2 g = m_1 a + m_2 a
\]
\[
    (m_1 - m_2)g = (m_1 + m_2)a
\]
\[
    a = \frac{(m_1 - m_2)g}{m_1 + m_2} = \frac{(5 - 3) \times 9{,}81}{5 + 3} = \frac{19{,}62}{8} = \boxed{\SI{2{,}45}{m/s^2}}
\]

Pour la tension, substituons dans (2) :
\[
    T = m_2(g + a) = 3 \times (9{,}81 + 2{,}45) = 3 \times 12{,}26 = \boxed{\SI{36{,}8}{N}}
\]

\textbf{Vérification :}
\begin{itemize}
    \item Si $m_1 = m_2$ : $a = 0$ (logique, les masses s'équilibrent)
    \item Si $m_2 = 0$ : $a = g$ (chute libre de $m_1$)
    \item $T$ est entre $m_2 g = \SI{29{,}4}{N}$ et $m_1 g = \SI{49{,}1}{N}$ \cmark
\end{itemize}
\end{exemple}

% -----------------------------------------------------------------------------
\subsubsection{Poulie avec plan incliné}
% -----------------------------------------------------------------------------

\begin{exemple}{Bloc sur plan incliné relié à une masse suspendue}{poulie-plan}
Un bloc de $m_1 = \SI{8}{kg}$ est posé sur un plan incliné à $\theta = 30°$ (sans frottement). Il est relié par une corde passant sur une poulie à une masse suspendue $m_2 = \SI{5}{kg}$.

Déterminez l'accélération du système et la tension dans la corde.

\tcblower

\begin{center}
\begin{tikzpicture}[scale=0.75]
    % Plan incliné
    \fill[gray!20] (0, 0) -- (5, 0) -- (5, 2.89) -- cycle;
    \draw[thick] (0, 0) -- (5, 2.89);
    \draw[thick] (0, 0) -- (5, 0);
    
    % Angle
    \draw[thick] (1.2, 0) arc (0:30:1.2);
    \node at (1.5, 0.3) {\small $\theta$};
    
    % Bloc m1
    \begin{scope}[rotate=30, shift={(2, 0)}]
        \draw[thick, fill=blue!30] (0, 0) rectangle (1, 0.7);
        \node at (0.5, 0.35) {$m_1$};
    \end{scope}
    
    % Poulie
    \draw[thick, fill=gray!20] (5, 3.2) circle (0.3);
    \fill (5, 3.2) circle (2pt);
    
    % Cordes
    \draw[thick, brown] ({2.5*cos(30) + 0.5*cos(30)}, {2.5*sin(30) + 0.5*sin(30) + 0.35}) -- (4.7, 3.2);
    \draw[thick, brown] (5.3, 3.2) -- (5.3, 1);
    
    % Masse m2
    \draw[thick, fill=blue!20] (4.8, 0.2) rectangle (5.8, 1);
    \node at (5.3, 0.6) {$m_2$};
    
    % Flèches
    \draw[-{Stealth}, thick, purple] (2.3, 2.2) -- (3.3, 2.78);
    \node[purple, above left] at (2.8, 2.5) {\small $\vec{a}$};
    
    \draw[-{Stealth}, thick, purple] (5.3, 0) -- (5.3, -0.6);
    \node[purple, right] at (5.4, -0.3) {\small $\vec{a}$};
\end{tikzpicture}
\end{center}

\textbf{DCL des deux masses :}

\begin{center}
\begin{tikzpicture}[scale=0.8]
    % DCL m1 (sur plan incliné)
    \begin{scope}[xshift=0cm]
        \node[above, font=\bfseries] at (0, 2.5) {DCL de $m_1$};
        
        % Point
        \fill[blue!30] (0, 0) circle (5pt);
        
        % Axes inclinés (x parallèle à la pente, vers le haut)
        \draw[-{Stealth}, thick] (0, 0) -- ({2*cos(30)}, {2*sin(30)}) node[right] {$x$};
        \draw[-{Stealth}, thick] (0, 0) -- ({-1.5*sin(30)}, {1.5*cos(30)}) node[above] {$y$};
        \node[font=\scriptsize, below right] at ({1.5*cos(30)}, {1.5*sin(30)}) {(+ vers haut)};
        
        % Poids (vertical vers le bas)
        \draw[-{Stealth[length=3mm]}, very thick, red] (0, 0) -- (0, -1.8);
        \node[red, right] at (0.1, -1.0) {$m_1\vect{g}$};
        
        % Normale (perpendiculaire au plan)
        \draw[-{Stealth[length=3mm]}, very thick, green!60!black] (0, 0) -- ({-1.3*sin(30)}, {1.3*cos(30)});
        \node[green!60!black, above left] at (-0.4, 1.0) {$\vect{N}$};
        
        % Tension (vers le haut du plan)
        \draw[-{Stealth[length=3mm]}, very thick, orange] (0, 0) -- ({1.2*cos(30)}, {1.2*sin(30)});
        \node[orange, above right] at (0.7, 0.6) {$\vect{T}$};
        
        % Composantes du poids en pointillés
        \draw[dashed, red!50] (0, -1.8) -- ({1.8*sin(30)*cos(30)}, {-1.8 + 1.8*sin(30)*sin(30)});
        \draw[dashed, red!50] (0, -1.8) -- ({-1.8*cos(30)*sin(30)}, {-1.8*cos(30)*cos(30)});
    \end{scope}
    
    % DCL m2 (suspendue)
    \begin{scope}[xshift=6cm]
        \node[above, font=\bfseries] at (0, 2.5) {DCL de $m_2$};
        
        % Point
        \fill[blue!20] (0, 0) circle (5pt);
        
        % Axe vertical (+ vers le bas)
        \draw[-{Stealth}, thick] (0, 0) -- (0, -2) node[below] {$+$};
        \node[font=\scriptsize, right] at (0.2, -1.5) {(+ vers bas)};
        
        % Poids
        \draw[-{Stealth[length=3mm]}, very thick, red] (0, 0) -- (0, -1.5);
        \node[red, right] at (0.1, -0.8) {$m_2\vect{g}$};
        
        % Tension (vers le haut)
        \draw[-{Stealth[length=3mm]}, very thick, orange] (0, 0) -- (0, 1.2);
        \node[orange, right] at (0.1, 0.6) {$\vect{T}$};
    \end{scope}
\end{tikzpicture}
\end{center}

\textbf{Question préliminaire :} Dans quel sens le système va-t-il bouger?

Comparons les « forces motrices » :
\begin{itemize}
    \item Composante du poids de $m_1$ le long de la pente : $m_1 g \sin\theta = 8 \times 9{,}81 \times 0{,}5 = \SI{39{,}2}{N}$
    \item Poids de $m_2$ : $m_2 g = 5 \times 9{,}81 = \SI{49{,}1}{N}$
\end{itemize}

Comme $\SI{49{,}1}{N} > \SI{39{,}2}{N}$, la masse $m_2$ descend et $m_1$ monte le plan.

\textbf{Équations de Newton :}

Pour $m_1$ (positif vers le haut du plan) :
\[
    T - m_1 g \sin\theta = m_1 a \tag{1}
\]

Pour $m_2$ (positif vers le bas) :
\[
    m_2 g - T = m_2 a \tag{2}
\]

\textbf{Résolution :}

Additionnons (1) et (2) :
\[
    m_2 g - m_1 g \sin\theta = (m_1 + m_2)a
\]
\[
    a = \frac{m_2 g - m_1 g \sin\theta}{m_1 + m_2} = \frac{g(m_2 - m_1 \sin\theta)}{m_1 + m_2}
\]
\[
    a = \frac{9{,}81 \times (5 - 8 \times 0{,}5)}{8 + 5} = \frac{9{,}81 \times 1}{13} = \boxed{\SI{0{,}75}{m/s^2}}
\]

Tension :
\[
    T = m_2(g - a) = 5 \times (9{,}81 - 0{,}75) = 5 \times 9{,}06 = \boxed{\SI{45{,}3}{N}}
\]
\end{exemple}

\begin{pratiqueautonome}[title=Système poulie avec frottement]
Reprenons l'exemple précédent, mais avec un coefficient de frottement cinétique $\mu_c = 0{,}15$ entre $m_1$ et le plan.

\begin{enumerate}
    \item Le système bouge-t-il toujours dans le même sens?
    \item Calculez la nouvelle accélération.
    \item Calculez la nouvelle tension.
\end{enumerate}

\textit{Indice :} Le frottement s'oppose au mouvement de $m_1$ (vers le haut), donc il est dirigé vers le bas de la pente.

\espaceresolution[5cm]

\tcblower
\textbf{Réponses :}
\begin{enumerate}
    \item Oui, $m_2$ descend toujours (mais il faut vérifier).
    \item $f_c = \mu_c N = \mu_c m_1 g \cos\theta = 0{,}15 \times 8 \times 9{,}81 \times 0{,}866 = \SI{10{,}2}{N}$
    
    Nouvelle équation pour $m_1$ : $T - m_1 g\sin\theta - f_c = m_1 a$
    
    $a = \frac{m_2 g - m_1 g\sin\theta - f_c}{m_1 + m_2} = \frac{49{,}1 - 39{,}2 - 10{,}2}{13} = \frac{-0{,}3}{13} \approx \SI{0}{m/s^2}$
    
    Le système est presque en équilibre! (En fait, avec ces valeurs, il ne bouge pas car le frottement statique suffit.)
    
    \item Si $a \approx 0$ : $T \approx m_2 g = \SI{49{,}1}{N}$
\end{enumerate}
\end{pratiqueautonome}