% =============================================================================
% SECTION 1 - LES FONDEMENTS
% =============================================================================

\section{Les fondements de la dynamique}
\label{sec:fondements}

% -----------------------------------------------------------------------------
\subsection{Introduction : pourquoi étudier la dynamique?}
\label{subsec:intro}
% -----------------------------------------------------------------------------

\subsubsection*{La dynamique au cœur du métier d'officier de marine}

En tant que futur officier de navigation, vous serez responsable de navires dont la masse peut atteindre des centaines de milliers de tonnes. Contrairement à une automobile qui peut s'arrêter en quelques mètres, un pétrolier lancé à pleine vitesse peut nécessiter plusieurs kilomètres pour s'immobiliser. Pourquoi? La réponse se trouve dans ce chapitre.

La dynamique — l'étude des causes du mouvement — vous permettra de comprendre :
\begin{itemize}
    \item Pourquoi un navire met si longtemps à s'arrêter ou à changer de cap
    \item Comment calculer les forces dans les amarres qui retiennent votre navire au quai
    \item Pourquoi la cargaison non arrimée se déplace dangereusement lors des manœuvres
    \item Les forces en jeu lorsque votre navire affronte une tempête
    \item La physique derrière les vents et les courants que vous utiliserez pour naviguer
\end{itemize}

Mais avant d'aborder les lois qui gouvernent le mouvement, il est essentiel de comprendre pourquoi notre intuition nous trompe si souvent en physique.

\subsubsection*{D'Aristote à Newton : une révolution de la pensée}

Pendant près de deux mille ans, la physique occidentale a été dominée par les idées d'Aristote (384--322 av. J.-C.). Sa vision du mouvement, bien que fausse, correspond remarquablement bien à notre intuition quotidienne. C'est précisément pour cette raison qu'elle est si difficile à abandonner.

\begin{remarque}[title=La physique d'Aristote]
Selon Aristote, le mouvement « naturel » des objets terrestres est de tomber vers le centre de la Terre (leur « lieu naturel »), puis de s'y immobiliser. Tout autre mouvement est « violent » et nécessite une cause permanente : \textbf{pour qu'un objet continue de bouger, il faut continuellement exercer une force sur lui}.

Cette vision semble logique! Quand vous poussez une boîte sur le plancher et que vous arrêtez de pousser, la boîte s'arrête. Quand vous pédalez sur un vélo et que vous arrêtez, le vélo ralentit puis s'immobilise. Il semble bien qu'il faille une force pour maintenir le mouvement.
\end{remarque}

Le problème avec la physique aristotélicienne, c'est qu'elle confond deux choses : le mouvement lui-même et les forces de friction qui s'opposent au mouvement. Aristote ne pouvait pas facilement éliminer la friction de ses expériences, alors il l'a intégrée inconsciemment dans sa théorie.

C'est Galilée (1564--1642) qui a eu l'intuition géniale de se demander : « Que se passerait-il si on pouvait éliminer complètement la friction? » En étudiant des billes roulant sur des plans inclinés de plus en plus lisses, il a conclu qu'un objet en mouvement sur une surface parfaitement horizontale et sans friction continuerait indéfiniment à la même vitesse.

\begin{definition}[title=Le mouvement naturel selon Galilée et Newton]
Contrairement à ce qu'Aristote pensait, le mouvement « naturel » d'un objet libre de toute force n'est pas le repos, mais le \textbf{mouvement rectiligne uniforme} — c'est-à-dire en ligne droite et à vitesse constante. Un objet au repos est simplement un cas particulier où cette vitesse constante est nulle.
\end{definition}

Cette idée révolutionnaire est au cœur de la mécanique newtonienne que nous allons étudier.

% -----------------------------------------------------------------------------
\subsection{La masse et l'inertie}
\label{subsec:masse_inertie}
% -----------------------------------------------------------------------------

Vous avez certainement remarqué qu'il est plus difficile de pousser une voiture qu'une bicyclette, même sur une surface parfaitement lisse. Cette résistance au changement de vitesse porte un nom : l'\textbf{inertie}.

\begin{definition}[title=L'inertie]
L'\textbf{inertie} est la tendance naturelle de tout objet matériel à résister aux changements de son état de mouvement. Un objet au repos tend à rester au repos. Un objet en mouvement tend à continuer en ligne droite à la même vitesse.

L'inertie n'est pas une force — c'est une \textbf{propriété fondamentale de la matière}.
\end{definition}

\begin{attention}[title=L'inertie n'est PAS une force!]
C'est une erreur très répandue de parler de « force d'inertie ». Quand vous êtes dans un autobus qui freine brusquement et que vous vous sentez « projeté » vers l'avant, \textbf{aucune force ne vous pousse}. 

Ce qui se passe réellement : l'autobus ralentit (une force de freinage agit sur lui), mais votre corps, lui, veut continuer à la même vitesse qu'avant — c'est son inertie. Vous n'êtes pas poussé vers l'avant; c'est l'autobus qui ralentit pendant que vous continuez tout droit. La « force » que vous ressentez n'est qu'une illusion due au fait que vous observez le monde depuis un référentiel (l'autobus) qui est en train d'accélérer.
\end{attention}

% -----------------------------------------------------------------------------
\subsection{La masse et l'inertie}
\label{subsec:masse_inertie}
% -----------------------------------------------------------------------------

Vous avez certainement remarqué qu'il est plus difficile de pousser une voiture qu'une bicyclette, même sur une surface parfaitement lisse. Cette résistance au changement de vitesse porte un nom : l'\textbf{inertie}.

\begin{definition}[title=L'inertie]
L'\textbf{inertie} est la tendance naturelle de tout objet matériel à résister aux changements de son état de mouvement. Un objet au repos tend à rester au repos. Un objet en mouvement tend à continuer en ligne droite à la même vitesse.

L'inertie n'est pas une force — c'est une \textbf{propriété fondamentale de la matière}.
\end{definition}

\begin{attention}[title=L'inertie n'est PAS une force!]
C'est une erreur très répandue de parler de « force d'inertie ». Quand vous êtes dans un autobus qui freine brusquement et que vous vous sentez « projeté » vers l'avant, \textbf{aucune force ne vous pousse}. 

Ce qui se passe réellement : l'autobus ralentit (une force de freinage agit sur lui), mais votre corps, lui, veut continuer à la même vitesse qu'avant — c'est son inertie. Vous n'êtes pas poussé vers l'avant; c'est l'autobus qui ralentit pendant que vous continuez tout droit. La « force » que vous ressentez n'est qu'une illusion due au fait que vous observez le monde depuis un référentiel (l'autobus) qui est en train d'accélérer.
\end{attention}

\subsubsection*{La masse : mesure de l'inertie}

Comment quantifier l'inertie d'un objet? C'est le rôle de la \textbf{masse}.

\begin{definition}[title=La masse]
La \textbf{masse} $m$ d'un objet est la mesure quantitative de son inertie — c'est-à-dire de sa résistance aux changements de vitesse. Plus un objet est massif, plus il est difficile de modifier sa vitesse.

L'unité de masse dans le système international est le \textbf{kilogramme} (kg).
\end{definition}

La masse est une propriété \textbf{intrinsèque} d'un objet : elle ne dépend pas de l'endroit où l'objet se trouve. Votre masse est la même sur Terre, sur la Lune, ou dans l'espace. En revanche, votre \textit{poids} (que nous définirons plus loin) varie selon l'intensité de la gravité locale.

% -----------------------------------------------------------------------------
\subsection{Les trois lois de Newton}
\label{subsec:lois_newton}
% -----------------------------------------------------------------------------

En 1687, Isaac Newton publia son œuvre majeure, les \textit{Philosophiae Naturalis Principia Mathematica} (Principes mathématiques de la philosophie naturelle). Dans cet ouvrage, il énonça trois lois qui décrivent avec une précision remarquable le mouvement de tous les objets, de la pomme qui tombe à la Lune qui orbite autour de la Terre.

\begin{remarque}[title=Des lois empiriques]
Les trois lois de Newton ne sont pas des vérités mathématiques qu'on peut démontrer. Ce sont des \textbf{lois empiriques}, c'est-à-dire des énoncés basés sur l'observation minutieuse de la nature et vérifiés expérimentalement d'innombrables fois.

Depuis plus de 300 ans, ces lois ont été testées dans des contextes extraordinairement variés : chute des corps, mouvement des planètes, collision de particules, fonctionnement des moteurs, trajectoire des projectiles... Elles fonctionnent remarquablement bien pour décrire tous les phénomènes de la vie quotidienne et de la navigation maritime.

Ce n'est qu'aux échelles extrêmes (très grandes vitesses proches de celle de la lumière, ou très petites dimensions subatomiques) que des corrections deviennent nécessaires. Pour tout ce qui concerne la navigation, les lois de Newton sont parfaitement adéquates.
\end{remarque}

% --- PREMIÈRE LOI ---

\begin{loinewton}[title=Première loi de Newton -- Principe d'inertie]
\textit{Tout objet persévère dans son état de repos ou de mouvement rectiligne uniforme, à moins qu'une force résultante non nulle ne le contraigne à changer d'état.}
\end{loinewton}

Cette loi nous dit que le mouvement « par défaut » d'un objet — en l'absence de toute force — est le mouvement rectiligne uniforme (MRU). Le repos n'est qu'un cas particulier du MRU, celui où la vitesse est nulle.

\subsubsection*{Pourquoi cette loi n'est-elle pas évidente?}

Dans notre expérience quotidienne, les objets en mouvement finissent toujours par s'arrêter. Une balle qui roule sur le sol ralentit puis s'immobilise. Un navire dont on coupe les moteurs finit par s'arrêter. Cela semble contredire la première loi!

La clé, c'est que ces objets ne sont \textbf{jamais} libres de toute force. La friction de l'air, la résistance de l'eau, le frottement avec le sol — ces forces s'opposent constamment au mouvement. C'est pourquoi les objets ralentissent. Mais si on pouvait éliminer toutes ces forces, l'objet continuerait indéfiniment à la même vitesse.

\begin{exemple}[title=L'outil qui tombe du mât]
Imaginez un navire qui avance à 10 nœuds (environ 5~m/s). Un matelot au sommet du mât, à 30 mètres de hauteur, échappe accidentellement une clé à molette. Où va-t-elle tomber?

Notre intuition aristotélicienne pourrait nous faire croire que la clé tombera à l'arrière du navire, puisque le navire « avance » pendant que la clé tombe. Mais ce n'est pas ce qui se passe!

La clé tombe \textbf{au pied du mât}, exactement comme si le navire était immobile.

Pourquoi? Avant d'être lâchée, la clé se déplaçait déjà avec le navire à 10 nœuds. Quand le matelot la lâche, aucune force horizontale n'agit sur elle (négligeons la résistance de l'air). Selon la première loi de Newton, elle conserve donc sa vitesse horizontale de 10 nœuds pendant toute sa chute. Le navire avance à 10 nœuds, la clé aussi — ils restent synchronisés.

C'est Galilée qui a compris ce phénomène, réfutant ainsi un argument courant contre le mouvement de la Terre.
\end{exemple}

\begin{exemple}[title=Le canon vers l'est et vers l'ouest]
Un argument historique contre la rotation de la Terre allait comme suit : si la Terre tourne vers l'est, alors un boulet de canon tiré vers l'ouest devrait aller plus loin (la cible « s'enfuit »), tandis qu'un boulet tiré vers l'est devrait aller moins loin (la cible « vient vers nous »).

Or, l'expérience montre que les deux boulets parcourent la même distance!

L'explication est similaire : le canon, le boulet et la cible partagent tous la même vitesse due à la rotation de la Terre. Le boulet conserve cette vitesse (première loi) en plus de la vitesse que le canon lui communique. Tout le système est cohérent.
\end{exemple}

\subsubsection*{Vers la notion de référentiel inertiel}

La première loi de Newton fait plus que décrire le comportement des objets : elle définit implicitement ce qu'est un « bon » référentiel pour faire de la physique.

\begin{definition}[title=Référentiel inertiel]
Un \textbf{référentiel inertiel} est un référentiel dans lequel la première loi de Newton est valide — c'est-à-dire un référentiel où tout objet libre de forces se déplace en mouvement rectiligne uniforme (ou reste au repos).
\end{definition}

Un référentiel attaché à la Terre est approximativement inertiel pour la plupart des applications pratiques. Un référentiel attaché à un manège en rotation ne l'est pas : dans ce référentiel, les objets semblent dévier de leur trajectoire rectiligne sans qu'aucune force « réelle » ne les y contraigne.

% --- DEUXIÈME LOI ---

\begin{loinewton}[title=Deuxième loi de Newton -- Principe fondamental de la dynamique]
\textit{L'accélération d'un objet est directement proportionnelle à la force résultante qui agit sur lui et inversement proportionnelle à sa masse.}
\begin{equation}
    \boxed{\sum \vect{F} = m\vect{a}}
    \label{eq:newton2}
\end{equation}
\end{loinewton}

Nous savons maintenant que la tendance naturelle d'un objet est de conserver sa vitesse (première loi). La question suivante est donc : \textbf{que se passe-t-il quand une force agit sur un objet?}

La réponse est simple et puissante : une force change la vitesse de l'objet. Elle lui communique une \textbf{accélération}.

Cette équation est \textbf{vectorielle}. Cela signifie que :
\begin{itemize}
    \item L'accélération $\vect{a}$ a la même direction que la force résultante $\sum\vect{F}$.
    \item Si vous poussez un objet vers le nord, il accélère vers le nord.
    \item Si plusieurs forces agissent simultanément, c'est leur \textit{somme vectorielle} qui détermine l'accélération.
\end{itemize}

En pratique, on décompose souvent cette équation selon les axes $x$ et $y$ :
\begin{align}
    \sum F_x &= ma_x \\
    \sum F_y &= ma_y
\end{align}

% --- TROISIÈME LOI ---

\begin{loinewton}[title=Troisième loi de Newton -- Principe d'action-réaction]
\textit{Lorsqu'un objet A exerce une force sur un objet B, l'objet B exerce simultanément sur l'objet A une force de même module et de direction opposée.}
\begin{equation}
    \boxed{\vect{F}_{A \to B} = -\vect{F}_{B \to A}}
    \label{eq:newton3}
\end{equation}
\end{loinewton}

La troisième loi de Newton est peut-être la moins bien comprise des trois, car elle défie souvent notre intuition.

\begin{remarque}[title=Seul dans un univers vide]
Imaginez que vous flottez seul dans un univers complètement vide. Pas de sol sous vos pieds. Pas d'air autour de vous. Pas d'étoiles, pas de planètes, rien. Juste vous, immobile dans le vide infini.

Comment pourriez-vous vous mettre en mouvement?

La réponse est troublante : \textbf{vous ne pouvez pas}. Il vous est absolument impossible de changer votre vitesse. Vous n'avez rien sur quoi pousser, rien à lancer, rien avec quoi interagir.

Cette expérience de pensée illustre un fait fondamental : \textbf{pour exercer une force, il faut deux objets}. Une force est toujours le résultat d'une interaction entre deux corps.
\end{remarque}

Et voici le point crucial : si deux objets interagissent, cette interaction est \textbf{nécessairement réciproque}. Vous ne pouvez pas pousser sur quelque chose sans que ce quelque chose vous pousse en retour.

Caractéristiques essentielles d'une paire action-réaction :
\begin{itemize}
    \item Les deux forces ont \textbf{exactement le même module}.
    \item Les deux forces ont des \textbf{directions exactement opposées}.
    \item Les deux forces agissent sur des \textbf{objets différents}.
    \item Les deux forces sont de \textbf{même nature} (toutes deux gravitationnelles, toutes deux de contact, etc.).
    \item Les deux forces existent \textbf{simultanément} — l'une n'existe pas sans l'autre.
\end{itemize}

\subsubsection*{Pourquoi notre intuition nous trompe}

\begin{exemple}[title=Le camion et la voiture]
Un camion de 10 tonnes percute une voiture de 1 tonne. La voiture est complètement détruite, tandis que le camion n'a qu'une bosse sur le pare-chocs. Il semble évident que le camion a frappé la voiture plus fort que la voiture n'a frappé le camion, non?

\textbf{Faux!} Les deux forces sont exactement égales en module : $F_{\text{camion} \to \text{voiture}} = F_{\text{voiture} \to \text{camion}}$.

Alors pourquoi la voiture est-elle plus endommagée? Parce que les \textit{accélérations} sont différentes! Selon la deuxième loi : $a = F/m$. Avec la même force $F$, la voiture (masse plus petite) subit une accélération 10 fois plus grande que le camion. C'est cette accélération violente qui cause les dégâts, pas une différence de force.
\end{exemple}

\begin{exemple}[title=Comment un navire avance-t-il?]
Les hélices d'un navire poussent l'eau vers l'arrière. Par la troisième loi, l'eau pousse les hélices (et donc le navire) vers l'avant. C'est cette réaction de l'eau qui propulse le navire!

La force exercée par les hélices sur l'eau est exactement égale à la force exercée par l'eau sur le navire. Mais l'eau, ayant accès à tout l'océan, ne semble pas accélérer de façon perceptible, tandis que le navire, lui, avance.
\end{exemple}

\begin{attention}[title=Le poids et la normale ne forment PAS une paire action-réaction]
Quand un livre est posé sur une table, deux forces agissent sur le livre : son poids (vers le bas) et la normale de la table (vers le haut). Ces deux forces sont égales en module et opposées en direction, donc le livre est en équilibre.

Mais attention! Ces deux forces ne forment \textbf{pas} une paire action-réaction. Elles agissent toutes les deux sur le \textit{même objet} (le livre), ce qui viole une des caractéristiques des paires action-réaction.

Les vraies paires action-réaction sont :
\begin{itemize}
    \item Le poids du livre (Terre $\to$ livre) et l'attraction du livre sur la Terre (livre $\to$ Terre)
    \item La normale de la table sur le livre (table $\to$ livre) et la force du livre sur la table (livre $\to$ table)
\end{itemize}
\end{attention}

% -----------------------------------------------------------------------------
\subsection{L'équilibre : un cas particulier de la première loi}
\label{subsec:equilibre}
% -----------------------------------------------------------------------------

La première loi de Newton nous dit qu'un objet reste au repos (ou en MRU) si aucune force résultante n'agit sur lui. Mais que se passe-t-il si \textit{plusieurs} forces agissent sur un objet, mais que leur somme vectorielle est nulle? L'objet reste également en équilibre!

\begin{definition}[title=Équilibre de translation]
Un objet est en \textbf{équilibre de translation} si et seulement si la somme vectorielle de toutes les forces qui agissent sur lui est nulle :
\begin{equation}
    \boxed{\sum \vect{F} = \vect{0}}
    \label{eq:equilibre}
\end{equation}
En composantes :
\begin{align}
    \sum F_x &= 0 \\
    \sum F_y &= 0
\end{align}
\end{definition}

Un objet en équilibre peut être :
\begin{itemize}
    \item \textbf{Au repos} — et il restera au repos (équilibre statique)
    \item \textbf{En mouvement rectiligne uniforme} — et il continuera ainsi (équilibre dynamique)
\end{itemize}

\begin{remarque}
L'équilibre ne signifie pas « immobile »! Un navire qui avance en ligne droite à vitesse constante est en équilibre : la poussée des moteurs est exactement compensée par la résistance de l'eau. Un parachutiste en chute à vitesse terminale est aussi en équilibre : son poids est compensé par la résistance de l'air.
\end{remarque}

L'étude des systèmes en équilibre statique (au repos) s'appelle la \textbf{statique}. C'est un cas particulier de la dynamique où l'accélération est nulle. Les conditions d'équilibre ($\sum F_x = 0$ et $\sum F_y = 0$) nous permettent de calculer des forces inconnues dans des systèmes comme les câbles de suspension, les amarres de navire, ou les structures en général.

% -----------------------------------------------------------------------------
% PRATIQUE AUTONOME - Section 1
% -----------------------------------------------------------------------------

\begin{pratiqueautonome}[title=Pratique autonome 2.1 — Les lois de Newton]
\textbf{Vrai ou Faux.} Justifiez brièvement vos réponses.

\begin{enumerate}
    \item Un objet au repos est nécessairement en équilibre.
    \item Un objet en mouvement est nécessairement soumis à une force résultante non nulle.
    \item L'inertie est une force qui s'oppose au mouvement.
    \item Dans une collision entre un camion et une voiture, le camion exerce une plus grande force sur la voiture que l'inverse.
    \item Le poids d'un objet et la normale exercée par une table sur cet objet forment une paire action-réaction.
    \item Un passager « projeté vers l'avant » dans un autobus qui freine est poussé par une force.
\end{enumerate}

\tcblower
\textit{Réponses :} 1) V \quad 2) F \quad 3) F \quad 4) F \quad 5) F \quad 6) F
\end{pratiqueautonome}
