% =============================================================================
% SECTION 1 - LES FONDEMENTS DE LA DYNAMIQUE
% Blocs 1 et 2 (4 heures - Semaine 5)
% =============================================================================

\section{Les fondements de la dynamique}
\label{sec:fondements}

% =============================================================================
\subsection{Introduction : pourquoi étudier la dynamique?}
\label{subsec:intro-dynamique}
% =============================================================================

Au chapitre précédent, nous avons appris à \textbf{décrire} le mouvement : position, vitesse, accélération. Mais nous n'avons jamais posé la question fondamentale : \textit{pourquoi} les objets bougent-ils? Qu'est-ce qui \textit{cause} le mouvement?

C'est exactement ce que la \textbf{dynamique} cherche à expliquer.

\begin{definition}[title=Dynamique]
La \textbf{dynamique} est la branche de la mécanique qui étudie les \textbf{causes} du mouvement. Elle répond aux questions :
\begin{itemize}
    \item Pourquoi un objet accélère-t-il?
    \item Quelle force faut-il pour produire un mouvement donné?
    \item Pourquoi certains objets restent-ils immobiles?
\end{itemize}
\end{definition}

\subsubsection*{La dynamique au cœur du métier d'officier de marine}

En tant que futur officier de navigation, vous serez responsable de navires dont la masse peut atteindre des centaines de milliers de tonnes. Contrairement à une automobile qui peut s'arrêter en quelques mètres, un pétrolier lancé à pleine vitesse peut nécessiter \textbf{plusieurs kilomètres} pour s'immobiliser.

Pourquoi une telle différence? La réponse se trouve dans ce chapitre.

\begin{remarque}[title=Questions auxquelles vous saurez répondre]
À la fin de ce chapitre, vous pourrez expliquer :
\begin{itemize}
    \item Pourquoi un navire met si longtemps à s'arrêter ou à changer de cap
    \item Comment calculer les forces dans les amarres qui retiennent votre navire au quai
    \item Pourquoi la cargaison mal arrimée se déplace dangereusement lors des manœuvres
    \item Les forces en jeu lorsque votre navire affronte une tempête
    \item Pourquoi un virage serré peut faire chavirer un navire
\end{itemize}
\end{remarque}

% =============================================================================
\subsection{D'Aristote à Newton : une révolution de la pensée}
\label{subsec:aristote-newton}
% =============================================================================

Avant d'énoncer les lois du mouvement, il est essentiel de comprendre pourquoi notre intuition nous trompe si souvent en physique. Pendant près de deux mille ans, la physique occidentale a été dominée par les idées d'Aristote --- et ces idées correspondent remarquablement bien à ce que nous observons au quotidien.

\subsubsection*{La physique d'Aristote (384--322 av. J.-C.)}

\begin{remarque}[title=La vision aristotélicienne du mouvement]
Selon Aristote, le mouvement « naturel » des objets terrestres est de tomber vers le centre de la Terre (leur « lieu naturel »), puis de s'y immobiliser. Tout autre mouvement est « violent » et nécessite une cause permanente.

\textbf{Conclusion d'Aristote :} Pour qu'un objet continue de bouger, il faut continuellement exercer une force sur lui.
\end{remarque}

Cette vision semble parfaitement logique! Observez autour de vous :
\begin{itemize}
    \item Quand vous poussez une boîte et que vous arrêtez de pousser, la boîte s'arrête.
    \item Quand vous pédalez sur un vélo et que vous arrêtez, le vélo ralentit puis s'immobilise.
    \item Un navire dont on coupe les moteurs finit par s'arrêter.
\end{itemize}

Il semble bien qu'il faille une force pour \textit{maintenir} le mouvement, n'est-ce pas?

\subsubsection*{Le problème caché : le frottement}

Le problème avec la physique aristotélicienne, c'est qu'elle \textbf{confond deux choses} : le mouvement lui-même et les forces de friction qui s'opposent au mouvement. Aristote ne pouvait pas facilement éliminer la friction de ses expériences, alors il l'a intégrée inconsciemment dans sa théorie.

\begin{center}
\begin{tikzpicture}[scale=1.0]
    % Titre
    \node[font=\bfseries] at (0, 3.2) {Ce qu'Aristote voyait};
    \node[font=\bfseries] at (7, 3.2) {Ce qui se passe réellement};
    
    % Gauche : vision d'Aristote
    \begin{scope}[xshift=0cm]
        % Sol
        \fill[gray!20] (-2, 0) rectangle (2, -0.3);
        \draw[thick] (-2, 0) -- (2, 0);
        
        % Boîte
        \draw[thick, fill=blue!20] (-0.5, 0) rectangle (0.5, 0.8);
        \node at (0, 0.4) {$m$};
        
        % Force de poussée
        \draw[-{Stealth[length=3mm]}, very thick, red] (-1.5, 0.4) -- (-0.6, 0.4);
        \node[red, above] at (-1.05, 0.5) {$\vec{F}$};
        
        % Mouvement
        \draw[-{Stealth[length=2mm]}, thick, purple] (0.6, 0.4) -- (1.4, 0.4);
        \node[purple, above] at (1.0, 0.5) {$\vec{v}$};
        
        % Texte
        \node[text width=4cm, align=center] at (0, -1.2) {\small « La force cause le mouvement »};
    \end{scope}
    
    % Droite : réalité
    \begin{scope}[xshift=7cm]
        % Sol rugueux
        \fill[gray!20] (-2, 0) rectangle (2, -0.3);
        \draw[thick] (-2, 0) -- (2, 0);
        \foreach \x in {-1.8, -1.4, -1.0, -0.6, -0.2, 0.2, 0.6, 1.0, 1.4, 1.8} {
            \draw[gray] (\x, 0) -- (\x+0.15, -0.15);
        }
        
        % Boîte
        \draw[thick, fill=blue!20] (-0.5, 0) rectangle (0.5, 0.8);
        \node at (0, 0.4) {$m$};
        
        % Force de poussée
        \draw[-{Stealth[length=3mm]}, very thick, red] (-1.5, 0.4) -- (-0.6, 0.4);
        \node[red, above] at (-1.05, 0.5) {$\vec{F}$};
        
        % Frottement (opposé au mouvement)
        \draw[-{Stealth[length=3mm]}, very thick, orange] (0.5, 0.15) -- (-0.3, 0.15);
        \node[orange, below] at (0.1, 0.1) {$\vec{f}$};
        
        % Mouvement
        \draw[-{Stealth[length=2mm]}, thick, purple] (0.6, 0.6) -- (1.4, 0.6);
        \node[purple, above] at (1.0, 0.7) {$\vec{v}$};
        
        % Texte
        \node[text width=4cm, align=center] at (0, -1.2) {\small La force $\vec{F}$ compense le frottement $\vec{f}$};
    \end{scope}
\end{tikzpicture}
\end{center}

\subsubsection*{Galilée et l'expérience de pensée}

Au début du XVIIe siècle, Galilée (1564--1642) a eu une idée révolutionnaire : \textit{et si on pouvait éliminer le frottement?}

Il a imaginé une bille roulant sur un plan incliné parfaitement lisse. Si on lance la bille vers le bas d'une pente, elle accélère. Si on la lance vers le haut, elle décélère. Mais si le plan est parfaitement horizontal et sans friction... la bille devrait continuer indéfiniment à la même vitesse!

\begin{center}
\begin{tikzpicture}[scale=0.9]
    % Plan incliné descendant
    \begin{scope}[xshift=-4cm]
        \draw[thick, fill=gray!10] (0, 2) -- (3, 0) -- (3, -0.2) -- (0, 1.8) -- cycle;
        \shade[ball color=blue!50] (1.0, 1.55) circle (0.2);
        \draw[-{Stealth}, thick, purple] (1.2, 1.45) -- (2.0, 0.92);
        \node[below, text width=2.5cm, align=center] at (1.5, -0.7) {\small Descente :\\ accélère};
    \end{scope}
    
    % Plan horizontal
    \begin{scope}[xshift=0cm]
        \draw[thick, fill=gray!10] (0, 0) rectangle (3, -0.2);
        \shade[ball color=blue!50] (0.8, 0.4) circle (0.2);
        \draw[-{Stealth}, thick, purple] (1.0, 0.4) -- (2.2, 0.4);
        \node[below, text width=2.5cm, align=center] at (1.5, -0.7) {\small Horizontal :\\ $v$ constante!};
    \end{scope}
    
    % Plan incliné montant
    \begin{scope}[xshift=4cm]
        \draw[thick, fill=gray!10] (0, 0) -- (3, 2) -- (3, 1.8) -- (0, -0.2) -- cycle;
        \shade[ball color=blue!50] (1.0, 0.87) circle (0.2);
        \draw[-{Stealth}, thick, purple] (1.2, 1.0) -- (2.0, 1.53);
        \node[below, text width=2.5cm, align=center] at (1.5, -0.7) {\small Montée :\\ décélère};
    \end{scope}
\end{tikzpicture}
\end{center}

\begin{attention}[title=La révolution galiléenne]
Galilée a compris que \textbf{le mouvement à vitesse constante ne nécessite aucune force}. Une force est nécessaire uniquement pour \textbf{changer} la vitesse (accélérer ou décélérer), pas pour la maintenir.

Quand vous poussez une boîte et qu'elle s'arrête après que vous cessez de pousser, ce n'est pas parce que « le mouvement a besoin d'une force » --- c'est parce que le \textbf{frottement} exerce une force qui décélère la boîte!
\end{attention}

% =============================================================================
\subsection{L'inertie}
\label{subsec:inertie}
% =============================================================================

La découverte de Galilée a mené à un concept fondamental : l'\textbf{inertie}.

\begin{definition}[title=Inertie]
L'\textbf{inertie} est la tendance naturelle d'un objet à \textbf{résister aux changements} de son état de mouvement.

\begin{itemize}
    \item Un objet au repos tend à rester au repos.
    \item Un objet en mouvement tend à continuer en ligne droite à vitesse constante.
\end{itemize}
\end{definition}

L'inertie explique de nombreux phénomènes quotidiens :

\begin{itemize}
    \item Quand un autobus freine brusquement, les passagers sont « projetés » vers l'avant --- en réalité, ils continuent simplement tout droit pendant que l'autobus décélère sous eux.
    
    \item Quand un navire vire à tribord, la cargaison non arrimée glisse vers bâbord --- elle tend à continuer en ligne droite.
    
    \item Un pétrolier de 300\,000 tonnes met plusieurs kilomètres à s'arrêter --- son énorme inertie résiste au changement de vitesse.
\end{itemize}

\begin{attention}[title=L'inertie n'est PAS une force!]
Erreur très fréquente : considérer l'inertie comme une « force » qui pousse les objets. 

\textbf{L'inertie n'est pas une force.} C'est une \textbf{propriété} de la matière. Dire qu'un passager est « poussé » vers l'avant quand l'autobus freine est incorrect : personne ne le pousse, il continue simplement tout droit par inertie pendant que l'autobus ralentit.

Dans un diagramme de corps libre, on ne dessine \textbf{jamais} l'inertie comme une force!
\end{attention}

\begin{pratiqueautonome}[title=Démystifier nos intuitions]

\textbf{Vrai ou Faux?} Pour chaque énoncé, indiquez s'il est vrai ou faux selon la physique newtonienne (pas selon l'intuition aristotélicienne!). Justifiez brièvement.

\begin{enumerate}
    \item Pour qu'un objet se déplace à vitesse constante, il faut exercer une force constante sur lui.
    
    \item Un objet qui avance est nécessairement soumis à une force qui le pousse.
    
    \item L'inertie est une force qui s'oppose au mouvement.
    
    \item Un passager « projeté vers l'avant » quand l'autobus freine est poussé par une force.
    
    \item Plus un objet est massif, plus il est difficile de changer sa vitesse.
    
    \item Un navire qui avance à 15 nœuds en ligne droite subit nécessairement une force résultante non nulle.
\end{enumerate}

\tcblower
\textbf{Réponses :}
\begin{enumerate}
    \item \textbf{FAUX} — À vitesse constante, la force résultante est nulle. Si vous devez pousser pour maintenir la vitesse, c'est pour compenser le frottement.
    \item \textbf{FAUX} — Un objet peut continuer à avancer par inertie, sans qu'aucune force ne le pousse. C'est exactement ce que Galilée a compris!
    \item \textbf{FAUX} — L'inertie est une propriété, pas une force. On ne la dessine jamais sur un DCL.
    \item \textbf{FAUX} — Aucune force ne le pousse. Il continue tout droit par inertie pendant que l'autobus décélère.
    \item \textbf{VRAI} — C'est la définition même de l'inertie : plus la masse est grande, plus la résistance au changement est grande.
    \item \textbf{FAUX} — Vitesse constante en ligne droite = MRU = force résultante nulle.
\end{enumerate}
\end{pratiqueautonome}

% =============================================================================
\subsection{La masse : mesure de l'inertie}
\label{subsec:masse}
% =============================================================================

Si l'inertie est la tendance à résister au changement, comment la quantifier? C'est là qu'intervient la \textbf{masse}.

\begin{definition}[title=Masse]
La \textbf{masse} est la mesure quantitative de l'inertie d'un objet. Plus un objet est massif, plus il résiste aux changements de son état de mouvement.

\begin{itemize}
    \item Symbole : $m$
    \item Unité SI : le \textbf{kilogramme} (kg)
    \item La masse est un \textbf{scalaire} (toujours positif)
    \item La masse est une propriété \textbf{intrinsèque} : elle ne dépend pas du lieu
\end{itemize}
\end{definition}

\begin{remarque}[title=Masse vs poids : première distinction]
La \textbf{masse} et le \textbf{poids} sont deux concepts différents :
\begin{itemize}
    \item La \textbf{masse} mesure l'inertie. Elle est la même partout dans l'univers.
    \item Le \textbf{poids} est la force gravitationnelle exercée sur un objet. Il dépend du lieu.
\end{itemize}

Un astronaute de masse $m = \SI{80}{kg}$ a la même masse sur Terre, sur la Lune ou dans l'espace. Mais son poids est différent à chaque endroit!

Nous reviendrons en détail sur cette distinction dans la section sur les forces.
\end{remarque}

% =============================================================================
\subsection{Les trois lois de Newton}
\label{subsec:lois-newton}
% =============================================================================

Isaac Newton (1642--1727) a synthétisé les travaux de Galilée et ses propres découvertes en trois lois fondamentales, publiées en 1687 dans les \textit{Principia Mathematica}. Ces trois lois constituent le fondement de toute la mécanique classique.

% -----------------------------------------------------------------------------
\subsubsection{Première loi de Newton (principe d'inertie)}
\label{subsubsec:premiere-loi}
% -----------------------------------------------------------------------------

\begin{definition}[title=Première loi de Newton]
\textbf{Tout objet persévère dans son état de repos ou de mouvement rectiligne uniforme, à moins qu'une force extérieure n'agisse sur lui.}

En d'autres termes : si la somme des forces sur un objet est nulle, alors :
\begin{itemize}
    \item S'il était au repos, il reste au repos.
    \item S'il était en mouvement, il continue en ligne droite à vitesse constante.
\end{itemize}

\begin{equationimportante}
\begin{equation}
    \sum \vect{F} = \vect{0} \quad \Longleftrightarrow \quad \vect{a} = \vect{0} \quad \text{(repos ou MRU)}
\end{equation}
\end{equationimportante}
\end{definition}

Cette loi formalise ce que Galilée avait compris : le mouvement à vitesse constante est l'état « naturel » d'un objet libre de toute force. Il n'y a pas besoin de force pour maintenir le mouvement --- seulement pour le \textit{changer}.

\begin{remarque}[title=Référentiels inertiels]
La première loi n'est valide que dans certains référentiels appelés \textbf{référentiels inertiels} (ou galiléens). Un référentiel inertiel est un référentiel dans lequel un objet libre de toute force reste au repos ou en MRU.

Pour nos applications, la surface de la Terre est une bonne approximation d'un référentiel inertiel (les effets de la rotation terrestre sont négligeables à notre échelle).
\end{remarque}

% -----------------------------------------------------------------------------
\subsubsection{Deuxième loi de Newton (loi fondamentale)}
\label{subsubsec:deuxieme-loi}
% -----------------------------------------------------------------------------

La première loi nous dit ce qui se passe quand la force résultante est nulle. Mais que se passe-t-il quand elle ne l'est pas?

\begin{definition}[title=Deuxième loi de Newton]
\textbf{L'accélération d'un objet est directement proportionnelle à la force résultante qui s'exerce sur lui, et inversement proportionnelle à sa masse.}

\begin{equationimportante}
\begin{equation}
    \boxed{\sum \vect{F} = m \vect{a}}
    \label{eq:deuxieme-loi}
\end{equation}
\end{equationimportante}

où :
\begin{itemize}
    \item $\sum \vect{F}$ est la somme vectorielle de toutes les forces (en newtons, N)
    \item $m$ est la masse de l'objet (en kilogrammes, kg)
    \item $\vect{a}$ est l'accélération résultante (en m/s²)
\end{itemize}
\end{definition}

Cette équation est probablement la plus importante de toute la mécanique. Elle établit le lien quantitatif entre force, masse et accélération :

\begin{itemize}
    \item Plus la force est grande, plus l'accélération est grande (proportionnalité directe).
    \item Plus la masse est grande, plus l'accélération est petite (proportionnalité inverse).
\end{itemize}

\begin{remarque}[title=Le newton]
L'unité de force, le \textbf{newton} (N), est définie à partir de la deuxième loi :
\[
    \SI{1}{N} = \SI{1}{kg \cdot m/s^2}
\]

Un newton est la force nécessaire pour donner à une masse de 1 kg une accélération de 1 m/s².

\textbf{Ordre de grandeur :} Une pomme pèse environ 1 N. Un humain de 70 kg pèse environ 700 N.
\end{remarque}

\begin{exemple}{Accélération d'un navire}{accel-navire}
Un remorqueur de masse $m = \SI{500}{tonnes} = \SI{5,0e5}{kg}$ est propulsé par une force de poussée $F = \SI{100}{kN}$. En négligeant la résistance de l'eau au démarrage, quelle est son accélération initiale?

\textbf{Solution :}

On applique la deuxième loi de Newton :
\[
    a = \frac{F}{m} = \frac{\SI{100e3}{N}}{\SI{5,0e5}{kg}} = \SI{0,20}{m/s^2}
\]

\textbf{Interprétation :} Le remorqueur gagne $\SI{0,20}{m/s}$ de vitesse chaque seconde. Pour atteindre $\SI{10}{m/s}$ (environ 20 nœuds), il lui faudrait $t = \frac{\SI{10}{m/s}}{\SI{0,20}{m/s^2}} = \SI{50}{s}$ en l'absence de résistance.

En réalité, la résistance de l'eau augmente avec la vitesse et limite la vitesse maximale atteignable.
\end{exemple}

% -----------------------------------------------------------------------------
\subsubsection{Troisième loi de Newton (action-réaction)}
\label{subsubsec:troisieme-loi}
% -----------------------------------------------------------------------------

\begin{definition}[title=Troisième loi de Newton]
\textbf{Quand un objet A exerce une force sur un objet B, l'objet B exerce simultanément sur l'objet A une force égale en module mais de sens opposé.}

\begin{equationimportante}
\begin{equation}
    \boxed{\vect{F}_{A \to B} = -\vect{F}_{B \to A}}
    \label{eq:troisieme-loi}
\end{equation}
\end{equationimportante}
\end{definition}

Les forces d'action et de réaction possèdent les caractéristiques suivantes :

\begin{enumerate}
    \item Elles ont le \textbf{même module} (même « force »).
    \item Elles ont la \textbf{même direction} (même ligne d'action).
    \item Elles ont des \textbf{sens opposés}.
    \item Elles agissent sur des \textbf{objets différents}.
    \item Elles apparaissent et disparaissent \textbf{simultanément}.
\end{enumerate}

\begin{center}
\begin{tikzpicture}[scale=1.0]
    % Navire (simplifié)
    \begin{scope}[xshift=0cm]
        \draw[thick, fill=gray!30, rounded corners=2pt] (-1.5, 0) -- (2, 0) -- (2.5, 0.3) -- (2, 0.6) -- (-1.5, 0.6) -- (-1.8, 0.3) -- cycle;
        \node at (0.3, 0.3) {\small Navire};
        
        % Hélice
        \draw[thick, fill=blue!20] (-1.8, 0.15) -- (-2.3, 0.3) -- (-1.8, 0.45) -- cycle;
    \end{scope}
    
    % Forces
    \draw[-{Stealth[length=4mm]}, very thick, red] (-2.3, 0.3) -- (-4, 0.3);
    \node[red, above] at (-3.15, 0.4) {$\vect{F}_{\text{hélice} \to \text{eau}}$};
    
    \draw[-{Stealth[length=4mm]}, very thick, blue] (-1.5, 0.3) -- (0.5, 0.3);
    \node[blue, below] at (-0.5, 0.2) {$\vect{F}_{\text{eau} \to \text{navire}}$};
    
    % Eau (vagues symboliques)
    \draw[thick, blue!50] (-4.5, -0.3) sin (-4, -0.1) cos (-3.5, -0.3) sin (-3, -0.1) cos (-2.5, -0.3);
    \draw[thick, blue!50] (2.5, -0.3) sin (3, -0.1) cos (3.5, -0.3) sin (4, -0.1) cos (4.5, -0.3);
    
    % Annotation
    \node[text width=8cm, align=center] at (0, -1.2) {\small L'hélice pousse l'eau vers l'arrière; en réaction, l'eau pousse le navire vers l'avant.};
\end{tikzpicture}
\end{center}

\begin{exemple}{Propulsion d'un navire}{propulsion}
Comment un navire avance-t-il? L'hélice, en tournant, pousse l'eau vers l'arrière. Par la troisième loi de Newton, l'eau exerce une force égale et opposée sur l'hélice (donc sur le navire), le propulsant vers l'avant.

La force exercée par l'hélice sur l'eau est exactement égale à la force exercée par l'eau sur le navire. Mais l'eau, ayant accès à tout l'océan, ne semble pas accélérer de façon perceptible, tandis que le navire, lui, avance.

C'est le même principe pour la marche : vous poussez le sol vers l'arrière avec votre pied, et le sol vous pousse vers l'avant.
\end{exemple}

\begin{attention}[title=Le poids et la force de la table ne forment PAS une paire action-réaction]
Quand un livre est posé sur une table, deux forces agissent \textbf{sur le livre} : son poids (vers le bas) et la force que la table exerce sur lui (vers le haut). Ces deux forces sont égales en module et opposées en direction, donc le livre reste immobile.

\textbf{Mais ces deux forces ne forment PAS une paire action-réaction!}

Elles agissent toutes les deux sur le \textbf{même objet} (le livre), ce qui viole la caractéristique fondamentale des paires action-réaction (qui doivent agir sur des objets \textit{différents}).

\begin{center}
\begin{tikzpicture}[scale=0.9]
    % Table
    \fill[brown!30] (-2, 0) rectangle (2, -0.3);
    \draw[thick] (-2, 0) -- (2, 0);
    \fill[pattern=north east lines, pattern color=brown!50] (-2, -0.3) rectangle (2, -0.5);
    
    % Livre
    \draw[thick, fill=blue!20] (-0.8, 0) rectangle (0.8, 0.5);
    \node at (0, 0.25) {\small Livre};
    
    % Forces sur le livre
    \draw[-{Stealth[length=3mm]}, very thick, red] (0, 0.25) -- (0, -1.0);
    \node[red, right] at (0.1, -0.4) {$\vect{F}_g$ (poids)};
    
    \draw[-{Stealth[length=3mm]}, very thick, green!60!black] (0, 0.25) -- (0, 1.5);
    \node[green!60!black, right] at (0.1, 0.9) {$\vect{F}_{\text{table}}$};
    
    % Annotation
    \node[text width=6cm, align=center] at (0, -1.8) {\small Ces deux forces agissent sur le \textbf{même objet}. Ce ne sont \textbf{pas} des forces action-réaction!};
\end{tikzpicture}
\end{center}

Les \textbf{vraies} paires action-réaction sont :
\begin{itemize}
    \item Le poids du livre (Terre $\to$ livre) et l'attraction du livre sur la Terre (livre $\to$ Terre)
    \item La force de la table sur le livre (table $\to$ livre) et la force du livre sur la table (livre $\to$ table)
\end{itemize}
\end{attention}

\begin{pratiqueautonome}[title=Identifier les paires action-réaction]

Pour chaque situation, identifiez la paire action-réaction associée à la force mentionnée.

\begin{enumerate}
    \item La Terre attire une pomme vers le bas.
    
    \item Un joueur de hockey frappe la rondelle avec son bâton.
    
    \item Un remorqueur tire un navire avec un câble.
    
    \item Les roues d'une voiture poussent le sol vers l'arrière.
    
    \item Un plongeur pousse sur le tremplin vers le bas.
\end{enumerate}

\tcblower
\textbf{Réponses :}
\begin{enumerate}
    \item La pomme attire la Terre vers le haut (avec la même force!).
    \item La rondelle exerce une force sur le bâton (c'est pourquoi on « sent » le choc).
    \item Le navire tire le remorqueur vers l'arrière (via le même câble).
    \item Le sol pousse les roues (donc la voiture) vers l'avant.
    \item Le tremplin pousse le plongeur vers le haut (c'est ce qui le propulse).
\end{enumerate}
\end{pratiqueautonome}

% =============================================================================
\subsection{L'équilibre : un cas particulier de la première loi}
\label{subsec:equilibre-intro}
% =============================================================================

La première loi de Newton nous dit qu'un objet reste au repos (ou en MRU) si aucune force résultante n'agit sur lui. Mais que se passe-t-il si \textit{plusieurs} forces agissent sur un objet, mais que leur somme vectorielle est nulle? L'objet reste également en équilibre!

\begin{definition}[title=Équilibre de translation]
Un objet est en \textbf{équilibre de translation} si et seulement si la somme vectorielle de toutes les forces qui agissent sur lui est nulle :

\begin{equationimportante}
\begin{equation}
    \boxed{\sum \vect{F} = \vect{0}}
    \label{eq:equilibre}
\end{equation}
\end{equationimportante}

En composantes cartésiennes :
\begin{align}
    \sum F_x &= 0 \\
    \sum F_y &= 0
\end{align}
\end{definition}

Un objet en équilibre peut être :
\begin{itemize}
    \item \textbf{Au repos} — et il restera au repos (\textbf{équilibre statique})
    \item \textbf{En mouvement rectiligne uniforme} — et il continuera ainsi (\textbf{équilibre dynamique})
\end{itemize}

\begin{remarque}[title=L'équilibre ne signifie pas « immobile »]
Un navire qui avance en ligne droite à vitesse constante est en équilibre : la poussée des moteurs est exactement compensée par la résistance de l'eau.

Un parachutiste en chute à vitesse terminale est aussi en équilibre : son poids est compensé par la résistance de l'air.

Dans les deux cas, $\sum \vect{F} = \vect{0}$ même si l'objet est en mouvement!
\end{remarque}

L'étude des systèmes en équilibre statique (au repos) s'appelle la \textbf{statique}. C'est un cas particulier de la dynamique où l'accélération est nulle. Les conditions d'équilibre ($\sum F_x = 0$ et $\sum F_y = 0$) nous permettent de calculer des forces inconnues dans des systèmes comme les câbles de suspension, les amarres de navire, ou les structures en général.

Nous approfondirons l'équilibre et la résolution de problèmes de statique après avoir étudié les différents types de forces.