% =============================================================================
% SECTION 3 - LE DIAGRAMME DE CORPS LIBRE ET L'ÉQUILIBRE
% Blocs 3 et 4 (4 heures - Semaine 6)
% =============================================================================

\section{Le diagramme de corps libre et l'équilibre}
\label{sec:dcl-equilibre}

% =============================================================================
\subsection{Le diagramme de corps libre (DCL)}
\label{subsec:dcl}
% =============================================================================

Le \textbf{diagramme de corps libre} (DCL) est l'outil le plus important pour résoudre des problèmes de mécanique. Avant de pouvoir appliquer les lois de Newton, vous devez identifier \textbf{toutes} les forces qui agissent sur l'objet étudié --- et le DCL est la méthode systématique pour y parvenir.

\begin{definition}[title=Diagramme de corps libre (DCL)]
Un \textbf{diagramme de corps libre} est un schéma simplifié montrant :
\begin{itemize}
    \item L'objet étudié, représenté de façon isolée (souvent par un point ou une forme simple)
    \item \textbf{Toutes} les forces extérieures agissant sur cet objet, représentées par des vecteurs
    \item Un système de coordonnées approprié
\end{itemize}

Le DCL ne montre \textbf{que} les forces sur l'objet étudié, pas les forces qu'il exerce sur d'autres objets.
\end{definition}

\begin{remarque}[title=Pourquoi le DCL est-il si important?]
Le DCL est à la mécanique ce que la liste d'épicerie est aux courses : on peut parfois s'en sortir sans, mais on oublie souvent quelque chose d'important!

Sans un DCL correct et complet :
\begin{itemize}
    \item Vous risquez d'oublier des forces
    \item Vous risquez d'ajouter des forces qui n'existent pas
    \item Vous ne pourrez pas écrire correctement les équations de Newton
    \item Votre solution sera probablement fausse
\end{itemize}

Investir du temps dans un bon DCL, c'est s'assurer une résolution correcte.
\end{remarque}

% -----------------------------------------------------------------------------
\subsubsection{Les cinq règles d'or du DCL}
\label{subsubsec:regles-dcl}
% -----------------------------------------------------------------------------

\begin{enumerate}
    \item \textbf{Isoler l'objet} --- Dessinez l'objet étudié \textit{seul}, séparé de son environnement. Représentez-le par un point ou une forme simple.
    
    \item \textbf{Représenter TOUTES les forces} --- Passez en revue mentalement toutes les forces possibles :
    \begin{itemize}
        \item[$\checkmark$] Le poids (toujours présent!)
        \item[$\checkmark$] La normale (s'il y a contact avec une surface)
        \item[$\checkmark$] La tension (s'il y a une corde ou un câble)
        \item[$\checkmark$] Le frottement (s'il y a contact et mouvement ou tendance au mouvement)
        \item[$\checkmark$] Toute autre force appliquée
    \end{itemize}
    
    \item \textbf{Partir du centre de masse} --- Dessinez toutes les forces comme partant du centre de l'objet (ou d'un point représentant l'objet).
    
    \item \textbf{Respecter les directions} --- Chaque force doit pointer dans la bonne direction :
    \begin{itemize}
        \item Poids : vers le bas (vertical)
        \item Normale : perpendiculaire à la surface
        \item Tension : le long de la corde, tirant l'objet
        \item Frottement : parallèle à la surface, opposé au mouvement
    \end{itemize}
    
    \item \textbf{Ne JAMAIS dessiner l'accélération} --- L'accélération n'est \textbf{pas} une force! Elle est la \textit{conséquence} des forces, pas une force elle-même.
\end{enumerate}

\begin{attention}[title=Erreurs fréquentes à éviter]
\begin{center}
\renewcommand{\arraystretch}{1.5}
\begin{tabular}{|L{6cm}|L{7cm}|}
\hline
\rowcolor{red!10} \textbf{Erreur} & \textbf{Correction} \\
\hline
Dessiner une « force du mouvement » & Le mouvement n'est pas une force! Un objet en MRU n'a pas besoin de force pour maintenir sa vitesse. \\
\hline
Dessiner l'inertie comme une force & L'inertie est une propriété, pas une force. \\
\hline
Oublier le poids & La Terre attire \textbf{toujours} l'objet. Le poids est toujours présent! \\
\hline
Confondre normale et poids & Ce sont deux forces distinctes. $N = mg$ seulement dans des cas particuliers. \\
\hline
Dessiner les forces sur d'autres objets & Le DCL ne montre que les forces \textbf{sur} l'objet étudié. \\
\hline
Oublier le frottement & S'il y a contact et mouvement (ou tendance), il y a frottement. \\
\hline
\end{tabular}
\end{center}
\end{attention}

% -----------------------------------------------------------------------------
\subsubsection{Construction d'un DCL : progression guidée}
\label{subsubsec:construction-dcl}
% -----------------------------------------------------------------------------

Voyons comment construire un DCL correct dans des situations de complexité croissante.

\paragraph{Niveau 1 : Objet au repos sur surface horizontale}

\begin{center}
\begin{tikzpicture}[scale=1.0]
    % Schéma de situation (gauche)
    \begin{scope}[xshift=-4cm]
        \node[above, font=\bfseries] at (0, 2.2) {Situation};
        
        % Table
        \fill[brown!30] (-1.5, 0) rectangle (1.5, -0.2);
        \draw[thick] (-1.5, 0) -- (1.5, 0);
        \fill[pattern=north east lines, pattern color=brown] (-1.5, -0.2) rectangle (1.5, -0.4);
        
        % Livre
        \draw[thick, fill=blue!20] (-0.6, 0) rectangle (0.6, 0.4);
        \node at (0, 0.2) {\small Livre};
    \end{scope}
    
    % DCL (droite)
    \begin{scope}[xshift=2cm]
        \node[above, font=\bfseries] at (0, 2.2) {DCL du livre};
        
        % Point représentant le livre
        \fill[blue!50] (0, 0.8) circle (5pt);
        
        % Axes
        \draw[-{Stealth}, thick] (0, 0.8) -- (2, 0.8) node[right] {$x$};
        \draw[-{Stealth}, thick] (0, 0.8) -- (0, 2.5) node[above] {$y$};
        
        % Poids
        \draw[-{Stealth[length=3mm]}, very thick, red] (0, 0.8) -- (0, -0.5);
        \node[red, right] at (0.1, 0.1) {$\vect{F}_g = m\vect{g}$};
        
        % Normale
        \draw[-{Stealth[length=3mm]}, very thick, green!60!black] (0, 0.8) -- (0, 2.1);
        \node[green!60!black, right] at (0.1, 1.5) {$\vect{N}$};
    \end{scope}
\end{tikzpicture}
\end{center}

\textbf{Forces identifiées :}
\begin{itemize}
    \item Poids $\vect{F}_g$ : vers le bas
    \item Normale $\vect{N}$ : vers le haut (perpendiculaire à la table)
\end{itemize}

\textbf{Équilibre :} $\sum F_y = N - mg = 0 \Rightarrow N = mg$

\paragraph{Niveau 2 : Objet tiré par une corde}

\begin{center}
\begin{tikzpicture}[scale=1.0]
    % Schéma de situation (gauche)
    \begin{scope}[xshift=-4cm]
        \node[above, font=\bfseries] at (0, 2.2) {Situation};
        
        % Sol
        \fill[gray!20] (-2, 0) rectangle (2, -0.2);
        \draw[thick] (-2, 0) -- (2, 0);
        
        % Caisse
        \draw[thick, fill=blue!20] (-0.5, 0) rectangle (0.5, 0.6);
        \node at (0, 0.3) {\small $m$};
        
        % Corde
        \draw[thick, brown] (0.5, 0.5) -- (1.8, 1.3);
        
        % Angle
        \draw[thick] (1.0, 0.5) arc (0:30:0.5);
        \node at (1.3, 0.7) {\small $\theta$};
        
        % Flèche mouvement
        \draw[-{Stealth}, thick, purple] (-0.8, 0.3) -- (-1.5, 0.3);
        \node[purple, above] at (-1.15, 0.4) {\small $\vect{v}$};
    \end{scope}
    
    % DCL (droite)
    \begin{scope}[xshift=2.5cm]
        \node[above, font=\bfseries] at (0, 2.2) {DCL de la caisse};
        
        % Point
        \fill[blue!50] (0, 0.8) circle (5pt);
        
        % Axes
        \draw[-{Stealth}, thick] (0, 0.8) -- (2.5, 0.8) node[right] {$x$};
        \draw[-{Stealth}, thick] (0, 0.8) -- (0, 2.5) node[above] {$y$};
        
        % Poids
        \draw[-{Stealth[length=3mm]}, very thick, red] (0, 0.8) -- (0, -0.4);
        \node[red, right] at (0.1, 0.2) {$\vect{F}_g$};
        
        % Normale
        \draw[-{Stealth[length=3mm]}, very thick, green!60!black] (0, 0.8) -- (0, 1.9);
        \node[green!60!black, right] at (0.1, 1.4) {$\vect{N}$};
        
        % Tension
        \draw[-{Stealth[length=3mm]}, very thick, orange] (0, 0.8) -- ({1.3*cos(30)}, {0.8 + 1.3*sin(30)});
        \node[orange, above right] at (0.8, 1.3) {$\vect{T}$};
        
        % Frottement
        \draw[-{Stealth[length=3mm]}, very thick, blue] (0, 0.8) -- (-1.0, 0.8);
        \node[blue, below] at (-0.5, 0.65) {$\vect{f}$};
        
        % Angle de la tension
        \draw[thick] (0.5, 0.8) arc (0:30:0.5);
        \node at (0.75, 1.0) {\small $\theta$};
    \end{scope}
\end{tikzpicture}
\end{center}

\textbf{Forces identifiées :}
\begin{itemize}
    \item Poids $\vect{F}_g$ : vers le bas
    \item Normale $\vect{N}$ : vers le haut
    \item Tension $\vect{T}$ : le long de la corde, à l'angle $\theta$
    \item Frottement $\vect{f}$ : opposé au mouvement (vers la gauche)
\end{itemize}

\textbf{Note :} Ici, $N \neq mg$ car la tension a une composante verticale!

\paragraph{Niveau 3 : Objet sur plan incliné}

\begin{center}
\begin{tikzpicture}[scale=1.0]
    % Schéma de situation (gauche)
    \begin{scope}[xshift=-4cm]
        \node[above, font=\bfseries] at (0.5, 2.5) {Situation};
        
        % Plan incliné
        \fill[gray!20] (0, 0) -- (3, 0) -- (3, 1.5) -- cycle;
        \draw[thick] (0, 0) -- (3, 1.5);
        \draw[thick] (0, 0) -- (3, 0);
        
        % Angle
        \draw[thick] (0.8, 0) arc (0:26.57:0.8);
        \node at (1.1, 0.2) {\small $\theta$};
        
        % Bloc
        \begin{scope}[rotate=26.57, shift={(1.2, 0)}]
            \draw[thick, fill=blue!20] (0, 0) rectangle (0.6, 0.5);
            \node at (0.3, 0.25) {\small $m$};
        \end{scope}
    \end{scope}
    
    % DCL (droite) - axes standards pour l'instant
    \begin{scope}[xshift=2.5cm]
        \node[above, font=\bfseries] at (0, 2.5) {DCL du bloc};
        
        % Point
        \fill[blue!50] (0, 0.8) circle (5pt);
        
        % Poids (vertical vers le bas)
        \draw[-{Stealth[length=3mm]}, very thick, red] (0, 0.8) -- (0, -0.6);
        \node[red, right] at (0.1, 0.1) {$\vect{F}_g$};
        
        % Normale (perpendiculaire au plan)
        \draw[-{Stealth[length=3mm]}, very thick, green!60!black] (0, 0.8) -- ({-1.1*sin(26.57)}, {0.8 + 1.1*cos(26.57)});
        \node[green!60!black, above left] at (-0.3, 1.6) {$\vect{N}$};
        
        % Frottement (parallèle au plan, vers le haut)
        \draw[-{Stealth[length=3mm]}, very thick, blue] (0, 0.8) -- ({-0.8*cos(26.57)}, {0.8 - 0.8*sin(26.57)});
        \node[blue, below left] at (-0.5, 0.5) {$\vect{f}$};
    \end{scope}
\end{tikzpicture}
\end{center}

\textbf{Forces identifiées :}
\begin{itemize}
    \item Poids $\vect{F}_g$ : vertical vers le bas
    \item Normale $\vect{N}$ : perpendiculaire au plan (vers l'extérieur)
    \item Frottement $\vect{f}$ : parallèle au plan, vers le haut (oppose la tendance à glisser)
\end{itemize}

\textbf{Note :} Le choix judicieux des axes et la décomposition des forces seront présentés dans les exemples de résolution.

% =============================================================================
\subsection{L'algorithme de résolution}
\label{subsec:algorithme}
% =============================================================================

Voici la méthode systématique pour résoudre \textbf{tout} problème de mécanique --- qu'il s'agisse d'équilibre (statique) ou de mouvement (dynamique).

\begin{definition}[title=Algorithme de résolution en 4 étapes]

\textbf{Étape 1 — SCHÉMA et DCL}
\begin{enumerate}[label=\alph*)]
    \item Dessiner un schéma de la situation physique
    \item Isoler l'objet d'intérêt (ou chaque objet, s'il y en a plusieurs)
    \item Tracer le DCL : représenter \textbf{toutes} les forces agissant sur l'objet
    \item Identifier les forces connues et inconnues
\end{enumerate}

\textbf{Étape 2 — AXES}
\begin{enumerate}[label=\alph*)]
    \item Choisir un système de coordonnées $(x, y)$ adapté au problème
    \item \textit{Conseil :} Aligner un axe avec l'accélération (si connue) ou avec la surface de contact
    \item Indiquer clairement la direction positive de chaque axe
\end{enumerate}

\textbf{Étape 3 — ÉQUATIONS DE NEWTON}
\begin{enumerate}[label=\alph*)]
    \item Décomposer chaque force selon les axes $x$ et $y$
    \item Appliquer la deuxième loi de Newton (ou les conditions d'équilibre) :
    \begin{align*}
        \sum F_x &= ma_x \qquad \text{(ou } = 0 \text{ si équilibre)} \\
        \sum F_y &= ma_y \qquad \text{(ou } = 0 \text{ si équilibre)}
    \end{align*}
    \item Écrire les équations explicitement avec les symboles des forces
\end{enumerate}

\textbf{Étape 4 — ALGÈBRE}
\begin{enumerate}[label=\alph*)]
    \item Compter les équations et les inconnues (il faut autant d'équations que d'inconnues!)
    \item Résoudre le système d'équations \textbf{algébriquement} (avec des symboles)
    \item Substituer les valeurs numériques \textit{à la fin}
    \item Vérifier : unités correctes? Ordre de grandeur raisonnable? Signe cohérent?
\end{enumerate}
\end{definition}

\begin{remarque}[title=Pourquoi résoudre algébriquement d'abord?]
Résoudre avec des symboles avant de substituer les nombres présente plusieurs avantages :
\begin{itemize}
    \item On peut vérifier les \textbf{unités} de la réponse finale
    \item On peut analyser les \textbf{cas limites} (que se passe-t-il si $\theta \to 0$? si $m \to \infty$?)
    \item On évite les erreurs de calcul intermédiaires
    \item La solution est \textbf{réutilisable} pour d'autres valeurs numériques
\end{itemize}
\end{remarque}

% -----------------------------------------------------------------------------
\subsubsection{Conseils pour le choix des axes}
\label{subsubsec:choix-axes}
% -----------------------------------------------------------------------------

Le choix du système de coordonnées est libre, mais certains choix simplifient grandement les calculs :

\begin{center}
\renewcommand{\arraystretch}{1.5}
\begin{tabular}{|L{4cm}|L{9cm}|}
\hline
\rowcolor{bleuclair} \textbf{Situation} & \textbf{Choix recommandé} \\
\hline
Mouvement horizontal & $x$ horizontal (sens du mouvement), $y$ vertical vers le haut \\
\hline
Plan incliné & $x$ parallèle à la pente, $y$ perpendiculaire à la pente \\
\hline
Objet suspendu & $x$ horizontal, $y$ vertical vers le haut \\
\hline
Mouvement circulaire & Un axe vers le centre du cercle \\
\hline
\end{tabular}
\end{center}

\begin{remarque}[title=Peu importe le choix, la physique est la même]
Vous pouvez toujours choisir n'importe quel système d'axes --- la réponse finale sera la même. Mais un bon choix réduit le nombre de forces à décomposer et simplifie les équations.

Sur un plan incliné, si vous choisissez des axes horizontaux/verticaux, vous devrez décomposer \textbf{trois} forces ($\vect{N}$, $\vect{f}$, et $\vect{F}_g$). Avec des axes parallèles/perpendiculaires à la pente, seul le poids doit être décomposé.
\end{remarque}

% =============================================================================
\subsection{Applications : problèmes d'équilibre (statique)}
\label{subsec:applications-statique}
% =============================================================================

Appliquons maintenant l'algorithme à des problèmes de statique, où $\vect{a} = \vect{0}$.

% -----------------------------------------------------------------------------
\begin{exemple}{Conteneur suspendu par deux câbles}{conteneur-cables}

Un conteneur de masse $m = \SI{2000}{kg}$ est suspendu dans la cale d'un navire par deux câbles. Le câble de bâbord fait un angle de $\alpha = 30°$ avec l'horizontale, celui de tribord fait un angle de $\beta = 45°$ avec l'horizontale. Déterminer les tensions $T_1$ et $T_2$ dans chaque câble.

\tcblower

\textbf{\underline{Étape 1 — SCHÉMA et DCL}}

\begin{center}
\begin{tikzpicture}[scale=0.9]
    % Schéma de situation (gauche)
    \begin{scope}[xshift=-4.5cm]
        % Supports
        \fill[gray!50] (-2.5, 2.5) rectangle (-2, 3);
        \fill[gray!50] (2, 2.5) rectangle (2.5, 3);
        
        % Câbles
        \draw[thick, brown] (-2.25, 2.5) -- (0, 0.8);
        \draw[thick, brown] (2.25, 2.5) -- (0, 0.8);
        
        % Conteneur
        \draw[thick, fill=blue!20] (-0.6, 0) rectangle (0.6, 0.8);
        \node at (0, 0.4) {\small $m$};
        
        % Angles
        \draw[thick] (-1.5, 2.5) arc (0:-30:0.75);
        \node at (-1.1, 2.2) {\small $\alpha$};
        \draw[thick] (1.5, 2.5) arc (180:225:0.75);
        \node at (1.1, 2.2) {\small $\beta$};
        
        % Labels
        \node[above left] at (-1.2, 1.7) {\small Câble 1};
        \node[above right] at (1.2, 1.7) {\small Câble 2};
        
        \node[below] at (0, -0.5) {\textit{Schéma de situation}};
    \end{scope}
    
    % DCL (droite)
    \begin{scope}[xshift=3cm]
        % Point
        \fill[blue!50] (0, 0) circle (5pt);
        
        % Axes
        \draw[-{Stealth}, thick] (0, 0) -- (2.5, 0) node[right] {$x$};
        \draw[-{Stealth}, thick] (0, 0) -- (0, 2.5) node[above] {$y$};
        
        % Poids
        \draw[-{Stealth[length=3mm]}, very thick, red] (0, 0) -- (0, -1.5);
        \node[red, right] at (0.1, -0.8) {$\vect{F}_g$};
        
        % Tension 1
        \draw[-{Stealth[length=3mm]}, very thick, orange] (0, 0) -- ({-1.5*cos(30)}, {1.5*sin(30)});
        \node[orange, above left] at (-0.9, 0.6) {$\vect{T}_1$};
        
        % Tension 2
        \draw[-{Stealth[length=3mm]}, very thick, green!60!black] (0, 0) -- ({1.5*cos(45)}, {1.5*sin(45)});
        \node[green!60!black, above right] at (0.8, 0.9) {$\vect{T}_2$};
        
        % Angles
        \draw[thick] ({-0.5*cos(30)}, {0.5*sin(30)}) arc (150:180:0.5);
        \node at (-0.65, 0.15) {\small $\alpha$};
        \draw[thick] ({0.5*cos(45)}, {0.5*sin(45)}) arc (45:0:0.5);
        \node at (0.65, 0.15) {\small $\beta$};
        
        \node[below] at (0, -2.0) {\textit{DCL du conteneur}};
    \end{scope}
\end{tikzpicture}
\end{center}

\textbf{Forces sur le conteneur :}
\begin{itemize}
    \item Poids : $F_g = mg$ (vers le bas)
    \item Tension 1 : $T_1$ (vers le câble de bâbord, à l'angle $\alpha$ au-dessus de l'horizontale)
    \item Tension 2 : $T_2$ (vers le câble de tribord, à l'angle $\beta$ au-dessus de l'horizontale)
\end{itemize}

\textbf{\underline{Étape 2 — MÉTHODE GÉOMÉTRIQUE (Triangle de forces)}}

Puisque le conteneur est en équilibre, la somme vectorielle des trois forces est nulle :
\[
    \vect{T}_1 + \vect{T}_2 + \vect{F}_g = \vect{0}
\]

Cela signifie que si on place les trois vecteurs bout à bout, ils forment un \textbf{triangle fermé}.

\begin{center}
\begin{tikzpicture}[scale=1.1]
    % Triangle de forces
    % On part du poids (vers le bas), puis T1, puis T2 qui revient au point de départ
    
    % Poids (vers le bas)
    \draw[-{Stealth[length=3mm]}, very thick, red] (0, 2.5) -- (0, 0);
    \node[red, left] at (-0.15, 1.25) {$\vect{F}_g = mg$};
    
    % T1 (fait un angle alpha avec l'horizontale)
    \draw[-{Stealth[length=3mm]}, very thick, orange] (0, 0) -- ({3*cos(30)}, {3*sin(30)});
    \node[orange, below right] at (1.4, 0.5) {$\vect{T}_1$};
    
    % T2 (revient au point de départ)
    \draw[-{Stealth[length=3mm]}, very thick, green!60!black] ({3*cos(30)}, {3*sin(30)}) -- (0, 2.5);
    \node[green!60!black, above right] at (1.6, 2.1) {$\vect{T}_2$};
    
    % Angles dans le triangle
    % Angle au sommet bas (entre Fg et T1) = 90° - alpha
    \draw[thick] (0, 0.6) arc (90:30:0.6);
    \node at (0.5, 0.8) {\small $90°{-}\alpha$};
    
    % Angle au sommet droit (entre T1 et T2) = 180° - alpha - beta
    \draw[thick] ({3*cos(30) - 0.5*cos(30+60)}, {3*sin(30) + 0.5*sin(30+60)}) arc (120:180-15:0.5);
    
    % Angle au sommet haut (entre T2 et Fg) = 90° - beta
    \draw[thick] (0, 1.9) arc (270:270+45:0.6);
    \node at (0.45, 1.65) {\small $90°{-}\beta$};
    
    % Annotation
    \node[text width=5cm, align=left] at (6.5, 1.25) {
        Les angles du triangle sont :
        \begin{itemize}
            \item En bas : $90° - \alpha$
            \item En haut : $90° - \beta$
            \item À droite : $\alpha + \beta$
        \end{itemize}
    };
\end{tikzpicture}
\end{center}

\textbf{\underline{Étape 3 — LOI DES SINUS}}

Dans un triangle, la loi des sinus stipule que le rapport entre un côté et le sinus de l'angle opposé est constant :
\[
    \frac{T_1}{\sin(90° - \beta)} = \frac{T_2}{\sin(90° - \alpha)} = \frac{mg}{\sin(\alpha + \beta)}
\]

En utilisant $\sin(90° - \theta) = \cos\theta$ :
\[
    \frac{T_1}{\cos\beta} = \frac{T_2}{\cos\alpha} = \frac{mg}{\sin(\alpha + \beta)}
\]

\textbf{\underline{Étape 4 — CALCULS}}

\textbf{Application numérique :}

Avec $m = \SI{2000}{kg}$, $\alpha = 30°$, $\beta = 45°$ :

\[
    \sin(\alpha + \beta) = \sin(75°) = 0{,}966
\]

\[
    T_1 = \frac{mg \cos\beta}{\sin(\alpha + \beta)} = \frac{2000 \times 9{,}81 \times \cos 45°}{\sin 75°} = \frac{19\,620 \times 0{,}707}{0{,}966} = \boxed{\SI{14{,}4}{kN}}
\]

\[
    T_2 = \frac{mg \cos\alpha}{\sin(\alpha + \beta)} = \frac{2000 \times 9{,}81 \times \cos 30°}{\sin 75°} = \frac{19\,620 \times 0{,}866}{0{,}966} = \boxed{\SI{17{,}6}{kN}}
\]

\textbf{Vérification :}
\begin{itemize}
    \item $T_2 > T_1$ : logique car le câble de tribord est plus vertical ($45° > 30°$), il supporte une plus grande partie du poids.
    \item On peut vérifier : $T_1 \sin\alpha + T_2 \sin\beta = 14\,400 \times 0{,}5 + 17\,600 \times 0{,}707 = 7\,200 + 12\,440 \approx mg$ ✓
\end{itemize}

\begin{remarque}[title=Avantage de la méthode géométrique]
La méthode du triangle de forces avec la loi des sinus donne directement les tensions sans avoir à résoudre un système d'équations. Elle est particulièrement efficace quand on a exactement trois forces en équilibre.
\end{remarque}
\end{exemple}

\begin{pratiqueautonome}[title=Feu de navigation suspendu]

Un feu de navigation de masse $m = \SI{15}{kg}$ est suspendu entre deux mâts par deux câbles. Le câble de bâbord fait un angle de $\alpha = 35°$ avec l'horizontale, celui de tribord fait un angle de $\beta = 50°$ avec l'horizontale.

\begin{enumerate}
    \item Dessinez le DCL du feu de navigation.
    \item Écrivez les équations d'équilibre.
    \item Calculez les tensions $T_1$ et $T_2$ dans chaque câble.
\end{enumerate}

\espaceresolution[6cm]

\tcblower
\textbf{Réponses :}

Équation (1) : $-T_1\cos 35° + T_2\cos 50° = 0$

Équation (2) : $T_1\sin 35° + T_2\sin 50° - mg = 0$

$T_1 = \SI{106}{N}$ ; $T_2 = \SI{135}{N}$
\end{pratiqueautonome}

% -----------------------------------------------------------------------------
\begin{exemple}{Bloc sur plan incliné à l'équilibre}{bloc-plan-equilibre}

Un bloc de masse $m = \SI{20}{kg}$ est posé sur un plan incliné à $\theta = 20°$ par rapport à l'horizontale. Le bloc reste immobile. Le coefficient de frottement statique est $\mu_s = 0{,}5$.

a) Calculer la force de frottement statique qui maintient le bloc en équilibre.

b) Quel est l'angle maximal avant que le bloc ne commence à glisser?

\tcblower

\textbf{\underline{Étape 1 — SCHÉMA et DCL}}

\begin{center}
\begin{tikzpicture}[scale=0.85]
    % Schéma de situation (gauche)
    \begin{scope}[xshift=-4.5cm]
        % Plan incliné (20°)
        \fill[gray!20] (0, 0) -- (4, 0) -- (4, 1.46) -- cycle;
        \draw[thick] (0, 0) -- (4, 1.46);
        \draw[thick] (0, 0) -- (4, 0);
        
        % Angle
        \draw[thick] (1, 0) arc (0:20:1);
        \node at (1.4, 0.2) {\small $20°$};
        
        % Bloc
        \begin{scope}[rotate=20, shift={(1.5, 0)}]
            \draw[thick, fill=blue!20] (0, 0) rectangle (0.8, 0.6);
            \node at (0.4, 0.3) {\small $m$};
        \end{scope}
        
        % Sol
        \fill[pattern=north east lines] (-0.2, -0.2) rectangle (4.2, 0);
        
        \node[below] at (2, -0.6) {\textit{Schéma de situation}};
    \end{scope}
    
    % DCL (droite)
    \begin{scope}[xshift=3cm]
        % Point
        \fill[blue!50] (0, 0) circle (5pt);
        
        % Axes inclinés
        \draw[-{Stealth}, thick] (0, 0) -- ({2.3*cos(20)}, {2.3*sin(20)}) node[right] {$x$};
        \draw[-{Stealth}, thick] (0, 0) -- ({-1.8*sin(20)}, {1.8*cos(20)}) node[above] {$y$};
        \node[below right, font=\scriptsize] at (1.7, 0.6) {(+ vers bas)};
        
        % Poids (vertical)
        \draw[-{Stealth[length=3mm]}, very thick, red] (0, 0) -- (0, -1.8);
        \node[red, right] at (0.1, -1.0) {$\vect{F}_g$};
        
        % Normale
        \draw[-{Stealth[length=3mm]}, very thick, green!60!black] (0, 0) -- ({-1.56*sin(20)}, {1.56*cos(20)});
        \node[green!60!black, above left] at (-0.3, 1.3) {$\vect{N}$};
        
        % Frottement (vers le haut de la pente = -x)
        \draw[-{Stealth[length=3mm]}, very thick, blue] (0, 0) -- ({-0.9*cos(20)}, {-0.9*sin(20)});
        \node[blue, below left] at (-0.7, -0.1) {$\vect{f}_s$};
        
        \node[below] at (0, -2.3) {\textit{DCL du bloc}};
    \end{scope}
\end{tikzpicture}
\end{center}

\textbf{Forces sur le bloc :}
\begin{itemize}
    \item Poids : $F_g = mg$ (vertical vers le bas)
    \item Normale : $N$ (perpendiculaire au plan, vers l'extérieur)
    \item Frottement statique : $f_s$ (parallèle au plan, vers le haut --- s'oppose à la tendance à glisser)
\end{itemize}

\textbf{\underline{Étape 2 — AXES}}

$x$ : parallèle à la pente, positif vers le bas

$y$ : perpendiculaire à la pente, positif vers l'extérieur

Ce choix est stratégique : seul le poids doit être décomposé.

\textbf{\underline{Étape 3 — ÉQUATIONS DE NEWTON}}

\textbf{Décomposition du poids :}
\begin{align*}
    F_{g,x} &= mg\sin\theta \quad \text{(vers le bas de la pente)} \\
    F_{g,y} &= -mg\cos\theta \quad \text{(vers la pente)}
\end{align*}

\textbf{Équilibre selon $y$ :} $\sum F_y = 0$
\begin{equation}
    N - mg\cos\theta = 0 \quad \Rightarrow \quad N = mg\cos\theta \tag{1}
\end{equation}

\textbf{Équilibre selon $x$ :} $\sum F_x = 0$
\begin{equation}
    mg\sin\theta - f_s = 0 \quad \Rightarrow \quad f_s = mg\sin\theta \tag{2}
\end{equation}

\textbf{\underline{Étape 4 — ALGÈBRE}}

\textbf{a) Force de frottement :}
\[
    f_s = mg\sin\theta = 20 \times 9{,}81 \times \sin 20° = 196{,}2 \times 0{,}342 = \boxed{\SI{67{,}1}{N}}
\]

\textbf{Vérification que le bloc peut rester en équilibre :}

Le frottement maximal possible est :
\[
    f_{s,max} = \mu_s N = \mu_s mg\cos\theta = 0{,}5 \times 20 \times 9{,}81 \times \cos 20° = 0{,}5 \times 196{,}2 \times 0{,}940 = \SI{92{,}2}{N}
\]

Comme $f_s = \SI{67{,}1}{N} < f_{s,max} = \SI{92{,}2}{N}$, le bloc reste bien en équilibre. ✓

\textbf{b) Angle maximal :}

Le bloc commence à glisser quand $f_s = f_{s,max}$, c'est-à-dire quand :
\[
    mg\sin\theta_{max} = \mu_s mg\cos\theta_{max}
\]
\[
    \tan\theta_{max} = \mu_s
\]
\[
    \theta_{max} = \arctan(\mu_s) = \arctan(0{,}5) = \boxed{26{,}6°}
\]

\textbf{Remarque :} L'angle limite de glissement ne dépend que de $\mu_s$, pas de la masse!
\end{exemple}

\begin{pratiqueautonome}[title=Caisse sur rampe de chargement]

Une caisse de $\SI{150}{kg}$ est posée sur une rampe de chargement inclinée à $25°$. Le coefficient de frottement statique entre la caisse et la rampe est $\mu_s = 0{,}6$.

\begin{enumerate}
    \item La caisse reste-t-elle en place ou glisse-t-elle?
    \item Si elle reste en place, quelle est la force de frottement?
    \item Quel angle maximal la rampe pourrait-elle avoir avant que la caisse ne glisse?
\end{enumerate}

\espaceresolution[5cm]

\tcblower
\textbf{Réponses :}
\begin{enumerate}
    \item $\tan 25° = 0{,}466 < \mu_s = 0{,}6$, donc la caisse \textbf{reste en place}.
    \item $f_s = mg\sin 25° = 150 \times 9{,}81 \times 0{,}423 = \SI{622}{N}$
    \item $\theta_{max} = \arctan(0{,}6) = 31{,}0°$
\end{enumerate}
\end{pratiqueautonome}