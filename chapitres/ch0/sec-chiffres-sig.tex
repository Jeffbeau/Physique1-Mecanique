% =============================================================================
\section{Scalaires et vecteurs}
\label{sec:vecteurs}
% =============================================================================

En physique, toutes les grandeurs ne se comportent pas de la même façon. Certaines sont entièrement décrites par un simple nombre : la température de l'eau, la masse d'un navire, le temps écoulé. D'autres nécessitent plus d'information : quand on parle du vent, dire qu'il souffle à 30 km/h ne suffit pas — il faut aussi savoir \textit{dans quelle direction} il souffle !

Cette distinction fondamentale divise les grandeurs physiques en deux catégories : les \textbf{scalaires} et les \textbf{vecteurs}. Comprendre cette différence est essentiel pour la suite du cours, car les vecteurs obéissent à des règles de calcul différentes des nombres ordinaires.

\begin{definition}
\begin{itemize}
    \item Un \textbf{scalaire} est une grandeur entièrement décrite par un nombre et une unité.
    
    \textit{Exemples : la masse (\SI{500}{\kilogram}), la température (\SI{15}{\celsius}), l'énergie (\SI{200}{\joule}), le temps (\SI{3}{\hour}), la pression (\SI{101}{\kilo\pascal}).}
    
    \item Un \textbf{vecteur} est une grandeur qui possède à la fois un \textbf{module} (valeur numérique), une \textbf{direction} et un \textbf{sens}.
    
    \textit{Exemples : la vitesse, l'accélération, la force, le déplacement, le courant marin.}
\end{itemize}
\end{definition}

% -----------------------------------------------------------------------------
\subsection{Représentation d'un vecteur}
% -----------------------------------------------------------------------------

Graphiquement, un vecteur est représenté par une \textbf{flèche} dans le plan cartésien.

\begin{figure}[H]
\centering
\begin{tikzpicture}[scale=0.8]
    % Axes
    \draw[axe] (-0.5,0) -- (7,0) node[right] {$x$};
    \draw[axe] (0,-0.5) -- (0,7) node[above] {$y$};
    % Grille
    \draw[help lines, gray!30] (0,0) grid (6,6);
    % Vecteur
    \draw[vecteur, line width=1.5pt] (0,0) -- (5,6) node[above right] {$\vect{A}$};
    % Points
    \fill (0,0) circle (2pt) node[below left] {$O$};
    \fill (5,6) circle (2pt) node[above left] {$A$};
    % Composantes en pointillé
    \draw[pointille] (5,0) -- (5,6);
    \draw[pointille] (0,6) -- (5,6);
    % Angle theta
    \draw[thick, red] (1.2,0) arc (0:50.2:1.2);
    \node[red] at (1.8,0.6) {$\theta = 50{,}2^\circ$};
    % Labels composantes
    \node[below] at (2.5,0) {$A_x = 5$};
    \node[left] at (0,3) {$A_y = 6$};
    % Module
    \node[blue, rotate=50] at (2.0,3.0) {$\|\vect{A}\| = 7{,}81$};
\end{tikzpicture}
\caption{Un vecteur $\vect{A}$ dans le plan cartésien. Module : $\|\vect{A}\| = 7{,}81$, orientation : $\theta = 50{,}2^\circ$}
\label{fig:vecteur-A}
\end{figure}

Par convention, on désigne les quantités vectorielles par des lettres chapeautées d'une flèche ($\vect{A}$) pour les distinguer des quantités scalaires ($A$). Le vecteur $\vect{A}$ possède une \textbf{origine} ($O$) et une \textbf{extrémité} ($A$).

% -----------------------------------------------------------------------------
\subsection{Caractéristiques d'un vecteur}
% -----------------------------------------------------------------------------

Un vecteur possède trois caractéristiques qui permettent de le définir complètement d'un point de vue mathématique :

\begin{definition}
\begin{itemize}
    \item Le \textbf{module} $\|\vect{A}\|$ : la « longueur » du vecteur, toujours positive ou nulle. C'est la grandeur de la quantité physique représentée.
    
    \item L'\textbf{orientation} $\theta$ : l'angle que fait le vecteur par rapport à une direction de référence (généralement l'axe des $x$ positifs).
    
    \item Les \textbf{composantes} $A_x$ et $A_y$ : les projections du vecteur sur les axes $x$ et $y$. Elles peuvent être positives ou négatives selon le sens du vecteur.
\end{itemize}
\end{definition}

Ces caractéristiques ne sont pas indépendantes : connaître le module et l'orientation permet de calculer les composantes, et vice versa. Ce sont simplement \textbf{deux façons différentes de décrire le même objet mathématique}.

\begin{figure}[H]
\centering
\begin{tikzpicture}[scale=0.6]
    % Exemple 1 : Vent
    \begin{scope}[xshift=0cm]
        \draw[axe] (-0.5,0) -- (3,0) node[right] {\small $x$};
        \draw[axe] (0,-0.5) -- (0,2.5) node[above] {\small $y$};
        \draw[vecteur, line width=1.3pt] (0,0) -- (2.1,2.1);
        \draw[thick, blue] (0.6,0) arc (0:45:0.6);
        \node[blue, font=\footnotesize] at (1.0,0.3) {$45^\circ$};
        \node[below, font=\footnotesize, align=center] at (1.2,-1.2) {Vent\\ $30 \angle 45^\circ$ km/h};
    \end{scope}
    % Exemple 2 : Déplacement
    \begin{scope}[xshift=7cm]
        \draw[axe] (-2.5,0) -- (0.5,0) node[right] {\small $x$};
        \draw[axe] (0,-0.5) -- (0,1) node[above] {\small $y$};
        \draw[vecteur rouge, line width=1.3pt] (0,0) -- (-2,0);
        \node[below, font=\footnotesize, align=center] at (-1,-1.2) {Déplacement\\ $5 \angle 180^\circ$ m};
    \end{scope}
    % Exemple 3 : Force
    \begin{scope}[xshift=13cm]
        \draw[axe] (-0.5,0) -- (3,0) node[right] {\small $x$};
        \draw[axe] (0,-2.5) -- (0,0.5) node[above] {\small $y$};
        \draw[vecteur vert, line width=1.3pt] (0,0) -- (1.7,-1);
        \draw[thick, green!60!black] (0.6,0) arc (0:-30:0.6);
        \node[green!60!black, font=\footnotesize] at (1.0,-0.25) {$-30^\circ$};
        \node[below, font=\footnotesize, align=center] at (1.2,-2.8) {Force\\ $100 \angle -30^\circ$ N};
    \end{scope}
\end{tikzpicture}
\caption{Exemples de vecteurs exprimés en module et orientation.}
\end{figure}

% -----------------------------------------------------------------------------
\subsection{Notation module-orientation}
% -----------------------------------------------------------------------------

La première façon d'exprimer un vecteur est d'utiliser son \textbf{module} (sa longueur) et son \textbf{orientation} (l'angle $\theta$ par rapport à l'axe des $x$ positifs).

\begin{equationimportante}
\textbf{Notation module-orientation}
\[
    \vect{A} = \|\vect{A}\| \angle \theta
\]
où $\|\vect{A}\|$ est le module et $\theta$ est l'orientation.
\end{equationimportante}

Sur la Figure~\ref{fig:vecteur-A}, le vecteur $\vect{A}$ a un module de 7,81 et une orientation de 50,2$^\circ$. On écrit alors : $\vect{A} = 7{,}81 \angle 50{,}2^\circ$

Cette notation est particulièrement utile lorsqu'on connaît directement la grandeur et la direction d'une quantité physique (par exemple, un vent de 30 km/h venant du nord-est).

% -----------------------------------------------------------------------------
\subsection{Notation en composantes}
% -----------------------------------------------------------------------------

La deuxième façon d'exprimer un vecteur est d'utiliser ses \textbf{composantes}, c'est-à-dire ses projections sur les axes.

\begin{definition}
Les \textbf{composantes} $A_x$ et $A_y$ du vecteur $\vect{A}$ sont ses projections sur l'axe des $x$ et l'axe des $y$ respectivement. Elles correspondent aux longueurs des côtés du \textbf{triangle rectangle} formé par le vecteur et les axes (voir Figure~\ref{fig:vecteur-A}).
\end{definition}

\textbf{Le signe d'une composante indique sa direction par rapport à l'axe de référence :}

\begin{figure}[H]
\centering
\begin{tikzpicture}[scale=0.7]
    % Axe x avec signes
    \begin{scope}[yshift=1.5cm]
        \draw[axe] (-4,0) -- (4,0) node[right] {$x$};
        \draw[vecteur, line width=1.5pt] (0,0) -- (2.5,0);
        \node[above, blue] at (1.25,0.1) {$A_x > 0$};
        \draw[vecteur rouge, line width=1.5pt] (0,0) -- (-2.5,0);
        \node[above, red] at (-1.25,0.1) {$A_x < 0$};
    \end{scope}
    % Axe y avec signes
    \begin{scope}[xshift=7cm]
        \draw[axe] (0,-2.5) -- (0,2.5) node[above] {$y$};
        \draw[vecteur, line width=1.5pt] (0,0) -- (0,1.8);
        \node[right, blue] at (0.1,0.9) {$A_y > 0$};
        \draw[vecteur rouge, line width=1.5pt] (0,0) -- (0,-1.8);
        \node[right, red] at (0.1,-0.9) {$A_y < 0$};
    \end{scope}
\end{tikzpicture}
\caption{Le signe d'une composante indique le sens par rapport à l'axe.}
\end{figure}

\begin{equationimportante}
\textbf{Notation en composantes}
\[
    \vect{A} = (A_x,\, A_y)
\]
Pour un vecteur en trois dimensions : $\vect{A} = (A_x,\, A_y,\, A_z)$
\end{equationimportante}

\begin{exemple}{Écriture en composantes}{ecriture-composantes}
Le vecteur $\vect{A}$ de la Figure~\ref{fig:vecteur-A} s'écrit :
\[
    \vect{A} = (5,\, 6)
\]
\end{exemple}

\begin{remarque}[title=Pourquoi peut-on décomposer un vecteur ?]
\small
À première vue, décomposer un vecteur en composantes peut sembler étrange. Après tout, marcher en diagonale de A vers B n'est pas la même chose que marcher d'abord vers l'est, puis vers le nord. Dans le monde réel, le trajet est différent!

Pourtant, \textbf{mathématiquement, le résultat est identique} : vous arrivez au même point final. Le vecteur déplacement — la flèche qui relie votre point de départ à votre point d'arrivée — est le même, peu importe le chemin emprunté.

\textbf{Analogie :} Imaginez que vous êtes à Rimouski et que vous décrivez la position de Matane :
\begin{itemize}
    \item Si vous regardez vers le fleuve, Matane est « à votre gauche »
    \item Si vous regardez vers les terres, Matane est « à votre droite »
\end{itemize}
La \textbf{description} change selon votre référentiel, mais \textbf{Matane n'a pas bougé}! Sa position réelle dans l'espace est la même — seule la façon de l'exprimer diffère.

C'est exactement ce qui se passe avec les vecteurs : $(A_x, A_y)$ et $\|\vect{A}\| \angle \theta$ sont deux façons de décrire \textbf{le même objet mathématique}. La décomposition en composantes nous permet d'utiliser l'algèbre ordinaire (additions et soustractions de nombres) plutôt que la géométrie, ce qui simplifie énormément les calculs.

\textbf{Un peu d'histoire :} Cette idée révolutionnaire remonte à \textbf{René Descartes} (1596-1650), mathématicien et philosophe français qui a inventé le \textit{plan cartésien}. Grâce à lui, un vecteur qui « existe » comme une flèche dans l'espace peut être parfaitement décrit par deux nombres : ses composantes.
\end{remarque}

\subsubsection*{Deux représentations, un seul objet}

Il est crucial de comprendre que $(A_x,\, A_y)$ et $\|\vect{A}\| \angle \theta$ sont \textbf{deux façons d'écrire exactement le même vecteur}. Ce n'est pas une approximation, ce n'est pas une simplification : c'est \textbf{mathématiquement identique}.

C'est comme écrire ``une douzaine'' ou ``12'' — même quantité, notation différente.

\begin{equationimportante}
\textbf{Équivalence des représentations}
\[
    \vect{A} = (A_x,\, A_y) \quad \Longleftrightarrow \quad \vect{A} = \|\vect{A}\| \angle \theta
\]
Ces deux écritures décrivent \textbf{le même vecteur}.
\end{equationimportante}

\subsubsection*{Passage module/orientation $\leftrightarrow$ composantes}

En utilisant la trigonométrie (SOH-CAH-TOA), on peut passer d'une représentation à l'autre :

\begin{equationimportante}
\textbf{Du module et orientation vers les composantes :}
\begin{align*}
    A_x &= \|\vect{A}\| \cos\theta \\
    A_y &= \|\vect{A}\| \sin\theta
\end{align*}
où $\|\vect{A}\|$ est le module et $\theta$ est l'angle par rapport à l'axe des $x$ positifs.
\end{equationimportante}

\begin{equationimportante}
\textbf{Des composantes vers le module et l'orientation :}
\begin{align*}
    \|\vect{A}\| &= \sqrt{A_x^2 + A_y^2} \\[0.5em]
    \alpha &= \arctan\left(\frac{|A_y|}{|A_x|}\right) \quad \text{(angle de référence, entre $0^\circ$ et $90^\circ$)}
\end{align*}
On interprète ensuite la position de l'angle selon les signes des composantes :
\begin{itemize}
    \item Q1 : $\theta = \alpha$
    \item Q2 : $\theta = 180^\circ - \alpha$
    \item Q3 : $\theta = 180^\circ + \alpha$
    \item Q4 : $\theta = -\alpha$ \quad (ou $360^\circ - \alpha$)
\end{itemize}
\end{equationimportante}

\begin{attention}
En utilisant les \textbf{valeurs absolues} des composantes, l'angle de référence $\alpha$ est toujours entre $0^\circ$ et $90^\circ$. Il suffit ensuite de regarder les signes de $A_x$ et $A_y$ pour déterminer le quadrant et calculer $\theta$.
\end{attention}

\begin{figure}[H]
\centering
\begin{tikzpicture}[scale=1.1]
    % Axes
    \draw[axe] (-4,0) -- (4,0) node[right] {$x$};
    \draw[axe] (0,-4) -- (0,4) node[above] {$y$};
    
    % Quadrant I
    \draw[vecteur, line width=1.2pt] (0,0) -- (2,1.5);
    \draw[thick, blue] (0.6,0) arc (0:37:0.6);
    \node[blue, font=\footnotesize] at (1.0,0.3) {$37^\circ$};
    \node[align=left, font=\footnotesize] at (3.0,2.0) {Q1 : $\theta = 37^\circ$};
    
    % Quadrant II
    \draw[vecteur rouge, line width=1.2pt] (0,0) -- (-2,1.5);
    \draw[thick, red] (-0.6,0) arc (180:143:0.6);
    \node[red, font=\footnotesize] at (-1.0,0.3) {$37^\circ$};
    \node[align=right, font=\footnotesize] at (-3.0,2.0) {Q2 : $\theta = 180^\circ - 37^\circ$};
    \node[align=right, font=\footnotesize] at (-3.0,1.4) {$= 143^\circ$};
    
    % Quadrant III
    \draw[vecteur vert, line width=1.2pt] (0,0) -- (-2,-1.5);
    \draw[thick, green!60!black] (-0.6,0) arc (180:217:0.6);
    \node[green!60!black, font=\footnotesize] at (-1.0,-0.35) {$37^\circ$};
    \node[align=right, font=\footnotesize] at (-3.0,-1.4) {Q3 : $\theta = 180^\circ + 37^\circ$};
    \node[align=right, font=\footnotesize] at (-3.0,-2.0) {$= 217^\circ$};
    
    % Quadrant IV
    \draw[vecteur orange, line width=1.2pt] (0,0) -- (2,-1.5);
    \draw[thick, orange] (0.6,0) arc (0:-37:0.6);
    \node[orange, font=\footnotesize] at (1.0,-0.4) {$37^\circ$};
    \node[align=left, font=\footnotesize] at (3.0,-1.4) {Q4 : $\theta = -37^\circ$};
    \node[align=left, font=\footnotesize] at (3.0,-2.0) {ou $360^\circ - 37^\circ = 323^\circ$};
    
    % Labels quadrants
    \node[gray, font=\small] at (1.5,0.6) {Q1};
    \node[gray, font=\small] at (-1.5,0.6) {Q2};
    \node[gray, font=\small] at (-1.5,-0.6) {Q3};
    \node[gray, font=\small] at (1.5,-0.6) {Q4};
\end{tikzpicture}
\caption{Méthode de l'angle de référence. On calcule d'abord $\alpha = \arctan\left(\frac{|A_y|}{|A_x|}\right)$, puis on place l'angle dans le bon quadrant.}
\label{fig:quadrants-arctan}
\end{figure}

\begin{exemple}{Module et orientation d'un vecteur (Q1)}{module-orientation-q1}
Soit le vecteur $\vect{A} = (5,\, 6)$. Calculons son module et son orientation.

\begin{minipage}{0.55\textwidth}
\begin{align*}
    \|\vect{A}\| &= \sqrt{5^2 + 6^2} = \sqrt{25 + 36} = \sqrt{61} \approx 7{,}81 \\[0.5em]
    \theta &= \arctan\left(\frac{6}{5}\right) = \arctan(1{,}2) \approx 50{,}2^\circ
\end{align*}
On peut donc écrire : $\vect{A} = 7{,}81 \angle 50{,}2^\circ$
\end{minipage}
\hfill
\begin{minipage}{0.4\textwidth}
\centering
\begin{tikzpicture}[scale=0.5]
    \draw[axe] (-0.5,0) -- (6,0) node[right] {\small $x$};
    \draw[axe] (0,-0.5) -- (0,7) node[above] {\small $y$};
    \draw[vecteur, line width=1.3pt] (0,0) -- (5,6);
    \draw[pointille] (5,0) -- (5,6);
    \node[below] at (2.5,0) {\small $5$};
    \node[right] at (5,3) {\small $6$};
    \draw[thick, red] (1,0) arc (0:50:1);
    \node[red] at (1.5,0.5) {\small $\theta$};
\end{tikzpicture}
\end{minipage}
\end{exemple}

\begin{exemple}{Module et orientation d'un vecteur (Q2)}{module-orientation-q2}
Soit le vecteur $\vect{B} = (-4,\, 3)$. Calculons son module et son orientation.

\begin{minipage}{0.55\textwidth}
\begin{align*}
    \|\vect{B}\| &= \sqrt{(-4)^2 + 3^2} = \sqrt{25} = 5{,}00 \\[0.5em]
    \alpha &= \arctan\left(\frac{|3|}{|-4|}\right) = \arctan(0{,}75) = 36{,}9^\circ
\end{align*}
Le vecteur est en \textbf{Q2} car $B_x < 0$ et $B_y > 0$.

Donc : $\theta = 180^\circ - 36{,}9^\circ = 143^\circ$
\end{minipage}
\hfill
\begin{minipage}{0.4\textwidth}
\centering
\begin{tikzpicture}[scale=0.5]
    \draw[axe] (-5,0) -- (1,0) node[right] {\small $x$};
    \draw[axe] (0,-1) -- (0,4.5) node[above] {\small $y$};
    \draw[vecteur rouge, line width=1.3pt] (0,0) -- (-4,3);
    \draw[pointille] (-4,0) -- (-4,3);
    \node[below] at (-2,0) {\small $-4$};
    \node[left] at (-4,1.5) {\small $3$};
    % Angle de référence
    \draw[thick, orange] (-0.8,0) arc (180:143:0.8);
    \node[orange] at (-1.3,0.4) {\small $\alpha$};
    % Vrai angle
    \draw[thick, red, ->] (0.4,0) arc (0:143:0.4);
    \node[red] at (0.2,0.8) {\small $143^\circ$};
\end{tikzpicture}
\end{minipage}
\end{exemple}

\begin{exemple}{Décomposition en composantes}{decomposition-composantes}
Décomposer les vecteurs suivants en composantes :
\begin{itemize}
    \item $\vect{A} = 5{,}0 \angle 60^\circ$ (Q1)
    \item $\vect{B} = 4{,}0 \angle 135^\circ$ (Q2)
    \item $\vect{C} = 6{,}0 \angle 210^\circ$ (Q3)
    \item $\vect{D} = 3{,}0 \angle -45^\circ$ (Q4)
\end{itemize}

\begin{center}
\begin{tikzpicture}[scale=0.45]
    \draw[axe] (-6,0) -- (6,0) node[right] {$x$};
    \draw[axe] (0,-6) -- (0,6) node[above] {$y$};
    \draw[help lines, gray!20] (-5,-5) grid (5,5);
    
    % Q1 : 5 ∠ 60$^\circ$
    \draw[vecteur, line width=1.3pt] (0,0) -- (2.5,4.33);
    \node[blue] at (2.2,3.0) {\small $\vect{A}$};
    
    % Q2 : 4 ∠ 135$^\circ$
    \draw[vecteur rouge, line width=1.3pt] (0,0) -- (-2.83,2.83);
    \node[red] at (-2.5,1.8) {\small $\vect{B}$};
    
    % Q3 : 6 ∠ 210$^\circ$
    \draw[vecteur vert, line width=1.3pt] (0,0) -- (-5.2,-3);
    \node[green!60!black] at (-4,-1.8) {\small $\vect{C}$};
    
    % Q4 : 3 ∠ -45$^\circ$
    \draw[vecteur orange, line width=1.3pt] (0,0) -- (2.12,-2.12);
    \node[orange] at (2.5,-1.2) {\small $\vect{D}$};
\end{tikzpicture}
\end{center}

\textbf{Solution :}

{\small
\begin{tabular}{@{}ll@{\hspace{1em}}l@{}}
\textbf{Q1 :} $\vect{A} = 5{,}0 \angle 60^\circ$ & $A_x = 5{,}0\cos(60^\circ) = 2{,}50$ ; $A_y = 5{,}0\sin(60^\circ) = 4{,}33$ & $\vect{A} = (2{,}50,\; 4{,}33)$ \\[0.3em]
\textbf{Q2 :} $\vect{B} = 4{,}0 \angle 135^\circ$ & $B_x = 4{,}0\cos(135^\circ) = -2{,}83$ ; $B_y = 4{,}0\sin(135^\circ) = 2{,}83$ & $\vect{B} = (-2{,}83,\; 2{,}83)$ \\[0.3em]
\textbf{Q3 :} $\vect{C} = 6{,}0 \angle 210^\circ$ & $C_x = 6{,}0\cos(210^\circ) = -5{,}20$ ; $C_y = 6{,}0\sin(210^\circ) = -3{,}00$ & $\vect{C} = (-5{,}20,\; -3{,}00)$ \\[0.3em]
\textbf{Q4 :} $\vect{D} = 3{,}0 \angle -45^\circ$ & $D_x = 3{,}0\cos(-45^\circ) = 2{,}12$ ; $D_y = 3{,}0\sin(-45^\circ) = -2{,}12$ & $\vect{D} = (2{,}12,\; -2{,}12)$
\end{tabular}
}
\end{exemple}

% -----------------------------------------------------------------------------
\subsection*{Pratique autonome — Décomposition de vecteurs}
% -----------------------------------------------------------------------------

\begin{pratiqueautonome}
Décomposer le vecteur force $\vect{F} = 120 \angle 55^\circ$ N en ses composantes $F_x$ et $F_y$.

\espaceresolution[5cm]

\reponsepratique{$F_x = \SI{68,8}{\newton}$, $F_y = \SI{98,3}{\newton}$, donc $\vect{F} = (68{,}8,\; 98{,}3)$ N}
\end{pratiqueautonome}

\begin{pratiqueautonome}
Décomposer le vecteur vitesse $\vect{v} = 8{,}5 \angle 210^\circ$ m/s en ses composantes.

\textit{Attention : Ce vecteur se trouve dans le troisième quadrant. Quels seront les signes des composantes?}

\espaceresolution[5cm]

\reponsepratique{$v_x = \SI{-7,36}{\meter/\second}$, $v_y = \SI{-4,25}{\meter/\second}$, donc $\vect{v} = (-7{,}36,\; -4{,}25)$ m/s}
\end{pratiqueautonome}

\begin{pratiqueautonome}
Un courant marin a des composantes $C_x = \SI{-2,4}{kn}$ et $C_y = \SI{1,8}{kn}$.

Trouver le module $\|\vect{C}\|$ et l'orientation $\theta$ de ce courant.

\textit{Indice : Le vecteur est dans le deuxième quadrant (Q2).}

\espaceresolution[6cm]

\reponsepratique{$\|\vect{C}\| = \SI{3,0}{kn}$, $\theta = 143^\circ$}
\end{pratiqueautonome}


% -----------------------------------------------------------------------------
\subsection{Opérations sur les vecteurs}
% -----------------------------------------------------------------------------

Tout comme il est possible d'additionner, de soustraire, de multiplier et de diviser des scalaires entre eux, il existe des opérations similaires pour les vecteurs.

\begin{attention}
On ne peut additionner, soustraire et égaler des grandeurs physiques que si elles ont la même dimension. Elles doivent aussi être \textbf{de même nature}. On ne peut donc pas additionner un vecteur avec un scalaire.

Par exemple, l'équation $A = \vect{B}$ et la somme $A + \vect{B}$ n'ont pas de sens.
\end{attention}

\subsubsection*{Multiplication d'un vecteur par un scalaire}

La seule exception est la \textbf{multiplication d'un vecteur par un scalaire}. Multiplier un vecteur par un scalaire revient à modifier le module du vecteur.

\begin{equationimportante}
Si $\vect{A} = (A_x,\, A_y)$, alors :
\[
    k\vect{A} = (kA_x,\, kA_y)
\]
\end{equationimportante}

Si le nombre $k$ est négatif, le \textbf{sens du vecteur s'inverse}. L'opposé du vecteur $\vect{A}$, que l'on écrit $-\vect{A}$, est de même module que $\vect{A}$, mais de sens contraire.

\begin{figure}[H]
\centering
\begin{tikzpicture}[scale=0.55]
    % Axes
    \draw[axe] (-3,0) -- (8,0) node[right] {$x$};
    \draw[axe] (0,-4) -- (0,5) node[above] {$y$};
    % Vecteur A
    \draw[vecteur, line width=1.5pt] (0,0) -- (2,1.5) node[above] {$\vect{A}$};
    % Vecteur 2A (côte à côte)
    \draw[vecteur vert, line width=1.5pt] (3,0) -- (7,3) node[above right] {$2\vect{A}$};
    % Vecteur -A
    \draw[vecteur rouge, line width=1.5pt] (0,0) -- (-2,-1.5) node[below left] {$-\vect{A}$};
\end{tikzpicture}
\caption{Les vecteurs $\vect{A}$, $2\vect{A}$ et $-\vect{A}$. Le vecteur $2\vect{A}$ a le double du module de $\vect{A}$.}
\label{fig:multiplication-scalaire}
\end{figure}

\subsubsection*{Addition de vecteurs}

La \textbf{somme de deux vecteurs} $\vect{A}$ et $\vect{B}$ est un nouveau vecteur appelé \textbf{résultante} $\vect{R}$.

\begin{figure}[H]
\centering
\begin{tikzpicture}[scale=0.9]
    % Vecteur A
    \draw[vecteur, line width=1.5pt] (0,0) -- (3,0) node[midway, below] {$\vect{A}$};
    % Vecteur B (à l'extrémité de A)
    \draw[vecteur rouge, line width=1.5pt] (3,0) -- (4.5,2.5) node[midway, right] {$\vect{B}$};
    % Résultante R
    \draw[vecteur vert, line width=1.5pt] (0,0) -- (4.5,2.5) node[midway, above left] {$\vect{R} = \vect{A} + \vect{B}$};
    % Points
    \fill (0,0) circle (2pt);
    \fill (3,0) circle (2pt);
    \fill (4.5,2.5) circle (2pt);
\end{tikzpicture}
\caption{Addition de vecteurs : la résultante $\vect{R}$ joint l'origine du premier vecteur à l'extrémité du dernier.}
\label{fig:addition-vecteurs}
\end{figure}

\begin{attention}
Lorsque l'on additionne des vecteurs, \textbf{on ne peut pas simplement additionner leurs modules}. En général :
\[
    \|\vect{A} + \vect{B}\| \neq \|\vect{A}\| + \|\vect{B}\|
\]

\textbf{Contre-exemple :} Soit $\vect{A} = 5{,}0 \angle 30^\circ$ et $\vect{B} = 3{,}0 \angle 110^\circ$.

Si on additionnait naïvement les modules : $\|\vect{R}\|_{\text{FAUX}} = 5{,}0 + 3{,}0 = 8{,}0$

Or, le calcul correct (que nous verrons plus loin) donne : $\|\vect{R}\|_{\text{VRAI}} = 6{,}26$

L'erreur est de \textbf{28\%}! Cela montre que l'addition des modules ne fonctionne que si les vecteurs sont \textbf{parallèles et de même sens} — ce qui est rarement le cas en pratique.
\end{attention}

\paragraph{Méthode géométrique (triangle)}

Pour additionner deux vecteurs graphiquement :
\begin{enumerate}
    \item Tracer le premier vecteur $\vect{A}$
    \item Placer l'origine du second vecteur $\vect{B}$ à l'extrémité de $\vect{A}$
    \item La résultante $\vect{R}$ va de l'origine de $\vect{A}$ à l'extrémité de $\vect{B}$
\end{enumerate}

Les trois vecteurs forment un triangle. On peut utiliser la \textbf{trigonométrie} (SOH-CAH-TOA si triangle rectangle) ou les \textbf{lois des sinus et des cosinus} pour trouver le module et l'orientation de la résultante.

\begin{exemple}{Addition de vecteurs perpendiculaires}{addition-perpendiculaires}
Un bateau se déplace de \SI{3}{NM} vers l'est puis de \SI{4}{NM} vers le nord. Quel est son déplacement total?

\begin{center}
\begin{tikzpicture}[scale=0.7]
    % Vecteur A (Est)
    \draw[vecteur, line width=1.3pt] (0,0) -- (3,0) node[midway, below] {$\vect{A} = \SI{3}{NM}$};
    % Vecteur B (Nord)
    \draw[vecteur rouge, line width=1.3pt] (3,0) -- (3,4) node[midway, right] {$\vect{B} = \SI{4}{NM}$};
    % Résultante
    \draw[vecteur vert, line width=1.3pt] (0,0) -- (3,4) node[midway, above left] {$\vect{R}$};
    % Angle droit
    \draw (2.7,0) -- (2.7,0.3) -- (3,0.3);
    % Angle theta
    \draw[thick, red] (0.8,0) arc (0:53:0.8);
    \node[red] at (1.1,0.4) {$\theta$};
\end{tikzpicture}
\end{center}

Le triangle est rectangle, on utilise Pythagore :
\[
    \|\vect{R}\| = \sqrt{3^2 + 4^2} = \sqrt{25} = \SI{5}{NM}
\]
L'orientation : $\theta = \arctan(4/3) = 53{,}1^\circ$ nord de l'est.
\end{exemple}

\begin{remarque}
L'addition vectorielle est \textbf{commutative} : $\vect{A} + \vect{B} = \vect{B} + \vect{A}$
\end{remarque}

\begin{exemple}{Méthode géométrique avec la loi des cosinus}{methode-geometrique}

Additionner les vecteurs $\vect{A} = 5{,}0 \angle 30^\circ$ et $\vect{B} = 3{,}0 \angle 110^\circ$ en utilisant la méthode géométrique.

\begin{center}
\begin{tikzpicture}[scale=0.8]
    \draw[axe] (-2,0) -- (6,0) node[right] {\small $x$};
    \draw[axe] (0,-0.5) -- (0,6) node[above] {\small $y$};
    % Vecteur A
    \draw[vecteur, line width=1.3pt] (0,0) -- (4.33,2.5) node[midway, below right] {\small $\vect{A}$};
    \draw[thick, blue] (1,0) arc (0:30:1);
    \node[blue] at (1.4,0.3) {\small $30^\circ$};
    % Vecteur B (à l'extrémité de A)
    \draw[vecteur rouge, line width=1.3pt] (4.33,2.5) -- (3.30,5.32) node[midway, right] {\small $\vect{B}$};
    % Angle entre A et B (à l'extrémité de A)
    \draw[thick, orange] (3.83,2.5) arc (180:140:0.5);
    % Prolongement de A en pointillé pour montrer l'angle
    \draw[pointille] (4.33,2.5) -- (5.5,3.2);
    \draw[thick, orange] (4.83,2.79) arc (30:110:0.5);
    \node[orange] at (5.0,3.4) {\small $80^\circ$};
    % Résultante
    \draw[vecteur vert, line width=1.3pt] (0,0) -- (3.30,5.32) node[midway, left] {\small $\vect{R}$};
    % Angle C dans le triangle
    \node[orange] at (3.5,3.5) {\small $C$};
\end{tikzpicture}
\end{center}

\textbf{Étape 1 :} Identifier l'angle entre les vecteurs

L'angle entre $\vect{A}$ et $\vect{B}$ est : $110^\circ - 30^\circ = 80^\circ$

\textbf{Attention :} Dans le triangle formé, l'angle $C$ (entre $\vect{A}$ et $\vect{B}$) est l'angle \textit{intérieur} au triangle. Il est égal à :
\[
    C = 180^\circ - 80^\circ = 100^\circ
\]

\textbf{Étape 2 :} Appliquer la loi des cosinus pour trouver le module

Dans le triangle formé par $\vect{A}$, $\vect{B}$ et $\vect{R}$, on applique la loi des cosinus :
\[
    \|\vect{R}\|^2 = \|\vect{A}\|^2 + \|\vect{B}\|^2 - 2\|\vect{A}\|\|\vect{B}\|\cos(C)
\]
\[
    \|\vect{R}\|^2 = 5{,}0^2 + 3{,}0^2 - 2(5{,}0)(3{,}0)\cos(100^\circ)
\]
\[
    \|\vect{R}\|^2 = 25 + 9 - 30 \times (-0{,}174) = 34 + 5{,}21 = 39{,}2
\]
\[
    \|\vect{R}\| = \sqrt{39{,}2} = 6{,}26
\]

\textbf{Étape 3 :} Appliquer la loi des sinus pour trouver l'orientation

On cherche l'angle $\alpha$ entre $\vect{R}$ et $\vect{A}$ :
\[
    \frac{\sin\alpha}{\|\vect{B}\|} = \frac{\sin C}{\|\vect{R}\|} \quad \Rightarrow \quad \sin\alpha = \frac{\|\vect{B}\|\sin C}{\|\vect{R}\|} = \frac{3{,}0 \times \sin(100^\circ)}{6{,}26} = 0{,}472
\]
\[
    \alpha = \arcsin(0{,}472) = 28{,}2^\circ
\]

L'orientation de $\vect{R}$ par rapport à l'axe des $x$ est :
\[
    \theta = 30^\circ + 28{,}2^\circ = 58{,}2^\circ
\]

\textbf{Réponse :} $\vect{R} = 6{,}26 \angle 58{,}2^\circ$
\end{exemple}

\paragraph{Méthode des composantes (algébrique)}

Cette méthode est plus rapide et plus précise pour les calculs.

\begin{equationimportante}
\textbf{Addition par composantes}

Si $\vect{A} = (A_x,\, A_y)$ et $\vect{B} = (B_x,\, B_y)$, alors :
\[
    \vect{R} = \vect{A} + \vect{B} = (A_x + B_x,\, A_y + B_y)
\]
\end{equationimportante}

\begin{attention}
Il ne faut \textbf{jamais additionner des composantes en $x$ avec des composantes en $y$}. On additionne les $x$ avec les $x$ et les $y$ avec les $y$.
\end{attention}

\begin{exemple}{Addition par composantes}{addition-composantes-simple}
Soit $\vect{A} = (3,\, 4)$ et $\vect{B} = (-1,\, 2)$. Calculons $\vect{R} = \vect{A} + \vect{B}$.

\begin{minipage}{0.5\textwidth}
\begin{align*}
    R_x &= A_x + B_x = 3 + (-1) = 2 \\
    R_y &= A_y + B_y = 4 + 2 = 6
\end{align*}
Donc $\vect{R} = (2,\, 6)$.

Module : $\|\vect{R}\| = \sqrt{2^2 + 6^2} = \sqrt{40} \approx 6{,}32$

Orientation : $\theta = \arctan(6/2) \approx 71{,}6^\circ$
\end{minipage}
\hfill
\begin{minipage}{0.45\textwidth}
\centering
\begin{tikzpicture}[scale=0.55]
    \draw[axe] (-2,0) -- (4,0) node[right] {\small $x$};
    \draw[axe] (0,-0.5) -- (0,7) node[above] {\small $y$};
    \draw[help lines, gray!30] (-1,0) grid (3,6);
    % Vecteur A
    \draw[vecteur, line width=1.3pt] (0,0) -- (3,4) node[midway, right] {\small $\vect{A}$};
    % Vecteur B (à l'extrémité de A)
    \draw[vecteur rouge, line width=1.3pt] (3,4) -- (2,6) node[midway, right] {\small $\vect{B}$};
    % Résultante
    \draw[vecteur vert, line width=1.3pt] (0,0) -- (2,6) node[midway, left] {\small $\vect{R}$};
\end{tikzpicture}
\end{minipage}
\end{exemple}

\begin{exemple}{Méthode des composantes pour le même problème}{methode-composantes}

Reprenons les vecteurs $\vect{A} = 5{,}0 \angle 30^\circ$ et $\vect{B} = 3{,}0 \angle 110^\circ$ et résolvons avec la méthode des composantes.

\textbf{Étape 1 :} Décomposer en composantes
\begin{align*}
    \vect{A} &: \quad A_x = 5{,}0\cos(30^\circ) = 4{,}33 \quad A_y = 5{,}0\sin(30^\circ) = 2{,}50 \\
    \vect{B} &: \quad B_x = 3{,}0\cos(110^\circ) = -1{,}03 \quad B_y = 3{,}0\sin(110^\circ) = 2{,}82
\end{align*}

\textbf{Étape 2 :} Additionner les composantes
\begin{align*}
    R_x &= 4{,}33 + (-1{,}03) = 3{,}30 \\
    R_y &= 2{,}50 + 2{,}82 = 5{,}32
\end{align*}

\textbf{Étape 3 :} Calculer module et orientation
\begin{align*}
    \|\vect{R}\| &= \sqrt{3{,}30^2 + 5{,}32^2} = \sqrt{39{,}2} = 6{,}26 \\
    \theta &= \arctan\left(\frac{5{,}32}{3{,}30}\right) = 58{,}2^\circ
\end{align*}

\textbf{Réponse :} $\vect{R} = 6{,}26 \angle 58{,}2^\circ$ ou $\vect{R} = (3{,}30,\; 5{,}32)$

\textit{On retrouve exactement le même résultat qu'avec la méthode géométrique, ce qui confirme l'équivalence des deux approches.}
\end{exemple}

\subsubsection*{Soustraction de vecteurs}

La \textbf{soustraction de deux vecteurs} revient à additionner le vecteur opposé :

\begin{equationimportante}
\[
    \vect{A} - \vect{B} = \vect{A} + (-\vect{B})
\]
\end{equationimportante}

En pratique, on utilise la \textbf{méthode des composantes} : on soustrait les composantes correspondantes.

\begin{equationimportante}
\textbf{Soustraction par composantes}

Si $\vect{A} = (A_x,\, A_y)$ et $\vect{B} = (B_x,\, B_y)$, alors :
\[
    \vect{A} - \vect{B} = (A_x - B_x,\, A_y - B_y)
\]
\end{equationimportante}

\begin{exemple}{Soustraction de vecteurs}{soustraction-vecteurs}
Soit $\vect{A} = (5,\, 3)$ et $\vect{B} = (2,\, 7)$. Calculons $\vect{D} = \vect{A} - \vect{B}$.

\begin{minipage}{0.5\textwidth}
\begin{align*}
    D_x &= A_x - B_x = 5 - 2 = 3 \\
    D_y &= A_y - B_y = 3 - 7 = -4
\end{align*}
Donc $\vect{D} = (3,\, -4)$.

Module : $\|\vect{D}\| = \sqrt{3^2 + (-4)^2} = \sqrt{25} = 5{,}00$

Orientation : $\theta = \arctan(-4/3) = -53{,}1^\circ$ (Q4)
\end{minipage}
\hfill
\begin{minipage}{0.45\textwidth}
\centering
\begin{tikzpicture}[scale=0.5]
    \draw[axe] (-1,0) -- (6,0) node[right] {\small $x$};
    \draw[axe] (0,-5) -- (0,8) node[above] {\small $y$};
    \draw[help lines, gray!30] (0,-4) grid (5,7);
    % Vecteur A
    \draw[vecteur, line width=1.3pt] (0,0) -- (5,3) node[above] {\small $\vect{A}$};
    % Vecteur -B (à l'extrémité de A)
    \draw[vecteur rouge, line width=1.3pt] (5,3) -- (3,-4) node[right] {\small $-\vect{B}$};
    % Résultante
    \draw[vecteur vert, line width=1.3pt] (0,0) -- (3,-4) node[below right] {\small $\vect{D}$};
    % Vecteur B en pointillé pour référence
    \draw[pointille, line width=1pt] (0,0) -- (2,7) node[left] {\small $\vect{B}$};
\end{tikzpicture}
\end{minipage}
\end{exemple}

% -----------------------------------------------------------------------------
\subsection*{Pratique autonome — Addition de vecteurs}
% -----------------------------------------------------------------------------

\begin{pratiqueautonome}
Un remorqueur exerce une force $\vect{F}_1 = 25 \angle 40^\circ$ kN sur un navire. Un second remorqueur exerce une force $\vect{F}_2 = 18 \angle 140^\circ$ kN.

Trouver la force résultante $\vect{R} = \vect{F}_1 + \vect{F}_2$ en utilisant la \textbf{méthode des composantes}.

\textit{Procédure : (1) Décomposer chaque vecteur, (2) Additionner les composantes, (3) Calculer module et orientation.}

\espaceresolution[8cm]

\reponsepratique{$\vect{R} = 29{,}5 \angle 81{,}4^\circ$ kN \quad (ou $\vect{R} = (4{,}41,\; 29{,}2)$ kN)}
\end{pratiqueautonome}

\begin{pratiqueautonome}
Un navire navigue à une vitesse $\vect{v}_{\text{navire}} = 12 \angle 60^\circ$ nœuds par rapport à l'eau. Le courant marin est $\vect{v}_{\text{courant}} = 3{,}0 \angle 180^\circ$ nœuds (vers l'ouest).

Quelle est la vitesse résultante $\vect{v}_{\text{fond}}$ du navire par rapport au fond marin?

\espaceresolution[8cm]

\reponsepratique{$\vect{v}_{\text{fond}} = 11{,}4 \angle 66{,}2^\circ$ kn \quad (ou $\vect{v}_{\text{fond}} = (3{,}00,\; 10{,}4)$ kn)}
\end{pratiqueautonome}

\begin{pratiqueautonome}[title={$\vartriangleright$ Pratique autonome~\thepratique{} (Défi)}]
\stepcounter{pratique}
Trois forces agissent simultanément sur une bouée d'amarrage :
\begin{itemize}
    \item $\vect{F}_1 = (150,\; 0)$ N \quad (tension du câble vers l'est)
    \item $\vect{F}_2 = 100 \angle 120^\circ$ N \quad (force du courant)
    \item $\vect{F}_3 = 80 \angle 240^\circ$ N \quad (force du vent)
\end{itemize}

Calculer la force résultante $\vect{R} = \vect{F}_1 + \vect{F}_2 + \vect{F}_3$.

\espaceresolution[9cm]

\reponsepratique{$\vect{R} = 64{,}0 \angle 40{,}9^\circ$ N \quad (ou $\vect{R} = (48{,}4,\; 41{,}9)$ N)}
\end{pratiqueautonome}

% -----------------------------------------------------------------------------
\subsection{Vecteurs dans les problèmes en une dimension}
% -----------------------------------------------------------------------------

On sera souvent amené à résoudre des problèmes à une seule dimension. Dans de telles situations, la distinction entre scalaires et vecteurs sera moins importante à faire. En effet, dans des problèmes en une seule dimension, on ne peut se déplacer que dans deux directions : vers la gauche ou vers la droite. La direction d'une grandeur est simplement encodée dans son \textbf{signe} (+ ou $-$).

\begin{exemple}{Vecteur en une dimension}{vecteur-1d}
Dans un système dont l'axe des $x$ est l'unique dimension, une vitesse de $v = \SI{-5}{\meter/\second}$ est une vitesse dont le module est de \SI{5}{\meter/\second} et dont la direction est vers les $x$ négatifs.
\end{exemple}

\begin{remarque}
Bien qu'il soit important d'être toujours conscient de la \textbf{nature des grandeurs physiques} (scalaire ou vecteur), il ne sera pas toujours nécessaire d'écrire les vecteurs avec une flèche dans les problèmes en une dimension. La direction sera encodée dans le signe, mais les grandeurs ne perdront pas leur statut de vecteur.
\end{remarque}