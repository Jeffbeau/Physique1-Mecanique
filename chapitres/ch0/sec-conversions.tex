% =============================================================================
\section{Précision et chiffres significatifs}
\label{sec:CS}
% =============================================================================

Les grandeurs physiques utilisées dans les calculs ont été mesurées avec des appareils qui ont un niveau de précision donné.

\begin{definition}
Le \textbf{nombre de chiffres significatifs} (C.S.) correspond au nombre de chiffres dont on est certain, plus le premier chiffre incertain.
\end{definition}

\subsubsection*{Les règles du zéro}
\begin{enumerate}
    \item Placés \textbf{au début} d'une valeur, les zéros ne sont \textbf{jamais significatifs}.
    \item Placés \textbf{à la fin} d'une valeur \textbf{avec décimales}, ils sont \textbf{toujours significatifs}.
    \item Placés \textbf{à la fin} d'une valeur \textbf{sans décimale}, ils \textbf{peuvent être significatifs ou non}.
\end{enumerate}
La notation scientifique permet de lever toute ambiguïté sur le nombre de C.S.

\subsubsection*{Règle pour ce cours}

\begin{equationimportante}
\textbf{Dans ce cours, toutes les réponses doivent être exprimées avec 3 chiffres significatifs}, en utilisant le \textbf{préfixe SI approprié} ou la \textbf{notation scientifique}.

\textbf{Important :} On n'écrit jamais plus de \textbf{deux zéros après la virgule}. Si une valeur est très petite, il faut utiliser un préfixe SI ou la notation scientifique.
\end{equationimportante}

\begin{exemple}{Arrondir à 3 chiffres significatifs}{arrondir-3cs}
\begin{enumerate}[label=\alph*)]
    \item $127{,}845$ m $\rightarrow$ \textbf{128 m}
    \item $0{,}004\,562\,3$ kg $\rightarrow$ $0{,}00456$ kg $\rightarrow$ \textbf{4,56 g} \quad (ou $4{,}56 \times 10^{-3}$ kg)
    \item $45\,230\,000$ W $\rightarrow$ $4{,}52 \times 10^7$ W $\rightarrow$ \textbf{45,2 MW}
\end{enumerate}
\end{exemple}

\begin{exemple}{Calcul complet avec grands nombres}{calcul-grands-nombres}

Un cargo de masse $m = \SI{52000}{t}$ accélère à $a = \SI{0,15}{\meter/\second^2}$. Calculer la force. Réponse en notation scientifique avec 3 C.S.

\textbf{Solution :}

\textit{Conversion :} $m = \SI{52000}{t} = 52\,000 \times 1000$ kg $= 5{,}2 \times 10^7$ kg

\textit{Calcul :}
\[
    F = ma = 5{,}2 \times 10^7 \text{ kg} \times 0{,}15 \text{ m/s}^2 = 7\,800\,000 \text{ N}
\]

\textit{Réponse (3 C.S.) :}
\[
    \boxed{F = 7{,}80 \times 10^6 \text{ N} = \SI{7,80}{\mega\newton}}
\]
\end{exemple}

% -----------------------------------------------------------------------------
\subsection*{Pratique autonome — Chiffres significatifs}
% -----------------------------------------------------------------------------

\begin{pratiqueautonome}
Combien de chiffres significatifs contiennent les valeurs suivantes?

\begin{enumerate}[label=\alph*)]
    \item $\SI{0,00340}{\kilogram}$
    \item $2{,}50 \times 10^4$ W
    \item $\SI{100,0}{\meter}$
\end{enumerate}

\espaceresolution[4cm]

\reponsepratique{a) 3 C.S. \quad b) 3 C.S. \quad c) 4 C.S.}
\end{pratiqueautonome}

\begin{pratiqueautonome}
Arrondir les valeurs suivantes à \textbf{3 chiffres significatifs}. Utiliser la notation scientifique ou les préfixes SI si nécessaire.

\begin{enumerate}[label=\alph*)]
    \item $\SI{45678}{\meter}$
    \item $\SI{0,0098765}{\second}$
    \item $\SI{123,45}{\kilo\newton}$
\end{enumerate}

\espaceresolution[5cm]

\reponsepratique{a) \SI{45,7}{\kilo\meter} \quad b) \SI{9,88}{\milli\second} \quad c) \SI{123}{\kilo\newton}}
\end{pratiqueautonome}

\begin{pratiqueautonome}
Un pétrolier de masse $m = \SI{85000}{t}$ navigue à une vitesse de $v = \SI{12,5}{kn}$.

Calculer son énergie cinétique $E_c = \frac{1}{2}mv^2$ en joules, avec les bons chiffres significatifs et une notation appropriée (préfixe SI ou notation scientifique).

\textit{Rappel : \SI{1}{kn} = \SI{0,5144}{\meter/\second}}

\espaceresolution[7cm]

\reponsepratique{$E_c = \SI{1,76}{\giga\joule}$ (ou $1{,}76 \times 10^9$ J)}
\end{pratiqueautonome}

% =============================================================================
\section{Trigonométrie}
\label{sec:trigo}
% =============================================================================

La trigonométrie est un outil mathématique \textbf{essentiel} en physique. Dans ce cours, vous l'utiliserez constamment pour :

\begin{itemize}
    \item \textbf{Décomposer les vecteurs} en leurs composantes horizontale et verticale — c'est la base du calcul vectoriel que nous verrons à la section suivante
    \item \textbf{Retrouver l'angle} d'un vecteur à partir de ses composantes
    \item \textbf{Résoudre des triangles} dans les problèmes de navigation, d'équilibre des forces, de trajectoires, etc.
\end{itemize}

Sans une maîtrise solide de la trigonométrie, il sera très difficile de progresser dans ce cours. Prenez le temps de revoir ces notions si elles ne sont pas fraîches dans votre mémoire.

% -----------------------------------------------------------------------------
\subsection{Le triangle quelconque}
% -----------------------------------------------------------------------------

Un triangle est un polygone possédant trois côtés et trois sommets. Dans un triangle, \textbf{la somme des angles internes totalise toujours 180°}.

\begin{figure}[H]
\centering
\begin{tikzpicture}[scale=1.5]
    % Triangle quelconque
    \coordinate (A) at (0,0);
    \coordinate (B) at (5,0);
    \coordinate (C) at (1.5,2.5);
    
    % Côtés avec labels
    \draw[thick] (A) -- (B) node[midway, below] {$c$};
    \draw[thick] (B) -- (C) node[midway, above right] {$a$};
    \draw[thick] (C) -- (A) node[midway, above left] {$b$};
    
    % Angle A (en bas à gauche)
    \pic[draw, thick, blue, angle radius=8mm, angle eccentricity=1.4, "$A$" font=\small] {angle = B--A--C};
    
    % Angle B (en bas à droite)
    \pic[draw, thick, red, angle radius=8mm, angle eccentricity=1.4, "$B$" font=\small] {angle = C--B--A};
    
    % Angle C (en haut)
    \pic[draw, thick, green!60!black, angle radius=6mm, angle eccentricity=1.5, "$C$" font=\small] {angle = A--C--B};
    
    % Sommets
    \fill (A) circle (1.5pt);
    \fill (B) circle (1.5pt);
    \fill (C) circle (1.5pt);
\end{tikzpicture}
\caption{Triangle quelconque : le côté $a$ est opposé à l'angle $A$, le côté $b$ est opposé à l'angle $B$, le côté $c$ est opposé à l'angle $C$.}
\label{fig:triangle-quelconque}
\end{figure}

Il est possible de définir toutes les propriétés d'un triangle quelconque à partir de trois informations :
\begin{enumerate}
    \item Deux côtés et l'angle entre eux
    \item Un côté et deux angles
    \item Trois côtés
\end{enumerate}

Les \textbf{lois des sinus} et \textbf{des cosinus} permettent de résoudre n'importe quel triangle quelconque. Ces notions, vues au secondaire, sont essentielles pour les problèmes de navigation et d'analyse de forces que vous rencontrerez dans ce cours.

\begin{equationimportante}
\textbf{Loi des sinus}
\[
    \frac{a}{\sin A} = \frac{b}{\sin B} = \frac{c}{\sin C}
\]
\end{equationimportante}

\begin{equationimportante}
\textbf{Loi des cosinus}
\[
    c^2 = a^2 + b^2 - 2ab\cos C
\]
(et les deux autres formes obtenues par permutation des lettres)
\end{equationimportante}

% -----------------------------------------------------------------------------
\subsection{Le triangle rectangle}
% -----------------------------------------------------------------------------

Le triangle rectangle possède un angle droit (90°). Cette propriété particulière permet d'utiliser des outils plus simples et plus rapides : le \textbf{théorème de Pythagore} et les \textbf{rapports trigonométriques}.

\begin{attention}
Les rapports trigonométriques (sin, cos, tan) sont \textbf{la clé} pour décomposer les vecteurs et retrouver leurs angles. Vous devez être capable de les utiliser rapidement et sans erreur.
\end{attention}

\begin{figure}[H]
\centering
\begin{tikzpicture}[scale=1.3]
    % Triangle
    \coordinate (A) at (0,0);
    \coordinate (B) at (4.5,0);
    \coordinate (C) at (4.5,2.8);
    
    % Côtés avec labels
    \draw[thick] (A) -- (B) node[midway, below] {\textbf{Adjacent} à $\theta$};
    \draw[thick] (B) -- (C) node[midway, right] {\textbf{Opposé} à $\theta$};
    \draw[thick] (C) -- (A) node[midway, above left] {\textbf{Hypoténuse}};
    
    % Angle droit
    \draw[thick] (4.2,0) -- (4.2,0.3) -- (4.5,0.3);
    
    % Angle theta
    \pic[draw, thick, blue, angle radius=10mm, angle eccentricity=1.3, "$\theta$" font=\large] {angle = B--A--C};
    
    % Sommets
    \fill (A) circle (2pt);
    \fill (B) circle (2pt);
    \fill (C) circle (2pt);
\end{tikzpicture}
\caption{Triangle rectangle : les côtés sont nommés \textbf{par rapport à l'angle} $\theta$ considéré.}
\label{fig:triangle-rectangle}
\end{figure}

Le truc mnémotechnique \textbf{SOH-CAH-TOA} permet de retenir les trois rapports trigonométriques :

\begin{equationimportante}
\textbf{SOH-CAH-TOA}
\begin{align*}
    \sin\theta &= \frac{\text{Opposé}}{\text{Hypoténuse}} & \text{(SOH)} \\[0.5em]
    \cos\theta &= \frac{\text{Adjacent}}{\text{Hypoténuse}} & \text{(CAH)} \\[0.5em]
    \tan\theta &= \frac{\text{Opposé}}{\text{Adjacent}} & \text{(TOA)}
\end{align*}
\end{equationimportante}

\begin{equationimportante}
\textbf{Théorème de Pythagore}
\[
    \text{Hypoténuse}^2 = \text{Opposé}^2 + \text{Adjacent}^2
\]
\end{equationimportante}

Il est possible de retrouver toutes les propriétés d'un triangle rectangle à partir de \textbf{seulement deux informations} :
\begin{enumerate}
    \item Un angle (autre que l'angle droit) et un côté
    \item Deux côtés
\end{enumerate}