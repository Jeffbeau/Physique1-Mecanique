% =============================================================================
\chapter{Notions fondamentales}
\label{chap:notions}
% =============================================================================

% =============================================================================
\section{Système international d'unités}
\label{sec:SI}
% =============================================================================

Le système international d'unités (SI) est le système d'unités le plus largement utilisé dans le monde. Officiellement, seulement trois pays ne l'ont pas adopté : les États-Unis, le Liberia et la Birmanie (Myanmar). Le système international est issu du système métrique, lequel est apparu pendant la Révolution française. Le français demeure à ce jour la langue officielle du système international, ce qui se dénote notamment par l'acronyme « SI », utilisé dans la plupart des langues.

La force du système international est qu'il couvre l'ensemble des domaines tout en étant simple à comprendre et à utiliser, grâce à sa structure qui tire profit de la numération décimale et qui est construite autour de sept unités de base.

% -----------------------------------------------------------------------------
\subsection{Pourquoi un système d'unités universel?}
% -----------------------------------------------------------------------------

L'histoire regorge d'exemples où l'absence d'un référent commun a mené à des erreurs coûteuses. Avant la Révolution française, chaque région avait ses propres unités : la « toise » de Paris ne correspondait pas à celle de Bordeaux, et le « pied » variait d'un pays à l'autre. Cette confusion rendait le commerce difficile et les calculs scientifiques hasardeux.

\begin{attention}
\textbf{La catastrophe du Mars Climate Orbiter (1999)}

Le 23 septembre 1999, la sonde spatiale \textit{Mars Climate Orbiter} de la NASA s'est désintégrée dans l'atmosphère de Mars. Coût de la mission : 125 millions de dollars américains.

La cause? Une erreur de conversion d'unités. L'équipe de Lockheed Martin, qui avait construit la sonde, transmettait les données de poussée des moteurs en \textbf{livre-force par seconde} (lbf$\cdot$s), une unité du système impérial. L'équipe de navigation de la NASA, elle, attendait ces données en \textbf{newton-seconde} (N$\cdot$s), l'unité du système international.

Personne n'a remarqué l'incompatibilité. Pendant des mois, les corrections de trajectoire ont été calculées avec les mauvaises unités. Résultat : au lieu de passer à \SI{150}{\kilo\meter} d'altitude pour se mettre en orbite, la sonde est descendue à \SI{57}{\kilo\meter} et s'est consumée dans l'atmosphère martienne.

Cette catastrophe illustre parfaitement pourquoi, en sciences et en ingénierie, \textbf{l'utilisation rigoureuse du système international est non négociable}.
\end{attention}

% -----------------------------------------------------------------------------
\subsection{Grandeur, valeur et unité}
% -----------------------------------------------------------------------------

Commençons par distinguer trois termes importants :

\begin{definition}
\begin{itemize}
    \item \textbf{Grandeur} : une propriété physique qui peut être mesurée ou calculée \textit{(ex.: la longueur, la force, la masse, l'énergie, l'angle)}.
    
    \item \textbf{Valeur} : le nombre obtenu pour quantifier la grandeur mesurée ou calculée.
    
    \item \textbf{Unité} : accompagne la valeur afin d'expliciter le système de mesures utilisé et la nature de la grandeur.
\end{itemize}
\end{definition}

Il est important de toujours accompagner une valeur de son unité lorsque l'on exprime une grandeur, sans quoi elle perd tout son sens. Les grandeurs ($l$, $m$, $g$) sont en italique, alors que les unités associées ne le sont pas. Cela permet d'éviter toute confusion.

% -----------------------------------------------------------------------------
\subsection{Unités de base du SI}
% -----------------------------------------------------------------------------

Toutes les unités au sein du SI sont définies à partir de sept \textbf{unités de base}, elles-mêmes définies à partir de constantes fondamentales de la nature depuis la révision de 2019 par le Bureau international des poids et mesures (BIPM).

\begin{table}[H]
\centering
\begin{tabular}{@{}llll@{}}
\toprule
\textbf{Grandeur} & \textbf{Unité} & \textbf{Symbole} & \textbf{Définition basée sur} \\
\midrule
Longueur & mètre & m & vitesse de la lumière ($c = \SI{299792458}{m/s}$) \\
Masse & kilogramme & kg & constante de Planck ($h$) \\
Temps & seconde & s & fréquence de l'atome de césium 133 \\
Courant électrique & ampère & A & charge élémentaire ($e$) \\
Température & kelvin & K & constante de Boltzmann ($k$) \\
Quantité de matière & mole & mol & nombre d'Avogadro ($N_A = 6{,}02 \times 10^{23}$) \\
Intensité lumineuse & candela & cd & efficacité lumineuse \\
\bottomrule
\end{tabular}
\caption{Les sept unités de base du SI}
\end{table}

\begin{remarque}
L'unité de base de la masse est le kilogramme et non le gramme ; c'est la seule unité du SI à posséder un préfixe dans sa version de base. Dans ce cours, nous utiliserons principalement le \textbf{mètre}, le \textbf{kilogramme} et la \textbf{seconde}.
\end{remarque}

Les équations utilisées en physique sont basées sur le SI. Il est donc toujours préférable de convertir les unités dans le système international avant de les insérer dans une équation.

% -----------------------------------------------------------------------------
\subsection{La notation scientifique}
% -----------------------------------------------------------------------------

En sciences, on travaille souvent avec des nombres extrêmement grands ou extrêmement petits. Par exemple, la distance Terre-Soleil est d'environ \SI{150000000000}{\meter}, tandis que le diamètre d'un atome est d'environ \SI{0,0000000001}{\meter}. Écrire ces nombres en notation décimale est peu pratique et source d'erreurs.

\begin{definition}
La \textbf{notation scientifique} exprime un nombre sous la forme :
\[
    a \times 10^n
\]
où $a$ est la \textbf{mantisse} (un nombre entre 1 et 10) et $n$ est l'\textbf{exposant} (un entier positif ou négatif).

\textbf{Exemples :}
\begin{itemize}
    \item Distance Terre-Soleil : $\SI{150000000000}{\meter} = 1{,}50 \times 10^{11}$ m
    \item Masse d'un électron : $9{,}11 \times 10^{-31}$ kg
    \item Nombre d'Avogadro : $N_A = 6{,}02 \times 10^{23}$
\end{itemize}
\end{definition}

% -----------------------------------------------------------------------------
\subsection{Les préfixes SI}
% -----------------------------------------------------------------------------

Le système de préfixes permet de simplifier l'écriture des nombres extrêmement petits ou grands. Chaque préfixe est associé à une puissance de dix. \textbf{L'utilisation des préfixes SI est obligatoire dans ce cours.}

\begin{table}[H]
\centering
\small
\begin{tabular}{@{}lcclcc@{}}
\toprule
\textbf{Préfixe} & \textbf{Symbole} & \textbf{Facteur} & \textbf{Préfixe} & \textbf{Symbole} & \textbf{Facteur} \\
\midrule
téra  & T & $10^{12}$  & milli & m & $10^{-3}$ \\
giga  & G & $10^{9}$   & micro & µ & $10^{-6}$ \\
méga  & M & $10^{6}$   & nano  & n & $10^{-9}$ \\
kilo  & k & $10^{3}$   & pico  & p & $10^{-12}$ \\
hecto & h & $10^{2}$   & centi & c & $10^{-2}$ \\
déca  & da & $10^{1}$  & déci  & d & $10^{-1}$ \\
\bottomrule
\end{tabular}
\caption{Préfixes des multiples et sous-multiples décimaux du SI}
\label{tab:prefixes}
\end{table}

\begin{exemple}
\textbf{Passage entre notation scientifique et préfixes SI}

\begin{enumerate}
    \item \textbf{De la notation scientifique vers le préfixe :}
    
    $d = 3{,}50 \times 10^{6}$ m $\rightarrow$ Exposant 6 = méga $\rightarrow$ $d = \SI{3,50}{\mega\meter}$ (ou \SI{3500}{\kilo\meter})
    
    \item \textbf{Du préfixe vers la notation scientifique :}
    
    $m = \SI{45,0}{\micro\gram}$ $\rightarrow$ micro = $10^{-6}$ $\rightarrow$ $m = 45{,}0 \times 10^{-6}$ g $= 4{,}50 \times 10^{-5}$ g
    
    \item \textbf{Conversion entre préfixes :}
    
    $P = \SI{2,35}{\giga\watt}$ en kilowatts?
    
    G = $10^9$ et k = $10^3$, donc G = $10^6 \times$ k
    
    $P = 2{,}35 \times 10^6$ kW $= \SI{2,35e6}{\kilo\watt}$
\end{enumerate}
\end{exemple}

% -----------------------------------------------------------------------------
\subsection{Équivalences utiles}
% -----------------------------------------------------------------------------

\begin{table}[H]
\centering
\small
\begin{tabular}{@{}llcl@{}}
\toprule
\textbf{Grandeur} & \textbf{Unité} & \textbf{Symbole} & \textbf{Équivalence} \\
\midrule
\multirow{4}{*}{Distance} 
    & mètre        & m   & -- \\
    & pouce        & po  & 1 po = 0,0254 m \\
    & pied         & pi  & 1 pi = 0,3048 m \\
    & mille marin  & NM  & 1 NM = 1852 m \\
\midrule
\multirow{2}{*}{Vitesse}
    & mètre/seconde & m/s & -- \\
    & nœud         & kn  & 1 kn = 1 NM/h = 0,5144 m/s \\
\midrule
\multirow{2}{*}{Masse}
    & kilogramme   & kg  & -- \\
    & livre-masse  & lb  & 1 lb $\approx$ 0,4536 kg \\
\midrule
\multirow{2}{*}{Force}
    & newton       & N   & 1 N = 1 kg$\cdot$m/s$^2$ \\
    & livre-force  & lbf & 1 lbf $\approx$ 4,448 N \\
\midrule
\multirow{2}{*}{Pression}
    & pascal       & Pa  & 1 Pa = 1 N/m$^2$ \\
    & psi          & psi & 1 psi $\approx$ 6895 Pa \\
\bottomrule
\end{tabular}
\caption{Équivalences courantes entre unités SI et autres systèmes}
\label{tab:equivalences}
\end{table}

% -----------------------------------------------------------------------------
\subsection{Unités dérivées du SI}
% -----------------------------------------------------------------------------

\begin{table}[H]
\centering
\small
\begin{tabular}{@{}llll@{}}
\toprule
\textbf{Unité} & \textbf{Symbole} & \textbf{Grandeur} & \textbf{Équivalence} \\
\midrule
radian      & rad  & angle plan              & m/m \\
hertz       & Hz   & fréquence               & s$^{-1}$ \\
newton      & N    & force                   & kg$\cdot$m/s$^2$ \\
pascal      & Pa   & pression                & N/m$^2$ \\
joule       & J    & énergie, travail        & N$\cdot$m \\
watt        & W    & puissance               & J/s \\
coulomb     & C    & charge électrique       & s$\cdot$A \\
volt        & V    & tension                 & W/A \\
ohm         & $\Omega$ & résistance électrique & V/A \\
\bottomrule
\end{tabular}
\caption{Unités dérivées du SI utilisées dans le cours}
\end{table}

% =============================================================================
\section{Conversions d'unités}
\label{sec:conversions}
% =============================================================================

Dans votre formation, vous aurez souvent à convertir des unités :
\begin{itemize}
    \item milles nautiques en kilomètres
    \item nœuds en mètres par seconde
    \item tours par minute (RPM) en radians par seconde
    \item livres par pouce carré en pascals
    \item pieds cubes en mètres cubes
\end{itemize}

Les équations que nous verrons sont basées sur le SI et il faudra convertir toute grandeur dans ce système avant de l'insérer dans une équation.

% -----------------------------------------------------------------------------
\subsection{Les facteurs de conversion}
% -----------------------------------------------------------------------------

Afin de convertir les unités d'une grandeur, il faut multiplier par le \textbf{facteur de conversion}. Les facteurs de conversion sont des quantités égales à 1, obtenues à partir des équivalences entre unités (voir Tableau~\ref{tab:equivalences}).

\begin{equationimportante}
\textbf{Procédure de conversion d'unités}
\begin{enumerate}
    \item \textbf{Écrire} la grandeur avec sa valeur et son unité de départ
    \item \textbf{Trouver} le(s) facteur(s) de conversion nécessaire(s) dans le Tableau~\ref{tab:equivalences}
    \item \textbf{Placer} les facteurs de sorte que les unités à éliminer s'annulent (unité à éliminer en haut $\rightarrow$ la mettre en bas, et vice versa)
    \item \textbf{Calculer} en multipliant les valeurs numériques
\end{enumerate}
\end{equationimportante}

\begin{exemple}
La distance entre Rimouski et Sept-Îles est de \SI{150}{NM}. Convertir en kilomètres.

\textbf{Étape 1 :} Écrire la grandeur
\[
    d = \SI{150}{NM}
\]

\textbf{Étape 2 :} Trouver le facteur de conversion (Tableau~\ref{tab:equivalences})
\begin{itemize}
    \item $\SI{1}{NM} = \SI{1852}{\meter} = \SI{1,852}{\kilo\meter}$
\end{itemize}

\textbf{Étape 3 :} Placer le facteur (NM en haut $\rightarrow$ NM en bas)
\[
    d = \SI{150}{NM} \cdot \frac{\SI{1,852}{\kilo\meter}}{\SI{1}{NM}}
\]

\textbf{Étape 4 :} Calculer
\[
    d = 150 \times 1{,}852\,\si{\kilo\meter} = \SI{278}{\kilo\meter}
\]
\end{exemple}

% -----------------------------------------------------------------------------
\subsection{Conversion d'unités composées}
% -----------------------------------------------------------------------------

Lorsque l'unité à convertir est composée de plusieurs unités (comme une vitesse), il faut convertir \textbf{chaque unité séparément}.

\begin{exemple}
Convertir une vitesse de $v = \SI{90}{\kilo\meter/\hour}$ en \si{\meter/\second}.

\textbf{Étape 1 :} Écrire la grandeur
\[
    v = \SI{90}{\kilo\meter/\hour} = 90 \cdot \frac{\SI{1}{\kilo\meter}}{\SI{1}{\hour}}
\]

\textbf{Étape 2 :} Trouver les facteurs de conversion
\begin{itemize}
    \item $\SI{1}{\kilo\meter} = \SI{1000}{\meter}$
    \item $\SI{1}{\hour} = \SI{3600}{\second}$
\end{itemize}

\textbf{Étape 3 :} Placer les facteurs (km en haut $\rightarrow$ km en bas ; h en bas $\rightarrow$ h en haut)
\[
    v = 90 \cdot \frac{\SI{1}{\kilo\meter}}{\SI{1}{\hour}} \cdot \frac{\SI{1000}{\meter}}{\SI{1}{\kilo\meter}} \cdot \frac{\SI{1}{\hour}}{\SI{3600}{\second}}
\]

\textbf{Étape 4 :} Calculer
\[
    v = 90 \cdot \frac{1000}{3600}\,\si{\meter/\second} = \SI{25,0}{\meter/\second}
\]
\end{exemple}

\begin{exemple}
Convertir une vitesse de \SI{25}{kn} en \si{\kilo\meter/\hour}.

\textbf{Étape 1 :} Écrire la grandeur (rappel : $\SI{1}{kn} = \SI{1}{NM/h}$)
\[
    v = \SI{25}{kn} = 25 \cdot \frac{\SI{1}{NM}}{\SI{1}{\hour}}
\]

\textbf{Étape 2 :} Trouver le facteur de conversion (Tableau~\ref{tab:equivalences})
\begin{itemize}
    \item $\SI{1}{NM} = \SI{1,852}{\kilo\meter}$
\end{itemize}

\textbf{Étape 3 :} Placer le facteur
\[
    v = 25 \cdot \frac{\SI{1}{NM}}{\SI{1}{\hour}} \cdot \frac{\SI{1,852}{\kilo\meter}}{\SI{1}{NM}}
\]

\textbf{Étape 4 :} Calculer
\[
    v = 25 \times 1{,}852\,\si{\kilo\meter/\hour} = \SI{46,3}{\kilo\meter/\hour}
\]
\end{exemple}

\begin{exemple}
Convertir une pression de $p = \SI{32,0}{psi}$ en kilopascals (\si{\kilo\pascal}).

\textbf{Étape 1 :} Écrire la grandeur (rappel : $\SI{1}{psi} = \SI{1}{lbf/po^2}$)
\[
    p = \SI{32,0}{psi} = 32{,}0 \cdot \frac{\SI{1}{lbf}}{\SI{1}{po^2}}
\]

\textbf{Étape 2 :} Trouver les facteurs de conversion (Tableau~\ref{tab:equivalences})
\begin{itemize}
    \item $\SI{1}{lbf} = \SI{4,448}{\newton}$
    \item $\SI{1}{po} = \SI{0,0254}{\meter}$, donc $\SI{1}{po^2} = (\SI{0,0254}{\meter})^2$
\end{itemize}

\textbf{Étape 3 :} Placer les facteurs
\[
    p = 32{,}0 \cdot \frac{\SI{1}{lbf}}{\SI{1}{po^2}} \cdot \frac{\SI{4,448}{\newton}}{\SI{1}{lbf}} \cdot \frac{\SI{1}{po^2}}{(\SI{0,0254}{\meter})^2}
\]

\textbf{Étape 4 :} Calculer
\[
    p = 32{,}0 \cdot \frac{4{,}448}{0{,}0254^2}\,\si{\newton/\meter^2} = \SI{221}{\kilo\pascal}
\]
\end{exemple}

% -----------------------------------------------------------------------------
\subsection{Unités avec exposant}
% -----------------------------------------------------------------------------

Lorsqu'une unité porte un exposant (aires, volumes), il faut se rappeler que cet exposant signifie que l'unité est \textbf{multipliée par elle-même}.

\begin{equationimportante}
\[
    \SI{1}{\meter^2} = \SI{1}{\meter} \times \SI{1}{\meter}
    \qquad\text{et}\qquad
    \SI{1}{\meter^3} = \SI{1}{\meter} \times \SI{1}{\meter} \times \SI{1}{\meter}
\]
Par conséquent, le facteur de conversion doit aussi être élevé à la même puissance !
\end{equationimportante}

\begin{exemple}
L'aire d'une écoutille est de $A = \SI{1500}{\centi\meter^2}$. Convertir en \si{\meter^2}.

Puisque $\si{\centi\meter^2} = \si{\centi\meter} \times \si{\centi\meter}$, on doit appliquer le facteur de conversion \textbf{deux fois} :
\begin{align*}
    A &= \SI{1500}{\centi\meter^2} = 1500 \cdot (\SI{1}{\centi\meter}) \cdot (\SI{1}{\centi\meter}) \\[0.5em]
      &= 1500 \cdot \left(\SI{1}{\centi\meter} \cdot \frac{\SI{1}{\meter}}{\SI{100}{\centi\meter}}\right) \cdot \left(\SI{1}{\centi\meter} \cdot \frac{\SI{1}{\meter}}{\SI{100}{\centi\meter}}\right) \\[0.5em]
      &= 1500 \cdot \frac{1}{100} \cdot \frac{1}{100}\,\si{\meter^2} = \SI{0,150}{\meter^2}
\end{align*}

\textbf{Raccourci :} On peut directement élever le facteur de conversion au carré :
\[
    A = \SI{1500}{\centi\meter^2} \cdot \left(\frac{\SI{1}{\meter}}{\SI{100}{\centi\meter}}\right)^2 = 1500 \cdot \frac{1}{100^2}\,\si{\meter^2} = \SI{0,150}{\meter^2}
\]
\end{exemple}

\begin{exemple}
Convertir un volume de $V = \SI{8,50}{pi^3}$ (pieds cubes) en \si{\meter^3}.

Sachant que $\SI{1}{pi} = \SI{0,3048}{\meter}$ (Tableau~\ref{tab:equivalences}) :
\begin{align*}
    V &= \SI{8,50}{pi^3} \cdot \left(\frac{\SI{0,3048}{\meter}}{\SI{1}{pi}}\right)^3 \\[0.5em]
      &= 8{,}50 \cdot (0{,}3048)^3\,\si{\meter^3} \\[0.5em]
      &= 8{,}50 \cdot 0{,}02832\,\si{\meter^3} = \SI{0,241}{\meter^3}
\end{align*}
\end{exemple}

% -----------------------------------------------------------------------------
\subsection*{Pratique autonome — Conversions d'unités}
% -----------------------------------------------------------------------------

\begin{pratiqueautonome}
Un cargo voyage à une vitesse de 18 nœuds. Convertir cette vitesse en mètres par seconde (\si{\meter/\second}).

\espaceresolution[6cm]

\reponsepratique{\SI{9,26}{\meter/\second}}
\end{pratiqueautonome}

\begin{pratiqueautonome}
La surface du pont principal d'un navire est de \SI{450}{\meter^2}. Convertir cette aire en pieds carrés (\si{pi^2}).

\textit{Indice : Le facteur de conversion doit être appliqué deux fois (une fois par dimension).}

\espaceresolution[6cm]

\reponsepratique{\SI{4840}{pi^2}}
\end{pratiqueautonome}

\begin{pratiqueautonome}
La consommation de carburant d'un moteur marin est de \SI{85}{\liter/\hour}. Convertir en gallons américains par minute (\si{gal(US)/min}).

\textit{Rappel : \SI{1}{gal(US)} = \SI{3,785}{\liter}}

\espaceresolution[7cm]

\reponsepratique{\SI{0,374}{gal(US)/min}}
\end{pratiqueautonome}

% =============================================================================
\section{Homogénéité des équations}
\label{sec:homogeneite}
% =============================================================================

En physique, les équations relient des grandeurs physiques entre elles. Pour qu'une équation soit valide, elle doit respecter une règle fondamentale : l'\textbf{homogénéité dimensionnelle}.

\begin{definition}
Une équation est \textbf{homogène} si les deux côtés de l'égalité ont les \textbf{mêmes unités} (ou dimensions). 

Autrement dit : on ne peut égaler, additionner ou soustraire que des grandeurs \textbf{de même nature}.
\end{definition}

\begin{attention}
\textbf{Règles fondamentales :}
\begin{itemize}
    \item On ne peut \textbf{pas additionner} des mètres et des secondes : $\SI{5}{m} + \SI{3}{s}$ n'a aucun sens!
    \item On ne peut \textbf{pas égaler} une vitesse et une accélération : $v = a$ est impossible.
    \item Les deux côtés d'une équation \textbf{doivent avoir les mêmes unités}.
\end{itemize}
\end{attention}

% -----------------------------------------------------------------------------
\subsection{Pourquoi vérifier l'homogénéité?}
% -----------------------------------------------------------------------------

Vérifier l'homogénéité d'une équation est un outil puissant pour :

\begin{enumerate}
    \item \textbf{Détecter des erreurs} : Si votre réponse n'a pas les bonnes unités, il y a forcément une erreur quelque part dans votre calcul.
    
    \item \textbf{Vérifier une formule} : Avant d'utiliser une équation, vous pouvez vérifier qu'elle est dimensionnellement correcte.
    
    \item \textbf{Retrouver une formule oubliée} : L'analyse dimensionnelle peut vous aider à reconstruire une équation dont vous avez oublié la forme exacte.
\end{enumerate}

% -----------------------------------------------------------------------------
\subsection{Comment vérifier l'homogénéité}
% -----------------------------------------------------------------------------

Pour vérifier l'homogénéité d'une équation, on remplace chaque grandeur par ses \textbf{unités de base} (m, kg, s) et on simplifie.

\begin{exemple}
Vérifions que l'équation de l'énergie cinétique $E_c = \frac{1}{2}mv^2$ est homogène.

\textbf{Côté gauche :} L'énergie se mesure en joules.
\[
    [E_c] = \si{\joule} = \si{\newton\cdot\meter} = \si{\kilogram\cdot\meter\per\second\squared\cdot\meter} = \si{\kilogram\cdot\meter^2\per\second^2}
\]

\textbf{Côté droit :} La masse est en kg, la vitesse en m/s.
\[
    \left[\frac{1}{2}mv^2\right] = \si{\kilogram} \cdot \left(\si{\meter\per\second}\right)^2 = \si{\kilogram\cdot\meter^2\per\second^2}
\]

Les deux côtés ont les mêmes unités $\rightarrow$ \textbf{l'équation est homogène} $\checkmark$
\end{exemple}

\begin{exemple}
Un étudiant propose la formule $v = \frac{1}{2}at^2$ pour la vitesse. Est-elle homogène?

\textbf{Côté gauche :} $[v] = \si{\meter\per\second}$

\textbf{Côté droit :} 
\[
    \left[\frac{1}{2}at^2\right] = \si{\meter\per\second^2} \cdot \si{\second^2} = \si{\meter}
\]

Le côté gauche est en m/s, le côté droit est en m $\rightarrow$ \textbf{l'équation n'est PAS homogène} $\times$

La formule correcte est $v = at$ (ou $x = \frac{1}{2}at^2$ pour la position).
\end{exemple}

\begin{equationimportante}
\textbf{Méthode de vérification}
\begin{enumerate}
    \item Écrire les unités de chaque grandeur en unités de base (m, kg, s)
    \item Simplifier les unités de chaque côté de l'équation
    \item Comparer : si les unités sont identiques, l'équation est homogène
\end{enumerate}

\textbf{Attention :} Une équation homogène n'est pas nécessairement correcte (il pourrait manquer un facteur numérique), mais une équation \textbf{non homogène est toujours fausse}.
\end{equationimportante}

% -----------------------------------------------------------------------------
\subsection*{Pratique autonome — Homogénéité des équations}
% -----------------------------------------------------------------------------

\begin{pratiqueautonome}
L'énergie cinétique de rotation d'un objet est donnée par $E_r = \frac{1}{2}I\omega^2$, où $I$ est le moment d'inertie (en \si{\kilogram\cdot\meter^2}) et $\omega$ est la vitesse angulaire (en \si{rad/\second}).

Vérifier que cette équation est homogène. \textit{(Rappel : l'énergie se mesure en joules, où \SI{1}{\joule} = \SI{1}{\kilogram\cdot\meter^2/\second^2})}

\espaceresolution[6cm]

\reponsepratique{Homogène : les deux côtés donnent \si{\kilogram\cdot\meter^2/\second^2}}
\end{pratiqueautonome}

\begin{pratiqueautonome}
Un étudiant propose la formule $t = \dfrac{v}{2a^2}$ pour calculer un temps, où $v$ est une vitesse (\si{\meter/\second}) et $a$ est une accélération (\si{\meter/\second^2}).

Cette formule est-elle homogène? Justifier.

\espaceresolution[6cm]

\reponsepratique{Non homogène : côté droit donne \si{\second^3/\meter}, pas des secondes}
\end{pratiqueautonome}

\begin{pratiqueautonome}
La période d'oscillation $T$ d'un pendule simple dépend de sa longueur $L$ et de l'accélération gravitationnelle $g$. On propose la formule :
\[
    T = 2\pi \sqrt{\frac{L^n}{g}}
\]
où $n$ est un exposant inconnu. En utilisant l'analyse dimensionnelle, déterminer la valeur de $n$ pour que l'équation soit homogène.

\textit{Indice : La période $T$ est en secondes, $L$ en mètres, et $g$ en \si{\meter/\second^2}.}

\espaceresolution[7cm]

\reponsepratique{$n = 1$}
\end{pratiqueautonome}
