% =============================================================================
\section*{Compétences}
\addcontentsline{toc}{section}{Compétences}
% =============================================================================

\begin{table}[H]
\centering
\small
\begin{tabular}{@{}p{10cm}cccc@{}}
\toprule
\textbf{Compétences} & \rotatebox{90}{\textbf{Difficile}} & \rotatebox{90}{\textbf{Familier}} & \rotatebox{90}{\textbf{Minimum}} & \rotatebox{90}{\textbf{Maîtrise}} \\
\midrule
Fournir une réponse complète à chaque question & $\square$ & $\square$ & $\square$ & $\square$ \\
Mettre les unités dans chaque nouvelle équation & $\square$ & $\square$ & $\square$ & $\square$ \\
Connaître les unités de base du SI utilisées dans Physique 1 & $\square$ & $\square$ & $\square$ & $\square$ \\
Convertir des valeurs ayant des unités simples & $\square$ & $\square$ & $\square$ & $\square$ \\
Convertir des valeurs ayant des unités composées (km/h, cm$^3$, RPM...) & $\square$ & $\square$ & $\square$ & $\square$ \\
Utiliser une équation pour déterminer les unités de base & $\square$ & $\square$ & $\square$ & $\square$ \\
S'assurer de l'homogénéité d'une équation & $\square$ & $\square$ & $\square$ & $\square$ \\
Déterminer le nombre de chiffres significatifs d'une valeur & $\square$ & $\square$ & $\square$ & $\square$ \\
Déterminer le nombre de chiffres significatifs d'une réponse & $\square$ & $\square$ & $\square$ & $\square$ \\
Utiliser sin, cos, tan et Pythagore dans un triangle rectangle & $\square$ & $\square$ & $\square$ & $\square$ \\
Utiliser les lois des sinus et du cosinus & $\square$ & $\square$ & $\square$ & $\square$ \\
Différencier une grandeur vectorielle d'une grandeur scalaire & $\square$ & $\square$ & $\square$ & $\square$ \\
Additionner et soustraire des vecteurs & $\square$ & $\square$ & $\square$ & $\square$ \\
\midrule
\multicolumn{5}{l}{\textbf{Mathématiques} (essentielles mais non enseignées dans le cours)} \\
\midrule
Isoler une variable dans une équation & $\square$ & $\square$ & $\square$ & $\square$ \\
Priorité des opérations & $\square$ & $\square$ & $\square$ & $\square$ \\
Mise en évidence & $\square$ & $\square$ & $\square$ & $\square$ \\
Distributivité & $\square$ & $\square$ & $\square$ & $\square$ \\
Proportionnalité, règle de trois, produit croisé & $\square$ & $\square$ & $\square$ & $\square$ \\
\bottomrule
\end{tabular}
\end{table}


