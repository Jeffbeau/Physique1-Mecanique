% =============================================================================
% SECTION - QUANTITÉ DE MOUVEMENT ET IMPULSION
% Chapitre 3 - Énergie, travail et quantité de mouvement
% Semaines 10-11
% =============================================================================

\section{Quantité de mouvement et impulsion}
\label{sec:quantite-mvt}

% =============================================================================
\subsection{Introduction : pourquoi une nouvelle grandeur?}
\label{subsec:intro-qdm}
% =============================================================================

Au chapitre précédent, nous avons vu que l'énergie est une grandeur conservée --- un outil puissant pour résoudre des problèmes. Mais l'énergie est une grandeur \textbf{scalaire} : elle ne contient aucune information sur la \textbf{direction} du mouvement. Or, dans de nombreuses situations, la direction est cruciale.

Considérons l'exemple suivant : un vraquier de 80\,000 tonnes naviguant à 12 nœuds vers l'est et un porte-conteneurs de 40\,000 tonnes naviguant à 24 nœuds vers l'ouest possèdent la \textbf{même énergie cinétique}. Pourtant, si ces deux navires entrent en collision, le résultat dépendra clairement de la direction de chaque navire --- pas seulement de l'énergie.

Nous avons besoin d'une grandeur qui combine la masse, la vitesse \textbf{et} la direction du mouvement. Cette grandeur, c'est la \textbf{quantité de mouvement}.

% =============================================================================
\subsection{Définition de la quantité de mouvement}
\label{subsec:def-qdm}
% =============================================================================

\begin{definition}[title=Quantité de mouvement]
La \textbf{quantité de mouvement} $\vect{p}$ d'un objet de masse $m$ se déplaçant à la vitesse $\vect{v}$ est définie par :

\begin{equationimportante}
\begin{equation}
\vect{p} = m\vect{v}
\label{eq:quantite_mvt}
\end{equation}
\end{equationimportante}

\begin{itemize}
    \item C'est une grandeur \textbf{vectorielle} : elle a un module, une direction et un sens
    \item Sa direction est la même que celle de la vitesse $\vect{v}$
    \item L'unité SI est le $\SI{}{kg \cdot m/s}$ (il n'y a pas de nom spécial)
\end{itemize}
\end{definition}

\begin{remarque}[title=Quantité de mouvement et inertie en mouvement]
La quantité de mouvement représente la « difficulté à arrêter un objet en mouvement ». Un objet massif et rapide possède une grande quantité de mouvement et est donc très difficile à arrêter. C'est pourquoi on l'appelle parfois \textbf{momentum} (terme anglais couramment utilisé en sciences).

En notation scalaire (1D), si l'on choisit un axe, la quantité de mouvement s'écrit simplement :
\[ p = mv \]
où $v$ peut être positif ou négatif selon le sens du mouvement.
\end{remarque}

\begin{exemple}{Quantité de mouvement de différents navires}{qdm-navires}
Comparons la quantité de mouvement de trois navires :

\begin{center}
\renewcommand{\arraystretch}{1.4}
\begin{tabular}{|l|c|c|c|}
\hline
\rowcolor{bleuclair}
\textbf{Navire} & \textbf{Masse} & \textbf{Vitesse} & \textbf{$|\vect{p}| = mv$} \\
\hline
Canot de sauvetage & $\SI{300}{kg}$ & $\SI{8}{m/s}$ & $\SI{2400}{kg \cdot m/s}$ \\
\hline
Traversier & $\SI{5000}{tonnes}$ & $\SI{8}{m/s}$ & $\SI{4,0e7}{kg \cdot m/s}$ \\
\hline
Pétrolier (VLCC) & $\SI{300000}{tonnes}$ & $\SI{8}{m/s}$ & $\SI{2,4e9}{kg \cdot m/s}$ \\
\hline
\end{tabular}
\end{center}

À la même vitesse, le pétrolier a une quantité de mouvement \textbf{un million de fois} plus grande que le canot. C'est pourquoi le pétrolier a besoin de plusieurs kilomètres pour s'arrêter, tandis que le canot s'arrête en quelques mètres.
\end{exemple}

\begin{pratiqueautonome}
Un porte-conteneurs de masse $m = \SI{60000}{tonnes}$ navigue à $\SI{14}{\knots}$ ($\approx \SI{7,2}{m/s}$) vers le nord. Un vraquier de masse $m = \SI{80000}{tonnes}$ navigue à $\SI{10}{\knots}$ ($\approx \SI{5,1}{m/s}$) vers l'est.

\begin{enumerate}[label=\alph*)]
    \item Calculez la quantité de mouvement de chaque navire.
    \item Lequel des deux navires sera le plus difficile à arrêter? Justifiez.
\end{enumerate}

\espaceresolution[5cm]
\reponsepratique{a) $p_1 = \SI{4,32e8}{kg \cdot m/s}$ (nord), $p_2 = \SI{4,08e8}{kg \cdot m/s}$ (est) \quad b) Le porte-conteneurs ($|\vect{p}_1| > |\vect{p}_2|$)}
\end{pratiqueautonome}

% =============================================================================
\subsection{L'impulsion}
\label{subsec:impulsion}
% =============================================================================

En pratique, les forces agissent pendant un certain \textbf{temps}. L'effet d'une force sur un objet dépend non seulement de son intensité, mais aussi de sa \textbf{durée d'application}. C'est la notion d'impulsion.

\begin{definition}[title=Impulsion]
L'\textbf{impulsion} $\vect{J}$ est le produit d'une force par l'intervalle de temps pendant lequel elle agit :

\begin{equationimportante}
\begin{equation}
\vect{J} = \vect{F} \cdot \Delta t
\label{eq:impulsion}
\end{equation}
\end{equationimportante}

\begin{itemize}
    \item C'est une grandeur \textbf{vectorielle}
    \item Sa direction est la même que celle de la force $\vect{F}$
    \item L'unité SI est le $\SI{}{N \cdot s}$, qui est équivalent au $\SI{}{kg \cdot m/s}$
\end{itemize}
\end{definition}

\begin{remarque}[title=Force moyenne]
En réalité, la force n'est pas toujours constante pendant la collision. L'impulsion est alors définie avec la \textbf{force moyenne} $\vect{F}_{moy}$ :
\[ \vect{J} = \vect{F}_{moy} \cdot \Delta t \]

Graphiquement, l'impulsion correspond à l'\textbf{aire sous la courbe} $F(t)$.

% --- DIAGRAMME À CRÉER ---
% TikZ : Graphique F(t) montrant une force variable typique d'une collision.
% - Axe horizontal : t (temps), de 0 à ~0,15 s
% - Axe vertical : F (force), de 0 à ~F_max
% - Courbe en cloche asymétrique (montée rapide, descente plus lente) représentant
%   la force durant une collision (ex. : navire frappant un quai)
% - Aire sous la courbe hachurée en bleu clair avec label "J = aire"
% - Ligne pointillée horizontale à F_moy avec label "F_moy"
% - Rectangle pointillé de hauteur F_moy et largeur Δt, même aire que la courbe
% - Annotations : Δt entre les points d'intersection avec l'axe t
\end{remarque}

% =============================================================================
\subsection{Le théorème de l'impulsion et de la quantité de mouvement}
\label{subsec:theoreme-impulsion}
% =============================================================================

Le lien fondamental entre l'impulsion et la quantité de mouvement s'obtient directement à partir de la deuxième loi de Newton.

\begin{definition}[title=Théorème de l'impulsion et de la quantité de mouvement]
L'impulsion totale exercée sur un objet est égale à la \textbf{variation de sa quantité de mouvement} :

\begin{equationimportante}
\begin{equation}
\vect{J} = \Delta \vect{p} = \vect{p}_f - \vect{p}_i = m\vect{v}_f - m\vect{v}_i
\label{eq:theoreme_impulsion}
\end{equation}
\end{equationimportante}

En notation scalaire (1D) :
\[ F_{moy} \cdot \Delta t = mv_f - mv_i \]
\end{definition}

\begin{remarque}[title=Démonstration]
Par la deuxième loi de Newton : $\vect{F} = m\vect{a}$

En remplaçant $\vect{a}$ par $\dfrac{\Delta \vect{v}}{\Delta t}$ :
\[ \vect{F} = m \cdot \frac{\Delta \vect{v}}{\Delta t} = \frac{m\vect{v}_f - m\vect{v}_i}{\Delta t} = \frac{\Delta \vect{p}}{\Delta t} \]

En multipliant les deux côtés par $\Delta t$ :
\[ \vect{F} \cdot \Delta t = \Delta \vect{p} \]

C'est la forme originale de la deuxième loi de Newton, telle que formulée par Newton lui-même!
\end{remarque}

\begin{remarque}[title=Pourquoi ce théorème est-il utile?]
Le théorème de l'impulsion est particulièrement utile pour analyser les \textbf{collisions} et les \textbf{interactions brèves}, car :
\begin{itemize}
    \item On connaît souvent les vitesses \textbf{avant} et \textbf{après} l'interaction
    \item On peut calculer la force moyenne même si la force réelle varie de façon complexe
    \item Il fait le lien entre le \textbf{temps} de contact et la \textbf{force} subie
\end{itemize}
\end{remarque}

\begin{exemple}{Accostage d'un traversier}{accostage}
Un traversier de masse $m = \SI{5000}{tonnes} = \SI{5,0e6}{kg}$ s'approche du quai à une vitesse de $\SI{0,5}{m/s}$. Le système d'amortissement (défenses de quai) immobilise le traversier en $\SI{4}{s}$.

Quelle est la force moyenne exercée par les défenses sur le traversier?

\textbf{Données :} $m = \SI{5,0e6}{kg}$, $v_i = \SI{0,5}{m/s}$, $v_f = \SI{0}{m/s}$, $\Delta t = \SI{4}{s}$

\textbf{Théorème de l'impulsion :}
\begin{align*}
F_{moy} \cdot \Delta t &= mv_f - mv_i \\
F_{moy} \cdot 4 &= \SI{5,0e6}{} \times 0 - \SI{5,0e6}{} \times 0{,}5 \\
F_{moy} &= \frac{-\SI{2,5e6}{}}{\SI{4}{}} = \SI{-625}{kN}
\end{align*}

Le signe négatif indique que la force s'oppose au mouvement du traversier. La force moyenne est de $\SI{625}{kN}$, soit l'équivalent du poids d'environ 64 tonnes!

% --- DIAGRAMME À CRÉER ---
% TikZ : Schéma en 3 étapes montrant l'accostage
% Panneau gauche : Traversier (rectangle gris) s'approchant du quai (mur hachuré)
%   avec vecteur v_i vers la droite et label "État initial"
% Panneau central : Contact — traversier contre défenses (ressorts stylisés)
%   avec F_moy vers la gauche et label "Pendant Δt = 4 s"
% Panneau droite : Traversier immobile contre le quai
%   avec v_f = 0 et label "État final"
\end{exemple}

\begin{exemple}{Temps de contact et force d'impact}{temps-contact}
Un conteneur de $\SI{8000}{kg}$ tombe de $\SI{5}{m}$ et frappe le pont du navire.

Comparez la force d'impact si le conteneur s'arrête en :
\begin{enumerate}[label=\alph*)]
    \item $\Delta t_1 = \SI{0,5}{s}$ (pont avec amortisseurs)
    \item $\Delta t_2 = \SI{0,02}{s}$ (pont rigide en acier)
\end{enumerate}

\textbf{Vitesse juste avant l'impact} (conservation de l'énergie ou chute libre) :
\[ v = \sqrt{2gh} = \sqrt{2 \times 9{,}81 \times 5} = \SI{9,9}{m/s} \]

\textbf{Quantité de mouvement avant l'impact :}
\[ p_i = mv = 8000 \times 9{,}9 = \SI{79200}{kg \cdot m/s} \]

\textbf{Impulsion nécessaire} (identique dans les deux cas) :
\[ |\Delta p| = |0 - 79200| = \SI{79200}{kg \cdot m/s} \]

\textbf{Force moyenne :}
\begin{enumerate}[label=\alph*)]
    \item Avec amortisseurs : $F_1 = \dfrac{79200}{0{,}5} = \SI{158}{kN}$
    \item Sans amortisseurs : $F_2 = \dfrac{79200}{0{,}02} = \SI{3960}{kN} \approx \SI{4,0}{MN}$
\end{enumerate}

\textbf{Conclusion :} En augmentant le temps de contact d'un facteur 25, la force est réduite d'un facteur 25. C'est le principe fondamental de \textbf{tous} les systèmes d'amortissement : airbags, défenses de quai, pare-chocs, zones de déformation.
\end{exemple}

\begin{attention}[title=La grande leçon de l'impulsion]
Pour une même variation de quantité de mouvement ($\Delta p$ fixé), il y a un compromis :
\begin{itemize}
    \item \textbf{Grand $\Delta t$} $\rightarrow$ \textbf{petite force} (atterrissage en parachute, défenses de quai)
    \item \textbf{Petit $\Delta t$} $\rightarrow$ \textbf{grande force} (coup de marteau, collision rigide)
\end{itemize}

Ce compromis est au cœur de la conception des systèmes de sécurité maritime et automobile.
\end{attention}

\begin{pratiqueautonome}
Un remorqueur de $\SI{500}{tonnes}$ se déplaçant à $\SI{3}{m/s}$ heurte une bouée d'amarrage qui l'immobilise en $\SI{2}{s}$.

\begin{enumerate}[label=\alph*)]
    \item Calculez la quantité de mouvement initiale du remorqueur.
    \item Calculez l'impulsion reçue par le remorqueur.
    \item Calculez la force moyenne exercée par la bouée sur le remorqueur.
    \item Si la bouée était rigide et que l'arrêt se faisait en $\SI{0,1}{s}$, quelle serait la force?
\end{enumerate}

\espaceresolution[6cm]
\reponsepratique{a) $p_i = \SI{1,5e6}{kg \cdot m/s}$ \quad b) $J = \SI{-1,5e6}{N \cdot s}$ \quad c) $F = \SI{-750}{kN}$ \quad d) $F = \SI{-15}{MN}$}
\end{pratiqueautonome}

% =============================================================================
\subsection{Conservation de la quantité de mouvement}
\label{subsec:conservation-qdm}
% =============================================================================

Le théorème de l'impulsion mène directement à l'un des principes les plus fondamentaux de la physique : la \textbf{conservation de la quantité de mouvement}.

\subsubsection{Système isolé et forces internes}

\begin{definition}[title=Système isolé]
Un \textbf{système} est un ensemble d'objets que l'on choisit d'étudier ensemble. Un système est dit \textbf{isolé} si la somme des forces \textbf{extérieures} agissant sur lui est nulle :
\[ \sum \vect{F}_{ext} = \vect{0} \]

Les forces que les objets du système exercent \textbf{entre eux} sont des forces \textbf{internes} --- elles ne changent pas la quantité de mouvement totale du système.
\end{definition}

\begin{remarque}[title=Forces internes vs externes]
Lors d'une collision entre deux navires :
\begin{itemize}
    \item Les forces de contact entre les deux navires sont des forces \textbf{internes} au système (navire A + navire B). Par la 3\textsuperscript{e} loi de Newton, elles sont égales et opposées : $\vect{F}_{A \to B} = -\vect{F}_{B \to A}$. Leur somme est nulle.
    \item La gravité, la poussée d'Archimède et la résistance de l'eau sont des forces \textbf{externes}. Si elles s'annulent (ou sont négligeables par rapport aux forces de collision), le système est isolé.
\end{itemize}
\end{remarque}

\subsubsection{Principe de conservation}

\begin{definition}[title=Conservation de la quantité de mouvement]
Si la résultante des forces extérieures sur un système est nulle ($\sum \vect{F}_{ext} = \vect{0}$), alors la quantité de mouvement totale du système est \textbf{conservée} :

\begin{equationimportante}
\begin{equation}
\vect{p}_{total,i} = \vect{p}_{total,f}
\label{eq:conservation_qdm}
\end{equation}
\end{equationimportante}

Pour un système de deux objets (en 1D) :
\begin{equation}
m_1 v_{1i} + m_2 v_{2i} = m_1 v_{1f} + m_2 v_{2f}
\label{eq:conservation_qdm_2corps}
\end{equation}
\end{definition}

\begin{remarque}[title=Démonstration pour deux objets]
Considérons deux objets qui interagissent. Par la 3\textsuperscript{e} loi de Newton :
\[ \vect{F}_{1 \to 2} = -\vect{F}_{2 \to 1} \]

Par le théorème de l'impulsion, appliqué à chaque objet pendant le temps $\Delta t$ de l'interaction :
\begin{align*}
\vect{F}_{2 \to 1} \cdot \Delta t &= \Delta \vect{p}_1 = m_1\vect{v}_{1f} - m_1\vect{v}_{1i} \\
\vect{F}_{1 \to 2} \cdot \Delta t &= \Delta \vect{p}_2 = m_2\vect{v}_{2f} - m_2\vect{v}_{2i}
\end{align*}

En additionnant ces deux équations et en utilisant $\vect{F}_{1 \to 2} = -\vect{F}_{2 \to 1}$ :
\[ \Delta \vect{p}_1 + \Delta \vect{p}_2 = \vect{0} \]

Donc : $\vect{p}_{1i} + \vect{p}_{2i} = \vect{p}_{1f} + \vect{p}_{2f}$
\end{remarque}

\begin{remarque}[title=Conditions d'application]
La conservation de la quantité de mouvement est \textbf{exacte} si le système est parfaitement isolé. En pratique, on peut l'appliquer de façon approximative lorsque :
\begin{itemize}
    \item Les forces \textbf{internes} (collision) sont beaucoup plus grandes que les forces \textbf{externes} (frottement, résistance)
    \item L'interaction est \textbf{brève} (le temps de la collision est si court que les forces externes n'ont pas le temps d'agir significativement)
\end{itemize}

Dans le cas d'une collision entre navires, les forces d'impact (millions de newtons) sont largement supérieures à la résistance de l'eau (milliers de newtons). La conservation s'applique donc très bien pendant la collision elle-même.
\end{remarque}

\begin{exemple}{Recul d'un canon naval}{recul-canon}
Un canon naval de masse $M = \SI{12000}{kg}$ tire un obus de masse $m = \SI{50}{kg}$ à une vitesse de $\SI{800}{m/s}$.

Quelle est la vitesse de recul du canon?

\textbf{Système :} Canon + obus (isolé dans la direction horizontale, car la force d'explosion est interne)

\textbf{État initial :} Tout est au repos $\rightarrow$ $p_i = 0$

\textbf{Conservation de $\vect{p}$ :}
\begin{align*}
p_i &= p_f \\
0 &= m \cdot v_{obus} + M \cdot v_{canon} \\
0 &= 50 \times 800 + 12000 \times v_{canon} \\
v_{canon} &= \frac{-50 \times 800}{12000} = \SI{-3,3}{m/s}
\end{align*}

Le signe négatif confirme que le canon recule dans la direction \textbf{opposée} à celle de l'obus. La vitesse de recul est relativement faible grâce à la grande masse du canon.

% --- DIAGRAMME À CRÉER ---
% TikZ : Deux panneaux (avant/après le tir)
% Panneau gauche "Avant" : Canon (grand rectangle gris) avec obus (petit rectangle
%   rouge) à l'intérieur. Label p_total = 0. Tout est au repos.
% Panneau droite "Après" : Canon reculant vers la gauche (vecteur v_canon, court)
%   et obus se déplaçant vers la droite (vecteur v_obus, long).
%   Labels : Mv_canon (gauche) et mv_obus (droite).
%   Annotation : "p_total = Mv_canon + mv_obus = 0" en dessous
\end{exemple}

\begin{exemple}{Largage de cargaison en mouvement}{largage}
Un navire de masse $M = \SI{20000}{tonnes}$ navigue à $\SI{6}{m/s}$ vers l'est. Il largue une ancre de masse $m = \SI{2000}{kg}$ verticalement (la composante horizontale de la vitesse de l'ancre au moment du largage est la même que celle du navire).

La vitesse du navire change-t-elle?

\textbf{Attention!} Au moment du largage, l'ancre conserve la même vitesse horizontale que le navire ($v_{ancre} = \SI{6}{m/s}$ vers l'est). Le largage est \textbf{vertical}, pas horizontal.

\textbf{Conservation de $p_x$ :}
\begin{align*}
p_{xi} &= p_{xf} \\
(M + m) \times 6 &= M \times v_{navire,f} + m \times 6
\end{align*}

Puisque $v_{ancre,x} = \SI{6}{m/s}$ (inchangé au moment du largage) :
\[ (M + m) \times 6 = M \times v_{navire,f} + m \times 6 \]
\[ v_{navire,f} = 6 \text{ m/s} \]

La vitesse horizontale du navire ne change \textbf{pas} immédiatement! C'est seulement lorsque l'ancre touchera le fond et exercera une force de freinage que le navire ralentira.
\end{exemple}

\begin{pratiqueautonome}
Un marin de $\SI{80}{kg}$ se tient debout à la poupe d'un canot de $\SI{120}{kg}$ initialement au repos sur l'eau (sans frottement). Le marin marche vers la proue à $\SI{2}{m/s}$ par rapport au sol.

\begin{enumerate}[label=\alph*)]
    \item Quelle est la vitesse du canot par rapport au sol?
    \item Quelle est la vitesse du marin par rapport au canot?
\end{enumerate}

\espaceresolution[6cm]
\reponsepratique{a) $v_{canot} = \SI{-1,33}{m/s}$ (vers la poupe) \quad b) $v_{rel} = 2 - (-1{,}33) = \SI{3,33}{m/s}$}
\end{pratiqueautonome}

\begin{pratiqueautonome}
Deux barges naviguent sur le même cap. La barge A ($m_A = \SI{500}{tonnes}$, $v_A = \SI{4}{m/s}$) rattrape la barge B ($m_B = \SI{300}{tonnes}$, $v_B = \SI{1}{m/s}$). Après le contact, les deux barges restent accrochées et se déplacent ensemble.

Quelle est leur vitesse commune après la collision?

\espaceresolution[5cm]
\reponsepratique{$v_f = \dfrac{m_A v_A + m_B v_B}{m_A + m_B} = \dfrac{500 \times 4 + 300 \times 1}{800} = \SI{2,9}{m/s}$}
\end{pratiqueautonome}

% =============================================================================
\subsection{Comparaison : énergie vs quantité de mouvement}
\label{subsec:comparaison-E-p}
% =============================================================================

Il est important de comprendre les différences entre ces deux grandeurs conservées :

\begin{center}
\renewcommand{\arraystretch}{1.5}
\begin{tabular}{|L{4.5cm}|C{4.5cm}|C{4.5cm}|}
\hline
\rowcolor{bleuclair}
& \textbf{Énergie cinétique} & \textbf{Quantité de mouvement} \\
\hline
\textbf{Formule} & $K = \frac{1}{2}mv^2$ & $\vect{p} = m\vect{v}$ \\
\hline
\textbf{Type de grandeur} & Scalaire & Vectorielle \\
\hline
\textbf{Toujours conservée?} & Non (dissipation possible) & Oui (si système isolé) \\
\hline
\textbf{Dépendance en $v$} & Quadratique ($v^2$) & Linéaire ($v$) \\
\hline
\textbf{Peut être négative?} & Non (toujours $\geq 0$) & Oui (signe = direction) \\
\hline
\textbf{Liée à...} & Travail ($W = \Delta K$) & Impulsion ($J = \Delta p$) \\
\hline
\end{tabular}
\end{center}

\begin{remarque}[title=Relation entre $K$ et $p$]
On peut exprimer l'énergie cinétique en fonction de la quantité de mouvement :
\[ K = \frac{1}{2}mv^2 = \frac{(mv)^2}{2m} = \frac{p^2}{2m} \]

Cette relation est utile pour analyser les collisions : si $\vect{p}$ est conservée, cela ne signifie \textbf{pas} que $K$ l'est aussi (puisque $K$ dépend de $p^2$ et de $m$ différemment pour chaque objet).
\end{remarque}
