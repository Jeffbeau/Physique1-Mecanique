% =============================================================================
% SECTION - INTRODUCTION À L'ÉNERGIE
% Chapitre 3 - Énergie et travail
% =============================================================================

\section{Introduction : qu'est-ce que l'énergie?}
% =============================================================================

Le concept d'\textbf{énergie} est l'un des plus fondamentaux et des plus puissants de toute la physique. Pourtant, il est difficile de donner une définition simple de l'énergie : on ne peut pas la voir, la toucher ou la peser directement. Ce que l'on peut observer, ce sont ses \textbf{manifestations} et ses \textbf{transformations}.

\begin{definition}[title=Énergie]
L'\textbf{énergie} est la capacité d'un système à produire un changement, que ce soit mettre un objet en mouvement, le déformer, le chauffer ou modifier son état d'une quelconque façon.

L'unité SI de l'énergie est le \textbf{joule} (J).
\[ \SI{1}{J} = \SI{1}{kg \cdot m^2/s^2} = \SI{1}{N \cdot m} \]
\end{definition}

\begin{remarque}[title=Pourquoi l'énergie est-elle si importante?]
Toute l'utilité de l'énergie en physique réside dans le fait qu'elle est une \textbf{quantité qui se conserve}. L'énergie peut changer de forme --- passer de cinétique à potentielle, de mécanique à thermique --- mais elle ne peut être ni créée ni détruite. Ce principe de conservation fait de l'énergie un outil de résolution de problèmes souvent plus puissant que les lois de Newton.
\end{remarque}

\subsection{L'énergie dans le contexte maritime}

Pour un officier de navigation, la notion d'énergie est omniprésente, même si elle n'est pas toujours formulée explicitement :

\begin{itemize}
    \item L'\textbf{énergie cinétique} d'un navire de 50~000 tonnes à 15 nœuds est colossale --- c'est cette énergie qu'il faut dissiper lors du freinage
    \item L'\textbf{énergie potentielle} de l'eau dans un réservoir surélevé permet d'alimenter les systèmes de bord par gravité
    \item La \textbf{puissance} des moteurs détermine la vitesse maximale et la capacité d'accélération du navire
    \item Le \textbf{travail} effectué par les treuils et les grues permet de charger et décharger la cargaison
\end{itemize}

\begin{exemple}{Énergie cinétique d'un vraquier}{}
Un vraquier de masse $m = \SI{80000}{tonnes} = \SI{80e6}{kg}$ navigue à une vitesse de $v = \SI{12}{\knots} \approx \SI{6,2}{m/s}$.

Son énergie cinétique est :
\[ K = \frac{1}{2}mv^2 = \frac{1}{2} \times \SI{80e6}{kg} \times (\SI{6,2}{m/s})^2 \approx \SI{1,5e9}{J} = \SI{1,5}{GJ} \]

Pour mettre ce chiffre en perspective, cette énergie équivaut à celle libérée par l'explosion d'environ \SI{360}{kg} de TNT! C'est pourquoi les distances de freinage des grands navires se mesurent en \textbf{kilomètres}.
\end{exemple}

\subsection{Les deux formes d'énergie mécanique}

En mécanique, on distingue deux formes principales d'énergie :

\begin{definition}[title=Énergie cinétique]
L'\textbf{énergie cinétique} $K$ est l'énergie associée au \textbf{mouvement} d'un corps. Plus un objet est massif et rapide, plus son énergie cinétique est grande.

\begin{equationimportante}
\begin{equation}
K = \frac{1}{2}mv^2
\label{eq:energie_cinetique}
\end{equation}
\end{equationimportante}

où $m$ est la masse (en kg) et $v$ est la vitesse (en m/s).
\end{definition}

\begin{definition}[title=Énergie potentielle]
L'\textbf{énergie potentielle} $U$ est l'énergie \textbf{emmagasinée} par un système en raison de sa position ou de sa configuration. C'est une énergie « en réserve » qui peut être convertie en énergie cinétique.

Exemples courants :
\begin{itemize}
    \item Énergie potentielle \textbf{gravitationnelle} : un conteneur soulevé par une grue
    \item Énergie potentielle \textbf{élastique} : un ressort comprimé ou étiré
\end{itemize}
\end{definition}

\begin{remarque}[title=Comment repérer l'énergie?]
\begin{itemize}
    \item \textbf{Énergie cinétique} : facile à repérer --- là où il y a du mouvement, il y a de l'énergie cinétique.
    \item \textbf{Énergie potentielle} : il faut être capable d'\textbf{anticiper le mouvement}. Si un objet peut se mettre en mouvement spontanément (une bille en haut d'une pente, un ressort comprimé), c'est qu'il possède de l'énergie potentielle.
\end{itemize}
\end{remarque}

