% =============================================================================
% SECTION - LE TRAVAIL
% Chapitre 3 - Énergie et travail
% =============================================================================

% =============================================================================
\section{Le travail}
\label{sec:travail}
% =============================================================================

On sait maintenant qu'un corps en mouvement possède de l'énergie cinétique. Or, en vertu de la première loi de Newton, les corps ne se mettent pas en mouvement d'eux-mêmes : une modification du mouvement nécessite l'intervention d'une \textbf{force}. Il doit donc y avoir un lien entre l'action d'une force et l'énergie. Ce lien, c'est le \textbf{travail}.

\subsection{Définition du travail}

\begin{definition}[title=Travail d'une force]
Le \textbf{travail} ($W$) est l'énergie transférée à un système (ou retirée d'un système) par l'action d'une force dont le point d'application se déplace.

Pour une force \textbf{constante} $\vect{F}$ appliquée sur un objet qui se déplace d'une distance $d$ :

\begin{equationimportante}
\begin{equation}
W = Fd\cos\theta
\label{eq:travail}
\end{equation}
\end{equationimportante}

où :
\begin{itemize}
    \item $F$ est le module de la force (en N)
    \item $d$ est la distance parcourue (en m)
    \item $\theta$ est l'angle entre la force et le déplacement
\end{itemize}

L'unité SI du travail est le \textbf{joule} (J), la même que pour l'énergie.
\end{definition}

\begin{center}
\begin{tikzpicture}[scale=1.2]
    % Sol
    \fill[gray!20] (-0.5,-0.3) rectangle (7,0);
    \draw[thick] (-0.5,0) -- (7,0);
    
    % Boîte position initiale (fantôme)
    \draw[dashed, gray] (0.5,0) rectangle (1.5,1);
    
    % Boîte position finale
    \fill[blue!20] (4.5,0) rectangle (5.5,1);
    \draw[thick, blue!70!black] (4.5,0) rectangle (5.5,1);
    \node at (5,0.5) {$m$};
    
    % Vecteur déplacement
    \draw[very thick, ->, green!60!black] (1,0.5) -- (4.5,0.5);
    \node[green!60!black, below] at (2.75,0.5) {$\vect{d}$};
    
    % Vecteur force (angle)
    \draw[vecteur rouge, very thick] (5,0.5) -- ++(35:2) node[above right] {$\vect{F}$};
    
    % Angle
    \draw[thick] (5,0.5) -- ++(1.5,0);
    \draw[thick, ->] (6,0.5) arc (0:35:1) node[midway, right] {$\theta$};
    
    % Annotation
    \node[below] at (3.5,-0.5) {Déplacement $d$};
\end{tikzpicture}
\end{center}

\subsection{Interprétation du travail}

Le travail peut être positif, négatif ou nul selon l'orientation de la force par rapport au déplacement :

\begin{attention}[title=Signe du travail]
\renewcommand{\arraystretch}{1.5}
\begin{center}
\begin{tabular}{|c|c|c|p{5cm}|}
\hline
\rowcolor{bleuclair}
\textbf{Angle $\theta$} & \textbf{$\cos\theta$} & \textbf{Travail} & \textbf{Interprétation} \\
\hline
$0°$ & $+1$ & $W = +Fd$ & Force dans le sens du mouvement $\rightarrow$ \textbf{énergie ajoutée} \\
\hline
$90°$ & $0$ & $W = 0$ & Force perpendiculaire $\rightarrow$ \textbf{aucun transfert d'énergie} \\
\hline
$180°$ & $-1$ & $W = -Fd$ & Force opposée au mouvement $\rightarrow$ \textbf{énergie retirée} \\
\hline
\end{tabular}
\end{center}
\end{attention}

\begin{remarque}[title=Le travail ne crée pas d'énergie]
Le travail fait sur un système ne \textbf{crée} pas de nouvelle énergie : il ne fait que \textbf{transférer} l'énergie d'un endroit à un autre ou d'une forme à une autre. Le travail est le \textbf{mécanisme} par lequel l'énergie est échangée entre un système et son environnement.
\end{remarque}

\subsection{Cas particuliers importants}

\subsubsection{Force dans le sens du mouvement ($\theta = 0°$)}

Lorsque la force est appliquée dans la \textbf{même direction} que le déplacement, le travail est \textbf{maximal et positif} :
\[ W = Fd\cos(0°) = Fd \]

\begin{exemple}{Remorqueur poussant un navire}{}
Un remorqueur exerce une force de poussée de $F = \SI{150}{kN}$ sur un navire, dans la direction du mouvement. Le navire se déplace de $d = \SI{200}{m}$.

\begin{center}
\begin{tikzpicture}[scale=0.8]
    % Eau
    \fill[blue!10] (-1,-0.5) rectangle (10,2);
    \draw[blue!50, thick, decorate, decoration={snake, amplitude=1mm, segment length=5mm}] (-1,0) -- (10,0);
    
    % Navire (simplifié)
    \fill[gray!50] (5,0.2) -- (8,0.2) -- (8.5,0.7) -- (8,1.2) -- (5,1.2) -- (4.5,0.7) -- cycle;
    \node at (6.5,0.7) {Navire};
    
    % Remorqueur
    \fill[orange!70] (1,0.3) -- (3,0.3) -- (3.2,0.6) -- (3,0.9) -- (1,0.9) -- (0.8,0.6) -- cycle;
    \node[below] at (2,-0.3) {Remorqueur};
    
    % Force
    \draw[vecteur rouge, very thick] (3.2,0.6) -- (4.5,0.6) node[above, midway] {$\vect{F}$};
    
    % Déplacement
    \draw[very thick, ->, green!60!black] (1,1.5) -- (5,1.5) node[midway, above] {$\vect{d} = \SI{200}{m}$};
\end{tikzpicture}
\end{center}

\textbf{Travail effectué par le remorqueur :}
\[ W = Fd = \SI{150e3}{N} \times \SI{200}{m} = \SI{30e6}{J} = \SI{30}{MJ} \]

Ce travail positif signifie que le remorqueur \textbf{ajoute} de l'énergie au navire (augmentation de l'énergie cinétique ou compensation des pertes par frottement).
\end{exemple}

\subsubsection{Force perpendiculaire au mouvement ($\theta = 90°$)}

Lorsque la force est \textbf{perpendiculaire} au déplacement, le travail est \textbf{nul} :
\[ W = Fd\cos(90°) = 0 \]

\begin{remarque}[title=Travail nul $\neq$ force inutile]
Une force perpendiculaire au mouvement n'effectue aucun travail, mais elle n'est pas sans effet! Elle peut modifier la \textbf{direction} du mouvement sans changer sa \textbf{vitesse}.

Exemples :
\begin{itemize}
    \item La force centripète dans un virage (change la direction, pas la vitesse)
    \item La force normale sur un objet glissant sur une surface horizontale
    \item La tension d'une corde retenant un pendule au point le plus bas de sa trajectoire
\end{itemize}
\end{remarque}

\subsubsection{Force opposée au mouvement ($\theta = 180°$)}

Lorsque la force est dans la direction \textbf{opposée} au déplacement, le travail est \textbf{négatif} :
\[ W = Fd\cos(180°) = -Fd \]

\begin{exemple}{Freinage d'un navire par inversion des moteurs}{}
Un cargo de \SI{20000}{tonnes} navigue à \SI{8}{m/s}. Pour freiner, on inverse les moteurs qui exercent une force de résistance de $F = \SI{400}{kN}$ pendant que le navire parcourt encore $d = \SI{500}{m}$.

\textbf{Travail effectué par la force de freinage :}
\[ W = Fd\cos(180°) = \SI{400e3}{N} \times \SI{500}{m} \times (-1) = -\SI{200}{MJ} \]

Ce travail \textbf{négatif} signifie que de l'énergie est \textbf{retirée} au navire (diminution de l'énergie cinétique).
\end{exemple}

\begin{pratiqueautonome}
Un matelot tire une caisse de matériel sur le pont d'un navire. La corde fait un angle de $30°$ avec l'horizontale et exerce une tension de $\SI{250}{N}$. La caisse se déplace horizontalement de $\SI{8}{m}$.

\begin{center}
\begin{tikzpicture}[scale=0.9]
    % Pont
    \fill[brown!30] (-0.5,-0.2) rectangle (8,0);
    \draw[thick] (-0.5,0) -- (8,0);
    
    % Caisse
    \fill[blue!20] (2,0) rectangle (3.5,1);
    \draw[thick] (2,0) rectangle (3.5,1);
    \node at (2.75,0.5) {Caisse};
    
    % Corde et force
    \draw[thick, brown] (3.5,0.8) -- ++(30:2);
    \draw[vecteur rouge] (3.5,0.8) -- ++(30:1.5) node[above right] {$\vect{T} = \SI{250}{N}$};
    
    % Angle
    \draw[thick] (3.5,0.8) -- ++(1,0);
    \draw[->] (4.2,0.8) arc (0:30:0.7) node[midway, right] {$30°$};
    
    % Déplacement
    \draw[very thick, ->, green!60!black] (2,1.5) -- (6,1.5) node[midway, above] {$d = \SI{8}{m}$};
\end{tikzpicture}
\end{center}

Calculez le travail effectué par la tension de la corde sur la caisse.

\espaceresolution[4cm]
\reponsepratique{$W = Td\cos\theta = 250 \times 8 \times \cos(30°) = 250 \times 8 \times 0,866 \approx \SI{1730}{J}$}
\end{pratiqueautonome}

\subsection{Travail effectué par plusieurs forces}

Lorsque plusieurs forces agissent simultanément sur un objet, le \textbf{travail total} (ou travail résultant) est la somme des travaux effectués par chacune des forces :

\begin{equationimportante}
\begin{equation}
W_{\text{total}} = W_1 + W_2 + W_3 + \ldots = \sum_i W_i
\label{eq:travail_total}
\end{equation}
\end{equationimportante}

\begin{exemple}{Chargement d'un conteneur sur un plan incliné}{}
Un conteneur de masse $m = \SI{2000}{kg}$ est tiré vers le haut d'une rampe inclinée à $\theta = 20°$ par un câble parallèle à la rampe. Le conteneur monte de $d = \SI{15}{m}$ le long de la rampe. La tension dans le câble est $T = \SI{12000}{N}$ et la force de frottement est $f = \SI{2000}{N}$.

\begin{center}
\begin{tikzpicture}[scale=0.7]
    % Plan incliné
    \fill[gray!20] (0,0) -- (10,0) -- (10,3.64) -- cycle;
    \draw[very thick] (0,0) -- (10,3.64);
    \draw[thick] (0,0) -- (10,0);
    
    % Angle
    \draw[thick] (2,0) arc (0:20:2) node[midway, right] {$20°$};
    
    % Conteneur
    \begin{scope}[shift={(4,1.456)}, rotate=20]
        \fill[blue!30] (0,0) rectangle (1.5,1);
        \draw[thick] (0,0) rectangle (1.5,1);
        \node at (0.75,0.5) {$m$};
        
        % Force de tension (parallèle à la rampe, vers le haut)
        \draw[vecteur rouge, very thick] (1.5,0.5) -- ++(2,0) node[above] {$\vect{T}$};
        
        % Force de frottement (parallèle à la rampe, vers le bas)
        \draw[vecteur, very thick] (0,0.5) -- ++(-1.2,0) node[above] {$\vect{f}$};
        
        % Poids (vertical vers le bas)
        \draw[vecteur vert, very thick] (0.75,0) -- ++(0,-1.5);
    \end{scope}
    
    % Annotation poids
    \node[green!60!black] at (6.5,-0.5) {$\vect{F}_g$};
    
    % Déplacement
    \draw[very thick, ->, green!60!black] (1,0.8) -- ++(20:4) node[midway, above left] {$d = \SI{15}{m}$};
    
    % Hauteur
    \draw[dashed] (10,0) -- (10,3.64);
    \node[right] at (10,1.8) {$h$};
\end{tikzpicture}
\end{center}

Calculons le travail effectué par chaque force :

\textbf{1. Travail de la tension} (parallèle au déplacement, $\theta = 0°$) :
\[ W_T = Td\cos(0°) = \SI{12000}{N} \times \SI{15}{m} \times 1 = +\SI{180000}{J} \]

\textbf{2. Travail du frottement} (opposé au déplacement, $\theta = 180°$) :
\[ W_f = fd\cos(180°) = \SI{2000}{N} \times \SI{15}{m} \times (-1) = -\SI{30000}{J} \]

\textbf{3. Travail de la gravité} (angle de $90° + 20° = 110°$ avec le déplacement) :
\[ W_g = F_g d\cos(110°) = mg \cdot d\cos(110°) \]
\[ W_g = \SI{2000}{kg} \times \SI{9,8}{m/s^2} \times \SI{15}{m} \times \cos(110°) \]
\[ W_g = \SI{294000}{N \cdot m} \times (-0,342) = -\SI{100500}{J} \]

\textbf{4. Travail de la normale} (perpendiculaire au déplacement, $\theta = 90°$) :
\[ W_N = Nd\cos(90°) = 0 \]

\textbf{Travail total :}
\[ W_{\text{total}} = W_T + W_f + W_g + W_N = 180000 - 30000 - 100500 + 0 = +\SI{49500}{J} \]

Le travail total est positif, ce qui signifie que le conteneur \textbf{gagne} de l'énergie (il accélère en montant).
\end{exemple}

\begin{pratiqueautonome}
Une chaloupe de sauvetage de masse $m = \SI{500}{kg}$ est hissée verticalement de $h = \SI{6}{m}$ à l'aide d'un treuil. La tension dans le câble est $T = \SI{5500}{N}$.

\begin{enumerate}[label=\alph*)]
    \item Calculez le travail effectué par la tension.
    \item Calculez le travail effectué par la gravité.
    \item Calculez le travail total sur la chaloupe.
    \item La chaloupe accélère-t-elle, ralentit-elle ou se déplace-t-elle à vitesse constante?
\end{enumerate}

\espaceresolution[6cm]
\reponsepratique{a) $W_T = +\SI{33000}{J}$ \quad b) $W_g = -\SI{29400}{J}$ \quad c) $W_{\text{total}} = +\SI{3600}{J}$ \quad d) Elle accélère (travail total positif)}
\end{pratiqueautonome}
\subsection{Travail effectué par la gravité}

La force gravitationnelle est une force particulièrement importante en physique. Son travail dépend uniquement de la \textbf{variation de hauteur} de l'objet, et non du chemin parcouru.

\begin{definition}[title=Travail de la gravité]
Le travail effectué par la force gravitationnelle sur un objet dont la hauteur varie de $y_i$ (position initiale) à $y_f$ (position finale) est :

\begin{equationimportante}
\begin{equation}
W_g = -mg(y_f - y_i) = -mg\Delta y
\label{eq:travail_gravite}
\end{equation}
\end{equationimportante}

où :
\begin{itemize}
    \item $m$ est la masse de l'objet (en kg)
    \item $g = \SI{9,8}{m/s^2}$ est l'accélération gravitationnelle
    \item $\Delta y = y_f - y_i$ est le changement de hauteur (positif vers le haut)
\end{itemize}
\end{definition}

\begin{remarque}[title=Interprétation du signe]
\begin{itemize}
    \item Si l'objet \textbf{monte} ($\Delta y > 0$) : $W_g < 0$ (la gravité s'oppose au mouvement)
    \item Si l'objet \textbf{descend} ($\Delta y < 0$) : $W_g > 0$ (la gravité aide le mouvement)
\end{itemize}
Le travail de la gravité est \textbf{indépendant du chemin} parcouru : seule la différence de hauteur compte!
\end{remarque}

\begin{center}
\begin{tikzpicture}[scale=0.8]
    % Axes
    \draw[axe, thick, ->] (-0.5,0) -- (8,0) node[right] {$x$};
    \draw[axe, thick, ->] (0,-0.5) -- (0,5) node[above] {$y$};
    
    % Point initial
    \fill[blue] (1,4) circle (4pt) node[above left] {Position initiale};
    \node[left] at (0,4) {$y_i$};
    \draw[dashed, gray] (0,4) -- (1,4);
    
    % Point final
    \fill[red] (6,1) circle (4pt) node[below right] {Position finale};
    \node[left] at (0,1) {$y_f$};
    \draw[dashed, gray] (0,1) -- (6,1);
    
    % Différents chemins
    \draw[thick, blue!60, ->] (1,4) -- (1,1) -- (6,1);
    \draw[thick, green!60!black, ->] (1,4) to[out=-30, in=150] (6,1);
    \draw[thick, orange, ->] (1,4) -- (6,4) -- (6,1);
    
    % Annotation
    \draw[<->, thick, red] (7,1) -- (7,4) node[midway, right] {$\Delta y = y_f - y_i$};
    
    % Légende
    \node[below] at (4,-0.8) {Peu importe le chemin, $W_g = -mg\Delta y$};
\end{tikzpicture}
\end{center}

\begin{exemple}{Conteneur descendu par une grue}{}
Une grue de port descend un conteneur de masse $m = \SI{15000}{kg}$ d'une hauteur de $h = \SI{12}{m}$ jusqu'au pont d'un navire.

\begin{center}
\begin{tikzpicture}[scale=0.6]
    % Quai
    \fill[gray!30] (-1,0) rectangle (3,6);
    \draw[thick] (3,0) -- (3,6);
    
    % Navire
    \fill[blue!20] (4,0) -- (4,2) -- (10,2) -- (10,0) -- cycle;
    \draw[thick] (4,0) -- (4,2) -- (10,2) -- (10,0);
    \node at (7,1) {Navire};
    
    % Eau
    \fill[blue!10] (3,-0.5) rectangle (11,0);
    
    % Grue (simplifiée)
    \draw[very thick, gray!70] (2,6) -- (2,8) -- (7,8);
    
    % Conteneur position haute
    \draw[dashed, blue!50] (6,6) rectangle (7.5,7);
    \node[blue!50, right] at (7.5,6.5) {$y_i = \SI{12}{m}$};
    
    % Conteneur position basse
    \fill[orange!70] (6,2) rectangle (7.5,3);
    \draw[thick] (6,2) rectangle (7.5,3);
    \node[right] at (7.5,2.5) {$y_f = \SI{0}{m}$};
    
    % Câble
    \draw[thick] (6.75,8) -- (6.75,3);
    
    % Flèche déplacement
    \draw[very thick, ->, red] (5,6.5) -- (5,2.5) node[midway, left] {$\Delta y = -\SI{12}{m}$};
    
    % Poids
    \draw[vecteur vert] (6.75,2) -- (6.75,0.5) node[right] {$\vect{F}_g$};
\end{tikzpicture}
\end{center}

\textbf{Changement de hauteur :}
\[ \Delta y = y_f - y_i = 0 - 12 = -\SI{12}{m} \]

\textbf{Travail de la gravité :}
\[ W_g = -mg\Delta y = -(\SI{15000}{kg})(\SI{9,8}{m/s^2})(-\SI{12}{m}) \]
\[ W_g = +\SI{1764000}{J} = +\SI{1,76}{MJ} \]

Le travail est \textbf{positif} car la gravité aide à descendre le conteneur (la force et le déplacement sont dans le même sens).
\end{exemple}

\begin{pratiqueautonome}
Un ascenseur de navire de croisière transporte des passagers ($m_{\text{total}} = \SI{800}{kg}$) entre deux ponts séparés de $\SI{4}{m}$.

\begin{enumerate}[label=\alph*)]
    \item Calculez le travail effectué par la gravité lorsque l'ascenseur monte.
    \item Calculez le travail effectué par la gravité lorsque l'ascenseur descend.
\end{enumerate}

\espaceresolution[4cm]
\reponsepratique{a) $W_g = -\SI{31360}{J}$ \quad b) $W_g = +\SI{31360}{J}$}
\end{pratiqueautonome}

\subsection{Travail effectué par un ressort}

La force de rappel d'un ressort ($\vect{F}_R = -k\vect{x}$) est une force \textbf{variable} : son module dépend de la déformation du ressort. Le calcul du travail nécessite donc une approche différente.

\begin{definition}[title=Travail d'un ressort]
Le travail effectué par la force de rappel d'un ressort lorsque sa déformation passe de $x_i$ à $x_f$ est :

\begin{equationimportante}
\begin{equation}
W_{\text{ressort}} = \frac{1}{2}kx_i^2 - \frac{1}{2}kx_f^2
\label{eq:travail_ressort}
\end{equation}
\end{equationimportante}

où $k$ est la constante de rappel du ressort (en N/m) et $x$ est la déformation par rapport à la position d'équilibre.
\end{definition}

\begin{remarque}[title=Interprétation]
\begin{itemize}
    \item Si le ressort \textbf{se détend} ($|x_f| < |x_i|$) : $W_{\text{ressort}} > 0$ (le ressort fournit de l'énergie)
    \item Si le ressort \textbf{se comprime davantage} ($|x_f| > |x_i|$) : $W_{\text{ressort}} < 0$ (le ressort absorbe de l'énergie)
\end{itemize}
\end{remarque}

