% =============================================================================
% SECTION 3 - MÉTHODOLOGIE
% =============================================================================

\section{Méthodologie de résolution}
\label{sec:methodologie}

Résoudre un problème de dynamique ou de statique n'est pas une question d'intuition ou de chance. C'est un processus \textbf{systématique} qui, lorsqu'il est suivi rigoureusement, mène presque toujours à la solution.

Cette section présente la méthode que vous devez suivre pour \textbf{tous} les problèmes de mécanique. Même si un problème vous semble simple, suivez ces étapes — elles vous éviteront des erreurs et vous prépareront aux problèmes plus complexes.

% -----------------------------------------------------------------------------
\subsection{Le diagramme de corps libre (DCL)}
\label{subsec:dcl}
% -----------------------------------------------------------------------------

L'outil fondamental de la mécanique est le \textbf{diagramme de corps libre} (DCL). C'est la première chose à faire dans \textit{tout} problème.

\begin{definition}[title=Diagramme de corps libre (DCL)]
Un \textbf{diagramme de corps libre} est un schéma qui représente un objet isolé de son environnement, avec toutes les forces qui agissent \textbf{sur} cet objet — et seulement celles-là.
\end{definition}

\begin{attention}[title=Règles pour un bon DCL]
Un DCL complet et correct doit inclure :
\begin{enumerate}
    \item \textbf{L'objet isolé} — représenté simplement (souvent par un point ou un rectangle)
    \item \textbf{Toutes les forces} qui agissent sur cet objet
    \item \textbf{Un système de coordonnées} — axes $x$ et $y$ clairement indiqués
    \item \textbf{Les angles pertinents} — entre les forces et les axes
\end{enumerate}

\textbf{Erreurs courantes à éviter :}
\begin{itemize}
    \item Inclure des forces exercées \textit{par} l'objet (elles agissent sur d'autres objets!)
    \item Oublier le poids (la Terre attire toujours l'objet)
    \item Confondre la normale et le poids (ce ne sont pas des paires action-réaction!)
    \item Dessiner une « force du mouvement » dans le sens du déplacement (le mouvement n'est pas une force!)
    \item Oublier le frottement quand il y a contact entre surfaces
\end{itemize}
\end{attention}

% -----------------------------------------------------------------------------
\subsection{L'algorithme de résolution}
\label{subsec:algorithme}
% -----------------------------------------------------------------------------

\begin{algorithme}[title=Les 4 étapes pour résoudre tout problème de mécanique]

\textbf{Étape 1 — SCHÉMA et DCL}
\begin{enumerate}[label=\alph*)]
    \item Dessiner un schéma de la situation physique
    \item Isoler l'objet d'intérêt (ou les objets, s'il y en a plusieurs)
    \item Tracer le DCL : représenter \textbf{toutes} les forces agissant sur l'objet
    \item Identifier les forces connues et inconnues
\end{enumerate}

\vspace{0.5em}
\textbf{Étape 2 — AXES}
\begin{enumerate}[label=\alph*)]
    \item Choisir un système de coordonnées $(x, y)$ adapté au problème
    \item \textit{Conseil :} Aligner un axe avec l'accélération (si connue) ou avec la surface de contact
    \item Sur un plan incliné : $x$ parallèle à la pente, $y$ perpendiculaire
    \item Indiquer clairement la direction positive de chaque axe
\end{enumerate}

\vspace{0.5em}
\textbf{Étape 3 — ÉQUATIONS DE NEWTON}
\begin{enumerate}[label=\alph*)]
    \item Décomposer chaque force selon les axes $x$ et $y$
    \item Appliquer la deuxième loi de Newton (ou les conditions d'équilibre) :
    \begin{align*}
        \sum F_x &= ma_x \qquad \text{(ou } = 0 \text{ si équilibre)} \\
        \sum F_y &= ma_y \qquad \text{(ou } = 0 \text{ si équilibre)}
    \end{align*}
    \item Écrire les équations explicitement avec les symboles des forces
\end{enumerate}

\vspace{0.5em}
\textbf{Étape 4 — ALGÈBRE}
\begin{enumerate}[label=\alph*)]
    \item Compter les équations et les inconnues (il faut autant d'équations que d'inconnues!)
    \item Résoudre le système d'équations algébriquement
    \item Substituer les valeurs numériques \textit{à la fin}
    \item Vérifier : unités correctes? Ordre de grandeur raisonnable? Signe cohérent?
\end{enumerate}

\end{algorithme}

% -----------------------------------------------------------------------------
\subsection{Conseils pratiques}
\label{subsec:conseils}
% -----------------------------------------------------------------------------

\subsubsection*{Choix des axes}

Le choix du système de coordonnées est libre, mais certains choix simplifient grandement les calculs :

\begin{itemize}
    \item \textbf{Mouvement horizontal :} $x$ horizontal (dans le sens du mouvement), $y$ vertical vers le haut.
    
    \item \textbf{Plan incliné :} $x$ parallèle à la pente (positif vers le bas de la pente), $y$ perpendiculaire à la pente (positif vers l'extérieur). Ainsi, l'accélération est uniquement selon $x$, et $a_y = 0$.
    
    \item \textbf{Mouvement circulaire :} Un axe radial (vers le centre), un axe tangentiel.
    
    \item \textbf{Équilibre :} Souvent, aligner un axe avec une des forces inconnues simplifie les équations.
\end{itemize}

\subsubsection*{Décomposition des forces}

Si une force $\vect{F}$ fait un angle $\theta$ avec l'axe $x$ :
\begin{align*}
    F_x &= F \cos\theta \\
    F_y &= F \sin\theta
\end{align*}

\textbf{Attention aux signes!} Si la composante pointe dans le sens négatif de l'axe, elle doit être négative dans l'équation.

\subsubsection*{Résolution symbolique d'abord}

Résistez à la tentation de substituer les valeurs numériques dès le début. Travaillez avec des symboles ($m$, $g$, $\theta$, $\mu$, etc.) aussi longtemps que possible. Cela permet de :
\begin{itemize}
    \item Repérer plus facilement les erreurs algébriques
    \item Vérifier que les unités sont cohérentes
    \item Comprendre comment la réponse dépend des paramètres
    \item Réutiliser la solution pour d'autres valeurs numériques
\end{itemize}

\subsubsection*{Vérification finale}

Une fois la réponse obtenue, posez-vous ces questions :
\begin{itemize}
    \item Les unités sont-elles correctes?
    \item L'ordre de grandeur est-il raisonnable?
    \item Le signe est-il cohérent avec la direction attendue?
    \item Dans les cas limites (par exemple $\theta = 0$ ou $\mu = 0$), la formule donne-t-elle le résultat attendu?
\end{itemize}

% -----------------------------------------------------------------------------
% PRATIQUE AUTONOME - Section 3
% -----------------------------------------------------------------------------

\begin{pratiqueautonome}[title=Pratique autonome 2.10 — Choix des axes et décomposition]
Pour chaque situation, indiquez le meilleur choix d'axes et décomposez le poids selon ces axes.

\begin{enumerate}
    \item Un bloc glisse sur un plan incliné à 40° vers la droite.
    \item Un pendule est à 25° de la verticale, tiré vers la gauche.
    \item Une voiture accélère horizontalement vers la droite.
    \item Une balle de masse $m$ est lancée à 60° au-dessus de l'horizontale (analysez au point de lancer).
\end{enumerate}

\vspace{2cm}

\tcblower
\textit{Réponses suggérées :}
\begin{enumerate}
    \item Axes : $x$ // pente (+ vers le bas), $y$ $\perp$ pente (+ vers l'extérieur). $F_{gx} = mg\sin 40°$, $F_{gy} = -mg\cos 40°$
    \item Axes : $x$ // fil (+ vers le bas), $y$ $\perp$ fil. $F_{gx} = mg\cos 25°$, $F_{gy} = -mg\sin 25°$
    \item Axes standard : $x$ horizontal (+$\to$), $y$ vertical (+$\uparrow$). $F_{gx} = 0$, $F_{gy} = -mg$
    \item Axes standard : $x$ horizontal, $y$ vertical. $F_{gx} = 0$, $F_{gy} = -mg$ (le poids ne dépend pas de l'angle de lancer!)
\end{enumerate}
\end{pratiqueautonome}
