% =============================================================================
% RÉSUMÉ, COMPÉTENCES ET EXERCICES
% =============================================================================

\section*{Résumé}
\addcontentsline{toc}{section}{Résumé}

\begin{center}
\renewcommand{\arraystretch}{2.0}
\begin{tabular}{|p{4.5cm}|p{9.5cm}|}
\hline
\rowcolor{bleuclair}
\textbf{Concept} & \textbf{Description / Formule} \\
\hline
\textbf{Inertie} & Tendance naturelle de la matière à résister aux changements de vitesse. L'inertie n'est \textbf{pas} une force. \\
\hline
\textbf{Masse} & Mesure quantitative de l'inertie. Unité : kg \\
\hline
\textbf{1\textsuperscript{re} loi de Newton} & Un objet libre de toute force reste au repos ou en MRU. \\
\hline
\textbf{2\textsuperscript{e} loi de Newton} & $\displaystyle\sum \vect{F} = m\vect{a}$ \\
\hline
\textbf{3\textsuperscript{e} loi de Newton} & $\vect{F}_{A \to B} = -\vect{F}_{B \to A}$ \\
\hline
\textbf{Équilibre} & $\sum F_x = 0$ et $\sum F_y = 0$ (cas particulier de la 1\textsuperscript{re} loi) \\
\hline
\textbf{Gravitation universelle} & $\displaystyle F_g = G\frac{m_1 m_2}{r^2}$ \\
\hline
\textbf{Poids} & $F_g = mg$ \quad où $g = 9{,}81$~m/s² (toujours vers le bas) \\
\hline
\textbf{Tension} & Force exercée par une corde, parallèle à celle-ci \\
\hline
\textbf{Normale} & Force perpendiculaire à une surface (réaction passive) \\
\hline
\textbf{Frottement statique} & $f_s \leq \mu_s N$ \\
\hline
\textbf{Frottement cinétique} & $f_c = \mu_c N$ \\
\hline
\textbf{Force centripète} & $\displaystyle F_c = m\frac{v^2}{r} = m\omega^2 r$ \quad (vers le centre) \\
\hline
\end{tabular}
\end{center}

\vspace{1em}

\begin{algorithme}[title=Rappel : Algorithme de résolution]
\begin{enumerate}
    \item \textbf{SCHÉMA et DCL} — Dessiner toutes les forces sur l'objet isolé
    \item \textbf{AXES} — Choisir un système de coordonnées adapté
    \item \textbf{ÉQUATIONS DE NEWTON} — $\sum F_x = ma_x$ et $\sum F_y = ma_y$
    \item \textbf{ALGÈBRE} — Résoudre, puis substituer les valeurs numériques
\end{enumerate}
\end{algorithme}

% =============================================================================
% COMPÉTENCES
% =============================================================================

\section*{Compétences}
\addcontentsline{toc}{section}{Compétences}

À la fin de ce chapitre, vous devriez être capable de :

\begin{itemize}
    \item Expliquer pourquoi la physique d'Aristote est incorrecte et comment Galilée et Newton l'ont réfutée
    \item Définir l'inertie et expliquer pourquoi ce n'est pas une force
    \item Expliquer la différence entre masse et poids
    \item Énoncer et appliquer les trois lois de Newton
    \item Identifier les paires action-réaction dans une situation donnée
    \item Expliquer l'origine de la normale (répulsion des nuages électroniques)
    \item Utiliser la loi de la gravitation universelle et montrer comment on obtient $P = mg$
    \item Tracer un diagramme de corps libre (DCL) complet et correct
    \item Appliquer l'algorithme de résolution en 4 étapes à tout problème de mécanique
    \item Résoudre des problèmes de dynamique à une et deux dimensions
    \item Calculer les forces de frottement statique et cinétique
    \item Calculer la force centripète nécessaire pour un mouvement circulaire
    \item Expliquer pourquoi la « force centrifuge » n'est pas une vraie force newtonienne
    \item Résoudre des problèmes d'équilibre (statique du point)
\end{itemize}
