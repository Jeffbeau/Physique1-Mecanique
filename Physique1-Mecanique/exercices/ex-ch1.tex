% =============================================================================
\section*{Exercices}
\addcontentsline{toc}{section}{Exercices}
% =============================================================================

\begin{remarque}[title=Niveaux de difficulté]
\begin{itemize}
    \item[$\star$] Application directe d'une formule
    \item[$\star\star$] Problème à plusieurs étapes ou avec conversion
    \item[$\star\star\star$] Problème complexe ou piège conceptuel
\end{itemize}
\end{remarque}

% -----------------------------------------------------------------------------
\subsection*{Position et déplacement}
% -----------------------------------------------------------------------------

\begin{enumerate}
\item[$\star$] \textbf{1.} Un navire-citerne part du terminal A (position $x = 0$), se rend au terminal B situé à $\SI{15}{km}$ à l'est, puis revient au terminal C situé à $\SI{8}{km}$ à l'est de A.
\begin{enumerate}[label=\alph*)]
    \item Quelle est la distance totale parcourue?
    \item Quel est le déplacement du navire?
\end{enumerate}

\item[$\star$] \textbf{2.} Un remorqueur effectue les déplacements suivants dans un port : $\SI{200}{m}$ vers l'est, $\SI{150}{m}$ vers l'ouest, puis $\SI{300}{m}$ vers l'est. Calculez la distance parcourue et le déplacement net.

\item[$\star\star$] \textbf{3.} Un patrouilleur part de sa base, navigue $\SI{12}{km}$ vers l'est puis $\SI{5}{km}$ vers le nord.
\begin{enumerate}[label=\alph*)]
    \item Quelle est la distance totale parcourue?
    \item Quel est le module du déplacement? (Utilisez le théorème de Pythagore)
\end{enumerate}

\item[$\star\star\star$] \textbf{4.} \textbf{Vrai ou faux?} Justifiez chaque réponse.
\begin{enumerate}[label=\alph*)]
    \item Le déplacement est toujours inférieur ou égal à la distance parcourue.
    \item Si un navire revient à son point de départ, sa vitesse moyenne est nulle mais sa vitesse scalaire moyenne ne l'est pas.
    \item Le déplacement dépend du trajet emprunté.
\end{enumerate}
\end{enumerate}

% -----------------------------------------------------------------------------
\subsection*{Vitesse moyenne et scalaire}
% -----------------------------------------------------------------------------

\begin{enumerate}[resume]
\item[$\star$] \textbf{5.} Un traversier effectue la traversée Rimouski--Forestville ($\SI{25}{km}$) en $\SI{55}{min}$. Calculez sa vitesse moyenne en km/h, en m/s et en n\oe{}uds.

\item[$\star$] \textbf{6.} Un cargo parcourt $\SI{450}{milles nautiques}$ en 30 heures. Quelle est sa vitesse moyenne en n\oe{}uds et en m/s?

\item[$\star\star$] \textbf{7.} Un patrouilleur des garde-côtes part de sa base, parcourt $\SI{40}{NM}$ vers le nord à $\SI{20}{n\oe{}uds}$, puis revient à sa base à $\SI{15}{n\oe{}uds}$.
\begin{enumerate}[label=\alph*)]
    \item Quel est le temps total de la patrouille?
    \item Quelle est la vitesse moyenne sur l'ensemble du trajet?
    \item Quelle est la vitesse scalaire moyenne?
\end{enumerate}

\item[$\star\star$] \textbf{8.} Un vraquier effectue un trajet de $\SI{1200}{NM}$. Il navigue à $\SI{12}{n\oe{}uds}$ pendant les 20 premières heures, puis à $\SI{15}{n\oe{}uds}$ pour le reste du trajet. Quelle est sa vitesse scalaire moyenne?

\item[$\star\star\star$] \textbf{9.} \textbf{Question piège!} Un navire parcourt la \textit{première moitié} d'un trajet à $\SI{10}{n\oe{}uds}$ et la \textit{seconde moitié} à $\SI{20}{n\oe{}uds}$. 
\begin{enumerate}[label=\alph*)]
    \item Quelle est sa vitesse scalaire moyenne? 
    \item Expliquez pourquoi ce n'est \textbf{pas} $\SI{15}{n\oe{}uds}$.
\end{enumerate}

\item[$\star\star\star$] \textbf{10.} Un autre navire parcourt la \textit{première moitié du temps} à $\SI{10}{n\oe{}uds}$ et la \textit{seconde moitié du temps} à $\SI{20}{n\oe{}uds}$. Quelle est sa vitesse scalaire moyenne? Comparez avec l'exercice précédent.
\end{enumerate}

% -----------------------------------------------------------------------------
\subsection*{Graphiques}
% -----------------------------------------------------------------------------

\begin{enumerate}[resume]
\item[$\star\star$] \textbf{11.} Le graphique suivant montre la position d'un navire en fonction du temps :

\begin{center}
\begin{tikzpicture}[scale=0.6]
\draw[axe, thick] (0,0) -- (8,0) node[right] {$t$ (h)};
\draw[axe, thick] (0,0) -- (0,5) node[above] {$x$ (NM)};
\foreach \x in {1,2,3,4,5,6,7} {\draw (\x,0.1) -- (\x,-0.1) node[below] {\x};}
\foreach \y in {1,2,3,4} {\draw (0.1,\y) -- (-0.1,\y) node[left] {\pgfmathparse{int(\y*20)}\pgfmathresult};}
\draw[very thick, blue] (0,0) -- (2,2) -- (4,2) -- (7,4);
\end{tikzpicture}
\end{center}

\begin{enumerate}[label=\alph*)]
    \item Décrivez qualitativement le mouvement du navire.
    \item Calculez la vitesse entre $t = 0$ et $t = 2$ h.
    \item Le navire est-il à l'arrêt à un certain moment? Si oui, quand?
    \item Quelle est la vitesse moyenne sur l'ensemble du trajet (0 à 7 h)?
\end{enumerate}

\item[$\star\star$] \textbf{12.} Le graphique suivant montre la vitesse d'un cargo en fonction du temps :

\begin{center}
\begin{tikzpicture}[scale=0.6]
\draw[axe, thick] (0,0) -- (8,0) node[right] {$t$ (min)};
\draw[axe, thick] (0,0) -- (0,5) node[above] {$v$ (m/s)};
\foreach \x in {2,4,6,8,10,12,14} {\draw (\x/2,0.1) -- (\x/2,-0.1) node[below] {\x};}
\foreach \y in {1,2,3,4} {\draw (0.1,\y) -- (-0.1,\y) node[left] {\y};}
\draw[very thick, blue] (0,0) -- (2,4) -- (5,4) -- (7,0);
\end{tikzpicture}
\end{center}

\begin{enumerate}[label=\alph*)]
    \item Calculez l'accélération pendant les 4 premières minutes.
    \item Pendant quelle phase le navire est-il en MRU?
    \item Calculez la distance totale parcourue (aire sous la courbe).
\end{enumerate}

\item[$\star\star\star$] \textbf{13.} \textbf{Expliquez} pourquoi, sur un graphique position-temps $x(t)$, une pente négative signifie que l'objet se déplace dans le sens négatif, et non pas qu'il recule nécessairement.

\item[$\star\star$] \textbf{14.} Un navire-citerne effectue une manœuvre d'approche. Le graphique suivant montre sa vitesse en fonction du temps :

\begin{center}
\begin{tikzpicture}[scale=0.6]
\draw[axe, thick] (0,0) -- (10,0) node[right] {$t$ (min)};
\draw[axe, thick] (0,0) -- (0,5) node[above] {$v$ (m/s)};
\foreach \x in {2,4,6,8,10,12,14,16,18} {\draw (\x/2,0.1) -- (\x/2,-0.1) node[below] {\small\x};}
\foreach \y in {1,2,3,4} {\draw (0.1,\y) -- (-0.1,\y) node[left] {\y};}
\draw[very thick, blue] (0,3) -- (2,3) -- (4,4) -- (6,4) -- (9,0);
\fill[blue] (0,3) circle (2pt);
\fill[blue] (2,3) circle (2pt);
\fill[blue] (4,4) circle (2pt);
\fill[blue] (6,4) circle (2pt);
\fill[blue] (9,0) circle (2pt);
\end{tikzpicture}
\end{center}

\begin{enumerate}[label=\alph*)]
    \item Décrivez qualitativement les différentes phases du mouvement.
    \item Calculez l'accélération pendant la phase d'accélération (entre $t = 4$ et $t = 8$ min).
    \item Calculez l'accélération pendant la phase de freinage.
    \item Calculez la distance totale parcourue pendant toute la manœuvre (aire sous la courbe).
    \item \textbf{Tracez} le graphique $a(t)$ correspondant à ce mouvement.
\end{enumerate}

\item[$\star\star\star$] \textbf{15.} Un cargo quitte le port de Montréal. Voici le graphique de son accélération en fonction du temps pendant les 10 premières minutes :

\begin{center}
\begin{tikzpicture}[scale=0.7]
\draw[axe, thick] (0,0) -- (8,0) node[right] {$t$ (min)};
\draw[axe, thick] (0,-1.5) -- (0,3) node[above] {$a$ ($\times 10^{-2}$ m/s²)};
% Graduations temps
\foreach \x in {2,4,6,8,10} {\draw (\x*0.7,0.1) -- (\x*0.7,-0.1) node[below] {\small\x};}
% Graduations accélération
\draw (0.1,2) -- (-0.1,2) node[left] {\small 2};
\draw (0.1,1) -- (-0.1,1) node[left] {\small 1};
\draw (0.1,-1) -- (-0.1,-1) node[left] {\small $-1$};
% Courbe: a = 0.02 de 0 à 4 min, puis a = 0 de 4 à 8 min, puis a = -0.01 de 8 à 10 min
\draw[very thick, blue] (0,2) -- (2.8,2);
\draw[very thick, blue, dashed] (2.8,2) -- (2.8,0);
\draw[very thick, blue] (2.8,0) -- (5.6,0);
\draw[very thick, blue, dashed] (5.6,0) -- (5.6,-1);
\draw[very thick, blue] (5.6,-1) -- (7,-1);
\fill[blue] (0,2) circle (2pt);
\fill[blue] (2.8,2) circle (2pt);
\fill[blue] (2.8,0) circle (2pt);
\fill[blue] (5.6,0) circle (2pt);
\fill[blue] (5.6,-1) circle (2pt);
\fill[blue] (7,-1) circle (2pt);
\end{tikzpicture}
\end{center}

Le cargo part du repos ($v_0 = 0$).
\begin{enumerate}[label=\alph*)]
    \item Calculez la vitesse du cargo à $t = 4$ min.
    \item Calculez la vitesse du cargo à $t = 8$ min.
    \item Calculez la vitesse du cargo à $t = 10$ min.
    \item \textbf{Tracez} le graphique $v(t)$ correspondant.
\end{enumerate}

\item[$\star\star\star$] \textbf{16.} Un traversier effectue la traversée entre deux quais. Voici le graphique de sa position en fonction du temps :

\begin{center}
\begin{tikzpicture}[scale=0.6]
\draw[axe, thick] (0,0) -- (10,0) node[right] {$t$ (min)};
\draw[axe, thick] (0,0) -- (0,6) node[above] {$x$ (m)};
\foreach \x in {2,4,6,8,10,12,14,16} {\draw (\x*0.55,0.1) -- (\x*0.55,-0.1) node[below] {\small\x};}
\foreach \y in {1,2,3,4,5} {\draw (0.1,\y) -- (-0.1,\y) node[left] {\pgfmathparse{int(\y*500)}\pgfmathresult};}
% Parabole d'accélération, puis droite, puis parabole de décélération
\draw[very thick, blue, domain=0:2.2, samples=50] plot (\x, {0.4*\x*\x});
\draw[very thick, blue] (2.2,1.936) -- (5.5,4.4);
\draw[very thick, blue, domain=5.5:8.8, samples=50] plot (\x, {5.5-0.4*(\x-8.8)*(\x-8.8)});
\end{tikzpicture}
\end{center}

\begin{enumerate}[label=\alph*)]
    \item Identifiez les trois phases du mouvement (accélération, vitesse constante, freinage).
    \item Estimez graphiquement la vitesse maximale atteinte (pente de la partie linéaire).
    \item Le traversier est-il à l'arrêt au début et à la fin? Justifiez par la forme de la courbe.
    \item \textbf{Tracez} le graphique $v(t)$ correspondant à ce mouvement.
\end{enumerate}

\end{enumerate}

% -----------------------------------------------------------------------------
\subsection*{Accélération et MRUA}
% -----------------------------------------------------------------------------

\begin{enumerate}[resume]
\item[$\star$] \textbf{17.} Un porte-conteneurs passe de $\SI{5}{n\oe{}uds}$ à $\SI{18}{n\oe{}uds}$ en $\SI{12}{min}$. Calculez son accélération moyenne en m/s².

\item[$\star$] \textbf{18.} Un pétrolier naviguant à $\SI{14}{n\oe{}uds}$ freine et s'arrête après $\SI{25}{min}$. Quelle est sa décélération moyenne?

\item[$\star\star$] \textbf{19.} Un traversier part du repos et accélère à $\SI{0,08}{m/s^2}$ jusqu'à atteindre $\SI{12}{n\oe{}uds}$.
\begin{enumerate}[label=\alph*)]
    \item Combien de temps dure la phase d'accélération?
    \item Quelle distance parcourt-il pendant cette phase?
\end{enumerate}

\item[$\star\star$] \textbf{20.} Un cargo navigue à $\SI{15}{n\oe{}uds}$. Le capitaine ordonne le freinage avec $a = \SI{-0,006}{m/s^2}$.
\begin{enumerate}[label=\alph*)]
    \item Quelle est la distance de freinage?
    \item Combien de temps faut-il pour s'arrêter?
\end{enumerate}

\item[$\star\star$] \textbf{21.} Une voiture roule à $\SI{90}{km/h}$ et freine avec $a = \SI{-8}{m/s^2}$. Calculez sa distance de freinage. Comparez avec un cargo à la même vitesse qui freine avec $a = \SI{-0,006}{m/s^2}$.

\item[$\star\star\star$] \textbf{22.} Un navire doit s'arrêter exactement à un quai situé à $\SI{800}{m}$. Il arrive à $\SI{8}{n\oe{}uds}$.
\begin{enumerate}[label=\alph*)]
    \item Quelle décélération constante doit-il appliquer?
    \item Combien de temps dure la manœuvre?
    \item Si le capitaine applique plutôt $a = \SI{-0,015}{m/s^2}$, à quelle distance du quai le navire s'arrêtera-t-il?
\end{enumerate}

\item[$\star\star\star$] \textbf{23.} \textbf{Vrai ou faux?} Justifiez.
\begin{enumerate}[label=\alph*)]
    \item Si l'accélération est négative, l'objet ralentit toujours.
    \item Un objet peut avoir une vitesse nulle et une accélération non nulle.
    \item Si la vitesse et l'accélération ont le même signe, l'objet accélère.
\end{enumerate}

\item[$\star\star\star$] \textbf{24.} Un remorqueur navigue vers l'est à vitesse constante de $\SI{8}{n\oe{}uds}$. Un cargo, initialement à $\SI{500}{m}$ derrière lui (à l'ouest), démarre du repos et accélère à $\SI{0,015}{m/s^2}$ dans la même direction.
\begin{enumerate}[label=\alph*)]
    \item Après combien de temps le cargo rattrape-t-il le remorqueur?
    \item Quelle distance chacun a-t-il parcourue à ce moment?
    \item Quelle est la vitesse du cargo au moment du rattrapage?
\end{enumerate}
\end{enumerate}

% -----------------------------------------------------------------------------
\subsection*{Chute libre}
% -----------------------------------------------------------------------------

\begin{enumerate}[resume]
\item[$\star$] \textbf{25.} Un matelot échappe un outil du haut d'un mât situé à $\SI{18}{m}$ au-dessus du pont.
\begin{enumerate}[label=\alph*)]
    \item Combien de temps l'outil met-il pour atteindre le pont?
    \item À quelle vitesse frappe-t-il le pont?
\end{enumerate}

\item[$\star\star$] \textbf{26.} Un marin sur un pont à $\SI{8}{m}$ au-dessus de l'eau lance une bouée de sauvetage verticalement vers le bas avec une vitesse initiale de $\SI{3}{m/s}$.
\begin{enumerate}[label=\alph*)]
    \item Combien de temps la bouée met-elle pour atteindre l'eau?
    \item À quelle vitesse entre-t-elle dans l'eau?
\end{enumerate}

\item[$\star\star$] \textbf{27.} Une fusée éclairante est lancée verticalement vers le haut depuis le pont d'un navire ($y_0 = \SI{5}{m}$) avec une vitesse de $\SI{25}{m/s}$.
\begin{enumerate}[label=\alph*)]
    \item Quelle hauteur maximale atteint-elle au-dessus de l'eau?
    \item Combien de temps reste-t-elle en l'air avant de retomber à l'eau ($y = 0$)?
\end{enumerate}

\item[$\star\star\star$] \textbf{28.} \textbf{Question contre-intuitive.} Un marin sur un mât laisse tomber une balle A. Au même instant, un autre marin au sol lance une balle B verticalement vers le haut. Les deux balles se croisent à mi-hauteur. 

Laquelle des deux balles a la plus grande vitesse au moment où elles se croisent? \textit{Justifiez sans calcul.}
\end{enumerate}

% -----------------------------------------------------------------------------
\subsection*{Mouvement en 2D et projectile}
% -----------------------------------------------------------------------------

\begin{enumerate}[resume]
\item[$\star$] \textbf{29.} Un navire navigue à $\SI{18}{n\oe{}uds}$ avec un cap de $40°$ nord de l'est.
\begin{enumerate}[label=\alph*)]
    \item Quelle est la composante est-ouest de sa vitesse?
    \item Quelle est la composante nord-sud de sa vitesse?
\end{enumerate}

\item[$\star\star$] \textbf{30.} Un lance-amarre projette une ligne à $\SI{25}{m/s}$ avec un angle de $40°$ depuis une hauteur de $\SI{6}{m}$.
\begin{enumerate}[label=\alph*)]
    \item Calculez les composantes $v_{0x}$ et $v_{0y}$.
    \item Quelle est la portée horizontale?
\end{enumerate}

\item[$\star\star$] \textbf{31.} Un conteneur tombe d'une grue qui se déplace horizontalement à $\SI{2}{m/s}$. Le conteneur est à $\SI{15}{m}$ de hauteur. À quelle distance horizontale (par rapport au point directement sous le largage) le conteneur touche-t-il le sol?

\item[$\star\star\star$] \textbf{32.} Une fusée éclairante est lancée à $\SI{40}{m/s}$ avec un angle de $60°$ depuis le niveau de l'eau.
\begin{enumerate}[label=\alph*)]
    \item Quelle hauteur maximale atteint-elle?
    \item Quelle est sa portée horizontale?
    \item À quel angle faudrait-il la lancer pour maximiser la portée? (Sans calcul)
\end{enumerate}

\item[$\star\star\star$] \textbf{33.} \textbf{Expliquez} pourquoi un objet lancé horizontalement et un objet lâché au même instant depuis la même hauteur touchent le sol en même temps.
\end{enumerate}

% -----------------------------------------------------------------------------
\subsection*{Cinématique de rotation}
% -----------------------------------------------------------------------------

\begin{enumerate}[resume]
\item[$\star$] \textbf{34.} Convertissez : a) $270°$ en radians \quad b) $\frac{3\pi}{4}$~rad en degrés \quad c) $\SI{90}{RPM}$ en rad/s

\item[$\star$] \textbf{35.} Une hélice de navire tourne à $\SI{150}{RPM}$. Si le rayon de l'hélice est de $\SI{2}{m}$, quelle est la vitesse linéaire en bout de pale (en m/s et km/h)?

\item[$\star\star$] \textbf{36.} Un treuil de rayon $\SI{15}{cm}$ doit enrouler $\SI{8}{m}$ de câble.
\begin{enumerate}[label=\alph*)]
    \item Combien de tours doit-il effectuer?
    \item S'il tourne à $\SI{25}{RPM}$, combien de temps faut-il?
\end{enumerate}

\item[$\star\star$] \textbf{37.} L'hélice d'un navire passe de $\SI{0}{RPM}$ à $\SI{120}{RPM}$ en $\SI{20}{s}$ avec une accélération angulaire constante.
\begin{enumerate}[label=\alph*)]
    \item Quelle est l'accélération angulaire en rad/s²?
    \item Combien de tours l'hélice effectue-t-elle pendant cette phase?
\end{enumerate}

\item[$\star\star\star$] \textbf{38.} Un cabestan de rayon $\SI{20}{cm}$ tourne à $\SI{45}{RPM}$. On applique les freins et il s'arrête après 8 tours.
\begin{enumerate}[label=\alph*)]
    \item Quelle est la décélération angulaire?
    \item Quelle longueur de câble a été halée pendant le freinage?
    \item Combien de temps a duré le freinage?
\end{enumerate}
\end{enumerate}

% =============================================================================
\subsection*{Problèmes de synthèse}
% =============================================================================

Ces problèmes intègrent plusieurs concepts du chapitre. Ils sont représentatifs du niveau attendu lors des évaluations.

\begin{enumerate}[resume]
\item[\textbf{S1.}] \textbf{Manœuvre d'accostage}

Un cargo de haute mer arrive au port de Montréal. À $\SI{2}{km}$ du quai, il navigue à $\SI{10}{n\oe{}uds}$. Le pilote du port ordonne une première phase de freinage avec $a_1 = \SI{-0,005}{m/s^2}$ jusqu'à atteindre $\SI{3}{n\oe{}uds}$, puis une seconde phase avec $a_2 = \SI{-0,008}{m/s^2}$ jusqu'à l'arrêt complet.

\begin{enumerate}[label=\alph*)]
    \item Quelle distance parcourt le navire pendant la première phase?
    \item Quelle distance parcourt-il pendant la seconde phase?
    \item Le navire s'arrête-t-il avant le quai? Si non, à quelle distance du quai aurait-il dû commencer à freiner?
\end{enumerate}

\item[\textbf{S2.}] \textbf{Homme à la mer}

Un marin tombe d'un navire qui se déplace à $\SI{14}{n\oe{}uds}$. L'équipage met $\SI{45}{s}$ avant de réagir (temps de réaction), puis le navire freine avec une décélération de $\SI{0,015}{m/s^2}$ jusqu'à l'arrêt.

\begin{enumerate}[label=\alph*)]
    \item Quelle distance le navire parcourt-il pendant le temps de réaction?
    \item Quelle distance parcourt-il pendant le freinage?
    \item À quelle distance totale du point de chute le navire s'arrête-t-il?
    \item Si le marin dérive à $\SI{1}{n\oe{}ud}$ dans le sens opposé au navire, à quelle distance du navire se trouve-t-il quand celui-ci s'arrête?
\end{enumerate}

\item[\textbf{S3.}] \textbf{Lance-amarre tactique}

Un officier doit envoyer une amarre sur un quai situé à $\SI{60}{m}$ horizontalement. Le pont du navire est à $\SI{8}{m}$ au-dessus du niveau du quai. Le lance-amarre peut projeter la ligne à $\SI{30}{m/s}$.

\begin{enumerate}[label=\alph*)]
    \item À quel angle minimal doit-il tirer pour atteindre le quai?
    \item Si le vent souffle vers le navire et réduit la portée de 15\%, peut-il encore atteindre le quai?
    \item Quel serait l'angle optimal pour maximiser la portée dans les conditions normales?
\end{enumerate}

\item[\textbf{S4.}] \textbf{Opération de levage}

Une grue portuaire soulève un conteneur de $\SI{20}{tonnes}$. Le treuil (rayon $\SI{25}{cm}$) accélère de 0 à $\SI{30}{RPM}$ en $\SI{5}{s}$, puis maintient cette vitesse constante.

\begin{enumerate}[label=\alph*)]
    \item Quelle est l'accélération angulaire pendant la phase d'accélération?
    \item À quelle vitesse linéaire le conteneur monte-t-il en régime permanent?
    \item Quelle hauteur le conteneur atteint-il après $\SI{5}{s}$?
    \item Si le câble casse à une hauteur de $\SI{12}{m}$, combien de temps le conteneur met-il pour toucher le sol?
\end{enumerate}

\item[\textbf{S5.}] \textbf{Situation de collision}

Deux navires se font face dans un chenal étroit. Le navire A ($\SI{12}{n\oe{}uds}$ vers l'est) et le navire B ($\SI{8}{n\oe{}uds}$ vers l'ouest) sont séparés de $\SI{2000}{m}$.

\begin{enumerate}[label=\alph*)]
    \item Dans combien de temps se croiseront-ils si aucun ne freine?
    \item Les deux capitaines freinent simultanément : $a_A = \SI{-0,006}{m/s^2}$ et $a_B = \SI{-0,004}{m/s^2}$. Calculez la distance de freinage de chaque navire.
    \item Y a-t-il collision? Si oui, calculez la vitesse relative d'impact.
\end{enumerate}

\item[\textbf{S6.}] \textbf{Analyse de navigation}

Un cargo quitte Sept-Îles à 6h00 pour Rimouski ($\SI{180}{NM}$). Il navigue d'abord à $\SI{12}{n\oe{}uds}$ pendant 8 heures, puis accélère à $\SI{0,002}{m/s^2}$ jusqu'à atteindre $\SI{16}{n\oe{}uds}$, vitesse qu'il maintient jusqu'à destination.

\begin{enumerate}[label=\alph*)]
    \item Quelle distance a-t-il parcourue pendant les 8 premières heures?
    \item Combien de temps dure la phase d'accélération?
    \item À quelle heure arrive-t-il à Rimouski?
    \item Tracez le graphique $v(t)$ de ce voyage.
\end{enumerate}
\end{enumerate}

% =============================================================================
\subsection*{Défis intégrateurs}
% =============================================================================

Ces problèmes demandent de combiner plusieurs concepts et de développer une stratégie de résolution. Ils représentent le niveau attendu pour bien maîtriser la cinématique.

\begin{enumerate}
\item[\textbf{D1.}] \textbf{Sauvetage héliporté} $(\star\star\star\star)$

Un hélicoptère de la Garde côtière vole horizontalement vers l'est à $\SI{90}{km/h}$ à une altitude de $\SI{40}{m}$ au-dessus de l'eau. Il doit larguer une bouée de sauvetage pour qu'elle tombe exactement sur un naufragé.

\begin{enumerate}[label=\alph*)]
    \item À quelle distance horizontale \textbf{avant} le naufragé l'hélicoptère doit-il larguer la bouée?
    \item Avec quelle vitesse (module et direction) la bouée touche-t-elle l'eau?
    \item Si le naufragé dérive vers l'est à $\SI{2}{m/s}$ (même direction que l'hélicoptère), comment cela modifie-t-il la distance de largage?
    \item L'hélicoptère décide plutôt de lancer la bouée vers le bas avec une vitesse initiale de $\SI{5}{m/s}$. Recalculez la distance de largage.
\end{enumerate}

\item[\textbf{D2.}] \textbf{Rendez-vous en mer} $(\star\star\star\star)$

Deux navires doivent se rejoindre pour un transfert de personnel. À $t = 0$ :
\begin{itemize}
    \item Le navire A est à l'origine, immobile, et commence à accélérer vers l'est à $\SI{0,02}{m/s^2}$
    \item Le navire B est à $\SI{5}{km}$ à l'est, navigue vers l'ouest à $\SI{8}{n\oe{}uds}$ et maintient cette vitesse
\end{itemize}

\begin{enumerate}[label=\alph*)]
    \item Écrivez les équations de position $x_A(t)$ et $x_B(t)$ pour chaque navire.
    \item À quel instant et à quelle position les navires se rencontrent-ils?
    \item Quelle est la vitesse du navire A au moment du rendez-vous?
    \item Quelle est la vitesse \textbf{relative} du navire A par rapport au navire B à cet instant? Interprétez physiquement.
    \item Si le navire A doit s'arrêter au point de rendez-vous avec une décélération de $\SI{0,015}{m/s^2}$, à quel moment doit-il commencer à freiner?
\end{enumerate}

\item[\textbf{D3.}] \textbf{Lance-amarre en conditions réelles} $(\star\star\star\star)$

Un officier sur le gaillard d'avant (hauteur $h_1 = \SI{8}{m}$) doit envoyer une amarre sur un quai situé à $\SI{25}{m}$ horizontalement. Le quai est à une hauteur $h_2 = \SI{3}{m}$ au-dessus de l'eau. Le lance-amarre propulse la ligne à $v_0 = \SI{30}{m/s}$.

\begin{enumerate}[label=\alph*)]
    \item Quelle est la différence de hauteur $\Delta y$ entre le point de tir et le point d'arrivée?
    \item Déterminez l'angle minimal $\theta_{min}$ permettant d'atteindre le quai. \textit{(Indice : vous devrez résoudre une équation du second degré en $\tan\theta$)}
    \item Pour cet angle minimal, calculez le temps de vol de l'amarre.
    \item Il existe un deuxième angle $\theta_{max}$ qui permet aussi d'atteindre la cible. Lequel des deux angles est préférable en pratique? Justifiez.
    \item Si le navire s'approche du quai à $\SI{0,5}{m/s}$ pendant le vol de l'amarre, de combien la portée effective est-elle réduite?
\end{enumerate}

\item[\textbf{D4.}] \textbf{Analyse complète d'une manœuvre} $(\star\star\star\star)$

Un vraquier de $\SI{200}{m}$ de long effectue la manœuvre suivante dans le chenal :
\begin{itemize}
    \item Phase 1 : Accélération de $\SI{0}{n\oe{}ud}$ à $\SI{6}{n\oe{}uds}$ avec $a_1 = \SI{0,01}{m/s^2}$
    \item Phase 2 : Vitesse constante de $\SI{6}{n\oe{}uds}$ pendant $\SI{15}{min}$
    \item Phase 3 : Accélération de $\SI{6}{n\oe{}uds}$ à $\SI{12}{n\oe{}uds}$ avec $a_3 = \SI{0,008}{m/s^2}$
    \item Phase 4 : Vitesse constante de $\SI{12}{n\oe{}uds}$ jusqu'à destination
\end{itemize}

La destination est à $\SI{10}{km}$ du point de départ.

\begin{enumerate}[label=\alph*)]
    \item Calculez la durée et la distance parcourue pendant chaque phase (1, 2 et 3).
    \item Quelle distance reste-t-il à parcourir en phase 4? Combien de temps dure cette phase?
    \item Calculez le temps total de la manœuvre.
    \item Tracez les graphiques $v(t)$ et $x(t)$ à l'échelle, en identifiant clairement chaque phase.
    \item Calculez la vitesse moyenne sur l'ensemble du trajet. Comparez avec la moyenne arithmétique des vitesses (6 et 12 nœuds).
\end{enumerate}

\item[\textbf{D5.}] \textbf{Système de treuils coordonnés} $(\star\star\star\star)$

Lors d'une opération de chargement, deux treuils travaillent en coordination :
\begin{itemize}
    \item Treuil A (rayon $r_A = \SI{20}{cm}$) : part du repos, accélère à $\alpha_A = \SI{0,5}{rad/s^2}$ pendant $\SI{4}{s}$, puis maintient sa vitesse
    \item Treuil B (rayon $r_B = \SI{30}{cm}$) : tourne à vitesse constante $\omega_B = \SI{2}{rad/s}$ dès $t = 0$
\end{itemize}

Les deux treuils enroulent des câbles reliés au même conteneur par un système de poulies (le conteneur monte à la moyenne des vitesses linéaires des deux câbles).

\begin{enumerate}[label=\alph*)]
    \item Calculez la vitesse angulaire finale du treuil A et sa vitesse linéaire en bout de tambour.
    \item Calculez la vitesse linéaire du câble B.
    \item À $t = \SI{4}{s}$, à quelle vitesse monte le conteneur?
    \item Quelle longueur de câble le treuil A a-t-il enroulée pendant les 4 premières secondes?
    \item Quelle longueur de câble le treuil B a-t-il enroulée pendant le même temps?
    \item De quelle hauteur le conteneur est-il monté pendant ces $\SI{4}{s}$? \textit{(Attention : la vitesse de montée varie!)}
\end{enumerate}

\item[\textbf{D6.}] \textbf{Poursuite et interception} $(\star\star\star\star\star)$

Un navire de patrouille P détecte un navire suspect S à $\SI{8}{km}$ au nord. Au moment de la détection :
\begin{itemize}
    \item Le navire S navigue vers l'est à $\SI{15}{n\oe{}uds}$ (vitesse constante)
    \item Le navire P est immobile mais peut atteindre $\SI{25}{n\oe{}uds}$ avec une accélération de $\SI{0,03}{m/s^2}$
\end{itemize}

Le capitaine de P veut intercepter S en naviguant en ligne droite vers un point d'interception.

\begin{enumerate}[label=\alph*)]
    \item Si P navigue directement vers la position actuelle de S, expliquez pourquoi il ne l'interceptera jamais.
    \item Supposons que P navigue vers un point situé à un angle $\theta$ est du nord. Écrivez les équations de position de P et S en fonction du temps.
    \item Pour une interception, les positions doivent coïncider. Établissez les deux équations (en $x$ et en $y$) qui doivent être satisfaites.
    \item Par essai-erreur ou graphiquement, estimez l'angle $\theta$ optimal et le temps d'interception. \textit{(La solution analytique exacte dépasse le cadre du cours.)}
    \item Quelle distance totale le navire P parcourt-il jusqu'à l'interception?
\end{enumerate}

\item[\textbf{D7.}] \textbf{Chute d'un conteneur en mouvement} $(\star\star\star\star)$

Une grue portuaire déplace un conteneur horizontalement à $\SI{3}{m/s}$ à une hauteur de $\SI{20}{m}$. Le câble casse et le conteneur tombe.

\begin{enumerate}[label=\alph*)]
    \item Combien de temps le conteneur met-il pour toucher le sol?
    \item À quelle distance horizontale du point de rupture le conteneur atterrit-il?
    \item Avec quelle vitesse (module) le conteneur frappe-t-il le sol?
    \item Un travailleur se trouve à $\SI{8}{m}$ horizontalement du point de rupture (dans la direction du mouvement). A-t-il le temps de s'écarter s'il lui faut $\SI{1,5}{s}$ pour réagir et courir $\SI{3}{m}$?
    \item Le conteneur a des dimensions $\SI{6}{m} \times \SI{2,4}{m}$ (longueur $\times$ largeur). En supposant qu'il ne tourne pas pendant la chute, quelle est la \og zone de danger \fg{} au sol (rectangle où le conteneur pourrait tomber)?
\end{enumerate}

\item[\textbf{D8.}] \textbf{Navigation avec courant variable} $(\star\star\star\star\star)$

Un navire doit traverser un détroit de $\SI{4}{km}$ de large (direction nord-sud). Le courant dans le détroit varie linéairement :
\begin{itemize}
    \item Au bord sud : courant nul
    \item Au centre : courant maximal de $\SI{3}{n\oe{}uds}$ vers l'est
    \item Au bord nord : courant nul
\end{itemize}

Le navire part du bord sud et navigue à $\SI{10}{n\oe{}uds}$ (vitesse surface) cap au nord.

\begin{enumerate}[label=\alph*)]
    \item Exprimez le courant $v_c(y)$ en fonction de la position $y$ (où $y = 0$ au sud et $y = \SI{4}{km}$ au nord).
    \item Expliquez qualitativement pourquoi le navire dérivera vers l'est, puis reviendra partiellement.
    \item À quelle distance à l'est de sa ligne de départ le navire se trouvera-t-il quand il atteindra le bord nord? \textit{(Indice : divisez la traversée en petits segments et sommez les dérives, ou utilisez l'intégration si vous la connaissez.)}
    \item Combien de temps dure la traversée?
    \item Quel cap initial (ouest du nord) le capitaine devrait-il prendre pour arriver exactement au point visé? \textit{(Question ouverte -- une réponse approximative est acceptable.)}
\end{enumerate}

\end{enumerate}

\bigskip
\begin{remarque}[title=Conseils pour les défis]
\begin{itemize}
    \item \textbf{Dessinez toujours} un schéma avec les axes, les positions initiales et les directions.
    \item \textbf{Identifiez les phases} du mouvement (MRU, MRUA, chute libre...) pour chaque objet.
    \item \textbf{Listez les inconnues} et comptez vos équations -- vous devez avoir autant d'équations que d'inconnues.
    \item \textbf{Le temps est souvent le lien} entre les composantes $x$ et $y$, ou entre deux objets.
    \item \textbf{Vérifiez vos unités} à chaque étape et la cohérence de vos réponses (ordres de grandeur).
\end{itemize}
\end{remarque}

% =============================================================================
\subsection*{Réponses}
% =============================================================================

\textbf{Position et déplacement}

\textbf{1.} a) $\SI{22}{km}$ \quad b) $\SI{+8}{km}$ vers l'est

\textbf{2.} $d = \SI{650}{m}$, $\Delta x = \SI{+350}{m}$ vers l'est

\textbf{3.} a) $\SI{17}{km}$ \quad b) $|\Delta\vec{r}| = \SI{13}{km}$

\textbf{4.} a) Faux (égalité si ligne droite sans demi-tour) \quad b) Vrai \quad c) Faux (ne dépend que des positions initiale et finale)

\medskip
\textbf{Vitesse moyenne et scalaire}

\textbf{5.} $\SI{27,3}{km/h}$ ; $\SI{7,58}{m/s}$ ; $\SI{14,7}{n\oe{}uds}$

\textbf{6.} $\SI{15}{n\oe{}uds}$ ; $\SI{7,72}{m/s}$

\textbf{7.} a) $\SI{4,67}{h}$ \quad b) $\SI{0}{n\oe{}ud}$ \quad c) $\SI{17,1}{n\oe{}uds}$

\textbf{8.} $\SI{13,3}{n\oe{}uds}$

\textbf{9.} a) $\SI{13,3}{n\oe{}uds}$ \quad b) Le temps passé à chaque vitesse n'est pas égal

\textbf{10.} $\SI{15}{n\oe{}uds}$ (moyenne arithmétique car temps égaux)

\medskip
\textbf{Graphiques}

\textbf{11.} a) MRU, repos, MRU \quad b) $\SI{20}{NM/h}$ \quad c) Oui, entre 2h et 4h \quad d) $\SI{11,4}{NM/h}$

\textbf{12.} a) $\SI{1}{m/s/min} = \SI{0,017}{m/s^2}$ \quad b) Entre 4 et 10 min \quad c) $\SI{1680}{m}$

\textbf{13.} Le signe dépend du choix de l'axe; une pente négative indique un mouvement vers les $x$ décroissants

\medskip
\textbf{Accélération et MRUA}

\textbf{17.} $\SI{0,0093}{m/s^2}$

\textbf{18.} $\SI{-0,0048}{m/s^2}$

\textbf{19.} a) $\SI{77}{s}$ \quad b) $\SI{238}{m}$

\textbf{20.} a) $\SI{4940}{m} \approx \SI{2,7}{NM}$ \quad b) $\SI{1286}{s} \approx \SI{21}{min}$

\textbf{21.} Voiture : $\SI{39}{m}$ ; Cargo : $\SI{52\,000}{m}$ (1300 fois plus!)

\textbf{22.} a) $\SI{-0,0106}{m/s^2}$ \quad b) $\SI{389}{s}$ \quad c) $\SI{234}{m}$ avant le quai

\textbf{23.} a) Faux (dépend du signe de $v$) \quad b) Vrai (ex: balle au sommet) \quad c) Vrai

\textbf{24.} a) $\SI{651}{s} \approx \SI{10,9}{min}$ \quad b) Remorqueur : $\SI{2,68}{km}$ ; Cargo : $\SI{3,18}{km}$ \quad c) $\SI{9,8}{m/s} \approx \SI{19}{n\oe{}uds}$

\medskip
\textbf{Chute libre}

\textbf{25.} a) $\SI{1,92}{s}$ \quad b) $\SI{18,8}{m/s}$

\textbf{26.} a) $\SI{1,05}{s}$ \quad b) $\SI{13,3}{m/s}$

\textbf{27.} a) $\SI{36,9}{m}$ \quad b) $\SI{5,3}{s}$

\textbf{28.} La balle B (lancée) a une plus grande vitesse car elle a eu le temps d'accélérer depuis le sol

\medskip
\textbf{Mouvement en 2D et projectile}

\textbf{29.} a) $\SI{13,8}{n\oe{}uds}$ \quad b) $\SI{11,6}{n\oe{}uds}$

\textbf{30.} a) $v_{0x} = \SI{19,2}{m/s}$, $v_{0y} = \SI{16,1}{m/s}$ \quad b) $\SI{69}{m}$

\textbf{31.} $\SI{3,5}{m}$

\textbf{32.} a) $\SI{61,2}{m}$ \quad b) $\SI{141}{m}$ \quad c) $45°$

\textbf{33.} Le mouvement vertical est indépendant du mouvement horizontal; les deux subissent la même accélération $g$

\medskip
\textbf{Cinématique de rotation}

\textbf{34.} a) $\frac{3\pi}{2} \approx \SI{4,71}{rad}$ \quad b) $135°$ \quad c) $\SI{9,42}{rad/s}$

\textbf{35.} $\SI{31,4}{m/s} \approx \SI{113}{km/h}$

\textbf{36.} a) $8,5$ tours \quad b) $\SI{20,4}{s}$

\textbf{37.} a) $\SI{0,628}{rad/s^2}$ \quad b) $20$ tours

\textbf{38.} a) $\SI{-0,221}{rad/s^2}$ \quad b) $\SI{10,1}{m}$ \quad c) $\SI{21,3}{s}$

\medskip
\textbf{Problèmes de synthèse}

\textbf{S1.} a) $\SI{1890}{m}$ \quad b) $\SI{118}{m}$ \quad c) Non, s'arrête à $\SI{8}{m}$ avant; aurait dû commencer à $\SI{2008}{m}$

\textbf{S2.} a) $\SI{324}{m}$ \quad b) $\SI{173}{m}$ \quad c) $\SI{497}{m}$ \quad d) $\SI{521}{m}$

\textbf{S3.} a) $\approx 25°$ \quad b) Oui (portée réduite à $\SI{59}{m}$, suffisant) \quad c) $\approx 40°$ (car lancé d'une hauteur)

\textbf{S4.} a) $\SI{0,628}{rad/s^2}$ \quad b) $\SI{0,785}{m/s}$ \quad c) $\SI{0,98}{m}$ \quad d) $\SI{1,56}{s}$

\textbf{S5.} a) $\SI{194}{s} \approx \SI{3,2}{min}$ \quad b) $A$: $\SI{1764}{m}$, $B$: $\SI{2117}{m}$ \quad c) Oui, collision; $v_{rel} \approx \SI{4,6}{m/s}$

\textbf{S6.} a) $\SI{96}{NM}$ \quad b) $\SI{1029}{s} \approx \SI{17}{min}$ \quad c) $\approx$ 17h30 \quad d) Graphique trapézoïdal

\medskip
\textbf{Défis intégrateurs}

\textbf{D1.} a) $\SI{71,4}{m}$ \quad b) $v = \SI{37,5}{m/s}$ à $41,8°$ sous l'horizontale \quad c) Réduit de $\SI{5,7}{m}$ (larguer à $\SI{65,7}{m}$) \quad d) $\SI{63,9}{m}$

\textbf{D2.} a) $x_A = \frac{1}{2}(0,02)t^2$; $x_B = 5000 - 4,12t$ \quad b) $t = \SI{650}{s}$, $x = \SI{4220}{m}$ \quad c) $\SI{13}{m/s} = \SI{25,3}{n\oe{}uds}$ \quad d) $v_{A/B} = \SI{17,1}{m/s}$ vers l'est (A s'approche rapidement de B) \quad e) Commencer à freiner à $t = \SI{217}{s}$

\textbf{D3.} a) $\Delta y = \SI{+5}{m}$ (tire vers le bas) \quad b) $\theta_{min} \approx 28°$ \quad c) $t \approx \SI{1,4}{s}$ \quad d) $\theta_{min}$ préférable (trajectoire plus tendue, moins affectée par le vent) \quad e) Portée réduite de $\SI{0,7}{m}$

\textbf{D4.} a) Phase 1: $\SI{309}{s}$, $\SI{476}{m}$; Phase 2: $\SI{900}{s}$, $\SI{2778}{m}$; Phase 3: $\SI{386}{s}$, $\SI{1736}{m}$ \quad b) $\SI{5010}{m}$ restants, $\SI{811}{s}$ \quad c) $\SI{2406}{s} \approx \SI{40}{min}$ \quad d) Voir graphique \quad e) $v_{moy} = \SI{15,0}{km/h} = \SI{8,1}{n\oe{}uds}$ (inférieure à $\SI{9}{n\oe{}uds}$, moyenne arithmétique)

\textbf{D5.} a) $\omega_A = \SI{2}{rad/s}$, $v_A = \SI{0,4}{m/s}$ \quad b) $v_B = \SI{0,6}{m/s}$ \quad c) $v_{cont} = \SI{0,5}{m/s}$ \quad d) $L_A = \SI{0,8}{m}$ \quad e) $L_B = \SI{2,4}{m}$ \quad f) $h = \SI{1,33}{m}$ (intégration de la vitesse moyenne)

\textbf{D6.} a) S se déplace; quand P atteint la position initiale de S, celui-ci est déjà ailleurs \quad b) $x_P = v_P(t)\sin\theta \cdot t$; $y_P = v_P(t)\cos\theta \cdot t$; $x_S = 7,72t$; $y_S = 8000$ \quad c) $x_P = x_S$ et $y_P = y_S$ \quad d) $\theta \approx 50°$ est du nord, $t \approx \SI{620}{s}$ \quad e) $\approx \SI{6,2}{km}$

\textbf{D7.} a) $\SI{2,02}{s}$ \quad b) $\SI{6,06}{m}$ \quad c) $\SI{20,0}{m/s}$ \quad d) Non! Le conteneur atterrit à $\SI{6,06}{m}$, le travailleur à $\SI{8}{m}$ est en sécurité même sans bouger \quad e) Zone de $\SI{6}{m} \times \SI{2,4}{m}$ centrée à $\SI{6,06}{m}$ du point de rupture (de $\SI{3,06}{m}$ à $\SI{9,06}{m}$)

\textbf{D8.} a) $v_c(y) = \SI{3}{nd} \times \sin\left(\frac{\pi y}{\SI{4}{km}}\right)$ ou approximation triangulaire \quad b) Dérive maximale au centre où le courant est maximal, puis le courant diminue mais ne ramène pas le navire \quad c) $\approx \SI{780}{m}$ vers l'est \quad d) $\SI{24}{min}$ \quad e) $\approx 5°$ à $7°$ ouest du nord (solution itérative)
