% =============================================================================
% EXERCICES - CHAPITRE 3 : ÉNERGIE ET TRAVAIL
% =============================================================================

\section{Exercices}
\label{sec:exercices_ch3}
% =============================================================================

\subsection{Travail}

\begin{enumerate}[label=\textbf{3.\arabic*}]

\item Un matelot pousse une caisse de $\SI{40}{kg}$ sur le pont avec une force horizontale de $\SI{120}{N}$ sur une distance de $\SI{5}{m}$. Calculez le travail effectué par :
\begin{enumerate}[label=\alph*)]
    \item la force de poussée
    \item la gravité
    \item la force normale
\end{enumerate}

\item Une grue exerce une tension de $\SI{25000}{N}$ sur un câble incliné à $15°$ par rapport à la verticale pour soulever un conteneur de $\SI{6}{m}$. Quel travail la tension effectue-t-elle?

\item Un remorqueur tire un chaland avec une force de $\SI{80}{kN}$. Le câble fait un angle de $25°$ avec la direction du mouvement. Quel travail est effectué sur une distance de $\SI{500}{m}$?

\item Un navire avance de $\SI{2}{km}$ sous l'effet d'une force de propulsion de $\SI{600}{kN}$. Pendant ce temps, la force de résistance de l'eau est de $\SI{550}{kN}$. Calculez :
\begin{enumerate}[label=\alph*)]
    \item le travail de la force de propulsion
    \item le travail de la résistance
    \item le travail total
\end{enumerate}

\item Un conteneur de $\SI{10000}{kg}$ est soulevé de $\SI{8}{m}$ par une grue, puis déplacé horizontalement de $\SI{15}{m}$, et finalement descendu de $\SI{3}{m}$. Quel est le travail total effectué par la gravité?

\end{enumerate}

\subsection{Énergie cinétique et théorème de l'énergie cinétique}

\begin{enumerate}[label=\textbf{3.\arabic*}]
\setcounter{enumi}{5}

\item Calculez l'énergie cinétique de :
\begin{enumerate}[label=\alph*)]
    \item un canot de sauvetage de $\SI{250}{kg}$ se déplaçant à $\SI{8}{m/s}$
    \item un pétrolier de $\SI{200000}{tonnes}$ naviguant à $\SI{8}{n\oe uds}$ ($\approx \SI{4,1}{m/s}$)
    \item un matelot de $\SI{75}{kg}$ courant à $\SI{5}{m/s}$
\end{enumerate}

\item Un navire de $\SI{50000}{tonnes}$ navigue à $\SI{12}{n\oe uds}$ ($\approx \SI{6,2}{m/s}$). Les moteurs sont coupés et le navire s'arrête après avoir parcouru $\SI{1,5}{km}$. Quelle est la force moyenne de résistance?

\item Une voiture de $\SI{1200}{kg}$ roule à $\SI{25}{m/s}$ sur une route horizontale. Le conducteur freine et la voiture s'arrête après $\SI{50}{m}$. Quelle est la force de freinage moyenne?

\item Un projectile de masse $\SI{0,02}{kg}$ est lancé avec une vitesse initiale de $\SI{300}{m/s}$. Il pénètre dans une cible et s'arrête après $\SI{0,15}{m}$. Quelle est la force moyenne exercée par la cible sur le projectile?

\item Un ascenseur de navire ($m = \SI{2000}{kg}$ avec passagers) accélère vers le haut de $v_i = 0$ à $v_f = \SI{3}{m/s}$ sur une distance de $\SI{4}{m}$. Le câble exerce une tension de $\SI{24000}{N}$. Vérifiez ce résultat à l'aide du théorème de l'énergie cinétique.

\end{enumerate}

\subsection{Énergie potentielle}

\begin{enumerate}[label=\textbf{3.\arabic*}]
\setcounter{enumi}{10}

\item Une ancre de $\SI{1500}{kg}$ est suspendue à $\SI{20}{m}$ au-dessus du fond marin. Calculez son énergie potentielle gravitationnelle par rapport au fond.

\item Un ressort de constante $k = \SI{2000}{N/m}$ est comprimé de $\SI{0,08}{m}$. Calculez :
\begin{enumerate}[label=\alph*)]
    \item l'énergie potentielle élastique stockée
    \item la force exercée par le ressort à cette compression
\end{enumerate}

\item Un conteneur de $\SI{8000}{kg}$ est déplacé par une grue de la position A (hauteur $\SI{2}{m}$) à la position B (hauteur $\SI{10}{m}$). Calculez la variation d'énergie potentielle gravitationnelle.

\item L'eau d'un réservoir de navire ($\SI{5000}{kg}$) est située à $\SI{12}{m}$ au-dessus du niveau de la mer. Calculez l'énergie potentielle gravitationnelle de cette eau. Si cette eau s'écoulait jusqu'au niveau de la mer, quelle serait l'énergie cinétique maximale qu'elle pourrait acquérir (en négligeant les pertes)?

\end{enumerate}

\subsection{Conservation de l'énergie mécanique}

\begin{enumerate}[label=\textbf{3.\arabic*}]
\setcounter{enumi}{14}

\item Un conteneur de $\SI{6000}{kg}$ tombe d'une hauteur de $\SI{5}{m}$. En négligeant la résistance de l'air :
\begin{enumerate}[label=\alph*)]
    \item Quelle est sa vitesse juste avant l'impact?
    \item Quelle est son énergie cinétique à ce moment?
\end{enumerate}

\item Une bille de $\SI{0,2}{kg}$ est lâchée du sommet d'une rampe sans frottement de hauteur $\SI{1,5}{m}$. La rampe se termine par une section horizontale.
\begin{enumerate}[label=\alph*)]
    \item Quelle est la vitesse de la bille au bas de la rampe?
    \item Si la bille remonte ensuite une autre rampe, quelle hauteur maximale atteindra-t-elle?
\end{enumerate}

\item Un pendule de longueur $\SI{0,8}{m}$ est écarté de $45°$ puis lâché. Calculez :
\begin{enumerate}[label=\alph*)]
    \item la hauteur initiale au-dessus du point le plus bas
    \item la vitesse au point le plus bas
\end{enumerate}

\item Un skieur nautique de $\SI{70}{kg}$ est tracté à $\SI{15}{m/s}$ lorsqu'il atteint une rampe inclinée à $20°$. S'il lâche la corde au bas de la rampe et monte sans frottement, quelle hauteur maximale atteindra-t-il?

\item Un bloc de $\SI{2}{kg}$ est projeté vers le haut d'un plan incliné sans frottement ($30°$) avec une vitesse initiale de $\SI{6}{m/s}$. Quelle distance parcourt-il le long du plan avant de s'arrêter momentanément?

\end{enumerate}

\subsection{Énergie avec forces non-conservatives}

\begin{enumerate}[label=\textbf{3.\arabic*}]
\setcounter{enumi}{19}

\item Une caisse de $\SI{30}{kg}$ glisse sur $\SI{6}{m}$ le long d'un plan incliné à $25°$. Le coefficient de frottement cinétique est $\mu_c = 0,20$. Si la caisse part du repos, quelle est sa vitesse au bas?

\item Un navire de $\SI{20000}{tonnes}$ ralentit de $\SI{8}{m/s}$ à $\SI{4}{m/s}$ sur une distance de $\SI{400}{m}$. Calculez :
\begin{enumerate}[label=\alph*)]
    \item la variation d'énergie cinétique
    \item la force moyenne de résistance
\end{enumerate}

\item Un traîneau de $\SI{50}{kg}$ est tiré sur $\SI{20}{m}$ sur une surface horizontale avec une force de $\SI{150}{N}$ (parallèle au sol). Le coefficient de frottement est $\mu_c = 0,25$. Si le traîneau part du repos, quelle est sa vitesse finale?

\item Un skieur de $\SI{75}{kg}$ descend une pente de $\SI{100}{m}$ de long inclinée à $15°$. Sa vitesse passe de $\SI{5}{m/s}$ à $\SI{18}{m/s}$. Calculez :
\begin{enumerate}[label=\alph*)]
    \item le travail effectué par la gravité
    \item le travail effectué par le frottement
    \item le coefficient de frottement cinétique
\end{enumerate}

\item Une balle de $\SI{0,5}{kg}$ est lancée verticalement vers le haut avec une vitesse de $\SI{20}{m/s}$. Elle atteint une hauteur maximale de $\SI{18}{m}$ (au lieu des $\SI{20,4}{m}$ théoriques sans résistance de l'air). Calculez le travail effectué par la résistance de l'air pendant la montée.

\end{enumerate}

\subsection{Puissance}

\begin{enumerate}[label=\textbf{3.\arabic*}]
\setcounter{enumi}{24}

\item Une grue soulève un conteneur de $\SI{15000}{kg}$ de $\SI{10}{m}$ en $\SI{30}{s}$. Calculez :
\begin{enumerate}[label=\alph*)]
    \item le travail effectué
    \item la puissance moyenne en watts et en chevaux-vapeur
\end{enumerate}

\item Un remorqueur exerce une force de $\SI{200}{kN}$ sur un navire qu'il pousse à $\SI{3}{m/s}$. Quelle puissance le remorqueur fournit-il?

\item Les moteurs d'un ferry ont une puissance de $\SI{12}{MW}$. Si le ferry navigue à $\SI{20}{n\oe uds}$ ($\approx \SI{10,3}{m/s}$), quelle est la force de propulsion?

\item Un cycliste de $\SI{80}{kg}$ (vélo inclus) monte une côte à $15°$ à vitesse constante de $\SI{4}{m/s}$. Quelle puissance développe-t-il?

\item Un ascenseur de $\SI{1500}{kg}$ (avec passagers) monte à $\SI{2}{m/s}$. Le moteur a une puissance de $\SI{40}{kW}$. Quel est le rendement du système?

\item Un navire consomme une puissance de $\SI{15}{MW}$ pendant une traversée de 6 heures.
\begin{enumerate}[label=\alph*)]
    \item Quelle énergie est consommée (en kWh et en GJ)?
    \item Si le carburant fournit $\SI{40}{MJ/L}$ et que le rendement est de 35\%, combien de litres sont consommés?
\end{enumerate}

\end{enumerate}

\subsection{Problèmes de synthèse}

\begin{enumerate}[label=\textbf{3.\arabic*}]
\setcounter{enumi}{30}

\item \textbf{(Problème de synthèse)} Un système de lancement de canot de sauvetage utilise un ressort ($k = \SI{15000}{N/m}$) comprimé de $\SI{0,4}{m}$. Le canot de $\SI{300}{kg}$ glisse ensuite le long d'une rampe inclinée à $20°$ sur $\SI{8}{m}$ avant d'atteindre l'eau. Le coefficient de frottement sur la rampe est $\mu_c = 0,10$.
\begin{enumerate}[label=\alph*)]
    \item Calculez l'énergie potentielle élastique initiale.
    \item Calculez le travail effectué par la gravité sur la rampe.
    \item Calculez le travail effectué par le frottement.
    \item Quelle est la vitesse du canot lorsqu'il atteint l'eau?
\end{enumerate}

\item \textbf{(Problème de synthèse)} Un vraquier de $\SI{40000}{tonnes}$ doit s'arrêter en urgence. Sa vitesse initiale est de $\SI{10}{n\oe uds}$ ($\approx \SI{5,1}{m/s}$). Les moteurs peuvent fournir une force de freinage (marche arrière) de $\SI{600}{kN}$, et la résistance de l'eau à cette vitesse est d'environ $\SI{400}{kN}$.
\begin{enumerate}[label=\alph*)]
    \item Quelle est l'énergie cinétique initiale du navire?
    \item Quelle est la force totale de freinage?
    \item Quelle distance le navire parcourt-il avant de s'arrêter?
    \item Combien de temps dure le freinage? (Supposez une décélération constante.)
\end{enumerate}

\item \textbf{(Problème de synthèse)} Une pompe de cale doit évacuer $\SI{500}{L}$ d'eau par minute d'une cale située $\SI{4}{m}$ sous le niveau de rejet.
\begin{enumerate}[label=\alph*)]
    \item Quelle masse d'eau est pompée par minute?
    \item Quel travail contre la gravité la pompe effectue-t-elle par minute?
    \item Quelle est la puissance minimale requise pour la pompe?
    \item Si le rendement de la pompe est de 60\%, quelle puissance électrique consomme-t-elle?
\end{enumerate}

\end{enumerate}

% =============================================================================
% RÉPONSES AUX EXERCICES (à placer en annexe ou feuille séparée)
% =============================================================================

\newpage
\subsection*{Réponses aux exercices}

\begin{multicols}{2}
\footnotesize
\textbf{3.1} a) 600 J, b) 0, c) 0

\textbf{3.2} 145 000 J

\textbf{3.3} 36,3 MJ

\textbf{3.4} a) 1200 MJ, b) $-1100$ MJ, c) 100 MJ

\textbf{3.5} $-490$ kJ

\textbf{3.6} a) 8000 J, b) 1,68 GJ, c) 937,5 J

\textbf{3.7} 64 kN

\textbf{3.8} 7500 N

\textbf{3.9} 6000 N

\textbf{3.10} $W_T = 96$ kJ, $\Delta K = 9$ kJ, vérifié

\textbf{3.11} 294 kJ

\textbf{3.12} a) 6,4 J, b) 160 N

\textbf{3.13} 627 kJ

\textbf{3.14} 588 kJ pour les deux

\textbf{3.15} a) 9,9 m/s, b) 294 kJ

\textbf{3.16} a) 5,4 m/s, b) 1,5 m

\textbf{3.17} a) 0,234 m, b) 2,1 m/s

\textbf{3.18} 11,5 m

\textbf{3.19} 3,67 m

\textbf{3.20} 5,6 m/s

\textbf{3.21} a) $-960$ MJ, b) 2,4 MN

\textbf{3.22} 6,6 m/s

\textbf{3.23} a) 19 kJ, b) $-6,4$ kJ, c) $\mu_c \approx 0,09$

\textbf{3.24} $-11,8$ J

\textbf{3.25} a) 1,47 MJ, b) 49 kW $\approx$ 67 ch

\textbf{3.26} 600 kW

\textbf{3.27} 1,17 MN

\textbf{3.28} 810 W

\textbf{3.29} 73,5\%

\textbf{3.30} a) 90 000 kWh = 324 GJ, b) 23 100 L

\textbf{3.31} a) 1200 J, b) 8000 J, c) $-277$ J, d) 7,6 m/s

\textbf{3.32} a) 520 MJ, b) 1 MN, c) 520 m, d) 204 s

\textbf{3.33} a) 500 kg/min, b) 19,6 kJ/min, c) 327 W, d) 545 W
\end{multicols}
