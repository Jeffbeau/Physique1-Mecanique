% =============================================================================
% EXERCICES - CHAPITRE 3
% Énergie, travail, quantité de mouvement et collisions
% 203-20B-QM  Physique 1 — Mécanique
% Institut Maritime du Québec
% =============================================================================

\section*{Exercices}
\addcontentsline{toc}{section}{Exercices}

\begin{remarque}[title=Niveaux de difficulté]
\begin{itemize}
    \item[$\star$] Application directe d'une formule
    \item[$\star\star$] Problème à plusieurs étapes ou avec conversion
    \item[$\star\star\star$] Problème complexe ou piège conceptuel
    \item[$\star\star\star\star$] Problème de synthèse intégrateur
\end{itemize}
\end{remarque}

% =============================================================================
% SECTION A — TRAVAIL ET THÉORÈME DE L'ÉNERGIE CINÉTIQUE
% =============================================================================
\subsection*{A — Travail et théorème de l'énergie cinétique}

% ---- EXERCICE 1 ----
\begin{enumerate}

\item[$\star$] \textbf{1.} Un remorqueur exerce une force de poussée de $F = \SI{150}{kN}$ sur un pétrolier, dans la direction du mouvement. Le pétrolier se déplace de $d = \SI{250}{m}$ sous l'effet de cette poussée.

\begin{enumerate}[label=\alph*)]
    \item Calculez le travail effectué par le remorqueur sur le pétrolier.
    \item Ce travail est-il positif ou négatif? Qu'est-ce que cela signifie physiquement?
\end{enumerate}


% ---- EXERCICE 2 ----
\item[$\star$] \textbf{2.} Pour chacune des situations suivantes, déterminez si le travail effectué par la force mentionnée est \textbf{positif}, \textbf{négatif} ou \textbf{nul}. Justifiez brièvement.

\begin{enumerate}[label=\alph*)]
    \item La force de propulsion d'un moteur sur un navire qui avance en ligne droite.
    \item La résistance de l'eau sur un navire qui avance.
    \item La force normale du pont sur un conteneur qui glisse horizontalement.
    \item La force gravitationnelle sur une ancre qu'on remonte vers le navire.
    \item La tension d'un câble de remorquage qui fait un angle de $20\si{\degree}$ avec la direction du mouvement du navire remorqué.
\end{enumerate}


% ---- EXERCICE 3 ----
\item[$\star\star$] \textbf{3.} Un matelot tire une caisse de matériel ($m = \SI{40}{kg}$) sur le pont d'un navire à l'aide d'une corde qui fait un angle de $\theta = 30\si{\degree}$ avec l'horizontale. La tension dans la corde est $T = \SI{200}{N}$. La caisse se déplace horizontalement sur une distance de $d = \SI{12}{m}$. Le coefficient de frottement cinétique entre la caisse et le pont est $\mu_c = 0{,}20$.

\begin{enumerate}[label=\alph*)]
    \item Calculez le travail effectué par la tension de la corde.
    \item Calculez la force normale, puis la force de frottement.
    \item Calculez le travail effectué par le frottement.
    \item Calculez le travail total effectué sur la caisse.
\end{enumerate}


% ---- EXERCICE 4 ----
\item[$\star\star$] \textbf{4.} Un conteneur de masse $m = \SI{500}{kg}$ est tiré vers le haut d'une rampe inclinée à $\theta = 25\si{\degree}$ par un câble parallèle à la rampe exercant une tension $T = \SI{4000}{N}$. Le conteneur parcourt une distance $d = \SI{8}{m}$ le long de la rampe. Le coefficient de frottement cinétique est $\mu_c = 0{,}15$.

\begin{enumerate}[label=\alph*)]
    \item Calculez le travail effectué par la tension.
    \item Calculez la variation de hauteur $\Delta y$ et le travail effectué par la gravité.
    \item Calculez le travail effectué par le frottement.
    \item Calculez le travail total effectué sur le conteneur. Le conteneur accélère-t-il?
\end{enumerate}


% ---- EXERCICE 5 ----
\item[$\star\star$] \textbf{5.} Un cargo de masse $m = \SI{20000}{tonnes}$ navigue à $v_i = \SI{10}{\knots}$ ($\approx \SI{5{,}14}{m/s}$). Le capitaine ordonne l'inversion des moteurs, qui exercent une force de freinage constante de $F = \SI{400}{kN}$.

\begin{enumerate}[label=\alph*)]
    \item Calculez l'énergie cinétique initiale du cargo.
    \item En utilisant le théorème de l'énergie cinétique, déterminez la distance de freinage.
    \item Exprimez cette distance en milles nautiques.
\end{enumerate}


\end{enumerate}

% =============================================================================
% SECTION B — ÉNERGIE POTENTIELLE ET CONSERVATION DE L'ÉNERGIE
% =============================================================================
\subsection*{B — Énergie potentielle et conservation de l'énergie}

% ---- EXERCICE 6 ----
\begin{enumerate}[resume]

\item[$\star$] \textbf{6.}
\begin{enumerate}[label=\alph*)]
    \item Une ancre de masse $m = \SI{1500}{kg}$ est suspendue à $\SI{20}{m}$ au-dessus du fond marin. Calculez son énergie potentielle gravitationnelle par rapport au fond marin.
    \item Un conteneur de masse $m = \SI{10000}{kg}$ est suspendu par une grue à $\SI{15}{m}$ au-dessus du quai. Calculez son énergie potentielle gravitationnelle par rapport au quai.
    \item La grue dépose le conteneur sur le pont du navire, à $\SI{5}{m}$ au-dessus du quai. Calculez la variation d'énergie potentielle $\Delta U_g$ et le travail effectué par la gravité lors de cette descente.
\end{enumerate}


% ---- EXERCICE 7 ----
\item[$\star\star$] \textbf{7.} Un outil de $m = \SI{2}{kg}$ tombe accidentellement du sommet d'un mât situé à $h = \SI{25}{m}$ au-dessus du pont. En négligeant la résistance de l'air :

\begin{enumerate}[label=\alph*)]
    \item Calculez la vitesse de l'outil juste avant qu'il ne touche le pont, en utilisant la conservation de l'énergie mécanique.
    \item Convertissez cette vitesse en km/h.
    \item La masse de l'outil influence-t-elle la vitesse d'impact? Justifiez.
\end{enumerate}


% ---- EXERCICE 8 ----
\item[$\star\star$] \textbf{8.} Un canot de sauvetage de masse $m = \SI{350}{kg}$ est lancé le long d'une glissière inclinée à $30\si{\degree}$ par rapport à l'horizontale. La glissière mesure $L = \SI{10}{m}$ de long. Le canot part du repos. En négligeant le frottement :

\begin{enumerate}[label=\alph*)]
    \item Calculez la hauteur $h$ du point de départ par rapport au point d'arrivée.
    \item En utilisant la conservation de l'énergie, déterminez la vitesse du canot au bas de la glissière.
    \item Si on veut que le canot atteigne une vitesse de $\SI{12}{m/s}$, quelle doit être la longueur minimale de la glissière?
\end{enumerate}


% ---- EXERCICE 9 ----
\item[$\star\star$] \textbf{9.} Un système de lancement de canot de sauvetage utilise un ressort de constante $k = \SI{5000}{N/m}$ comprimé de $x = \SI{0{,}40}{m}$.

\begin{enumerate}[label=\alph*)]
    \item Calculez l'énergie potentielle élastique emmagasinée dans le ressort.
    \item Si toute cette énergie est transférée à un chariot de $m = \SI{2}{kg}$, quelle sera sa vitesse?
    \item Quelle compression serait nécessaire pour doubler cette vitesse de lancement?
\end{enumerate}


% ---- EXERCICE 10 ----
\item[$\star\star\star$] \textbf{10.} Une caisse de matériel de masse $m = \SI{100}{kg}$ glisse du repos le long d'une rampe de chargement inclinée à $\theta = 30\si{\degree}$, sur une distance de $d = \SI{6}{m}$ le long de la rampe. Le coefficient de frottement cinétique est $\mu_c = 0{,}20$.

\begin{enumerate}[label=\alph*)]
    \item Calculez la hauteur de descente $h$.
    \item Calculez la force de frottement et le travail effectué par le frottement.
    \item En utilisant le bilan énergétique ($E_i + W_{nc} = E_f$), déterminez la vitesse au bas de la rampe.
    \item Quelle serait la vitesse sans frottement? Comparez.
\end{enumerate}


\end{enumerate}

% =============================================================================
% SECTION C — PUISSANCE
% =============================================================================
\subsection*{C — Puissance}

% ---- EXERCICE 11 ----
\begin{enumerate}[resume]

\item[$\star$] \textbf{11.} Une grue portuaire soulève un conteneur de masse $m = \SI{8000}{kg}$ sur une hauteur de $h = \SI{10}{m}$ en $t = \SI{25}{s}$.

\begin{enumerate}[label=\alph*)]
    \item Calculez le travail effectué contre la gravité.
    \item Calculez la puissance mécanique moyenne de la grue.
    \item Exprimez cette puissance en chevaux-vapeur ($\SI{1}{ch} = \SI{735{,}5}{W}$).
\end{enumerate}


% ---- EXERCICE 12 ----
\item[$\star\star$] \textbf{12.} Un cargo de \SI{25000}{tonnes} navigue à vitesse constante de $v = \SI{14}{\knots}$ ($\approx \SI{7{,}20}{m/s}$). La résistance totale (eau + air) est estimée à $F_r = \SI{650}{kN}$.

\begin{enumerate}[label=\alph*)]
    \item Quelle puissance les moteurs doivent-ils fournir pour maintenir cette vitesse?
    \item Exprimez cette puissance en MW et en chevaux-vapeur.
    \item Si le navire réduit sa vitesse à $\SI{10}{\knots}$ ($\approx \SI{5{,}14}{m/s}$) et que la résistance diminue à $\SI{350}{kN}$, quelle est la nouvelle puissance requise? Comparez avec la réponse en a).
\end{enumerate}


% ---- EXERCICE 13 ----
\item[$\star\star$] \textbf{13.} Un ferry consomme une puissance motrice moyenne de $P = \SI{6}{MW}$ pendant une traversée de $t = \SI{1{,}5}{h}$.

\begin{enumerate}[label=\alph*)]
    \item Calculez l'énergie consommée en MWh, en kWh et en MJ.
    \item Si le diesel marin fournit $\SI{42}{MJ}$ par litre et que le rendement global des moteurs est de $\eta = 38\%$, combien de litres de carburant sont consommés pendant la traversée?
    \item Si le ferry effectue 4 traversées par jour, quel est le coût quotidien en carburant si le diesel marin coûte $\SI{1{,}20}{\$/L}$?
\end{enumerate}


\end{enumerate}

% =============================================================================
% SECTION D — QUANTITÉ DE MOUVEMENT ET IMPULSION
% =============================================================================
\subsection*{D — Quantité de mouvement et impulsion}

% ---- EXERCICE 14 ----
\begin{enumerate}[resume]

\item[$\star$] \textbf{14.} Calculez la quantité de mouvement de chacun des navires suivants et classez-les par ordre décroissant de difficulté à les arrêter.

\begin{center}
\renewcommand{\arraystretch}{1.3}
\begin{tabular}{|l|c|c|}
\hline
\rowcolor{bleuclair}
\textbf{Navire} & \textbf{Masse} & \textbf{Vitesse} \\
\hline
Chalutier & $\SI{200}{tonnes}$ & $\SI{3}{m/s}$ \\
\hline
Traversier & $\SI{5000}{tonnes}$ & $\SI{8}{m/s}$ \\
\hline
Pétrolier (VLCC) & $\SI{150000}{tonnes}$ & $\SI{7}{m/s}$ \\
\hline
\end{tabular}
\end{center}


% ---- EXERCICE 15 ----
\item[$\star\star$] \textbf{15.} Un traversier de masse $m = \SI{10000}{tonnes}$ accoste un quai à une vitesse de $v_i = \SI{0{,}5}{m/s}$. Les défenses du quai l'immobilisent en $\Delta t = \SI{8}{s}$.

\begin{enumerate}[label=\alph*)]
    \item Calculez la variation de quantité de mouvement du traversier.
    \item Calculez la force moyenne exercée par les défenses sur le traversier.
    \item Si un accostage brutal réduit le temps de contact à $\SI{2}{s}$, quelle serait la force moyenne? Commentez.
\end{enumerate}


% ---- EXERCICE 16 ----
\item[$\star\star$] \textbf{16.} Un canon naval de masse $M = \SI{8000}{kg}$ tire un obus de masse $m = \SI{30}{kg}$ à une vitesse de $\SI{600}{m/s}$.

\begin{enumerate}[label=\alph*)]
    \item En utilisant la conservation de la quantité de mouvement, calculez la vitesse de recul du canon.
    \item Calculez l'énergie cinétique de l'obus et celle du canon. Comparez et expliquez la différence.
\end{enumerate}


% ---- EXERCICE 17 ----
\item[$\star\star$] \textbf{17.} Un marin de $\SI{75}{kg}$ se tient debout à la poupe d'un canot de $\SI{150}{kg}$, initialement au repos sur l'eau (frottement négligeable). Le marin marche vers la proue à $\SI{1{,}5}{m/s}$ par rapport au sol.

\begin{enumerate}[label=\alph*)]
    \item Quelle est la vitesse du canot par rapport au sol?
    \item Quelle est la vitesse du marin par rapport au canot?
\end{enumerate}


\end{enumerate}

% =============================================================================
% SECTION E — COLLISIONS
% =============================================================================
\subsection*{E — Collisions}

% ---- EXERCICE 18 ----
\begin{enumerate}[resume]

\item[$\star\star$] \textbf{18.} Une barge chargée ($m_A = \SI{400}{tonnes}$, $v_A = \SI{3}{m/s}$) rattrape une barge plus lourde ($m_B = \SI{600}{tonnes}$, $v_B = \SI{1}{m/s}$) se déplaçant dans la même direction. Les deux barges restent accrochées après le contact (collision parfaitement inélastique).

\begin{enumerate}[label=\alph*)]
    \item Calculez la vitesse commune après la collision.
    \item Calculez l'énergie cinétique avant et après la collision.
    \item Quel pourcentage de l'énergie cinétique a été perdu? Où est passée cette énergie?
\end{enumerate}


% ---- EXERCICE 19 ----
\item[$\star\star$] \textbf{19.} Un cargo ($m_1 = \SI{25000}{tonnes}$, $v_1 = \SI{4}{m/s}$ vers l'est) entre en collision frontale avec un vraquier ($m_2 = \SI{15000}{tonnes}$, $v_2 = \SI{2}{m/s}$ vers l'ouest). Les navires restent soudés après l'impact.

\begin{enumerate}[label=\alph*)]
    \item Définissez un axe et attribuez les signes appropriés aux vitesses.
    \item Calculez la vitesse de l'ensemble après la collision.
    \item Calculez le pourcentage d'énergie cinétique perdue.
    \item Si la collision dure $\Delta t = \SI{3}{s}$, calculez la force moyenne subie par le vraquier.
\end{enumerate}


% ---- EXERCICE 20 ----
\item[$\star\star\star$] \textbf{20.} Deux chariots sur rail se déplacent sur le pont d'un navire. Le chariot A ($m_A = \SI{200}{kg}$, $v_A = \SI{4}{m/s}$) frappe le chariot B ($m_B = \SI{100}{kg}$) initialement au repos. La collision est parfaitement \textbf{élastique}.

\begin{enumerate}[label=\alph*)]
    \item Calculez les vitesses des deux chariots après la collision en utilisant la conservation de $\vect{p}$ et la relation des vitesses relatives ($v_{1i} - v_{2i} = -(v_{1f} - v_{2f})$).
    \item Vérifiez que l'énergie cinétique est effectivement conservée.
\end{enumerate}


% ---- EXERCICE 21 ----
\item[$\star\star\star$] \textbf{21.} Un traversier de masse $m = \SI{5000}{tonnes}$ heurte un quai (considéré comme infiniment massif) à une vitesse de $v_i = \SI{0{,}8}{m/s}$. Le coefficient de restitution des défenses est $e = 0{,}40$.

\begin{enumerate}[label=\alph*)]
    \item Calculez la vitesse du traversier après le rebond.
    \item Calculez l'énergie cinétique avant et après le rebond.
    \item Quel pourcentage de l'énergie cinétique est absorbé par les défenses?
    \item Montrez que la fraction d'énergie conservée est $e^2$.
\end{enumerate}


% ---- EXERCICE 22 ----
\item[$\star\star\star$] \textbf{22.} Deux navires entrent en collision en pleine mer (collision parfaitement inélastique en 2D) :
\begin{itemize}
    \item Cargo : $m_1 = \SI{15000}{tonnes}$, $v_1 = \SI{5}{m/s}$, cap $090\si{\degree}$ (vers l'est)
    \item Traversier : $m_2 = \SI{5000}{tonnes}$, $v_2 = \SI{8}{m/s}$, cap $000\si{\degree}$ (vers le nord)
\end{itemize}

\begin{enumerate}[label=\alph*)]
    \item Choisissez un système d'axes et décomposez les quantités de mouvement initiales selon $x$ et $y$.
    \item Calculez les composantes $v_{fx}$ et $v_{fy}$ de la vitesse finale de l'ensemble.
    \item Calculez le module et la direction (cap) de la vitesse finale.
    \item Calculez le pourcentage d'énergie cinétique perdue lors de la collision.
\end{enumerate}


\end{enumerate}

% =============================================================================
% SECTION F — PROBLÈMES DE SYNTHÈSE (niveau examen)
% =============================================================================
\subsection*{F — Problèmes de synthèse}

% ---- EXERCICE 23 ----
\begin{enumerate}[resume]

\item[$\star\star\star\star$] \textbf{23.} \textbf{Opération de chargement au port}

Une grue portuaire soulève un conteneur de masse $m = \SI{10000}{kg}$ d'une hauteur de $h = \SI{12}{m}$ à vitesse constante de $v = \SI{0{,}3}{m/s}$. Le rendement mécanique de la grue est de $\eta = 70\%$.

\begin{enumerate}[label=\alph*)]
    \item Calculez le travail mécanique minimal effectué contre la gravité.
    \item Calculez l'énergie totale consommée par la grue, en tenant compte du rendement.
    \item Calculez la puissance mécanique fournie par la grue et la puissance électrique consommée.
    \item Le câble cède lorsque le conteneur est au sommet ($h = \SI{12}{m}$). Calculez la vitesse d'impact sur le quai.
    \item Si le conteneur s'immobilise en s'enfonçant de $d = \SI{0{,}50}{m}$ dans le quai, calculez la force moyenne d'impact. Combien de fois le poids du conteneur cela représente-t-il?
\end{enumerate}


% ---- EXERCICE 24 ----
\item[$\star\star\star\star$] \textbf{24.} \textbf{Accostage et amarrage}

Un traversier de masse $m = \SI{8000}{tonnes}$ approche un quai à $v_i = \SI{0{,}6}{m/s}$.

\begin{enumerate}[label=\alph*)]
    \item Calculez l'énergie cinétique du traversier à l'approche.
    \item Les défenses du quai ont un coefficient de restitution $e = 0{,}30$. Calculez la vitesse du traversier après le rebond et l'énergie absorbée par les défenses.
    \item Si la durée du contact avec les défenses est $\Delta t = \SI{4}{s}$, calculez la force moyenne exercée sur le traversier.
    \item Après le rebond, le traversier s'éloigne du quai. La résistance de l'eau exerce une force moyenne de $F_r = \SI{20}{kN}$. Quelle distance parcourt-il avant de s'immobiliser?
    \item Quelle puissance les moteurs du traversier devraient-ils fournir pour maintenir une vitesse de $\SI{0{,}6}{m/s}$ contre cette même résistance de $\SI{20}{kN}$?
\end{enumerate}


% ---- EXERCICE 25 ----
\item[$\star\star\star\star$] \textbf{25.} \textbf{Scénario d'abordage}

Un cargo ($m_1 = \SI{15000}{tonnes}$, cap $090\si{\degree}$, vitesse $\SI{5}{m/s}$) et un traversier ($m_2 = \SI{5000}{tonnes}$, cap $000\si{\degree}$, vitesse $\SI{8}{m/s}$) entrent en collision. Les navires restent enchevêtrés (collision parfaitement inélastique).

\begin{enumerate}[label=\alph*)]
    \item Calculez la vitesse (module et cap) de l'ensemble après la collision.
    \item Calculez l'énergie cinétique perdue. Identifiez les formes sous lesquelles cette énergie s'est transformée.
    \item Si la collision dure $\Delta t = \SI{2}{s}$, calculez le module de la force moyenne subie par le traversier pendant l'impact.
    \item Après la collision, la résistance de l'eau exerce une force de $F_r = \SI{100}{kN}$ sur l'ensemble. En utilisant le théorème de l'énergie cinétique, calculez la distance parcourue avant l'arrêt complet.
    \item Convertissez cette distance en milles nautiques et commentez les implications pour les opérations de sauvetage.
\end{enumerate}


\end{enumerate}

% =============================================================================
% RÉPONSES
% =============================================================================
\newpage
\subsection*{Réponses}

\textbf{Section A — Travail et théorème de l'énergie cinétique}


\textbf{1.} a) $W = Fd = \SI{150e3}{N} \times \SI{250}{m} = \SI{37{,}5}{MJ}$ \quad b) Positif : le remorqueur ajoute de l'énergie au pétrolier.

\textbf{2.} a) Positif ($\theta = 0\si{\degree}$). \quad b) Négatif ($\theta = 180\si{\degree}$). \quad c) Nul ($\theta = 90\si{\degree}$). \quad d) Négatif (force vers le bas, déplacement vers le haut, $\theta = 180\si{\degree}$). \quad e) Positif ($\cos 20\si{\degree} > 0$).

\textbf{3.} a) $W_T = Td\cos\theta = 200 \times 12 \times \cos 30\si{\degree} = 200 \times 12 \times 0{,}866 \approx \SI{2078}{J}$
    
    b) $N = mg - T\sin\theta = 40 \times 9{,}8 - 200 \times 0{,}5 = 392 - 100 = \SI{292}{N}$; \quad $f = \mu_c N = 0{,}20 \times 292 = \SI{58{,}4}{N}$
    
    c) $W_f = -fd = -58{,}4 \times 12 = \SI{-701}{J}$
    
    d) $W_{total} = 2078 - 701 = \SI{+1377}{J}$ (la caisse accélère)

\textbf{4.} a) $W_T = Td = 4000 \times 8 = \SI{32\,000}{J}$
    
    b) $\Delta y = d\sin 25\si{\degree} = 8 \times 0{,}4226 = \SI{3{,}38}{m}$; \quad $W_g = -mg\Delta y = -500 \times 9{,}8 \times 3{,}38 = \SI{-16\,562}{J}$
    
    c) $N = mg\cos 25\si{\degree} = 500 \times 9{,}8 \times 0{,}9063 = \SI{4441}{N}$; \quad $f = 0{,}15 \times 4441 = \SI{666}{N}$; \quad $W_f = -666 \times 8 = \SI{-5330}{J}$
    
    d) $W_{total} = 32\,000 - 16\,562 - 5330 = \SI{+10\,108}{J}$ (le conteneur accélère)

\textbf{5.} a) $K_i = \frac{1}{2}mv_i^2 = \frac{1}{2} \times \SI{20e6}{kg} \times (5{,}14)^2 = \SI{264}{MJ}$
    
    b) $W_{total} = \Delta K \Rightarrow -Fd = 0 - K_i \Rightarrow d = \dfrac{K_i}{F} = \dfrac{\SI{264e6}{}}{\SI{400e3}{}} = \SI{660}{m}$
    
    c) $d = \SI{660}{m} \div 1852 \approx \SI{0{,}36}{NM}$


\medskip
\textbf{Section B — Énergie potentielle et conservation de l'énergie}


\textbf{6.} a) $U_g = mgy = 1500 \times 9{,}8 \times 20 = \SI{294}{kJ}$
    
    b) $U_g = 10\,000 \times 9{,}8 \times 15 = \SI{1{,}47}{MJ}$
    
    c) $\Delta U_g = mg(y_f - y_i) = 10\,000 \times 9{,}8 \times (5 - 15) = \SI{-980}{kJ}$; \quad $W_g = -\Delta U_g = \SI{+980}{kJ}$

\textbf{7.} a) $\frac{1}{2}mv^2 = mgh \Rightarrow v = \sqrt{2gh} = \sqrt{2 \times 9{,}8 \times 25} = \sqrt{490} = \SI{22{,}1}{m/s}$
    
    b) $v = 22{,}1 \times 3{,}6 = \SI{79{,}6}{km/h}$
    
    c) Non, la masse s'annule dans $v = \sqrt{2gh}$.

\textbf{8.} a) $h = L\sin 30\si{\degree} = 10 \times 0{,}5 = \SI{5}{m}$
    
    b) $v = \sqrt{2gh} = \sqrt{2 \times 9{,}8 \times 5} = \sqrt{98} = \SI{9{,}9}{m/s}$
    
    c) $\frac{1}{2}mv^2 = mg L\sin 30\si{\degree} \Rightarrow L = \dfrac{v^2}{2g\sin 30\si{\degree}} = \dfrac{144}{2 \times 9{,}8 \times 0{,}5} = \SI{14{,}7}{m}$

\textbf{9.} a) $U_e = \frac{1}{2}kx^2 = \frac{1}{2} \times 5000 \times (0{,}40)^2 = \SI{400}{J}$
    
    b) $\frac{1}{2}mv^2 = U_e \Rightarrow v = \sqrt{2U_e/m} = \sqrt{2 \times 400 / 2} = \sqrt{400} = \SI{20}{m/s}$
    
    c) $v \propto x$, donc pour doubler $v$, il faut doubler $x$ : $x = \SI{0{,}80}{m}$.

\textbf{10.} a) $h = d\sin 30\si{\degree} = 6 \times 0{,}5 = \SI{3}{m}$
    
    b) $N = mg\cos 30\si{\degree} = 100 \times 9{,}8 \times 0{,}866 = \SI{849}{N}$; \quad $f = 0{,}20 \times 849 = \SI{170}{N}$; \quad $W_f = -170 \times 6 = \SI{-1020}{J}$
    
    c) $mgh + W_f = \frac{1}{2}mv^2 \Rightarrow 100 \times 9{,}8 \times 3 - 1020 = 50\,v^2 \Rightarrow 1920 = 50\,v^2 \Rightarrow v = \SI{6{,}2}{m/s}$
    
    d) $v = \sqrt{2gh} = \sqrt{58{,}8} = \SI{7{,}7}{m/s}$. Le frottement réduit la vitesse d'environ 19\%.


\medskip
\textbf{Section C — Puissance}


\textbf{11.} a) $W = mgh = 8000 \times 9{,}8 \times 10 = \SI{784\,000}{J} = \SI{784}{kJ}$
    
    b) $P = W/t = 784\,000/25 = \SI{31{,}4}{kW}$
    
    c) $P = 31\,360/735{,}5 \approx \SI{42{,}6}{ch}$

\textbf{12.} a) $P = F_r \times v = \SI{650e3}{} \times 7{,}20 = \SI{4{,}68}{MW}$
    
    b) $\SI{4{,}68}{MW} \approx \SI{6360}{ch}$
    
    c) $P' = 350\,000 \times 5{,}14 = \SI{1{,}80}{MW} \approx \SI{2450}{ch}$. Réduire la vitesse de 30\% diminue la puissance requise de 62\%.

\textbf{13.} a) $E = 6 \times 1{,}5 = \SI{9}{MWh} = \SI{9000}{kWh} = 9000 \times 3{,}6 = \SI{32\,400}{MJ}$
    
    b) Énergie carburant nécessaire $= 32\,400/0{,}38 = \SI{85\,263}{MJ}$; \quad $V = 85\,263/42 \approx \SI{2030}{L}$
    
    c) Coût quotidien $= 4 \times 2030 \times 1{,}20 = \SI{9744}{\$}$


\medskip
\textbf{Section D — Quantité de mouvement et impulsion}


\textbf{14.} Chalutier : $p = 200\,000 \times 3 = \SI{6{,}0e5}{kg \cdot m/s}$; \quad Traversier : $p = \SI{5e6}{} \times 8 = \SI{4{,}0e7}{kg \cdot m/s}$; \quad Pétrolier : $p = \SI{150e6}{} \times 7 = \SI{1{,}05e9}{kg \cdot m/s}$. 
    
    Ordre : Pétrolier $\gg$ Traversier $\gg$ Chalutier.

\textbf{15.} a) $\Delta p = mv_f - mv_i = 0 - \SI{10e6}{} \times 0{,}5 = \SI{-5{,}0e6}{kg \cdot m/s}$
    
    b) $F_{moy} = \Delta p / \Delta t = -5{,}0 \times 10^6 / 8 = \SI{-625}{kN}$; $|\vect{F}| = \SI{625}{kN}$
    
    c) $F_{moy} = -5{,}0 \times 10^6 / 2 = \SI{-2{,}5}{MN}$. La force est 4 fois plus grande! L'amortissement réduit les forces d'impact.

\textbf{16.} a) $0 = m v_{obus} + M v_{canon} \Rightarrow v_{canon} = -\dfrac{30 \times 600}{8000} = \SI{-2{,}25}{m/s}$ (recul)
    
    b) $K_{obus} = \frac{1}{2}(30)(600)^2 = \SI{5{,}40}{MJ}$; \quad $K_{canon} = \frac{1}{2}(8000)(2{,}25)^2 = \SI{20{,}3}{kJ}$. 
    
    L'obus reçoit 99{,}6\% de l'énergie : à quantité de mouvement égale, l'objet le plus léger a la plus grande énergie cinétique ($K = p^2/2m$).

\textbf{17.} a) $0 = 75 \times 1{,}5 + 150 \times v_{canot} \Rightarrow v_{canot} = \SI{-0{,}75}{m/s}$ (vers la poupe)
    
    b) $v_{rel} = 1{,}5 - (-0{,}75) = \SI{2{,}25}{m/s}$


\medskip
\textbf{Section E — Collisions}


\textbf{18.} a) $v_f = \dfrac{m_A v_A + m_B v_B}{m_A + m_B} = \dfrac{400\,000 \times 3 + 600\,000 \times 1}{1\,000\,000} = \SI{1{,}8}{m/s}$
    
    b) $K_i = \frac{1}{2}(400\,000)(9) + \frac{1}{2}(600\,000)(1) = 1800 + 300 = \SI{2100}{kJ}$; \quad $K_f = \frac{1}{2}(1\,000\,000)(1{,}8)^2 = \SI{1620}{kJ}$
    
    c) $\%_{\text{perdu}} = \dfrac{2100 - 1620}{2100} \times 100 = 22{,}9\%$. Énergie transformée en déformation des structures et chaleur.

\textbf{19.} a) Axe $x$ positif vers l'est : $v_1 = +\SI{4}{m/s}$, $v_2 = \SI{-2}{m/s}$.
    
    b) $v_f = \dfrac{25 \times 10^6 \times 4 + 15 \times 10^6 \times (-2)}{40 \times 10^6} = \dfrac{100 - 30}{40} = \SI{+1{,}75}{m/s}$ (vers l'est)
    
    c) $K_i = \frac{1}{2}(25 \times 10^6)(16) + \frac{1}{2}(15 \times 10^6)(4) = 200 + 30 = \SI{230}{MJ}$

    $K_f = \frac{1}{2}(40 \times 10^6)(1{,}75)^2 = \SI{61{,}3}{MJ}$; \quad $\%_{\text{perdu}} = \dfrac{230 - 61{,}3}{230} \times 100 = 73{,}3\%$
    
    d) $\Delta p_{vr} = m_2(v_f - v_{2i}) = 15 \times 10^6 \times (1{,}75 - (-2)) = 15 \times 10^6 \times 3{,}75 = \SI{56{,}3 \times 10^6}{N \cdot s}$
    
    $F_{moy} = 56{,}3 \times 10^6 / 3 = \SI{18{,}8}{MN}$

\textbf{20.} a) Conservation de $\vect{p}$ : $200 \times 4 + 100 \times 0 = 200\,v_{Af} + 100\,v_{Bf}$ \quad $\Rightarrow 800 = 200\,v_{Af} + 100\,v_{Bf}$ \quad (1)
    
    Vitesses relatives : $v_{Ai} - v_{Bi} = -(v_{Af} - v_{Bf}) \Rightarrow 4 - 0 = -(v_{Af} - v_{Bf}) \Rightarrow v_{Bf} - v_{Af} = 4$ \quad (2)
    
    De (2) : $v_{Bf} = v_{Af} + 4$. Dans (1) : $800 = 200\,v_{Af} + 100(v_{Af} + 4) = 300\,v_{Af} + 400$
    
    $v_{Af} = 400/300 = \SI{1{,}33}{m/s}$; \quad $v_{Bf} = 1{,}33 + 4 = \SI{5{,}33}{m/s}$
    
    b) $K_i = \frac{1}{2}(200)(16) = \SI{1600}{J}$; \quad $K_f = \frac{1}{2}(200)(1{,}33)^2 + \frac{1}{2}(100)(5{,}33)^2 = 177 + 1420 = \SI{1597}{J} \approx \SI{1600}{J}$ \checkmark

\textbf{21.} a) Contre un quai ($M \to \infty$) : $v_f = -e \cdot v_i = -0{,}40 \times 0{,}8 = \SI{-0{,}32}{m/s}$ (rebond)
    
    b) $K_i = \frac{1}{2}(5 \times 10^6)(0{,}8)^2 = \SI{1600}{kJ}$; \quad $K_f = \frac{1}{2}(5 \times 10^6)(0{,}32)^2 = \SI{256}{kJ}$
    
    c) $\%_{\text{absorbé}} = \dfrac{1600 - 256}{1600} \times 100 = 84\%$
    
    d) $\dfrac{K_f}{K_i} = \dfrac{\frac{1}{2}m(ev_i)^2}{\frac{1}{2}mv_i^2} = e^2 = (0{,}40)^2 = 0{,}16$. Donc fraction conservée $= e^2$, fraction perdue $= 1 - e^2 = 84\%$ \checkmark

\textbf{22.} Axe $x$ vers l'est, axe $y$ vers le nord.
    
    a) $p_{xi} = 15 \times 10^6 \times 5 = \SI{75 \times 10^6}{kg \cdot m/s}$; \quad $p_{yi} = 5 \times 10^6 \times 8 = \SI{40 \times 10^6}{kg \cdot m/s}$
    
    b) $v_{fx} = \dfrac{75 \times 10^6}{20 \times 10^6} = \SI{3{,}75}{m/s}$; \quad $v_{fy} = \dfrac{40 \times 10^6}{20 \times 10^6} = \SI{2{,}0}{m/s}$
    
    c) $v_f = \sqrt{3{,}75^2 + 2{,}0^2} = \sqrt{18{,}06} = \SI{4{,}25}{m/s}$; \quad $\theta = \arctan(2{,}0 / 3{,}75) = 28{,}1\si{\degree}$ nord de l'est; \quad Cap $\approx 062\si{\degree}$
    
    d) $K_i = \frac{1}{2}(15 \times 10^6)(25) + \frac{1}{2}(5 \times 10^6)(64) = 187{,}5 + 160 = \SI{347{,}5}{MJ}$
    
    $K_f = \frac{1}{2}(20 \times 10^6)(18{,}06) = \SI{180{,}6}{MJ}$; \quad $\%_{\text{perdu}} = \dfrac{347{,}5 - 180{,}6}{347{,}5} \times 100 \approx 48\%$


\medskip
\textbf{Section F — Problèmes de synthèse}


\textbf{23.} a) $W = mgh = 10\,000 \times 9{,}8 \times 12 = \SI{1{,}18}{MJ}$
    
    b) $E_{total} = W/\eta = 1{,}18/0{,}70 = \SI{1{,}68}{MJ}$
    
    c) $t = h/v = 12/0{,}3 = \SI{40}{s}$; \quad $P_{m\acute{e}c} = mgh/t = 1\,176\,000/40 = \SI{29{,}4}{kW}$; \quad $P_{\acute{e}lec} = 29{,}4/0{,}70 = \SI{42{,}0}{kW}$
    
    d) $v = \sqrt{2gh} = \sqrt{2 \times 9{,}8 \times 12} = \sqrt{235{,}2} = \SI{15{,}3}{m/s}$ ($\approx \SI{55}{km/h}$)
    
    e) $Fd = \frac{1}{2}mv^2 = mgh \Rightarrow F = mgh/d = 10\,000 \times 9{,}8 \times 12 / 0{,}50 = \SI{2{,}35}{MN}$. Ratio : $F/(mg) = 2\,352\,000/98\,000 = 24$ fois le poids.

\textbf{24.} a) $K_i = \frac{1}{2}mv_i^2 = \frac{1}{2}(8 \times 10^6)(0{,}6)^2 = \SI{1{,}44}{MJ}$
    
    b) $v_f = -e \cdot v_i = -0{,}30 \times 0{,}6 = \SI{-0{,}18}{m/s}$; \quad $K_f = \frac{1}{2}(8 \times 10^6)(0{,}18)^2 = \SI{129{,}6}{kJ}$
    
    Énergie absorbée $= 1440 - 129{,}6 = \SI{1310}{kJ} \approx \SI{1{,}31}{MJ}$
    
    c) $\Delta p = m(v_f - v_i) = 8 \times 10^6 \times (-0{,}18 - 0{,}6) = \SI{-6{,}24 \times 10^6}{N \cdot s}$; \quad $F_{moy} = \dfrac{6{,}24 \times 10^6}{4} = \SI{1{,}56}{MN}$
    
    d) $\frac{1}{2}mv_f^2 = F_r \times d \Rightarrow d = \dfrac{129\,600}{20\,000} = \SI{6{,}5}{m}$
    
    e) $P = F_r \times v = 20\,000 \times 0{,}6 = \SI{12}{kW}$

\textbf{25.} a) $v_{fx} = \dfrac{15 \times 10^6 \times 5}{20 \times 10^6} = \SI{3{,}75}{m/s}$; \quad $v_{fy} = \dfrac{5 \times 10^6 \times 8}{20 \times 10^6} = \SI{2{,}0}{m/s}$
    
    $v_f = \sqrt{3{,}75^2 + 2{,}0^2} = \SI{4{,}25}{m/s} \approx \SI{8{,}3}{\knots}$; \quad Cap $\approx 062\si{\degree}$
    
    b) $K_i = \SI{347{,}5}{MJ}$, $K_f = \SI{180{,}6}{MJ}$. Perte $= \SI{166{,}9}{MJ}$ (48\%). Transformée en déformation des coques, chaleur et son.
    
    c) $\Delta p_x = 5 \times 10^6 \times (3{,}75 - 0) = \SI{18{,}75 \times 10^6}{}$; \quad $\Delta p_y = 5 \times 10^6 \times (2{,}0 - 8) = \SI{-30 \times 10^6}{}$
    
    $|\Delta \vect{p}| = \sqrt{(18{,}75)^2 + (30)^2} \times 10^6 = \SI{35{,}4 \times 10^6}{N \cdot s}$; \quad $F_{moy} = 35{,}4 \times 10^6 / 2 = \SI{17{,}7}{MN}$
    
    d) $d = K_f / F_r = 180{,}6 \times 10^6 / 100\,000 = \SI{1806}{m}$
    
    e) $d = 1806/1852 \approx \SI{0{,}98}{NM}$. Les épaves dérivent près d'un mille nautique, ce qui définit le périmètre initial de recherche pour les opérations de sauvetage.
