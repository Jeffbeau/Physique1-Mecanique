% =============================================================================
% EXERCICES — CHAPITRE 2 : DYNAMIQUE ET STATIQUE
% =============================================================================

\section*{Exercices}
\addcontentsline{toc}{section}{Exercices}
\begin{remarque}[title=Niveaux de difficulté]
\begin{itemize}
    \item[$\star$] Application directe d'une formule
    \item[$\star\star$] Problème à plusieurs étapes ou avec conversion
    \item[$\star\star\star$] Problème complexe ou piège conceptuel
\end{itemize}
\end{remarque}

\textbf{Données utiles :}
\begin{itemize}
    \item Accélération gravitationnelle : $g = \SI{9,81}{m/s^2}$
    \item Constante de gravitation : $G = \SI{6,674e-11}{N \cdot m^2/kg^2}$
    \item Masse de la Terre : $M_T = \SI{5,972e24}{kg}$
    \item Rayon de la Terre : $R_T = \SI{6,378e6}{m}$
    \item 1 nœud $= \SI{0,514}{m/s}$
\end{itemize}

% =============================================================================
\subsection*{A — Concepts fondamentaux et lois de Newton}
% =============================================================================

\begin{enumerate}

% ---- EXERCICE 1 — Vrai ou faux conceptuel ★ --------------------------------
\item[$\star$] \textbf{1. Vrai ou faux.} Pour chaque affirmation, indiquez si elle est vraie ou fausse et \textbf{justifiez} votre réponse en une ou deux phrases.

\begin{enumerate}[label=\alph*)]
    \item Si aucune force nette n'agit sur un objet, celui-ci est nécessairement immobile.
    \item Un navire qui se déplace à vitesse constante en eau calme n'a aucune force qui agit sur lui.
    \item Un objet plus lourd tombe plus vite qu'un objet plus léger (en négligeant la résistance de l'air).
    \item Lorsqu'un remorqueur pousse un navire, le navire exerce sur le remorqueur une force de réaction plus petite, car le navire est plus massif.
    \item La force normale exercée par une surface sur un objet est toujours égale au poids de cet objet.
    \item Si la force résultante sur un objet est nulle, la vitesse de cet objet est nécessairement nulle.
\end{enumerate}

\textit{Réponses :}
\begin{enumerate}[label=\alph*)]
    \item \textbf{Faux.} Selon la première loi de Newton, si la force nette est nulle, l'objet conserve son état de mouvement : il peut être immobile \textit{ou} se déplacer à vitesse constante (MRU).
    \item \textbf{Faux.} Les forces sont présentes (poids, poussée d'Archimède, poussée des moteurs, résistance de l'eau), mais leur somme vectorielle est nulle. C'est un état d'équilibre dynamique.
    \item \textbf{Faux.} La deuxième loi donne $a = F_g/m = mg/m = g$. L'accélération gravitationnelle est la même pour tous les objets, indépendamment de leur masse.
    \item \textbf{Faux.} Selon la troisième loi de Newton, les forces d'action et de réaction sont \textit{toujours} égales en grandeur et opposées en direction, peu importe les masses des objets.
    \item \textbf{Faux.} La normale est égale au poids uniquement sur une surface horizontale sans autres forces verticales. Sur un plan incliné ou si une force a une composante verticale, $N \neq mg$.
    \item \textbf{Faux.} C'est une reformulation du (a). Force résultante nulle signifie accélération nulle, pas vitesse nulle.
\end{enumerate}

% ---- EXERCICE 2 — Identification des forces ★ ------------------------------
\item[$\star$] \textbf{2. Identification des forces.} Pour chacune des situations suivantes, identifiez \textbf{toutes} les forces qui agissent sur l'objet souligné et indiquez leur direction. Ne faites aucun calcul.

\begin{enumerate}[label=\alph*)]
    \item Un \underline{conteneur} est posé sur le pont d'un navire en mer calme. Le navire se déplace à vitesse constante.
    \item Un \underline{navire} est tiré par un remorqueur à vitesse constante dans un port.
    \item Une \underline{caisse} est tirée sur un quai par un câble formant un angle au-dessus de l'horizontale. La caisse accélère et il y a du frottement cinétique.
    \item Un \underline{conteneur} est suspendu, immobile, au câble d'une grue au-dessus du quai.
\end{enumerate}

\textit{Réponses :}
\begin{enumerate}[label=\alph*)]
    \item \textbf{Conteneur sur le pont :} poids $\vect{F}_g$ (vers le bas) et force normale $\vect{N}$ du pont (vers le haut). Deux forces.
    \item \textbf{Navire remorqué :} poids $\vect{F}_g$ (vers le bas), poussée d'Archimède $\vect{F}_A$ (vers le haut), tension du câble de remorquage $\vect{T}$ (vers l'avant), résistance de l'eau $\vect{f}$ (vers l'arrière). Quatre forces. (La somme est nulle puisque $v$ = constante.)
    \item \textbf{Caisse tirée :} poids $\vect{F}_g$ (vers le bas), force normale $\vect{N}$ (vers le haut), tension $\vect{T}$ (à angle vers le haut et l'avant), frottement cinétique $\vect{f}_c$ (vers l'arrière, parallèle à la surface). Quatre forces. (La somme n'est pas nulle : la caisse accélère.)
    \item \textbf{Conteneur suspendu :} poids $\vect{F}_g$ (vers le bas) et tension du câble $\vect{T}$ (vers le haut). Deux forces. ($T = F_g$ en équilibre.)
\end{enumerate}

% ---- EXERCICE 3 — Paires action-réaction ★ ---------------------------------
\item[$\star$] \textbf{3. Paires action-réaction (3\textsuperscript{e} loi de Newton).} Pour chaque force ci-dessous, identifiez la force de réaction : sur quel objet s'exerce-t-elle, par quel objet est-elle exercée, et dans quelle direction agit-elle?

\begin{enumerate}[label=\alph*)]
    \item La Terre exerce une force gravitationnelle vers le \textit{bas} sur un navire.
    \item L'hélice d'un navire pousse l'eau vers l'\textit{arrière}.
    \item Une amarre exerce une tension sur le navire, dirigée vers le quai.
    \item Le pont du navire exerce une force normale vers le \textit{haut} sur un conteneur.
\end{enumerate}

\textit{Réponses :}
\begin{enumerate}[label=\alph*)]
    \item \textbf{Réaction :} Le navire exerce une force gravitationnelle vers le \textit{haut} sur la Terre. (Même grandeur, mais totalement négligeable pour la Terre vu sa masse énorme.)
    \item \textbf{Réaction :} L'eau pousse l'hélice (et donc le navire) vers l'\textit{avant}. C'est cette réaction qui propulse le navire!
    \item \textbf{Réaction :} Le navire exerce une tension sur l'amarre, dirigée \textit{vers le large} (opposée à la direction du quai).
    \item \textbf{Réaction :} Le conteneur exerce une force vers le \textit{bas} sur le pont du navire. (C'est le « poids apparent » du conteneur sur le pont.)
\end{enumerate}

\end{enumerate}

% =============================================================================
\subsection*{B — Diagramme de corps libre et équilibre (statique)}
% =============================================================================

\begin{enumerate}[resume]

% ---- EXERCICE 4 — Tracé de DCL (sans calcul) ★ -----------------------------
\item[$\star$] \textbf{4. Tracé de DCL (sans calcul).} Pour chacune des situations suivantes, tracez le diagramme de corps libre (DCL) de l'objet indiqué. Identifiez chaque force par son symbole, indiquez sa direction et précisez l'objet qui l'exerce.

\begin{enumerate}[label=\alph*)]
    \item Un moteur de \SI{250}{kg} est suspendu au plafond de la salle des machines par un câble vertical unique.
    \item Une caisse de \SI{100}{kg} est immobile sur une rampe de chargement inclinée à $30°$, retenue par le frottement statique.
    \item Un conteneur est retenu en l'air par deux câbles : le premier fait un angle de $20°$ avec la verticale et le second fait un angle de $45°$ avec la verticale.
\end{enumerate}

\textit{Réponses :}
\begin{enumerate}[label=\alph*)]
    \item Deux forces : poids $\vect{F}_g$ (vers le bas) et tension $\vect{T}$ (vers le haut). Le point d'application est le centre de masse du moteur.
    \item Trois forces : poids $\vect{F}_g$ (verticalement vers le bas, au centre de masse), force normale $\vect{N}$ (perpendiculaire à la rampe, vers l'extérieur de la surface) et frottement statique $\vect{f}_s$ (parallèle à la rampe, vers le haut de la pente).
    \item Trois forces : poids $\vect{F}_g$ (vers le bas) et deux tensions $\vect{T}_1$ (le long du premier câble, à $20°$ de la verticale) et $\vect{T}_2$ (le long du second câble, à $45°$ de la verticale).
\end{enumerate}

% ---- EXERCICE 5 — Câbles dans la salle des machines ★★ ---------------------
\item[$\star\star$] \textbf{5. (Maritime)} Un moteur diesel de \SI{300}{kg} est suspendu au point de jonction A par un câble vertical dans la salle des machines d'un navire. Au point A, deux câbles maintiennent le système en équilibre :
\begin{itemize}
    \item le câble AB est fixé au plafond et fait un angle de $40°$ avec l'horizontale;
    \item le câble AD est horizontal et fixé à une cloison.
\end{itemize}

\begin{center}
\begin{tikzpicture}[scale=1.0, >=Stealth]
    % --- Coordonnées ---
    \coordinate (A) at (0, 0);
    \coordinate (B) at ({-3.5*cos(40)}, {3.5*sin(40)});
    \coordinate (D) at (3.2, 0);
    % --- Plafond ---
    \pgfmathsetmacro{\yPlaf}{3.5*sin(40)}
    \fill[pattern=north east lines, pattern color=gray!60] 
        ({-3.5*cos(40) - 0.5}, \yPlaf) rectangle (3.5, {\yPlaf + 0.3});
    \draw[thick] ({-3.5*cos(40) - 0.5}, \yPlaf) -- (3.5, \yPlaf);
    % --- Cloison droite ---
    \fill[pattern=north east lines, pattern color=gray!60] 
        (3.2, -2.6) rectangle (3.5, \yPlaf);
    \draw[thick] (3.2, -2.6) -- (3.2, \yPlaf);
    % --- Câble AB ---
    \draw[thick] (A) -- (B);
    \node[above left=2pt] at ($(A)!0.55!(B)$) {$T_{AB}$};
    % --- Câble AD ---
    \draw[thick] (A) -- (D);
    \node[above=3pt] at ($(A)!0.5!(D)$) {$T_{AD}$};
    % --- Câble vertical ---
    \draw[thick] (A) -- (0, -1.5);
    % --- Point A ---
    \filldraw[black] (A) circle (2.5pt);
    \node[below left=4pt] at (A) {A};
    % --- Points B et D ---
    \filldraw[black] (B) circle (2pt);
    \node[below left=2pt, font=\small] at (B) {B};
    \filldraw[black] (D) circle (2pt);
    \node[above left=2pt, font=\small] at (D) {D};
    % --- Angle 40° ---
    \draw[dashed, gray] (A) -- +(-2.8, 0);
    \draw[thick, blue!70] ($(A) + (180:0.9)$) arc (180:140:0.9);
    \node[blue!70, font=\small] at ($(A) + (160:1.25)$) {$40^\circ$};
    % --- Moteur ---
    \draw[thick, fill=gray!25] (-0.7, -2.3) rectangle (0.7, -1.5);
    \node at (0, -1.9) {M};
    \node[below, font=\small] at (0, -2.4) {\SI{300}{kg}};
\end{tikzpicture}
\end{center}

\begin{enumerate}[label=\alph*)]
    \item Tracez le DCL du point A.
    \item Déterminez la tension dans le câble AB.
    \item Déterminez la tension dans le câble AD.
\end{enumerate}

\textit{Réponses : b) $T_{AB} = \SI{4,58}{kN}$ \quad c) $T_{AD} = \SI{3,51}{kN}$}

% ---- EXERCICE 6 — Grue de chargement ★★ ------------------------------------
\item[$\star\star$] \textbf{6. (Maritime)} Un conteneur de \SI{2500}{kg} est soulevé par la grue d'un navire. Le câble de la grue fait un angle de $20°$ avec la verticale. Une amarre horizontale est attachée au conteneur pour l'empêcher de balancer.

\begin{center}
\begin{tikzpicture}[scale=1.0, >=Stealth]
    % --- Coordonnées ---
    \coordinate (P) at (0, 0);
    \pgfmathsetmacro{\xG}{4.5*sin(20)}
    \pgfmathsetmacro{\yG}{4.5*cos(20)}
    \coordinate (grue) at (\xG, \yG);
    % --- Bras de grue ---
    \draw[very thick, gray!60] (\xG, \yG) -- ({\xG + 2.5}, {\yG + 0.3});
    \draw[very thick, gray!60] ({\xG + 2.5}, {\yG + 0.3}) -- ({\xG + 2.5}, {\yG - 1.0});
    % --- Câble de la grue ---
    \draw[thick] (P) -- (grue);
    \node[right=6pt] at ($(P)!0.5!(grue)$) {$T$};
    % --- Amarre horizontale ---
    \draw[thick] (P) -- (-3.8, 0);
    \node[above=2pt, font=\small] at (-1.9, 0) {amarre};
    \node[below=2pt, font=\small] at (-1.9, 0) {$T_h$};
    % --- Point de fixation de l'amarre ---
    \fill[pattern=north east lines, pattern color=gray!60] 
        (-4.1, -0.4) rectangle (-3.8, 0.4);
    \draw[thick] (-3.8, -0.4) -- (-3.8, 0.4);
    % --- Référence verticale ---
    \draw[dashed, gray] (P) -- (0, 4.0);
    % --- Angle 20° ---
    \draw[thick, blue!70] ($(P) + (90:1.8)$) arc (90:70:1.8);
    \node[left=1pt, blue!70, font=\small] at ($(P) + (82:2.15)$) {$20^\circ$};
    % --- Point d'attache ---
    \filldraw[black] (P) circle (2.5pt);
    % --- Conteneur ---
    \draw[thick, fill=red!12] (-1.1, -1.9) rectangle (1.1, -0.15);
    \node at (0, -1.0) {\SI{2500}{kg}};
\end{tikzpicture}
\end{center}

\begin{enumerate}[label=\alph*)]
    \item Tracez le DCL du conteneur.
    \item Calculez la tension dans le câble de la grue.
    \item Calculez la tension dans l'amarre horizontale.
\end{enumerate}

\textit{Réponses : b) $T = \SI{26,1}{kN}$ \quad c) $T_h = \SI{8,93}{kN}$}

% ---- EXERCICE 7 — Système poulie sur quai ★★ -------------------------------
\item[$\star\star$] \textbf{7. (Maritime)} Sur un quai, une caisse de \SI{40}{kg} est posée sur une surface sans frottement. Un câble attaché à la caisse passe par une poulie fixée en hauteur au bord du quai, puis descend verticalement pour supporter un fût de \SI{25}{kg} au-dessus de l'eau. Le segment de câble entre la caisse et la poulie fait un angle de $30°$ avec l'horizontale. Un câble de retenue horizontal, attaché à un bollard, maintient le système en équilibre.

\begin{center}
\begin{tikzpicture}[scale=0.85, >=Stealth]
    % --- Paramètres ---
    \def\bordQuai{5.0}
    \def\rP{0.35}
    % --- Surface du quai ---
    \fill[pattern=north east lines, pattern color=gray!50] 
        (-4.0, -0.3) rectangle (\bordQuai, 0);
    \draw[thick] (-4.0, 0) -- (\bordQuai, 0);
    % --- Bord du quai ---
    \draw[thick] (\bordQuai, 0) -- (\bordQuai, -3.5);
    \fill[pattern=north east lines, pattern color=gray!50] 
        (\bordQuai, -3.5) rectangle ({\bordQuai + 0.3}, 0);
    % --- Eau ---
    \draw[blue!50, thick, decorate, 
        decoration={snake, amplitude=1.5pt, segment length=8pt}]
        (3.0, -3.5) -- (\bordQuai, -3.5);
    \node[blue!50, font=\footnotesize] at (4.0, -3.8) {eau};
    % --- Caisse ---
    \draw[thick, fill=blue!12] (0.3, 0) rectangle (2.1, 1.1);
    \node[font=\small] at (1.2, 0.55) {\SI{40}{kg}};
    \coordinate (attC) at (2.1, 0.75);
    % --- Poulie ---
    \pgfmathsetmacro{\yP}{0.75 + (\bordQuai - 2.1)*tan(30)}
    \coordinate (poulie) at (\bordQuai, \yP);
    % Support poulie
    \fill[pattern=north east lines, pattern color=gray!50] 
        ({\bordQuai - 0.3}, {\yP + \rP + 0.15}) rectangle ({\bordQuai + 0.3}, {\yP + \rP + 0.45});
    \draw[thick] ({\bordQuai - 0.3}, {\yP + \rP + 0.15}) -- ({\bordQuai + 0.3}, {\yP + \rP + 0.15});
    \draw[thick, gray] (\bordQuai, {\yP + \rP + 0.15}) -- (poulie);
    % Poulie cercle
    \draw[thick, fill=white] (poulie) circle (\rP);
    \filldraw[black] (poulie) circle (1.5pt);
    % --- Câble caisse → poulie ---
    \pgfmathsetmacro{\xTg}{\bordQuai - \rP*sin(30)}
    \pgfmathsetmacro{\yTg}{\yP - \rP*cos(30)}
    \draw[thick] (attC) -- (\xTg, \yTg);
    % --- Câble poulie → fût ---
    \pgfmathsetmacro{\xF}{\bordQuai + \rP}
    \draw[thick] (\xF, \yP) -- (\xF, -0.5);
    % --- Fût ---
    \draw[thick, fill=orange!12] ({\xF - 0.5}, -1.7) rectangle ({\xF + 0.5}, -0.5);
    \node[font=\small] at (\xF, -1.1) {\SI{25}{kg}};
    % --- Angle 30° ---
    \draw[dashed, gray] (attC) -- +(2.2, 0);
    \draw[thick, blue!70] ($(attC) + (0:0.8)$) arc (0:30:0.8);
    \node[right=2pt, blue!70, font=\small] at ($(attC) + (15:1.15)$) {$30^\circ$};
    % --- Bollard ---
    \draw[thick, fill=gray!50] (-2.3, 0) -- (-2.3, 0.65) 
        -- (-2.6, 0.65) -- (-2.6, 0.55) -- (-2.4, 0.55) -- (-2.4, 0) -- cycle;
    \node[above, font=\footnotesize] at (-2.45, 0.70) {bollard};
    % --- Câble de retenue ---
    \draw[thick] (0.3, 0.55) -- (-2.3, 0.55);
    \node[below=1pt, font=\small] at (-1.0, 0.50) {$T_{\text{ret}}$};
\end{tikzpicture}
\end{center}

\begin{enumerate}[label=\alph*)]
    \item Déterminez la tension dans le câble de retenue.
    \item Quelle est la force normale exercée par le quai sur la caisse?
\end{enumerate}

\textit{Réponses : a) $T_{\text{ret}} = \SI{212}{N}$ \quad b) $N = \SI{270}{N}$}

% ---- EXERCICE 8 — Cargaison sur rampe (statique) ★★ ------------------------
\item[$\star\star$] \textbf{8. (Maritime)} Une caisse de \SI{500}{kg} est immobile sur la passerelle de chargement d'un navire, inclinée à $25°$ par rapport à l'horizontale.

\begin{enumerate}[label=\alph*)]
    \item Quel est le coefficient de frottement statique minimal $\mu_{s,\text{min}}$ entre la caisse et la passerelle pour que la caisse ne glisse pas?
    \item Si le coefficient de frottement statique réel est $\mu_s = 0{,}60$, quelle force maximale pourrait-on exercer sur la caisse, parallèlement à la pente et vers le bas, sans qu'elle ne se mette à glisser?
\end{enumerate}

\textit{Réponses : a) $\mu_{s,\text{min}} = 0{,}466$ \quad b) $F_{\text{max}} = \SI{594}{N}$}

\end{enumerate}

% =============================================================================
\subsection*{C — Dynamique : la deuxième loi en action}
% =============================================================================

\begin{enumerate}[resume]

% ---- EXERCICE 9 — Accélération d'un navire ★ -------------------------------
\item[$\star$] \textbf{9. (Maritime)} Le NM \textit{Saaremaa I} ($m = \SI{8500}{tonnes}$) accélère de 0 à 12 nœuds en \SI{90}{s}.

\begin{enumerate}[label=\alph*)]
    \item En supposant l'accélération constante et en négligeant la résistance de l'eau, calculez la force nette requise.
    \item Si la résistance de l'eau est de \SI{150}{kN} pendant cette phase d'accélération, quelle est la poussée réelle des moteurs?
\end{enumerate}

\textit{Réponses : a) $F_{\text{net}} = \SI{583}{kN}$ \quad b) $F_{\text{poussée}} = \SI{733}{kN}$}

% ---- EXERCICE 10 — Freinage d'urgence ★ ------------------------------------
\item[$\star$] \textbf{10. (Maritime)} Un conteneur de \SI{8}{tonnes} est posé sur le pont d'un navire. Lors d'un freinage d'urgence, le navire décélère à raison de \SI{0,5}{m/s^2}.

\begin{enumerate}[label=\alph*)]
    \item Quelle force de frottement minimale est requise pour empêcher le conteneur de glisser sur le pont?
    \item Si le coefficient de frottement statique entre le conteneur et le pont est $\mu_s = 0{,}40$, le conteneur glisse-t-il?
\end{enumerate}

\textit{Réponses : a) $f = \SI{4,00}{kN}$ \quad b) $f_{s,\text{max}} = \SI{31,4}{kN} \gg \SI{4,00}{kN}$ : le conteneur ne glisse pas.}

% ---- EXERCICE 11 — Force d'arrêt (dynamique + cinématique) ★★ ---------------
\item[$\star\star$] \textbf{11. (Maritime)} Lors de l'amarrage, un navire de \SI{5000}{tonnes} se déplaçant à 2 nœuds est immobilisé sur une distance de \SI{5}{m} par les amarres et les défenses du quai.

\begin{enumerate}[label=\alph*)]
    \item Calculez la décélération du navire.
    \item Calculez la force moyenne exercée par les amarres et les défenses.
\end{enumerate}

\textit{Réponses : a) $a = \SI{0,106}{m/s^2}$ \quad b) $F = \SI{528}{kN}$}

% ---- EXERCICE 12 — Treuil à angle sur quai ★★ ------------------------------
\item[$\star\star$] \textbf{12. (Maritime)} Un treuil tire une caisse de \SI{80}{kg} sur un quai à l'aide d'un câble faisant un angle de $25°$ au-dessus de l'horizontale. Le coefficient de frottement cinétique entre la caisse et le quai est $\mu_c = 0{,}30$. La tension dans le câble est de \SI{350}{N}.

\begin{enumerate}[label=\alph*)]
    \item Calculez la force normale.
    \item Calculez la force de frottement cinétique.
    \item Calculez l'accélération de la caisse.
\end{enumerate}

\textit{Réponses : a) $N = \SI{637}{N}$ \quad b) $f_c = \SI{191}{N}$ \quad c) $a = \SI{1,58}{m/s^2}$}

% ---- EXERCICE 13 — Descente de rampe ★★ ------------------------------------
\item[$\star\star$] \textbf{13. (Maritime)} Une palette de \SI{200}{kg} est relâchée du repos au sommet d'une rampe de chargement inclinée à $30°$ et longue de \SI{4}{m}. Le coefficient de frottement cinétique est $\mu_c = 0{,}20$.

\begin{enumerate}[label=\alph*)]
    \item Calculez l'accélération de la palette sur la rampe.
    \item Quelle est la vitesse de la palette au bas de la rampe?
\end{enumerate}

\textit{Réponses : a) $a = \SI{3,21}{m/s^2}$ \quad b) $v = \SI{5,06}{m/s}$}

% ---- EXERCICE 14 — Rampe puis plancher ★★ -----------------------------------
\item[$\star\star$] \textbf{14. (Maritime)} Un baril de \SI{40}{kg} part du repos et descend une rampe inclinée à $25°$, longue de \SI{2}{m}, avec un coefficient de frottement cinétique $\mu_c = 0{,}15$. Au bas de la rampe, le baril continue sur le plancher horizontal de la cale, où le coefficient de frottement cinétique est $\mu_c = 0{,}20$. À quelle distance du bas de la rampe le baril s'immobilise-t-il?

\textit{Indice : Ce problème se résout en deux étapes. Trouvez d'abord la vitesse au bas de la rampe, puis utilisez-la comme vitesse initiale sur le plancher.}

\textit{Réponse : $d = \SI{2,87}{m}$}

% ---- EXERCICE 15 — Système poulie avec rampe ★★★ ----------------------------
\item[$\star\star\star$] \textbf{15. (Maritime)} Lors du chargement d'un navire, une caisse de \SI{50}{kg} est posée sur une rampe inclinée à $30°$ avec un coefficient de frottement cinétique $\mu_c = 0{,}15$. Elle est reliée par un câble passant par une poulie sans frottement au sommet de la rampe à un contrepoids de \SI{40}{kg} suspendu dans le vide. Le système est relâché du repos.

\begin{center}
\begin{tikzpicture}[scale=0.85, >=Stealth]
    % --- Paramètres ---
    \def\angR{30}
    \def\Lramp{7.0}
    \def\rP{0.35}
    % --- Coordonnées de la rampe ---
    \coordinate (bas) at (0, 0);
    \pgfmathsetmacro{\xH}{\Lramp*cos(\angR)}
    \pgfmathsetmacro{\yH}{\Lramp*sin(\angR)}
    \coordinate (haut) at (\xH, \yH);
    % --- Sol ---
    \fill[pattern=north east lines, pattern color=gray!50] 
        (-1.5, -0.3) rectangle ({\xH + 2.0}, 0);
    \draw[thick] (-1.5, 0) -- (bas);
    \draw[thick] (\xH, 0) -- ({\xH + 2.0}, 0);
    % --- Rampe ---
    \draw[very thick] (bas) -- (haut);
    \pgfmathsetmacro{\dx}{0.15*sin(\angR)}
    \pgfmathsetmacro{\dy}{0.15*cos(\angR)}
    \fill[gray!15] (bas) -- (haut) -- ({\xH + \dx}, {\yH - \dy}) 
        -- (\dx, {-\dy}) -- cycle;
    \draw[thick, gray!70] (\dx, {-\dy}) -- ({\xH + \dx}, {\yH - \dy});
    % --- Pilier vertical ---
    \draw[thick] (\xH, 0) -- (haut);
    % --- Angle de la rampe ---
    \draw[thick, blue!70] ($(bas) + (0:1.5)$) arc (0:\angR:1.5);
    \node[right=2pt, blue!70, font=\small] at ($(bas) + (15:1.85)$) {$30^\circ$};
    % --- Caisse sur la rampe ---
    \pgfmathsetmacro{\posF}{0.38}
    \pgfmathsetmacro{\xC}{\posF*\Lramp*cos(\angR)}
    \pgfmathsetmacro{\yC}{\posF*\Lramp*sin(\angR)}
    \coordinate (cBase) at (\xC, \yC);
    \begin{scope}[shift={(cBase)}, rotate=\angR]
        \draw[thick, fill=blue!12] (-0.6, 0) rectangle (0.6, 0.85);
    \end{scope}
    % Label 50 kg — horizontal, à gauche de la caisse
    \pgfmathsetmacro{\xLab}{\xC - 0.6*cos(\angR) - 0.3}
    \pgfmathsetmacro{\yLab}{\yC - 0.6*sin(\angR) + 0.85}
    \node[left, font=\small] at (\xLab, \yLab) {\SI{50}{kg}};
    % --- Annotation frottement ---
    \node[font=\footnotesize, gray!60] at (-1.0, 0.8) {$\mu_c = 0{,}15$};
    % --- Point d'attache câble ---
    \pgfmathsetmacro{\xCab}{\xC + 0.6*cos(\angR)}
    \pgfmathsetmacro{\yCab}{\yC + 0.6*sin(\angR)}
    \coordinate (cabStart) at (\xCab, \yCab);
    % --- Poulie au sommet ---
    \pgfmathsetmacro{\xPl}{\xH}
    \pgfmathsetmacro{\yPl}{\yH + \rP + 0.1}
    \coordinate (poul) at (\xPl, \yPl);
    \draw[thick, gray] (haut) -- (poul);
    \draw[thick, fill=white] (poul) circle (\rP);
    \filldraw[black] (poul) circle (1.5pt);
    % --- Câble caisse → poulie ---
    \pgfmathsetmacro{\xTr}{\xPl - \rP*sin(\angR)}
    \pgfmathsetmacro{\yTr}{\yPl - \rP*cos(\angR)}
    \draw[thick] (cabStart) -- (\xTr, \yTr);
    \node[above left=2pt, font=\small] at ($(cabStart)!0.55!(\xTr, \yTr)$) {câble};
    % --- Câble poulie → contrepoids ---
    \pgfmathsetmacro{\xDr}{\xPl + \rP}
    \draw[thick] (\xDr, \yPl) -- (\xDr, 1.0);
    % --- Contrepoids ---
    \draw[thick, fill=orange!12] ({\xDr - 0.5}, -0.1) rectangle ({\xDr + 0.5}, 1.0);
    \node[font=\small] at (\xDr, 0.45) {\SI{40}{kg}};
\end{tikzpicture}
\end{center}

\begin{enumerate}[label=\alph*)]
    \item Dans quelle direction le système se met-il en mouvement? Justifiez par une comparaison des forces.
    \item Calculez l'accélération du système.
    \item Quelle est la tension dans le câble?
\end{enumerate}

\textit{Indice : Comparez le poids du contrepoids avec la somme de la composante du poids de la caisse parallèle à la rampe et du frottement.}

\textit{Réponses : a) Le contrepoids descend (la caisse monte). \quad b) $a = \SI{0,93}{m/s^2}$ \quad c) $T = \SI{355}{N}$}

\end{enumerate}

% =============================================================================
\subsection*{D — Mouvement circulaire}
% =============================================================================

\begin{enumerate}[resume]

% ---- EXERCICE 16 — Hélice de navire ★ --------------------------------------
\item[$\star$] \textbf{16. (Maritime)} L'hélice d'un cargo a un diamètre de \SI{5}{m} et tourne à \SI{150}{RPM}.

\begin{enumerate}[label=\alph*)]
    \item Calculez la vitesse angulaire de l'hélice en rad/s.
    \item Calculez la vitesse tangentielle à l'extrémité d'une pale.
    \item Calculez l'accélération centripète subie par un point à l'extrémité d'une pale.
\end{enumerate}

\textit{Réponses : a) $\omega = \SI{15,7}{rad/s}$ \quad b) $v = \SI{39,3}{m/s}$ (\SI{141}{km/h}) \quad c) $a_c = \SI{617}{m/s^2}$}

% ---- EXERCICE 17 — Navire en virage ★★ -------------------------------------
\item[$\star\star$] \textbf{17. (Maritime)} Un pétrolier de \SI{60\,000}{tonnes} effectue un virage avec un rayon de giration de \SI{700}{m}. Sa vitesse est de 12 nœuds.

\begin{enumerate}[label=\alph*)]
    \item Calculez la force centripète nécessaire pour maintenir ce virage.
    \item En combien de temps le navire effectue-t-il un demi-tour ($180°$)?
    \item Si le capitaine réduit la vitesse à 8 nœuds, quelle est la nouvelle force centripète?
\end{enumerate}

\textit{Réponses : a) $F_c = \SI{3,26}{MN}$ \quad b) $t = $ 5~min~57~s \quad c) $F_c = \SI{1,45}{MN}$}

% ---- EXERCICE 18 — Cargaison en virage ★★ ----------------------------------
\item[$\star\star$] \textbf{18. (Maritime)} Un conteneur de \SI{15}{tonnes} est arrimé sur le pont d'un navire. Le coefficient de frottement statique entre le conteneur et le pont est $\mu_s = 0{,}45$.

\begin{enumerate}[label=\alph*)]
    \item Si le navire effectue un virage de rayon \SI{500}{m}, quelle est la vitesse maximale pour que le conteneur ne glisse pas?
    \item À 10 nœuds, quel est le rayon de virage minimal sécuritaire?
\end{enumerate}

\textit{Réponses : a) $v_{\text{max}} = \SI{47,0}{m/s}$ (91 nœuds) \quad b) $r_{\text{min}} = \SI{5,98}{m}$}

\begin{remarque}[title=Interprétation physique]
Ces résultats montrent que, pour les vitesses typiques de navigation, le frottement statique sur le pont est largement suffisant pour retenir la cargaison dans un virage. Le vrai danger pour la cargaison vient des accélérations en tangage et en roulis (mouvements du navire dans les vagues), qui sont beaucoup plus violentes qu'un simple virage.
\end{remarque}

% ---- EXERCICE 19 — Satellite de navigation ★★ ------------------------------
\item[$\star\star$] \textbf{19. (Maritime)} Un satellite GPS de \SI{2000}{kg} orbite à une altitude de \SI{20\,200}{km} au-dessus de la surface terrestre. Ces satellites sont essentiels à la navigation maritime moderne.

\begin{enumerate}[label=\alph*)]
    \item Calculez l'accélération gravitationnelle à cette altitude.
    \item Calculez la vitesse orbitale du satellite.
    \item Calculez la période orbitale et comparez avec la valeur connue d'environ 12~h.
\end{enumerate}

\textit{Réponses : a) $g = \SI{0,564}{m/s^2}$ \quad b) $v = \SI{3870}{m/s}$ (\SI{13\,900}{km/h}) \quad c) $T = \SI{43\,100}{s} \approx \SI{12,0}{h}$}

% ---- EXERCICE 20 — Sommet de vague ★★★ -------------------------------------
\item[$\star\star\star$] \textbf{20. (Maritime)} Un navire passe au sommet d'une vague dont le profil a un rayon de courbure de \SI{40}{m}.

\begin{enumerate}[label=\alph*)]
    \item À quelle vitesse un matelot de \SI{80}{kg} situé sur le pont aurait-il un poids apparent nul (sensation d'apesanteur)?
    \item Si le navire se déplace à 15 nœuds, quel est le poids apparent du matelot au sommet de la vague? Exprimez-le en newtons et en pourcentage de son poids normal.
\end{enumerate}

\textit{Réponses : a) $v = \SI{19,8}{m/s}$ (38,5 nœuds) \quad b) $N = \SI{666}{N}$, soit $85\%$ du poids normal.}

\end{enumerate}

% =============================================================================
\subsection*{E — Synthèse et intégration}
% =============================================================================

\begin{enumerate}[resume]

% ---- EXERCICE 21 — Manœuvre complète ★★★ ------------------------------------
\item[$\star\star\star$] \textbf{21. (Maritime — Manœuvre complète)} Le traversier \textit{F.A. Gauthier} ($m = \SI{5000}{tonnes}$) quitte le port de Matane.

\textbf{Phase 1 :} Le navire accélère de 0 à 14 nœuds en 2 minutes.

\textbf{Phase 2 :} À vitesse constante de 14 nœuds, le navire effectue un virage de rayon \SI{500}{m}.

\textbf{Phase 3 :} Sur le pont du navire, un véhicule de \SI{2000}{kg} n'est pas arrimé. Le coefficient de frottement statique entre les pneus et le pont est $\mu_s = 0{,}40$.

\begin{enumerate}[label=\alph*)]
    \item Calculez la force de poussée des moteurs pendant la phase 1. (Négligez la résistance de l'eau.)
    \item Calculez la force centripète nécessaire pendant le virage.
    \item Le véhicule glisse-t-il pendant le virage? Justifiez par un calcul.
    \item Quelle serait la vitesse maximale du traversier pour que le véhicule ne glisse pas dans un virage de ce rayon?
\end{enumerate}

\textit{Réponses : a) $F = \SI{300}{kN}$ \quad b) $F_c = \SI{518}{kN}$ \quad c) Force centripète requise sur le véhicule : \SI{207}{N}; frottement statique maximal disponible : \SI{7850}{N}. Non, le véhicule ne glisse pas. \quad d) $v_{\text{max}} = \SI{44,3}{m/s}$ (86 nœuds, bien au-delà des vitesses de navigation).}

% ---- EXERCICE 22 — Opération de chargement ★★★ ------------------------------
\item[$\star\star\star$] \textbf{22. (Maritime — Opération de chargement)} Une caisse de \SI{2000}{kg} doit être chargée sur un navire.

\textbf{Étape 1 :} La caisse est soulevée par une grue. Le câble fait un angle de $15°$ avec la verticale et une amarre horizontale empêche la caisse de balancer.

\textbf{Étape 2 :} La caisse est déposée sur le pont, puis poussée horizontalement par un chariot élévateur avec une force de \SI{5000}{N}. Le coefficient de frottement cinétique entre la caisse et le pont est $\mu_c = 0{,}20$.

\begin{enumerate}[label=\alph*)]
    \item Calculez les tensions dans le câble de la grue et dans l'amarre horizontale (étape 1).
    \item Calculez l'accélération de la caisse sur le pont (étape 2).
    \item Si la caisse part du repos et est poussée sur une distance de \SI{10}{m}, quelle est sa vitesse finale?
\end{enumerate}

\textit{Réponses : a) $T_{\text{câble}} = \SI{20,3}{kN}$; $T_{\text{amarre}} = \SI{5,26}{kN}$ \quad b) $a = \SI{0,538}{m/s^2}$ \quad c) $v = \SI{3,28}{m/s}$}

% ---- EXERCICE 23 — Estimation : distance de freinage ★★ ---------------------
\item[$\star\star$] \textbf{23. (Maritime — Estimation)} Un porte-conteneurs de \SI{100\,000}{tonnes} navigue à 20 nœuds. La force de freinage maximale (inversion des moteurs et résistance de l'eau) est estimée à \SI{2000}{kN}.

\begin{enumerate}[label=\alph*)]
    \item Estimez la distance de freinage.
    \item Estimez le temps d'arrêt.
    \item Comparez avec une automobile de \SI{1500}{kg} roulant à \SI{100}{km/h} et freinant avec une force de \SI{10\,000}{N}. Commentez les implications pour la navigation maritime.
\end{enumerate}

\textit{Réponses : a) $d \approx \SI{2,6}{km}$ \quad b) $t \approx $ 8~min~34~s \quad c) Pour l'automobile : $d \approx \SI{58}{m}$, $t \approx \SI{4}{s}$. La distance de freinage du navire est environ 46 fois plus grande. Cela explique pourquoi les règles d'anticollision en mer exigent des manœuvres préventives bien à l'avance et pourquoi la veille maritime est si importante.}

\end{enumerate}